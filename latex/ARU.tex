%% Generated by Sphinx.
\def\sphinxdocclass{report}
\documentclass[letterpaper,10pt,russian,openany]{sphinxmanual}
\ifdefined\pdfpxdimen
   \let\sphinxpxdimen\pdfpxdimen\else\newdimen\sphinxpxdimen
\fi \sphinxpxdimen=.75bp\relax
\ifdefined\pdfimageresolution
    \pdfimageresolution= \numexpr \dimexpr1in\relax/\sphinxpxdimen\relax
\fi
%% let collapsible pdf bookmarks panel have high depth per default
\PassOptionsToPackage{bookmarksdepth=5}{hyperref}
%% turn off hyperref patch of \index as sphinx.xdy xindy module takes care of
%% suitable \hyperpage mark-up, working around hyperref-xindy incompatibility
\PassOptionsToPackage{hyperindex=false}{hyperref}
%% memoir class requires extra handling
\makeatletter\@ifclassloaded{memoir}
{\ifdefined\memhyperindexfalse\memhyperindexfalse\fi}{}\makeatother

\PassOptionsToPackage{warn}{textcomp}

\catcode`^^^^00a0\active\protected\def^^^^00a0{\leavevmode\nobreak\ }
\usepackage{cmap}
\usepackage{fontspec}
\defaultfontfeatures[\rmfamily,\sffamily,\ttfamily]{}
\usepackage{amsmath,amssymb,amstext}
\usepackage{polyglossia}
\setmainlanguage{russian}



\setmainfont{DejaVu Serif}
\setsansfont{DejaVu Sans}
\setmonofont{DejaVu Sans Mono}



\usepackage[Sonny]{fncychap}
\ChNameVar{\Large\normalfont\sffamily}
\ChTitleVar{\Large\normalfont\sffamily}
\usepackage{sphinx}

\fvset{fontsize=\small}
\usepackage{geometry}


% Include hyperref last.
\usepackage{hyperref}
% Fix anchor placement for figures with captions.
\usepackage{hypcap}% it must be loaded after hyperref.
% Set up styles of URL: it should be placed after hyperref.
\urlstyle{same}

\addto\captionsrussian{\renewcommand{\contentsname}{Содержание:}}

\usepackage{sphinxmessages}
\setcounter{tocdepth}{4}
\setcounter{secnumdepth}{4}


\title{Arch Linux Optimization Guide (RU)}
\date{июн. 06, 2022}
\release{2022.02.28}
\author{Arch Linux Optimization Guide (RU)}
\newcommand{\sphinxlogo}{\vbox{}}
\renewcommand{\releasename}{Выпуск}
\makeindex
\begin{document}

\pagestyle{empty}
\sphinxmaketitle
\pagestyle{plain}
\sphinxtableofcontents
\pagestyle{normal}
\phantomsection\label{\detokenize{index::doc}}


\sphinxAtStartPar
Это помощник по настройке вашей системы Arch Linux с целью получить максимальную производительность и настроить систему для комфортной игры.
Здесь вы также сможете найти руководства по оптимизации DE (рабочих окружений) GNOME/KDE/Cinnamon и немного для Xfce.
Проект не претендует на замену Arch Wiki, он лишь является практическим руководством, написанным на основе личного опыта его авторов.

\sphinxAtStartPar
Данный репозиторий является зеркалом, а ныне и основным хранилищем одноименного руководства из Google Docs.

\sphinxstepscope


\chapter{Предисловие}
\label{\detokenize{source/preface:preface}}\label{\detokenize{source/preface:id1}}\label{\detokenize{source/preface::doc}}
\sphinxAtStartPar
Привет, неизвестный мне чувак из интернета, раз ты тут,
то возможно жаждешь настроить свою систему на максимальный выхлоп, но прежде чем начать \sphinxhyphen{} знай:
\sphinxstylestrong{Все манипуляции на вашей совести и авторы не несут никакой ответственности},
но если вам нужна помощь или что\sphinxhyphen{}то не понятно  \sphinxhyphen{} создайте тему на \sphinxhref{https://github.com/ventureoo/ARU/issues}{GitHub Issues}

\sphinxAtStartPar
На текущий момент руководство находится в активной переработке,
поэтому если вы нашли ошибку или просто хотите внести свой вклад в проект \sphinxhyphen{} отправьте на рассмотрение Pull Request в наш GitHub репозиторий.
Мы будем признательны за ваше участие.

\index{basics@\spxentry{basics}}\index{commands@\spxentry{commands}}\index{pacman@\spxentry{pacman}}\ignorespaces 

\section{Основные консольные команды}
\label{\detokenize{source/preface:basic-commands}}\label{\detokenize{source/preface:index-0}}\label{\detokenize{source/preface:id2}}
\begin{sphinxVerbatim}[commandchars=\\\{\}]
sudo pacman \PYGZhy{}S    \PYG{c+c1}{\PYGZsh{} Установить программу из основных репозиториев.}
sudo pacman \PYGZhy{}Syu  \PYG{c+c1}{\PYGZsh{} Выполнить полное обновление системы и репозиториев}
sudo pacman \PYGZhy{}R    \PYG{c+c1}{\PYGZsh{} Удалить пакет}
sudo pacman \PYGZhy{}Rsn  \PYG{c+c1}{\PYGZsh{} Удалить пакет и его зависимости}
git clone         \PYG{c+c1}{\PYGZsh{} Клонирует внешний git репозиторий, например AUR пакет}
makepkg \PYGZhy{}si       \PYG{c+c1}{\PYGZsh{} Осуществляет сборку пакета и его установку из PKGBUILD}
\PYG{n+nb}{cd}                \PYG{c+c1}{\PYGZsh{} Перейти в директорию, например: cd tools.}
ls                \PYG{c+c1}{\PYGZsh{} Показать файлы и папку внутри другой папки.}
\end{sphinxVerbatim}

\index{aur@\spxentry{aur}}\index{helpers@\spxentry{helpers}}\index{packaging@\spxentry{packaging}}\ignorespaces 

\subsection{Пару слов об AUR помощниках}
\label{\detokenize{source/preface:aur}}\label{\detokenize{source/preface:aur-helpers}}\label{\detokenize{source/preface:index-1}}
\sphinxAtStartPar
Далее в руководстве все пакеты из \sphinxhref{https://wiki.archlinux.org/title/Arch\_User\_Repository}{AUR}
(Arch Linux Repository) будут устанавливаться и собираться, если так можно выразиться, "дедовским" способом,
т.е. через стандартные утилиты git и makepkg, без применения так называемых "\sphinxhref{https://wiki.archlinux.org/title/AUR\_helpers}{AUR Помощников}".
Это сделано по причине их быстрой сменяемости, и тот помощник который был актуален раньше, может стать устаревшим и никому не нужным.
Для примера, так было с AUR\sphinxhyphen{}помощником yaourt. И кроме того, согласно Arch Wiki AUR\sphinxhyphen{}помощники "официально" не поддерживаются дистрибутивом.
А "старый" метод, через обычное клонирование git репозитория из AUR командой git clone и сборка пакета через makepkg, будет работать всегда.
Тем не менее, обращаем ваше внимание, что возможность установки пакетов через AUR помощник возможна,
и вы можете её использовать для всех AUR пакетов о которых пойдет речь далее.
Подробнее об этом можно почитать \sphinxhref{https://wiki.archlinux.org/index.php/AUR\_helpers}{здесь}.

\sphinxstepscope


\chapter{Первые шаги}
\label{\detokenize{source/first-steps:first-steps}}\label{\detokenize{source/first-steps:id1}}\label{\detokenize{source/first-steps::doc}}
\index{pacman@\spxentry{pacman}}\index{settings@\spxentry{settings}}\ignorespaces 

\section{Настройка pacman}
\label{\detokenize{source/first-steps:pacman}}\label{\detokenize{source/first-steps:pacman-settings}}\label{\detokenize{source/first-steps:index-0}}
\index{pacman@\spxentry{pacman}}\index{key@\spxentry{key}}\index{gpg@\spxentry{gpg}}\ignorespaces 

\subsection{Обновление ключей Arch Linux}
\label{\detokenize{source/first-steps:arch-linux}}\label{\detokenize{source/first-steps:gpg-update}}\label{\detokenize{source/first-steps:index-1}}
\sphinxAtStartPar
Обновление ключей необходимо во избежание дальнейших проблем с установкой пакетов:

\begin{sphinxVerbatim}[commandchars=\\\{\}]
sudo pacman\PYGZhy{}key \PYGZhy{}\PYGZhy{}init               \PYG{c+c1}{\PYGZsh{} Инициализация}
sudo pacman\PYGZhy{}key \PYGZhy{}\PYGZhy{}populate archlinux \PYG{c+c1}{\PYGZsh{} Получить ключи из репозитория}
sudo pacman\PYGZhy{}key \PYGZhy{}\PYGZhy{}refresh\PYGZhy{}keys       \PYG{c+c1}{\PYGZsh{} Проверить текущие ключи на актуальность}
sudo pacman \PYGZhy{}Sy                      \PYG{c+c1}{\PYGZsh{} Обновить ключи для всей системы}
\end{sphinxVerbatim}

\sphinxAtStartPar
Данная операция может занять продолжительное время.

\index{pacman@\spxentry{pacman}}\index{multilib@\spxentry{multilib}}\index{wine@\spxentry{wine}}\index{steam@\spxentry{steam}}\ignorespaces 

\subsection{Включение 32\sphinxhyphen{}битного репозитория}
\label{\detokenize{source/first-steps:multilib-repository}}\label{\detokenize{source/first-steps:index-2}}\label{\detokenize{source/first-steps:id2}}
\sphinxAtStartPar
Убедимся, что конфигурация пакетного менеджера Pacman настроена для получения доступа к 32\sphinxhyphen{}битным зависимостям, нужным в частности для установки Wine и Steam.

\sphinxAtStartPar
Для этого раскомментируем так называемый \sphinxstyleemphasis{multilib} репозиторий:

\begin{sphinxVerbatim}[commandchars=\\\{\}]
sudo nano /etc/pacman.conf           \PYG{c+c1}{\PYGZsh{} Раскоментируйте последние две строчки как на скриншоте}
\end{sphinxVerbatim}

\noindent\sphinxincludegraphics{{first-steps-1}.png}

\begin{sphinxVerbatim}[commandchars=\\\{\}]
sudo pacman \PYGZhy{}Suy                     \PYG{c+c1}{\PYGZsh{} Обновление репозиториев и всех программ (пакетов)}
\end{sphinxVerbatim}

\index{pacman@\spxentry{pacman}}\index{mirrorlist@\spxentry{mirrorlist}}\index{reflector@\spxentry{reflector}}\ignorespaces 

\subsection{Ускорение обновления системы}
\label{\detokenize{source/first-steps:speed-up-system-updates}}\label{\detokenize{source/first-steps:index-3}}\label{\detokenize{source/first-steps:id3}}
\sphinxAtStartPar
Утилита Reflector отсортирует доступные репозитории по скорости:

\begin{sphinxVerbatim}[commandchars=\\\{\}]
sudo pacman \PYGZhy{}S reflector rsync curl  \PYG{c+c1}{\PYGZsh{} Установка reflector и его зависимостей}
\end{sphinxVerbatim}

\sphinxAtStartPar
Если вы из Европейской части России, то всегда советуем использовать зеркала Германии,
так как их больше всего и они имеют оптимальную свежесть/скорость:

\begin{sphinxVerbatim}[commandchars=\\\{\}]
sudo reflector \PYGZhy{}\PYGZhy{}verbose \PYGZhy{}\PYGZhy{}country \PYG{l+s+s1}{\PYGZsq{}Germany\PYGZsq{}} \PYGZhy{}l \PYG{l+m}{25} \PYGZhy{}\PYGZhy{}sort rate \PYGZhy{}\PYGZhy{}save /etc/pacman.d/mirrorlist
\end{sphinxVerbatim}

\sphinxAtStartPar
Если вы проживаете не на территории Европейской части РФ или в иной стране, то просто измените \sphinxstyleemphasis{Germany} на \sphinxstyleemphasis{Russia} или ваше государство.

\sphinxAtStartPar
Можно также вручную отредактировать список зеркал, добавив туда зеркала из постоянно обновляющегося перечня на сайте Arch Linux (\sphinxurl{https://archlinux.org/mirrorlist/})

\begin{sphinxVerbatim}[commandchars=\\\{\}]
sudo nano /etc/pacman.d/mirrorlist \PYG{c+c1}{\PYGZsh{} Рекомендуем прописывать зеркала от Яндекса}
\end{sphinxVerbatim}

\index{installation@\spxentry{installation}}\index{basic@\spxentry{basic}}\index{packages@\spxentry{packages}}\ignorespaces 

\section{Установка базовых пакетов и набора программ}
\label{\detokenize{source/first-steps:basic-software-installation}}\label{\detokenize{source/first-steps:index-4}}\label{\detokenize{source/first-steps:id4}}
\sphinxAtStartPar
Вот основной набор программ который мы можем вам порекомендовать к установке первым делом:

\begin{sphinxVerbatim}[commandchars=\\\{\}]
sudo pacman \PYGZhy{}Syu base\PYGZhy{}devel \PYG{l+s+se}{\PYGZbs{} }\PYG{c+c1}{\PYGZsh{} Обязательная к установке группа!}
                 nano \PYG{l+s+se}{\PYGZbs{} }\PYG{c+c1}{\PYGZsh{} Минималистичный консольный редактор}
                 git \PYG{l+s+se}{\PYGZbs{} }\PYG{c+c1}{\PYGZsh{} Консольный Git, нужен для установки AUR пакетов}
                 chromium \PYG{l+s+se}{\PYGZbs{} }\PYG{c+c1}{\PYGZsh{} Браузер, на ваш выбор}
                 gvfs \PYG{l+s+se}{\PYGZbs{} }\PYG{c+c1}{\PYGZsh{} Нужен для корзины}
                 ccache \PYG{l+s+se}{\PYGZbs{} }\PYG{c+c1}{\PYGZsh{} Ускоряет дальнейшее перекомпилирование больших пакетов}
                 vlc \PYG{l+s+se}{\PYGZbs{} }\PYG{c+c1}{\PYGZsh{} Плеер, на ваш выбор}
                 steam \PYG{l+s+se}{\PYGZbs{} }\PYG{c+c1}{\PYGZsh{} Steam, можно также установить steam\PYGZhy{}runtime}
                 bleachbit  \PYG{l+s+se}{\PYGZbs{} }\PYG{c+c1}{\PYGZsh{} Программа очистки системы}
                 grub\PYGZhy{}customizer \PYG{l+s+se}{\PYGZbs{} }\PYG{c+c1}{\PYGZsh{} Графический менеджер настройки GRUB}
                 unrar \PYG{l+s+se}{\PYGZbs{} }\PYG{c+c1}{\PYGZsh{} Для поддержки архивов формата rar}
                 unzip \PYG{l+s+se}{\PYGZbs{} }\PYG{c+c1}{\PYGZsh{} Для поддержки архивов формата zip}
                 file\PYGZhy{}roller \PYG{l+s+se}{\PYGZbs{} }\PYG{c+c1}{\PYGZsh{} Минималистичный менеджер архивов}
                 qbittorrent \PYG{l+s+se}{\PYGZbs{} }\PYG{c+c1}{\PYGZsh{} Торрент\PYGZhy{}клиент, на ваш выбор}
                 unace \PYG{l+s+se}{\PYGZbs{} }\PYG{c+c1}{\PYGZsh{} Для поддержки архивов формата ace}
                 lrzip \PYG{l+s+se}{\PYGZbs{} }\PYG{c+c1}{\PYGZsh{} Для поддержки сжатия через rzip/lcma/lzo и т.д.}
                 squashfs\PYGZhy{}tools
\end{sphinxVerbatim}

\sphinxAtStartPar
Дополнительно можно отметить легковесный файловый менеджер PCManFM:

\begin{sphinxVerbatim}[commandchars=\\\{\}]
sudo pacman \PYGZhy{}S pcmanfm\PYGZhy{}gtk3 gvfs
\end{sphinxVerbatim}

\sphinxAtStartPar
Мы установили набор джентльмена и парочку программ, что понадобятся нам в дальнейшем.
Но если вас не устраивает тот или иной компонент, вы всегда можете найти любой пакет по адресу \sphinxurl{https://www.archlinux.org/packages/}
или установить из AUR (т.е. скомпилировать) по адресу \sphinxurl{https://aur.archlinux.org/packages/}.

\index{installation@\spxentry{installation}}\index{drivers@\spxentry{drivers}}\index{nvidia@\spxentry{nvidia}}\index{amd@\spxentry{amd}}\index{intel@\spxentry{intel}}\ignorespaces 

\subsection{Установка актуальных драйверов для видеокарты}
\label{\detokenize{source/first-steps:drivers-installation}}\label{\detokenize{source/first-steps:index-5}}\label{\detokenize{source/first-steps:id5}}
\sphinxAtStartPar
В установке драйверов для Linux\sphinxhyphen{}систем нет ничего сложного, главное просто учитывать, что от свежести ядра и версии драйвера,
будет зависеть получите ли вы чёрный экран смерти или нет (Шутка).

\sphinxAtStartPar
И да, \sphinxstylestrong{устанавливайте драйвера ТОЛЬКО через пакетный менеджер вашего дистрибутива!}

\sphinxAtStartPar
Забудьте про скачивание драйвера с сайта NVIDIA/AMD, это поможет вам избежать кучу проблем в дальнейшем.


\subsubsection{NVIDIA}
\label{\detokenize{source/first-steps:nvidia}}
\sphinxAtStartPar
В гайде мы установим драйвер версии DKMS, который сам подстроится под нужное ядро и не позволит убить систему при обновлении
(не касается свободных драйверов Mesa).

\sphinxAtStartPar
Перед установкой рекомендуется отключить \sphinxstyleemphasis{"Secure Boot"} в UEFI, ибо из\sphinxhyphen{}за этого модули драйвера могут не загрузиться.

\begin{sphinxVerbatim}[commandchars=\\\{\}]
sudo pacman \PYGZhy{}S nvidia\PYGZhy{}dkms nvidia\PYGZhy{}utils lib32\PYGZhy{}nvidia\PYGZhy{}utils nvidia\PYGZhy{}settings vulkan\PYGZhy{}icd\PYGZhy{}loader lib32\PYGZhy{}vulkan\PYGZhy{}icd\PYGZhy{}loader lib32\PYGZhy{}opencl\PYGZhy{}nvidia opencl\PYGZhy{}nvidia libxnvctrl
sudo mkinitcpio \PYGZhy{}P \PYG{c+c1}{\PYGZsh{} Обновляем образы ядра}
\end{sphinxVerbatim}


\subsubsection{Nouveau (\sphinxstyleemphasis{Только для старых видеокарт})}
\label{\detokenize{source/first-steps:nouveau}}
\sphinxAtStartPar
Для старых видеокарт Nvidia (ниже GeForce 600) рекомендуется использовать свободную альтернативу драйвера NVIDIA — Nouveau, входящую в состав Mesa.
Она имеет официальную поддержку и обновления в отличии от старых версий закрытого драйвера NVIDIA (340, 390) и отлично справляется с 2D ускорением.
Вдобавок, Nouveau хорошо работает с Wayland.

\begin{sphinxVerbatim}[commandchars=\\\{\}]
sudo pacman \PYGZhy{}S mesa lib32\PYGZhy{}mesa xf86\PYGZhy{}video\PYGZhy{}nouveau vulkan\PYGZhy{}icd\PYGZhy{}loader lib32\PYGZhy{}vulkan\PYGZhy{}icd\PYGZhy{}loader
\end{sphinxVerbatim}


\subsubsection{AMD}
\label{\detokenize{source/first-steps:amd}}
\begin{sphinxVerbatim}[commandchars=\\\{\}]
sudo pacman \PYGZhy{}S mesa lib32\PYGZhy{}mesa vulkan\PYGZhy{}radeon lib32\PYGZhy{}vulkan\PYGZhy{}radeon vulkan\PYGZhy{}icd\PYGZhy{}loader lib32\PYGZhy{}vulkan\PYGZhy{}icd\PYGZhy{}loader
\end{sphinxVerbatim}


\subsubsection{Intel}
\label{\detokenize{source/first-steps:intel}}
\begin{sphinxVerbatim}[commandchars=\\\{\}]
sudo pacman \PYGZhy{}S mesa lib32\PYGZhy{}mesa vulkan\PYGZhy{}intel lib32\PYGZhy{}vulkan\PYGZhy{}intel vulkan\PYGZhy{}icd\PYGZhy{}loader lib32\PYGZhy{}vulkan\PYGZhy{}icd\PYGZhy{}loader
\end{sphinxVerbatim}

\sphinxAtStartPar
Данные команды выполнят установку полного набора драйверов для вашей видеокарты и всех зависимостей,
но внимание: автор использует проприетарный драйвер NVIDIA, поэтому если вы заметили ошибку или желаете более проверенный источник: \sphinxhref{https://github.com/lutris/docs/blob/master/InstallingDrivers.md}{GitHub}.

\begin{sphinxadmonition}{attention}{Внимание:}
\sphinxAtStartPar
У авторов отсутствует оборудование AMD, поэтому в данном руководстве основной акцент будет сделан именно на настройке оборудования от компании NVIDIA.
Если у вас есть желание дополнить это руководство специфичными для открытых драйверов Mesa твиками/оптимизациями,
вы можете отправить нам свои изменения в качестве \sphinxhref{https://github.com/ventureoo/ARU/pulls}{Pull Request'a} на рассмотрение.
\end{sphinxadmonition}

\index{modules@\spxentry{modules}}\index{mkinitcpio@\spxentry{mkinitcpio}}\index{initramfs@\spxentry{initramfs}}\ignorespaces 

\section{Добавление важных модулей в образы ядра}
\label{\detokenize{source/first-steps:important-modules}}\label{\detokenize{source/first-steps:index-6}}\label{\detokenize{source/first-steps:id6}}
\sphinxAtStartPar
Прежде чем мы начнем, необходимо добавить важные модули в загрузочный образ
нашего ядра.
Это позволит нам избежать проблем в дальнейшем, и снизить риск словить
"чёрный экран" при загрузке из\sphinxhyphen{}за того что какие\sphinxhyphen{}либо модули не были подгружены во время или просто отсутствуют.

\sphinxAtStartPar
Для этого отредактируем параметры сборки наших образов: \sphinxcode{\sphinxupquote{sudo nano /etc/mkinitcpio.conf}}

\sphinxAtStartPar
Отредактируйте строку \sphinxstyleemphasis{MODULES} как показано на изображении и выполните команды ниже.

\sphinxAtStartPar
В массив (ограничен скобками) вы можете прописать любые модули ядра которые считаете наиболее важными и нужными.
Ниже мы указали модули файловой системы Btrfs.

\sphinxAtStartPar
Если у вас видеокарта от AMD/Intel, то можно прописать дополнительно указать модули соответствующих драйверов AMD/Intel:
\sphinxstyleemphasis{amdgpu radeon} или \sphinxstyleemphasis{crc32c\sphinxhyphen{}intel intel\_agp i915}.

\sphinxAtStartPar
Так же если у вас другая файловая система, то прописывать модули для Btrfs не нужно.

\begin{sphinxVerbatim}[commandchars=\\\{\}]
\PYG{n+nv}{MODULES}\PYG{o}{=}\PYG{o}{(}crc32c libcrc32c zlib\PYGZus{}deflate btrfs\PYG{o}{)}
\end{sphinxVerbatim}

\noindent{\hspace*{\fill}\sphinxincludegraphics{{image4}.png}\hspace*{\fill}}

\begin{sphinxVerbatim}[commandchars=\\\{\}]
sudo mkinitcpio \PYGZhy{}P                                 \PYG{c+c1}{\PYGZsh{} Пересобираем наши образы ядра.}
\end{sphinxVerbatim}

\index{cpu@\spxentry{cpu}}\index{intel@\spxentry{intel}}\index{amd@\spxentry{amd}}\index{microcode@\spxentry{microcode}}\ignorespaces 

\section{Установка микрокода}
\label{\detokenize{source/first-steps:microcode-installation}}\label{\detokenize{source/first-steps:index-7}}\label{\detokenize{source/first-steps:id7}}
\sphinxAtStartPar
Микрокод \sphinxhyphen{} программа реализующая набор инструкций процессора.
Она уже встроена в материнскую плату вашего компьютера,
но скорее всего вы его либо не обновляли вовсе, либо делаете это не часто вместе с обновлением BIOS (UEFI).

\sphinxAtStartPar
Однако у ядра Linux есть возможность применять его обновления прямо во время загрузки.
Обновления микрокода содержат множественные исправления ошибок и улучшения стабильности,
поэтому настоятельно рекомендуется их периодически устанавливать.

\sphinxAtStartPar
Осуществляется это следующими командами:

\begin{sphinxVerbatim}[commandchars=\\\{\}]
sudo pacman \PYGZhy{}S intel\PYGZhy{}ucode                  \PYG{c+c1}{\PYGZsh{} Установить микрокод Intel}
sudo pacman \PYGZhy{}S amd\PYGZhy{}ucode                    \PYG{c+c1}{\PYGZsh{} Установить микрокод AMD}
sudo mkinitcpio \PYGZhy{}P                          \PYG{c+c1}{\PYGZsh{} Пересобираем образы ядра.}
sudo grub\PYGZhy{}mkconfig \PYGZhy{}o /boot/grub/grub.cfg   \PYG{c+c1}{\PYGZsh{} Обновляем загрузчик, можно так же через grub\PYGZhy{}customizer.}
\end{sphinxVerbatim}

\index{nvidia@\spxentry{nvidia}}\index{driver@\spxentry{driver}}\index{xorg@\spxentry{xorg}}\ignorespaces 

\section{Настройка драйвера NVIDIA}
\label{\detokenize{source/first-steps:nvidia-driver-setup}}\label{\detokenize{source/first-steps:index-8}}\label{\detokenize{source/first-steps:id8}}
\sphinxAtStartPar
После установки драйвера обязательно перезагрузитесь, откройте панель nvidia\sphinxhyphen{}settings, и выполните все шаги как показано на изображениях:

\begin{sphinxVerbatim}[commandchars=\\\{\}]
nvidia\PYGZhy{}settings \PYG{c+c1}{\PYGZsh{} Открыть панель Nvidia}
\end{sphinxVerbatim}

\noindent\sphinxincludegraphics{{nvidia-settings-1}.png}

\sphinxAtStartPar
(Если у вас больше одного монитора, то выбирайте здесь тот, который имеет большую частоту обновления)

\noindent\sphinxincludegraphics{{nvidia-settings-2}.png}

\sphinxAtStartPar
(Это изменение профиля питания видеокарты работает только до перезагрузки.
Если вы хотите зафиксировать профиль производительности,
то установите пакет nvidia\sphinxhyphen{}tweaks с параметром \sphinxstyleemphasis{\_powermizer\_scheme=1}, как описано в следующем подразделе.)

\noindent\sphinxincludegraphics{{nvidia-settings-3}.png}

\sphinxAtStartPar
(Не забудьте здесь настроить все мониторы которые у вас есть, задать им правильное разрешение и частоту обновления.)

\begin{sphinxadmonition}{attention}{Внимание:}
\sphinxAtStartPar
Советуем вам не использовать параметры \sphinxstyleemphasis{"Force composition Pipeline"} и \sphinxstyleemphasis{"Force Full composition Pipeline"}.
Несмотря на то, что эти два параметра действительно могут полностью вылечить тиринг (разрывы экрана), они также создают сильные задержки ввода (input lag).
Вместо этого рекомендуем вам выполнить настройку композитора вашего DE (WM) как это описано в разделе "\sphinxhref{https://ventureoo.github.io/ARU/source/de-optimizations.html}{Оптимизация рабочего окружения (DE)}".
\end{sphinxadmonition}

\noindent\sphinxincludegraphics{{nvidia-settings-4}.png}

\sphinxAtStartPar
Теперь переместите ранее сохраненый файл настройки в \sphinxstyleemphasis{/etc/X11/xorg.conf}, чтобы примененные вами настройки для мониторов
работали для всей системы и не слетали после перезагрузки:

\begin{sphinxVerbatim}[commandchars=\\\{\}]
sudo mv \PYGZti{}/xorg.conf /etc/X11/xorg.conf
\end{sphinxVerbatim}

\begin{sphinxadmonition}{attention}{Внимание:}
\sphinxAtStartPar
Если вы используете GNOME/Plasma, то помните, что эти окружения могут игнорировать настройки для мониторов которые вы указали здесь,
и использовать свои собственные. В этом случае настраивать мониторы нужно именно в настройках вашего рабочего окружения.
\end{sphinxadmonition}

\index{nvidia@\spxentry{nvidia}}\index{tweaks@\spxentry{tweaks}}\index{driver@\spxentry{driver}}\ignorespaces 

\subsection{Твики драйвера NVIDIA}
\label{\detokenize{source/first-steps:nvidia-tweaking}}\label{\detokenize{source/first-steps:index-9}}\label{\detokenize{source/first-steps:id9}}
\sphinxAtStartPar
По умолчанию в закрытом NVIDIA драйвере не используются некоторые скрытые оптимизации которые могут помочь с улучшением производительности и работоспособности видеокарты.

\sphinxAtStartPar
Поэтому, для того чтобы вы могли их активировать удобным способом, мы сделали пакет который включает в себя все эти твики для драйвера
\sphinxhyphen{} \sphinxhref{https://aur.archlinux.org/packages/nvidia-tweaks/}{nvidia\sphinxhyphen{}tweaks}. Прежде чем устанавливать выполните установку самого драйвера NVIDIA как это было описано выше.

\sphinxAtStartPar
\sphinxstylestrong{Установка}

\begin{sphinxVerbatim}[commandchars=\\\{\}]
git clone https://aur.archlinux.org/nvidia\PYGZhy{}tweaks.git
\PYG{n+nb}{cd} nvidia\PYGZhy{}tweaks
nano PKGBUILD \PYG{c+c1}{\PYGZsh{} В PKGBUILD вы можете найти больше опций для настройки, например настройку питания через PowerMizer}
makepkg \PYGZhy{}sric
\end{sphinxVerbatim}

\sphinxAtStartPar
При возникновении следующей ошибки:

\begin{sphinxVerbatim}[commandchars=\\\{\}]
\PYG{o}{=}\PYG{o}{=}\PYGZgt{} ОШИБКА: Cannot find the fakeroot binary.
\PYG{o}{=}\PYG{o}{=}\PYGZgt{} ОШИБКА: Cannot find the strip binary required \PYG{k}{for} object file stripping.
\end{sphinxVerbatim}

\sphinxAtStartPar
Выполните: \sphinxcode{\sphinxupquote{sudo pacman \sphinxhyphen{}S base\sphinxhyphen{}devel}}

\index{nvidia@\spxentry{nvidia}}\index{environment@\spxentry{environment}}\index{variables@\spxentry{variables}}\index{latency@\spxentry{latency}}\ignorespaces 

\subsection{Специфические переменные окружения для драйвера NVIDIA}
\label{\detokenize{source/first-steps:nvidia-env-vars}}\label{\detokenize{source/first-steps:index-10}}\label{\detokenize{source/first-steps:id10}}
\sphinxAtStartPar
Указать вы их можете либо в Lutris для конкретных игр, либо в \sphinxstyleemphasis{"Параметрах Запуска"} игры в Steam
(\sphinxstyleemphasis{"Свойства"} \sphinxhyphen{}> \sphinxstyleemphasis{"Параметры запуска"}. После указания всех переменных обязательно добавьте в конце "\sphinxstyleemphasis{\%command\%}",
для того чтобы Steam понимал, что вы указали именно системные переменные окружения для запуска игры, а не параметры специфичные для этой самой игры).

\sphinxAtStartPar
\sphinxcode{\sphinxupquote{\_\_GL\_THREADED\_OPTIMIZATIONS=1}} \sphinxstylestrong{(По умолчанию выключено)} \sphinxhyphen{}  Активируем многопоточную обработку OpenGL.
Используете выборочно для нативных игр/приложений, ибо иногда может наоборот вызывать регрессию производительности.
Некоторые игры и вовсе могут не запускаться с данной переменной (К примеру, некоторые нативно\sphinxhyphen{}запускаемые части Metro).

\sphinxAtStartPar
\sphinxcode{\sphinxupquote{\_\_GL\_MaxFramesAllowed=1}} \sphinxstylestrong{(По умолчанию \sphinxhyphen{} 2)} \sphinxhyphen{} Задает тип буферизации кадров драйвером.
Можете указать значение \sphinxstyleemphasis{"3"} (Тройная буферизация) для большего количества FPS и улучшения производительности в приложениях/играх с VSync.
Мы рекомендуем задавать вовсе \sphinxstyleemphasis{"1"} (т.е. не использовать буферизацию, подавать кадры так как они есть).
Это может заметно уменьшить значение FPS в играх, но взамен вы получите лучшие задержки отрисовки и реальный физический отклик,
т.к. кадр будет отображаться вам сразу на экран без лишних этапов его обработки.

\sphinxAtStartPar
\sphinxcode{\sphinxupquote{\_\_GL\_YIELD="USLEEP"}} \sphinxstylestrong{(По умолчанию без значения)} \sphinxhyphen{} Довольно специфичный параметр, \sphinxstyleemphasis{"USLEEP"} \sphinxhyphen{} снижает нагрузку на CPU и некоторым образом помогает в борьбе с тирингом,
а \sphinxstyleemphasis{"NOTHING"} дает больше FPS при этом увеличивая нагрузку на процессор.

\index{nvidia@\spxentry{nvidia}}\index{hybrid\sphinxhyphen{}graphics@\spxentry{hybrid\sphinxhyphen{}graphics}}\index{laptops@\spxentry{laptops}}\ignorespaces 

\section{Гибридная графика в ноутбуках}
\label{\detokenize{source/first-steps:hybrid-graphics}}\label{\detokenize{source/first-steps:index-11}}\label{\detokenize{source/first-steps:id11}}
\sphinxAtStartPar
Одной из самых больных проблем при использовании Linux на домашнем ноутбуке является гибридная графика.
Конечно, в этой теме уже есть прогресс, и все не так плохо как кажется, но графическая подсистема по
прежнему одна из самых (если не самая) проблемных частей любой Linux\sphinxhyphen{}системы.

\sphinxAtStartPar
Тема сложная и с кучей подводных камней, поэтому сначала разберемся с основными понятиями.
Гибридная графика \sphinxhyphen{} это когда у вас есть два графических процессора, которые могут работать
одновременно. Такая конфигурация чаще всего представлена в ноутбуках, когда есть интегрированный
(т.е. встроенный, iGPU) в процессор видеочип и дискретная видеокарта (dGPU), которая
превосходит встроенную по характеристикам и нацелена на использование в высокопроизводительных
задачах.

\sphinxAtStartPar
Смысл такого разделения состоит в том, что мы можем использовать для малопрофильных задач встроенный видеочип,
а когда появляется, так скажем, "рыба покрупнее", и нужно выдавать максимальный FPS \sphinxhyphen{} используем дискретную графику.
На ноутбуках это позволяет сильно экономить энергию и, следовательно, повысить время своей работы.

\sphinxAtStartPar
Однако на практике такая система содержит много проблем. Главная из которых, это вопрос о том,
как эти два GPU будут взаимодействовать между собой. И если в Windows эту проблему как\sphinxhyphen{}то решили,
то в Linux к сожалению все не так просто. По итогу мы имеем несколько отдельных комбинаций
производителей встроенных видеочипов и дискретной видеокарты. Вот три наиболее встречаемых случая
(сначала встроенная графика, затем дискретная):
\begin{enumerate}
\sphinxsetlistlabels{\arabic}{enumi}{enumii}{}{.}%
\item {} 
\sphinxAtStartPar
Intel + NVIDIA

\item {} 
\sphinxAtStartPar
AMD + NVIDIA

\item {} 
\sphinxAtStartPar
Intel + AMD

\end{enumerate}

\sphinxAtStartPar
Самыми распространенными из них являются первый и второй случай. Они же самые проблемные.
Третий случай не должен вызывать у вас всяких проблем, ибо для обеих GPU могут использоваться
открытые драйвера Mesa, которые должны работать из коробки. Вам нужно будет лишь использовать
переменную окружения \sphinxcode{\sphinxupquote{DRI\_PRIME=1}} чтобы форсировать использование дискретной графики для нужного
вам приложения. Например для игры в Steam вам достаточно в её свойствах указать \sphinxcode{\sphinxupquote{DRI\_PRIME=1 \%command\%}}.

\sphinxAtStartPar
Далее мы будем рассматривать только первые два случая, имеющие между собой один и тот же алгоритм действий.

\sphinxAtStartPar
Итак, есть две возможные стратегии при связке NVIDIA + Intel, либо NVIDIA + AMD:
\begin{enumerate}
\sphinxsetlistlabels{\arabic}{enumi}{enumii}{}{.}%
\item {} 
\sphinxAtStartPar
Мы используем встроенный механизм работы с гибридной графикой драйвера NVIDIA

\item {} 
\sphinxAtStartPar
Мы уходим от гибридной графики, отключая один из возможных GPU и используем только
дискретный/встроенный видеочип.

\end{enumerate}

\sphinxAtStartPar
Прежде чем мы начнем рассматривать первый и второй план\sphinxhyphen{}капкан, стоит выполнить некоторые обязательные
шаги, если вы хотите чтобы графика в вашем ноутбуке работала правильно.
\begin{itemize}
\item {} 
\sphinxAtStartPar
Удостоверьтесь, что вы установили все драйвера правильно, как для встроенной видеокарты, так
и для NVIDIA (обязательно для NVIDIA!)

\item {} 
\sphinxAtStartPar
Проверьте, правильно ли загружаются модули драйвера NVIDIA. Для этого выполните команду \sphinxcode{\sphinxupquote{lsmod | grep nvidia}}.
Если вывод команды НЕ пустой, то все в порядке.

\item {} 
\sphinxAtStartPar
Включите DRM KMS для драйвера NVIDIA. Сделать это можно двумя способами: добавить параметр ядра \sphinxcode{\sphinxupquote{nvidia\sphinxhyphen{}drm.modeset=1}}
в конфигурацию вашего загрузчика, либо при помощи файла настройки. Создайте файл \sphinxcode{\sphinxupquote{/etc/modprobe.d/nvidia.conf}} и пропишите
в него следующее: \sphinxcode{\sphinxupquote{options nvidia\_drm modeset=1}}. И да, \sphinxstylestrong{обязательно выполните обновление образов ядра через команду
sudo mkinitcpio \sphinxhyphen{}P}! Не забывайте об этом пожалуйста. Кроме того, вы можете целиком пропустить данный шаг, если ранее установили
пакет nvidia\sphinxhyphen{}tweaks.

\item {} 
\sphinxAtStartPar
Отключите параметр \sphinxcode{\sphinxupquote{Secure Boot}} в настройках UEFI если вы ещё этого не сделали. Он может мешать загрузке драйвера NVIDIA.

\item {} 
\sphinxAtStartPar
Установите утилиту XRandr: \sphinxcode{\sphinxupquote{sudo pacman \sphinxhyphen{}S xorg\sphinxhyphen{}xrandr}}

\end{itemize}

\sphinxAtStartPar
Теперь рассмотрим первый вариант, т. е. использование встроенного механизма работы с гибридной графикой.
Как ни странно, но если вы имеете дискретную видеокарту NVIDIA, у которой есть поддержка версии драйвера
выше 435, то все должно работать прямо из коробки. Просто вы можете об этом не догадываться.

\sphinxAtStartPar
Тем не менее, лучше все таки проверить, что все работает правильно, и вы можете сделать
это через утилиту nvidia\sphinxhyphen{}prime:

\begin{sphinxVerbatim}[commandchars=\\\{\}]
sudo pacman \PYGZhy{}S nvidia\PYGZhy{}prime
prime\PYGZhy{}run glxinfo \PYG{p}{|} grep \PYG{l+s+s2}{\PYGZdq{}OpenGL renderer\PYGZdq{}}
\end{sphinxVerbatim}

\sphinxAtStartPar
Если вывод последней команды даёт вам упоминание вашей дискретной видеокарты, значит
вы всё сделали правильно. При возникновении проблем, советуем вам перепройти шаги
указанные выше.

\sphinxAtStartPar
Вот и всё. Данный вариант ещё называют \sphinxstyleemphasis{"Reverse PRIME"} (обратный PRIME).
После этого у вас будет использоваться встроенная графика по умолчанию, а использовать дискретную графику
вы можете выборочно, указав перед командой запуска желаемой программы уже упомянутую команду \sphinxcode{\sphinxupquote{prime\sphinxhyphen{}run}}.
Например: \sphinxcode{\sphinxupquote{prime\sphinxhyphen{}run glxgears}}. Для игр в Steam добавляете команду в \sphinxstyleemphasis{"Свойствах"} игры: \sphinxcode{\sphinxupquote{prime\sphinxhyphen{}run \%command\%}}.
В рабочем окружении GNOME, начиная с версии 3.36 есть дополнительный пункт в контекстном меню, который также позволяет
вам запускать приложения с использованием дискретной графики.

\begin{sphinxadmonition}{warning}{Предупреждение:}
\sphinxAtStartPar
Обращаем ваше внимание, что некоторые возможности дискретной графики в таком режиме несколько урезаны.
Так, вы не сможете настроить ваши мониторы через nvidia\sphinxhyphen{}settings как это было указано в предыдущем разделе, ибо за подключение
и обслуживание внешних мониторов отвечает встроенная графика. Исключается возможность разгона и ручного
управления питанием дискретной видеокарты.
\end{sphinxadmonition}

\sphinxAtStartPar
Теперь второй вариант. Его я могу порекомендовать всем тем, кто:
а) не хочет возни и возможных проблем с предыдущем вариантом
б) хочет получить максимальную производительность

\sphinxAtStartPar
По сути, здесь мы делаем все тоже самое, что и в прошлом в варианте, просто меняя
дискретную графику со встроенной местами. Для этого необходимо создать конфигурационный файл
\sphinxcode{\sphinxupquote{sudo nano /etc/X11/xorg.conf.d/10\sphinxhyphen{}gpu.conf}} и прописать в него следующее:

\begin{sphinxVerbatim}[commandchars=\\\{\}]
Section \PYG{l+s+s2}{\PYGZdq{}ServerLayout\PYGZdq{}}
  Identifier \PYG{l+s+s2}{\PYGZdq{}layout\PYGZdq{}}
  Screen \PYG{l+m}{0} \PYG{l+s+s2}{\PYGZdq{}nvidia\PYGZdq{}}
  Inactive \PYG{l+s+s2}{\PYGZdq{}intel\PYGZdq{}}
EndSection

Section \PYG{l+s+s2}{\PYGZdq{}Device\PYGZdq{}}
    Identifier  \PYG{l+s+s2}{\PYGZdq{}nvidia\PYGZdq{}}
    Driver      \PYG{l+s+s2}{\PYGZdq{}nvidia\PYGZdq{}}
    BusID       \PYG{l+s+s2}{\PYGZdq{}PCI:x:x:x\PYGZdq{}} \PYG{c+c1}{\PYGZsh{} Например: \PYGZdq{}PCI:1:0:0\PYGZdq{}}
EndSection

Section \PYG{l+s+s2}{\PYGZdq{}Screen\PYGZdq{}}
    Identifier \PYG{l+s+s2}{\PYGZdq{}nvidia\PYGZdq{}}
    Device \PYG{l+s+s2}{\PYGZdq{}nvidia\PYGZdq{}}
    Option \PYG{l+s+s2}{\PYGZdq{}AllowEmptyInitialConfiguration\PYGZdq{}}
EndSection

Section \PYG{l+s+s2}{\PYGZdq{}Device\PYGZdq{}}
    Identifier  \PYG{l+s+s2}{\PYGZdq{}intel\PYGZdq{}}
    Driver      \PYG{l+s+s2}{\PYGZdq{}intel\PYGZdq{}}
    BusID       \PYG{l+s+s2}{\PYGZdq{}PCI:x:x:x\PYGZdq{}}  \PYG{c+c1}{\PYGZsh{} Например: \PYGZdq{}PCI:0:2:0\PYGZdq{}}
EndSection

Section \PYG{l+s+s2}{\PYGZdq{}Screen\PYGZdq{}}
    Identifier \PYG{l+s+s2}{\PYGZdq{}intel\PYGZdq{}}
    Device \PYG{l+s+s2}{\PYGZdq{}intel\PYGZdq{}}
EndSection
\end{sphinxVerbatim}

\sphinxAtStartPar
В полях \sphinxstylestrong{"BusID"} вы должны указать собственные значения PCI ID в том формате, в котором они указаны в примере.
Их вы можете узнать при помощи следующей команды: \sphinxcode{\sphinxupquote{lspci | grep VGA}} (для каждой видеокарты PCI ID будет первым набором цифр в строке).

\sphinxAtStartPar
Кроме того, если в качестве встроенной графики у вас видеочип от AMD, то в поле \sphinxstyleemphasis{"Driver"} вместо Intel вы должны
указать либо \sphinxstyleemphasis{"ati"}, либо \sphinxstyleemphasis{"amdgpu"}, в зависимости от того, какой из них поддерживает ваш видеочип (и предварительно
установив пакеты \sphinxcode{\sphinxupquote{xf86\sphinxhyphen{}video\sphinxhyphen{}ati}} и \sphinxcode{\sphinxupquote{xf86\sphinxhyphen{}video\sphinxhyphen{}amdgpu}} соответственно).

\sphinxAtStartPar
Перезагружаемся, и снова смотрим выхлоп: \sphinxcode{\sphinxupquote{glxinfo | grep "OpenGL renderer"}}
(в этот раз без nvidia\sphinxhyphen{}prime). У вас так же должно появиться упоминание
вашей дискретной видеокарты.

\sphinxAtStartPar
В этом случае вся графика будет на плечах дискретной видеокарты, благодаря
чему достигается максимальная производительность и снимаются ряд ограничений
(панель nvidia\sphinxhyphen{}settings должна прибавить в возможностях).

\sphinxAtStartPar
Стоит отметить, что всё, что мы проделали выше \sphinxhyphen{} работает только для версии драйвера 435.17 и выше.
При использовании драйвера ниже этой версии у вас по умолчанию должна использоваться только дискретная графика (?).

\begin{sphinxadmonition}{attention}{Внимание:}
\sphinxAtStartPar
Да, многие на этом моменте могут сказать, что есть Bumblebee. Однако он признан морально устаревшим
и более неподдерживаемым. Потому он имеет целый ряд проблем, в частности с производительностью. Автор не советует
его использовать при любом раскладе. Лучше поиграться с частотами вашей дискретной видеокарты, дабы снизить энергопотребление.
\end{sphinxadmonition}

\index{nvidia@\spxentry{nvidia}}\index{hybrid\sphinxhyphen{}graphics@\spxentry{hybrid\sphinxhyphen{}graphics}}\index{laptops@\spxentry{laptops}}\ignorespaces 

\subsection{Альтернатива попроще: optimus\sphinxhyphen{}manager}
\label{\detokenize{source/first-steps:optimus-manager}}\label{\detokenize{source/first-steps:index-12}}\label{\detokenize{source/first-steps:id12}}
\sphinxAtStartPar
Если вы не хотите разбираться в этой теме подробно, и хотите просто поставить и забыть, то
есть специальный помощник в этом \sphinxhyphen{} \sphinxhref{https://github.com/Askannz/optimus-manager}{optimus\sphinxhyphen{}manager},
а также графическая обертка для него optimus\sphinxhyphen{}manager\sphinxhyphen{}qt.

\sphinxAtStartPar
Эта программа позволит вам быстро переключаться между различными режимами описанными выше и без танцев с бубном.
Программа работает как для новых версий драйвера (выше 435.17), так и для старых (правда без гибридного режима).

\noindent\sphinxincludegraphics{{tray-menu}.png}

\sphinxAtStartPar
\sphinxstylestrong{Установка}

\sphinxAtStartPar
Для правильной работы перед установкой выполните ряд шагов:
\begin{itemize}
\item {} 
\sphinxAtStartPar
Вы должны использовать один из популярных менеджеров входа: LightDM, SDDM или GDM (подробнее о нем ниже).

\item {} 
\sphinxAtStartPar
Если ваше рабочее окружение это GNOME, то вам необходимо установить модифицированный пакет \sphinxhref{https://aur.archlinux.org/packages/gdm-prime}{gdm\sphinxhyphen{}prime} из AUR.
Не забудьте отредактировать \sphinxcode{\sphinxupquote{sudo nano /etc/gdm/custom.conf}} и добавить строку \sphinxcode{\sphinxupquote{WaylandEnable=false}} чтобы форсировать отключение Wayland сессии.
Напоминаю, что режим гибридной графики на данный момент не работает в Wayland. Совсем. Вообще.

\item {} 
\sphinxAtStartPar
Полностью удалите \sphinxcode{\sphinxupquote{/etc/X11/xorg.conf}} или удалите в нем все строки связанные с настройкой GPU.
Optimus\sphinxhyphen{}manager использует собственные настройки Xorg для правильной работы всех доступных режимов.

\end{itemize}

\sphinxAtStartPar
Перейдем непосредственно к установке:

\begin{sphinxVerbatim}[commandchars=\\\{\}]
git clone https://aur.archlinux.org/optimus\PYGZhy{}manager.git \PYG{c+c1}{\PYGZsh{} Скачивание исходников}
\PYG{n+nb}{cd} optimus\PYGZhy{}manager                                      \PYG{c+c1}{\PYGZsh{} Переход в директорию}
makepkg \PYGZhy{}sric                                           \PYG{c+c1}{\PYGZsh{} Сборка и установка}

sudo systemctl \PYG{n+nb}{enable} optimus\PYGZhy{}manager.service \PYG{c+c1}{\PYGZsh{} Запускаем службу}
\end{sphinxVerbatim}

\sphinxAtStartPar
Дополнительно советуем установить графическую обертку:

\begin{sphinxVerbatim}[commandchars=\\\{\}]
git clone https://github.com/Shatur/optimus\PYGZhy{}manager\PYGZhy{}qt  \PYG{c+c1}{\PYGZsh{} Скачивание исходников}
\PYG{n+nb}{cd} optimus\PYGZhy{}manager\PYGZhy{}qt                                   \PYG{c+c1}{\PYGZsh{} Переход в директорию}
\PYG{c+c1}{\PYGZsh{} Перед сборкой можете отредактировать PKGBUILD, заменив строку \PYGZus{}plasma=false на \PYGZus{}plasma=true.}
\PYG{c+c1}{\PYGZsh{} Это улучшит совместимость с Plasma (если вы её используете).}
makepkg \PYGZhy{}sric                                           \PYG{c+c1}{\PYGZsh{} Сборка и установка}
\end{sphinxVerbatim}

\sphinxAtStartPar
После этого перезагрузитесь и запустив optimus\sphinxhyphen{}manager\sphinxhyphen{}qt выполните переключение в нужный вам режим.

\index{monitor@\spxentry{monitor}}\index{overlocking@\spxentry{overlocking}}\index{refresh\sphinxhyphen{}rate@\spxentry{refresh\sphinxhyphen{}rate}}\ignorespaces 

\section{Разгон монитора \sphinxstyleemphasis{(Для опытных пользователей)}}
\label{\detokenize{source/first-steps:monitor-overlocking}}\label{\detokenize{source/first-steps:index-13}}\label{\detokenize{source/first-steps:id13}}
\sphinxAtStartPar
Вопреки мнению многих людей, в Linux таки возможно выполнить разгон монитора.
Пусть и с небольшим количеством манипуляций мы попробуем это сделать в данном разделе
для разных конфигураций оборудования.

\begin{sphinxadmonition}{warning}{Предупреждение:}
\sphinxAtStartPar
Описанные ниже способы не работают для Wayland сессий.
\end{sphinxadmonition}

\index{monitor@\spxentry{monitor}}\index{overlocking@\spxentry{overlocking}}\index{refresh\sphinxhyphen{}rate@\spxentry{refresh\sphinxhyphen{}rate}}\index{amd@\spxentry{amd}}\index{intel@\spxentry{intel}}\index{mesa@\spxentry{mesa}}\ignorespaces 

\subsection{Для видеокарт AMD/Intel}
\label{\detokenize{source/first-steps:amd-intel}}\label{\detokenize{source/first-steps:monitor-overlocking-mesa}}\label{\detokenize{source/first-steps:index-14}}
\sphinxAtStartPar
Данный способ работает только для драйверов Mesa и Xorg.

\sphinxAtStartPar
Установим все необходимые компоненты:

\begin{sphinxVerbatim}[commandchars=\\\{\}]
sudo pacman \PYGZhy{}S xorg\PYGZhy{}xrandr libxcvt
\end{sphinxVerbatim}

\sphinxAtStartPar
Для начала сгенерируем модельную линию, которая предоставляет Xorg серверу информацию о подключенном мониторе компьютера.
Выполните следующую команду, где сначала указываете желаемое разрешение через пробел, а затем и желаемую частоту обновления:

\begin{sphinxVerbatim}[commandchars=\\\{\}]
cvt \PYG{l+m}{1920} \PYG{l+m}{1080} \PYG{l+m}{75}
\end{sphinxVerbatim}

\sphinxAtStartPar
Теперь зарегистрируем полученную модельную линию в Xorg через утилиту xrandr.
Скопируйте выведенную cvt строку и вставьте все после \sphinxstyleemphasis{Modeline} в эту команду:

\begin{sphinxVerbatim}[commandchars=\\\{\}]
xrandr \PYGZhy{}\PYGZhy{}newmode \PYG{l+s+s2}{\PYGZdq{}1920x1080\PYGZus{}75.00\PYGZdq{}}  \PYG{l+m}{220}.75  \PYG{l+m}{1920} \PYG{l+m}{2064} \PYG{l+m}{2264} \PYG{l+m}{2608}  \PYG{l+m}{1080} \PYG{l+m}{1083} \PYG{l+m}{1088} \PYG{l+m}{1130} \PYGZhy{}hsync +vsync
\end{sphinxVerbatim}

\sphinxAtStartPar
Теперь применим полученный Modeline для нужного монитора:

\begin{sphinxVerbatim}[commandchars=\\\{\}]
xrandr \PYGZhy{}\PYGZhy{}addmode HDMI\PYGZhy{}0 1920x1080\PYGZus{}75.00
xrandr \PYGZhy{}\PYGZhy{}output HDMI\PYGZhy{}0 \PYGZhy{}\PYGZhy{}mode 1920x1080\PYGZus{}75.00
\end{sphinxVerbatim}

\sphinxAtStartPar
(Где \sphinxstyleemphasis{HDMI\sphinxhyphen{}0} \sphinxhyphen{} тип подключения вашего монитора, его можно узнать через команду xrandr без аргументов)

\sphinxAtStartPar
Теперь вы можете в таком порядке выполнять эти операции постепенно повышая частоту обновления монитора, и результат в виде
модельной линии с максимальной рабочей частотой обновления добавить в файл настройки Xorg. Например:

\begin{sphinxVerbatim}[commandchars=\\\{\}]
sudo nano /etc/X11/xorg.conf.d/10\PYGZhy{}monitor.conf \PYG{c+c1}{\PYGZsh{} Прописываем строчки ниже}

Section \PYG{l+s+s2}{\PYGZdq{}Monitor\PYGZdq{}}
    Identifier \PYG{l+s+s2}{\PYGZdq{}VGA1\PYGZdq{}} \PYG{c+c1}{\PYGZsh{} Здесь указываем тип подключения вашего монитора}
    Modeline \PYG{l+s+s2}{\PYGZdq{}1280x1024\PYGZus{}60.00\PYGZdq{}}  \PYG{l+m}{109}.00  \PYG{l+m}{1280} \PYG{l+m}{1368} \PYG{l+m}{1496} \PYG{l+m}{1712}  \PYG{l+m}{1024} \PYG{l+m}{1027} \PYG{l+m}{1034} \PYG{l+m}{1063} \PYGZhy{}hsync +vsync \PYG{c+c1}{\PYGZsh{} Здесь указываем модельную линию которая у вас получилась}
    Option \PYG{l+s+s2}{\PYGZdq{}PreferredMode\PYGZdq{}} \PYG{l+s+s2}{\PYGZdq{}1280x1024\PYGZus{}60.00\PYGZdq{}} \PYG{c+c1}{\PYGZsh{} Здесь заменяем на название полученной модельной линии}
EndSection

Section \PYG{l+s+s2}{\PYGZdq{}Screen\PYGZdq{}}
    Identifier \PYG{l+s+s2}{\PYGZdq{}Screen0\PYGZdq{}}
    Monitor \PYG{l+s+s2}{\PYGZdq{}VGA1\PYGZdq{}} \PYG{c+c1}{\PYGZsh{} Здесь указываем тип подключения вашего монитора}
    DefaultDepth \PYG{l+m}{24}
    SubSection \PYG{l+s+s2}{\PYGZdq{}Display\PYGZdq{}}
      Modes \PYG{l+s+s2}{\PYGZdq{}1280x1024\PYGZus{}60.00\PYGZdq{}} \PYG{c+c1}{\PYGZsh{} Здесь меняем на название полученной модельной линии}
    EndSubSection
EndSection

Section \PYG{l+s+s2}{\PYGZdq{}Device\PYGZdq{}}
    Identifier \PYG{l+s+s2}{\PYGZdq{}Device0\PYGZdq{}}
    Driver \PYG{l+s+s2}{\PYGZdq{}intel\PYGZdq{}}      \PYG{c+c1}{\PYGZsh{} Здесь меняем на драйвер вашей видеокарты}
EndSection
\end{sphinxVerbatim}

\begin{sphinxadmonition}{attention}{Внимание:}
\sphinxAtStartPar
Обратите внимание на комментарии в привиденном примере файла настройки!
\end{sphinxadmonition}

\sphinxAtStartPar
После перезагрузки все настройки должны работать правильно.

\sphinxAtStartPar
Отдельным случаем стоит рассмотреть разгон матрицы ноутбука с графикой Intel.
Об этом вы можете прочитать \sphinxhref{https://www.lushnikov.net/2021/07/31/\%D0\%A0\%D0\%B0\%D0\%B7\%D0\%B3\%D0\%BE\%D0\%BD\%D1\%8F\%D0\%B5\%D0\%BC-\%D0\%BC\%D0\%B0\%D1\%82\%D1\%80\%D0\%B8\%D1\%86\%D1\%83-\%D0\%BD\%D0\%BE\%D1\%83\%D1\%82\%D0\%B1\%D1\%83\%D0\%BA\%D0\%B0-\%D1\%81-\%D0\%B3\%D1\%80\%D0\%B0\%D1\%84\%D0\%B8\%D0\%BA\%D0\%BE/}{в данной статье}.

\index{monitor@\spxentry{monitor}}\index{overlocking@\spxentry{overlocking}}\index{refresh\sphinxhyphen{}rate@\spxentry{refresh\sphinxhyphen{}rate}}\index{nvidia@\spxentry{nvidia}}\ignorespaces 

\subsection{Для видеокарт NVIDIA}
\label{\detokenize{source/first-steps:monitor-overlocking-nvidia}}\label{\detokenize{source/first-steps:index-15}}\label{\detokenize{source/first-steps:id15}}
\sphinxAtStartPar
Сейчас мы будем рассматривать вопрос разгона монитора только для видеокарт NVIDIA,
т. к. у этого производителя есть некоторые проблемы с применением модельных линий Xorg напрямую через XRandr.

\sphinxAtStartPar
Прежде всего, нужно узнать какой тип подключения у вашего монитора, сделать это можно при помощи утилиты xrandr:

\begin{sphinxVerbatim}[commandchars=\\\{\}]
sudo pacman \PYGZhy{}S xorg\PYGZhy{}xrandr \PYG{c+c1}{\PYGZsh{} Установка}
xrandr                     \PYG{c+c1}{\PYGZsh{} Запуск}
\end{sphinxVerbatim}

\sphinxAtStartPar
Из информации о наших мониторах, выводимой xrandr, нас интересует:
\begin{enumerate}
\sphinxsetlistlabels{\arabic}{enumi}{enumii}{}{.}%
\item {} 
\sphinxAtStartPar
Тип подключения монитора который вы хотите разогнать (HDMI\sphinxhyphen{}0/DP\sphinxhyphen{}0 и т.д.)

\item {} 
\sphinxAtStartPar
Строчка с разрешением монитора для разгона.
Необходимо чтобы рядом со значением его частоты обновления был знак звездочки (*).
Это означает, что монитор способен выдавать большее количество Герц чем указано, т.е. его можно разогнать.

\end{enumerate}

\sphinxAtStartPar
Затем переходим в панель управления NVIDIA X Settings:

\begin{sphinxVerbatim}[commandchars=\\\{\}]
sudo nvidia\PYGZhy{}settings
\end{sphinxVerbatim}

\sphinxAtStartPar
В ней нам нужно полностью настроить наш разгоняемый монитор с соответствующим типом подключения во вкладке  \sphinxstyleemphasis{"X Server Display Configuration"}.
Задайте разрешение монитора и его частоту обновления согласно тем значениям,
что нам вывел xrandr и сохраните все настройки в xorg.conf через кнопку снизу: \sphinxstyleemphasis{"Save X Configuration File"}.

\sphinxAtStartPar
После этого переходим во вкладку с названием монитора который вы хотите разогнать.
К примеру: \sphinxstyleemphasis{"HDMI\sphinxhyphen{}0 \sphinxhyphen{} (Samsung S24R35x)"}. И жмакаем на кнопку \sphinxstyleemphasis{"Acquire EDID..."} \sphinxhyphen{}>
И сохраняем EDID файл вашего монитора в домашнюю директорию (Это \sphinxstylestrong{обязательный шаг}, сохранять нужно только в домашнюю папку вашего пользователя).

\sphinxAtStartPar
Итак, теперь нам нужно отредактировать наш edid.bin файл монитора.
Чтобы это сделать установим свободно распространяемую утилиту \sphinxhref{https://sourceforge.net/projects/wxedid/}{wxedid}:

\begin{sphinxVerbatim}[commandchars=\\\{\}]
git clone https://aur.archlinux.org/wxedid.git \PYG{c+c1}{\PYGZsh{} Скачивание исходников}
\PYG{n+nb}{cd} wxedid                                      \PYG{c+c1}{\PYGZsh{} Переход в директорию}
makepkg \PYGZhy{}sric                                  \PYG{c+c1}{\PYGZsh{} Сборка и установка}
\end{sphinxVerbatim}

\sphinxAtStartPar
Запустив эту программу откроем через меню наш сохраненный edid файл.

\noindent\sphinxincludegraphics{{wxedid-1}.png}

\sphinxAtStartPar
Затем перейдем в \sphinxstyleemphasis{"DTD: Detailed Timing Descriptor"}.

\noindent\sphinxincludegraphics{{wxedid-2}.png}

\sphinxAtStartPar
Здесь нужно переключится на вкладку \sphinxstyleemphasis{"DTD Constructor"},
и в поле "Pixel clock" постепенно повышать частоту обновления монитора до необходимого значения.

\noindent\sphinxincludegraphics{{wxedid-3}.png}

\sphinxAtStartPar
О том, как найти нужное значение для вашего монитора \sphinxhyphen{} думайте сами и ищите на специализированных ресурсах.
Для разных мониторов \sphinxhyphen{} разные значения.

\sphinxAtStartPar
Сохраняем уже измененный EDID файл (так же в домашнюю директорию) и закрываем программу.

\noindent\sphinxincludegraphics{{wxedid-4}.png}

\sphinxAtStartPar
Теперь в настройках Xorg нужно указать путь до измененного EDID файла в секции с тем монитором который мы разгоняем:

\begin{sphinxVerbatim}[commandchars=\\\{\}]
sudo nano /etc/X11/xorg.conf \PYG{c+c1}{\PYGZsh{} Редактируем ранее сохраненный xorg.conf}
\end{sphinxVerbatim}

\sphinxAtStartPar
И добавляем туда опцию с полным путем к измененному EDID файлу в таком формате:

\begin{sphinxVerbatim}[commandchars=\\\{\}]
Option     \PYG{l+s+s2}{\PYGZdq{}CustomEDID\PYGZdq{}} \PYG{l+s+s2}{\PYGZdq{}HDMI\PYGZhy{}0:/home/ваше\PYGZus{}имя\PYGZus{}пользователя/edid.bin\PYGZdq{}}
\end{sphinxVerbatim}

\sphinxAtStartPar
(Где \sphinxstyleemphasis{HDMI\sphinxhyphen{}0} \sphinxhyphen{} ваш тип подключения, а \sphinxstyleemphasis{edid.bin} ваш файл для разгона)

\sphinxAtStartPar
Все. Теперь нужно перезагрузиться и наслаждаться плавностью.
(При условии что вы указали правильное значение).

\begin{sphinxadmonition}{warning}{Предупреждение:}
\sphinxAtStartPar
Пользователи с VGA подключением монитора (и не только) могут испытывать проблему с черным экраном после перезагрузки.
Поэтому, просим вас заранее сделать себе флешку с записанным на нее любым LiveCD окружением, для того чтобы можно было откатить изменения в случае возникновения проблем.
\end{sphinxadmonition}

\sphinxAtStartPar
\sphinxstylestrong{Видео версия (Немного устарела)}

\sphinxAtStartPar
\sphinxurl{https://www.youtube.com/watch?v=B9o5b2A2qN0}

\sphinxstepscope


\chapter{Базовое ускорение системы}
\label{\detokenize{source/generic-system-acceleration:generic-system-acceleration}}\label{\detokenize{source/generic-system-acceleration:id1}}\label{\detokenize{source/generic-system-acceleration::doc}}
\sphinxAtStartPar
Переходя к базовой оптимизации системы мне сто́ит напомнить, что чистый Arch Linux \sphinxhyphen{} это фундамент, и требуется уйма надстроек для нормальной работы системы.
Установить компоненты, которые будут отвечать за электропитание, чистку, оптимизацию и тому подобные вещи, что и описывается в данном разделе.

\index{makepkg\sphinxhyphen{}conf@\spxentry{makepkg\sphinxhyphen{}conf}}\index{native\sphinxhyphen{}compilation@\spxentry{native\sphinxhyphen{}compilation}}\index{flags@\spxentry{flags}}\index{lto@\spxentry{lto}}\ignorespaces 

\section{Настройка makepkg.conf}
\label{\detokenize{source/generic-system-acceleration:makepkg-conf}}\label{\detokenize{source/generic-system-acceleration:index-0}}\label{\detokenize{source/generic-system-acceleration:id2}}
\sphinxAtStartPar
Прежде чем приступать к сборке пакетов, мы должны изменить так называемые флаги компиляции,
что являются указателями для компилятора, какие инструкции и оптимизации использовать при сборке программ.

\sphinxAtStartPar
\sphinxcode{\sphinxupquote{sudo nano /etc/makepkg.conf}} \# Редактируем (Где "\sphinxhyphen{}O2" \sphinxhyphen{} \sphinxstylestrong{Это не нуль/ноль})

\sphinxAtStartPar
\sphinxstylestrong{Изменить ваши значения на данные:}

\begin{sphinxVerbatim}[commandchars=\\\{\}]
\PYG{n+nv}{CFLAGS}\PYG{o}{=}\PYG{l+s+s2}{\PYGZdq{}\PYGZhy{}march=native \PYGZhy{}mtune=native \PYGZhy{}O2 \PYGZhy{}pipe \PYGZhy{}fno\PYGZhy{}plt \PYGZhy{}fexceptions \PYGZbs{}}
\PYG{l+s+s2}{      \PYGZhy{}Wp,\PYGZhy{}D\PYGZus{}FORTIFY\PYGZus{}SOURCE=2 \PYGZhy{}Wformat \PYGZhy{}Werror=format\PYGZhy{}security \PYGZbs{}}
\PYG{l+s+s2}{      \PYGZhy{}fstack\PYGZhy{}clash\PYGZhy{}protection \PYGZhy{}fcf\PYGZhy{}protection\PYGZdq{}}
\PYG{n+nv}{CXXFLAGS}\PYG{o}{=}\PYG{l+s+s2}{\PYGZdq{}}\PYG{n+nv}{\PYGZdl{}CFLAGS}\PYG{l+s+s2}{ \PYGZhy{}Wp,\PYGZhy{}D\PYGZus{}GLIBCXX\PYGZus{}ASSERTIONS}\PYG{l+s+s2}{\PYGZdq{}}
\PYG{n+nv}{RUSTFLAGS}\PYG{o}{=}\PYG{l+s+s2}{\PYGZdq{}\PYGZhy{}C opt\PYGZhy{}level=3\PYGZdq{}}
\PYG{n+nv}{MAKEFLAGS}\PYG{o}{=}\PYG{l+s+s2}{\PYGZdq{}}\PYG{l+s+s2}{\PYGZhy{}j}\PYG{k}{\PYGZdl{}(}nproc\PYG{k}{)}\PYG{l+s+s2}{ \PYGZhy{}l}\PYG{k}{\PYGZdl{}(}nproc\PYG{k}{)}\PYG{l+s+s2}{\PYGZdq{}}
\PYG{n+nv}{OPTIONS}\PYG{o}{=}\PYG{o}{(}strip docs !libtool !staticlibs emptydirs zipman purge !debug lto\PYG{o}{)}
\end{sphinxVerbatim}

\sphinxAtStartPar
Данные флаги компилятора выжимают максимум производительности при компиляции, но могут вызывать ошибки сборки в очень редких приложениях.
Если такое случится, то отключите параметр ‘lto’ в строке options добавив перед ним символ восклицательного знака  ! (\sphinxstyleemphasis{"!lto"}).

\index{makepkg@\spxentry{makepkg}}\index{clang@\spxentry{clang}}\index{native\sphinxhyphen{}compilation@\spxentry{native\sphinxhyphen{}compilation}}\index{flags@\spxentry{flags}}\ignorespaces 

\subsection{Форсирование использования Clang при сборке пакетов}
\label{\detokenize{source/generic-system-acceleration:clang}}\label{\detokenize{source/generic-system-acceleration:force-clang-usage}}\label{\detokenize{source/generic-system-acceleration:index-1}}
\sphinxAtStartPar
В системах на базе ядра Linux различают две основных группы компиляторов, это LLVM и GCC.
И те, и другие хорошо справляются с возложенными на них задачами,
но LLVM имеет чуть большее преимущество с точки зрения производительности при меньших потерях в качестве конечного кода.
Поэтому, в целом, применение компиляторов LLVM для сборки различных пакетов при задании флага \sphinxhyphen{}O2
(максимальная производительность) является совершенно оправданным, и может дать реальный прирост при работе программ.

\sphinxAtStartPar
Компилятором для языков C/C++ в составе LLVM является Clang и Clang++ соответственно.
Его использование при сборке пакетов мы и будем форсировать через makepkg.conf

\sphinxAtStartPar
Для начала выполним их установку:

\begin{sphinxVerbatim}[commandchars=\\\{\}]
sudo pacman \PYGZhy{}Syu llvm clang lld
\end{sphinxVerbatim}

\sphinxAtStartPar
Теперь клонируем уже готовый конфигурационный файл /etc/makepkg.conf под новыми именем /etc/makepkg\sphinxhyphen{}clang.conf:

\begin{sphinxVerbatim}[commandchars=\\\{\}]
sudo cp \PYGZhy{}r /etc/makepkg.conf /etc/makepkg\PYGZhy{}clang.conf
\end{sphinxVerbatim}

\sphinxAtStartPar
Это поможет нам в случае чего откатиться к использованию компиляторов GCC если возникнут проблемы со сборкой пакетов через LLVM/Clang.

\sphinxAtStartPar
Теперь откроем выше скопированный файл и добавим туда после строки \sphinxcode{\sphinxupquote{CHOST="x86\_64\sphinxhyphen{}pc\sphinxhyphen{}linux\sphinxhyphen{}gnu"}} следующее:

\begin{sphinxVerbatim}[commandchars=\\\{\}]
\PYG{n+nb}{export} \PYG{n+nv}{CC}\PYG{o}{=}clang
\PYG{n+nb}{export} \PYG{n+nv}{CXX}\PYG{o}{=}clang++
\PYG{n+nb}{export} \PYG{n+nv}{LD}\PYG{o}{=}ld.lld
\PYG{n+nb}{export} \PYG{n+nv}{CC\PYGZus{}LD}\PYG{o}{=}lld
\PYG{n+nb}{export} \PYG{n+nv}{CXX\PYGZus{}LD}\PYG{o}{=}lld
\PYG{n+nb}{export} \PYG{n+nv}{AR}\PYG{o}{=}llvm\PYGZhy{}ar
\PYG{n+nb}{export} \PYG{n+nv}{NM}\PYG{o}{=}llvm\PYGZhy{}nm
\PYG{n+nb}{export} \PYG{n+nv}{STRIP}\PYG{o}{=}llvm\PYGZhy{}strip
\PYG{n+nb}{export} \PYG{n+nv}{OBJCOPY}\PYG{o}{=}llvm\PYGZhy{}objcopy
\PYG{n+nb}{export} \PYG{n+nv}{OBJDUMP}\PYG{o}{=}llvm\PYGZhy{}objdump
\PYG{n+nb}{export} \PYG{n+nv}{READELF}\PYG{o}{=}llvm\PYGZhy{}readelf
\PYG{n+nb}{export} \PYG{n+nv}{RANLIB}\PYG{o}{=}llvm\PYGZhy{}ranlib
\PYG{n+nb}{export} \PYG{n+nv}{HOSTCC}\PYG{o}{=}clang
\PYG{n+nb}{export} \PYG{n+nv}{HOSTCXX}\PYG{o}{=}clang++
\PYG{n+nb}{export} \PYG{n+nv}{HOSTAR}\PYG{o}{=}llvm\PYGZhy{}ar
\PYG{n+nb}{export} \PYG{n+nv}{HOSTLD}\PYG{o}{=}ld.lld
\end{sphinxVerbatim}

\sphinxAtStartPar
Отлично, теперь вы можете собрать нужные вам пакеты (программы) через LLVM/Clang просто добавив к уже известной команде makepkg следующие параметры:

\begin{sphinxVerbatim}[commandchars=\\\{\}]
makepkg \PYGZhy{}\PYGZhy{}config /etc/makepkg\PYGZhy{}clang.conf \PYGZhy{}sric
\end{sphinxVerbatim}

\begin{sphinxadmonition}{attention}{Внимание:}
\sphinxAtStartPar
Далеко не все пакеты так уж гладко собираются через Clang, в частности не пытайтесь собирать им Wine/DXVK,
т.к. это официально не поддерживается и с 98\% вероятностью приведет к ошибке сборки.
Но в случае неудачи вы всегда можете использовать компиляторы GCC, которые у вас заданы в настройках makepkg.conf по умолчанию,
т.е. просто уберите опцию \sphinxcode{\sphinxupquote{\sphinxhyphen{}\sphinxhyphen{}config /etc/makepkg\sphinxhyphen{}clang.conf}} из команды \sphinxcode{\sphinxupquote{makepkg}}.
\end{sphinxadmonition}

\sphinxAtStartPar
Дальнейшеная пересборка пакетов из официальных репозиториев осуществима через следующее команды:

\begin{sphinxVerbatim}[commandchars=\\\{\}]
git clone \PYGZhy{}\PYGZhy{}depth \PYG{l+m}{1} \PYGZhy{}\PYGZhy{}branch packages/package https://github.com/archlinux/svntogit\PYGZhy{}packages.git package
\PYG{n+nb}{cd} package/trunk
makepkg \PYGZhy{}\PYGZhy{}config /etc/makepkg\PYGZhy{}clang.conf \PYGZhy{}sric \PYGZhy{}\PYGZhy{}skippgpcheck
\end{sphinxVerbatim}

\sphinxAtStartPar
Где \sphinxstyleemphasis{package} \sphinxhyphen{} название нужного вам пакета.

\sphinxAtStartPar
Мы рекомендуем вам пересобрать наиболее важные пакеты. Например такие как драйвера (то есть \sphinxhref{https://archlinux.org/packages/extra/x86\_64/mesa/}{mesa}, \sphinxhref{https://archlinux.org/packages/multilib/x86\_64/lib32-mesa/}{lib32\sphinxhyphen{}mesa}, если у вас Intel/AMD),
\sphinxhref{https://archlinux.org/packages/extra/x86\_64/xorg-server/}{Xorg сервер}, а также связанные с ним компоненты, или \sphinxhref{https://archlinux.org/packages/extra/x86\_64/wayland/}{Wayland},
критически важные пакеты вашего DE/WM, например: \sphinxhref{https://aur.archlinux.org/packages/gnome-shell-performance}{gnome\sphinxhyphen{}shell}, \sphinxhref{https://archlinux.org/packages/extra/x86\_64/plasma-desktop/}{plasma\sphinxhyphen{}desktop}.
А также композиторы \sphinxhref{https://archlinux.org/packages/extra/x86\_64/kwin/}{kwin}, \sphinxhref{https://aur.archlinux.org/packages/mutter-performance}{mutter}, picom и т.д. в зависимости от того, чем именно вы пользуетесь.

\sphinxAtStartPar
Альтернативно, вы можете использовать уже подготовленный репозиторий \sphinxhref{https://github.com/h0tc0d3/arch-packages}{arch\sphinxhyphen{}packages}
с полной поддержкой сборки пакетов через LLVM/Clang. В этом репозитории представлены не все возможные пакеты, но самые
важные компоненты системы там есть, включая сам llvm\sphinxhyphen{}git, который вы тоже можете пересобрать:

\begin{sphinxVerbatim}[commandchars=\\\{\}]
git clone https://github.com/h0tc0d3/arch\PYGZhy{}packages
\PYG{n+nb}{cd} arch\PYGZhy{}packages
\PYG{n+nb}{cd} llvm\PYGZhy{}git
makepkg \PYGZhy{}sric
\end{sphinxVerbatim}

\sphinxAtStartPar
(Вместо \sphinxstyleemphasis{llvm\sphinxhyphen{}git} может быть любой другой пакет, доступный в данном репозитории)

\sphinxAtStartPar
Больше подробностей по теме вы можете найти в данной статье:

\sphinxAtStartPar
\sphinxurl{https://habr.com/ru/company/ruvds/blog/561286/}

\index{clang@\spxentry{clang}}\index{native\sphinxhyphen{}compilation@\spxentry{native\sphinxhyphen{}compilation}}\index{llvm\sphinxhyphen{}bolt\sphinxhyphen{}builds@\spxentry{llvm\sphinxhyphen{}bolt\sphinxhyphen{}builds}}\index{lto@\spxentry{lto}}\index{pgo@\spxentry{pgo}}\ignorespaces 

\subsubsection{Ускорение работы компиляторов LLVM/Clang}
\label{\detokenize{source/generic-system-acceleration:llvm-clang}}\label{\detokenize{source/generic-system-acceleration:speeding-up-clang-llvm-compilers}}\label{\detokenize{source/generic-system-acceleration:index-2}}
\sphinxAtStartPar
Дополнительно можно отметить, что после установки Clang вы можете перекомпилировать его самого через себя,
т.е. выполнить пересборку Clang с помощью бинарного Clang из репозиториев.
Это позволит оптимизировать уже сам компилятор под ваше железо и тем самым ускорить
его работу при сборке уже других программ. Аналогичную операцию вы можете проделать и с GCC.

\sphinxAtStartPar
Делается это так же, как и с любыми другими пакетами из официальных репозиториев:

\begin{sphinxVerbatim}[commandchars=\\\{\}]
git clone \PYGZhy{}\PYGZhy{}depth \PYG{l+m}{1} \PYGZhy{}\PYGZhy{}branch packages/clang https://github.com/archlinux/svntogit\PYGZhy{}packages.git clang
\PYG{n+nb}{cd} clang/trunk
makepkg \PYGZhy{}\PYGZhy{}config /etc/makepkg\PYGZhy{}clang.conf \PYGZhy{}sric \PYGZhy{}\PYGZhy{}skippgpcheck
\end{sphinxVerbatim}

\index{makepkg@\spxentry{makepkg}}\index{ccache@\spxentry{ccache}}\index{native\sphinxhyphen{}compilation@\spxentry{native\sphinxhyphen{}compilation}}\ignorespaces 

\subsection{Включение ccache}
\label{\detokenize{source/generic-system-acceleration:ccache}}\label{\detokenize{source/generic-system-acceleration:enabling-ccache}}\label{\detokenize{source/generic-system-acceleration:index-3}}
\sphinxAtStartPar
В Linux системах есть не так много программ, сборка которых может занять больше двух часов,
но они все таки есть. Потому, было бы неплохо ускорить повторную компиляцию таких программ как Wine/Proton\sphinxhyphen{}GE и т.д.

\sphinxAtStartPar
ccache \sphinxhyphen{} это кэш для компиляторов C/C++, в частности совместимый с компиляторами GCC/Clang,
цель которого состоит в ускорении повторного процесса компиляции одного и того же кода.
Это значит, что если при сборке программы новой версии, будут замечены полностью идентичные блоки исходного кода в сравнении с его старой версией,
то компиляция этих исходных текстов производиться не будет. Вместо этого, уже готовый, скомпилированный код старой версии будет вынут из кэша ccache.
За счёт этого и достигается многократное ускорение процесса компиляции.

\sphinxAtStartPar
\sphinxstylestrong{Установка}

\begin{sphinxVerbatim}[commandchars=\\\{\}]
sudo pacman \PYGZhy{}S ccache
\end{sphinxVerbatim}

\sphinxAtStartPar
После установки его ещё нужно активировать в ваших настройках makepkg.
Для этого отредактируем конфигурационный файл:

\begin{sphinxVerbatim}[commandchars=\\\{\}]
sudo nano /etc/makepkg.conf

\PYG{c+c1}{\PYGZsh{} Найдите данную строку в собственных настройках, затем уберите восклицательный знак перед *\PYGZdq{}ccache\PYGZdq{}*}
\PYG{n+nv}{BUILDENV}\PYG{o}{=}\PYG{o}{(}!distcc color ccache check !sign\PYG{o}{)}
\end{sphinxVerbatim}

\sphinxAtStartPar
После этого повторная пересборка желаемых программ и их обновление должны значительно ускориться.

\begin{sphinxadmonition}{attention}{Внимание:}
\sphinxAtStartPar
ccache может ломать сборку некоторых программ, поэтому будьте внимательны с его применением.
\end{sphinxadmonition}

\index{installation@\spxentry{installation}}\index{ananicy@\spxentry{ananicy}}\index{zram@\spxentry{zram}}\index{nohang@\spxentry{nohang}}\index{rng\sphinxhyphen{}tools@\spxentry{rng\sphinxhyphen{}tools}}\index{haveged@\spxentry{haveged}}\index{trim@\spxentry{trim}}\index{dbus\sphinxhyphen{}broker@\spxentry{dbus\sphinxhyphen{}broker}}\ignorespaces 

\section{Установка полезных служб и демонов}
\label{\detokenize{source/generic-system-acceleration:daemons-and-services}}\label{\detokenize{source/generic-system-acceleration:index-4}}\label{\detokenize{source/generic-system-acceleration:id3}}
\sphinxAtStartPar
\sphinxstylestrong{1.} \sphinxhref{https://aur.archlinux.org/packages/zramswap/}{Zramswap} — это специальный демон,
который сжимает оперативную память ресурсами центрального процессора и создает в ней файл подкачки.
Очень ускоряет систему вне зависимости от количества памяти, однако добавляет нагрузку на процессор, т.к. его ресурсами и происходит сжатие памяти.
Поэтому, на слабых компьютерах с малым количеством ОЗУ, это может негативно повлиять на производительность в целом.

\begin{sphinxVerbatim}[commandchars=\\\{\}]
git clone https://aur.archlinux.org/zramswap.git  \PYG{c+c1}{\PYGZsh{} Скачивание исходников.}
\PYG{n+nb}{cd} zramswap                                       \PYG{c+c1}{\PYGZsh{} Переход в zramswap.}
makepkg \PYGZhy{}sric                                     \PYG{c+c1}{\PYGZsh{} Сборка и установка.}
sudo systemctl \PYG{n+nb}{enable} \PYGZhy{}\PYGZhy{}now zramswap.service      \PYG{c+c1}{\PYGZsh{} Включаем службу.}
\end{sphinxVerbatim}

\sphinxAtStartPar
\sphinxstylestrong{1.1} \sphinxhref{https://github.com/hakavlad/nohang}{Nohang}  — это демон повышающий производительность путём обработки и слежки за потреблением памяти.

\begin{sphinxVerbatim}[commandchars=\\\{\}]
git clone https://aur.archlinux.org/nohang\PYGZhy{}git.git \PYG{c+c1}{\PYGZsh{} Скачивание исходников.}
\PYG{n+nb}{cd} nohang\PYGZhy{}git                                      \PYG{c+c1}{\PYGZsh{} Переход в nohang\PYGZhy{}git}
makepkg \PYGZhy{}sric                                      \PYG{c+c1}{\PYGZsh{} Сборка и установка.}
sudo systemctl \PYG{n+nb}{enable} \PYGZhy{}\PYGZhy{}now nohang\PYGZhy{}desktop         \PYG{c+c1}{\PYGZsh{} Включаем службу.}
\end{sphinxVerbatim}

\sphinxAtStartPar
\sphinxstylestrong{1.2} \sphinxhref{https://gitlab.com/ananicy-cpp/ananicy-cpp}{Ananicy CPP} — это форк одноименного демона, распределяющий приоритет задач. Его установка очень сильно повышает отклик системы. В отличии от оригинального Ananicy, данный форк переписан полностью на C++, из\sphinxhyphen{}за чего достигается прирост в скорости работы.

\begin{sphinxVerbatim}[commandchars=\\\{\}]
git clone https://aur.archlinux.org/ananicy\PYGZhy{}cpp.git \PYG{c+c1}{\PYGZsh{} Скачивание исходников.}
\PYG{n+nb}{cd} ananicy\PYGZhy{}cpp                                      \PYG{c+c1}{\PYGZsh{} Переход в ananicy\PYGZhy{}cpp.}
makepkg \PYGZhy{}sric                                       \PYG{c+c1}{\PYGZsh{} Сборка и установка.}
sudo systemctl \PYG{n+nb}{enable} \PYGZhy{}\PYGZhy{}now ananicy\PYGZhy{}cpp             \PYG{c+c1}{\PYGZsh{} Включаем службу.}

\PYG{c+c1}{\PYGZsh{} Далее описывается установка дополнительных правил по перераспределению приоритетов процессов}
git clone https://aur.archlinux.org/ananicy\PYGZhy{}rules\PYGZhy{}git.git \PYG{c+c1}{\PYGZsh{} Скачивание исходников}
\PYG{n+nb}{cd} ananicy\PYGZhy{}rules\PYGZhy{}git                                      \PYG{c+c1}{\PYGZsh{} Переход в директорию}
makepkg \PYGZhy{}sric                                             \PYG{c+c1}{\PYGZsh{} Сборка и установка}
sudo systemctl restart ananicy\PYGZhy{}cpp                        \PYG{c+c1}{\PYGZsh{} Перезапускаем службу}
\end{sphinxVerbatim}

\sphinxAtStartPar
\sphinxstylestrong{1.3} Включаем \sphinxhref{https://ru.wikipedia.org/wiki/Trim\_(команда\_для\_накопителей)}{TRIM} — очень полезно для SSD.

\begin{sphinxVerbatim}[commandchars=\\\{\}]
sudo systemctl \PYG{n+nb}{enable} fstrim.timer    \PYG{c+c1}{\PYGZsh{} Включаем службу.}
sudo fstrim \PYGZhy{}v /                      \PYG{c+c1}{\PYGZsh{} Ручной метод.}
sudo fstrim \PYGZhy{}va /                     \PYG{c+c1}{\PYGZsh{} Если первый метод не тримит весь диск.}
\end{sphinxVerbatim}

\sphinxAtStartPar
\sphinxstylestrong{1.4} \sphinxhref{https://wiki.archlinux.org/title/cron}{Сron} — это демон, который поможет вам очищать вашу систему от мусора полностью автономно.

\begin{sphinxVerbatim}[commandchars=\\\{\}]
sudo pacman \PYGZhy{}S cronie                         \PYG{c+c1}{\PYGZsh{} Установить cron.}
sudo systemctl \PYG{n+nb}{enable} \PYGZhy{}\PYGZhy{}now cronie.service    \PYG{c+c1}{\PYGZsh{} Запускает и включает службу.}
sudo \PYG{n+nv}{EDITOR}\PYG{o}{=}nano crontab \PYGZhy{}e                   \PYG{c+c1}{\PYGZsh{} Редактируем параметр.}
\end{sphinxVerbatim}

\sphinxAtStartPar
И прописываем:

\sphinxAtStartPar
\sphinxstyleemphasis{15 10 * * sun /sbin/pacman \sphinxhyphen{}Scc \sphinxhyphen{}\sphinxhyphen{}noconfirm}

\sphinxAtStartPar
Таким образом наша система будет чистить свой кэш раз в неделю, в воскресенье в 15:10.

\sphinxAtStartPar
\sphinxstylestrong{1.5} \sphinxhref{https://wiki.archlinux.org/title/Haveged\_(Русский)}{haveged} \sphinxhyphen{} это демон, что следит за энтропией системы.
Необходим для ускорения запуска системы при высоких показателях в: \sphinxstyleemphasis{systemd\sphinxhyphen{}analyze blame} (Больше 1 секунды).

\begin{sphinxVerbatim}[commandchars=\\\{\}]
sudo pacman \PYGZhy{}S haveged        \PYG{c+c1}{\PYGZsh{} Установка}
sudo systemctl \PYG{n+nb}{enable} haveged \PYG{c+c1}{\PYGZsh{} Включает и запускает службу.}
\end{sphinxVerbatim}

\sphinxAtStartPar
\sphinxstylestrong{1.5.1} \sphinxhref{https://wiki.archlinux.org/title/Rng-tools}{rng\sphinxhyphen{}tools} \sphinxhyphen{} демон, что также следит за энтропией системы, но в отличие от haveged уже через аппаратный таймер.
Необходим для ускорения запуска системы при высоких показателях \sphinxstyleemphasis{systemd\sphinxhyphen{}analyze blame} (Больше 1 секунды).

\begin{sphinxVerbatim}[commandchars=\\\{\}]
sudo pacman \PYGZhy{}S rng\PYGZhy{}tools         \PYG{c+c1}{\PYGZsh{} Установка}
sudo systemctl \PYG{n+nb}{enable} \PYGZhy{}\PYGZhy{}now rngd \PYG{c+c1}{\PYGZsh{} Включает и запускает службу.}
\end{sphinxVerbatim}

\sphinxAtStartPar
\sphinxstylestrong{1.6} \sphinxhref{https://github.com/bus1/dbus-broker}{dbus\sphinxhyphen{}broker} \sphinxhyphen{} Это реализация шины сообщений в соответствии со спецификацией D\sphinxhyphen{}Bus.
Её цель \sphinxhyphen{} обеспечить высокую производительность и надежность при сохранении совместимости с эталонной реализацией D\sphinxhyphen{}Bus.
Обеспечивает чуть более быстрое общение с видеокартой через PCIe.

\begin{sphinxVerbatim}[commandchars=\\\{\}]
sudo pacman \PYGZhy{}S dbus\PYGZhy{}broker                         \PYG{c+c1}{\PYGZsh{} Уставновка}
sudo systemctl \PYG{n+nb}{enable} \PYGZhy{}\PYGZhy{}now dbus\PYGZhy{}broker.service    \PYG{c+c1}{\PYGZsh{} Включает и запускает службу.}
sudo systemctl \PYGZhy{}\PYGZhy{}global \PYG{n+nb}{enable} dbus\PYGZhy{}broker.service \PYG{c+c1}{\PYGZsh{} Включает и запускает службу для всех пользователей.}
\end{sphinxVerbatim}

\sphinxAtStartPar
Если у вас ещё возникает вопрос: "Что действительно нужно установить из вышеперечисленного?",
то просто посмотрите на следующую схему:

\noindent\sphinxincludegraphics{{generic-system-acceleration-1}.png}

\index{lowlatency@\spxentry{lowlatency}}\index{audio@\spxentry{audio}}\index{pusleaudio@\spxentry{pusleaudio}}\ignorespaces 

\section{Низкие задержки звука}
\label{\detokenize{source/generic-system-acceleration:lowlatency-audio}}\label{\detokenize{source/generic-system-acceleration:index-5}}\label{\detokenize{source/generic-system-acceleration:id4}}
\sphinxAtStartPar
Установите следующие пакеты для понижения задержек звука в PulseAudio,
а также удобную графическую панель управления звуком \sphinxhyphen{}  \sphinxstyleemphasis{pavucontrol}.

\begin{sphinxVerbatim}[commandchars=\\\{\}]
sudo pacman \PYGZhy{}S jack2 pulseaudio\PYGZhy{}alsa pulseaudio\PYGZhy{}jack pavucontrol jack2\PYGZhy{}dbus realtime\PYGZhy{}privileges
\end{sphinxVerbatim}

\index{installation@\spxentry{installation}}\index{lowlatency@\spxentry{lowlatency}}\index{audio@\spxentry{audio}}\index{pipewire@\spxentry{pipewire}}\ignorespaces 

\subsection{Новая альтернатива PulseAudio}
\label{\detokenize{source/generic-system-acceleration:pulseaudio}}\label{\detokenize{source/generic-system-acceleration:pipewire-installation}}\label{\detokenize{source/generic-system-acceleration:index-6}}
\sphinxAtStartPar
\sphinxhref{https://wiki.archlinux.org/title/PipeWire\_(Русский)}{PipeWire} \sphinxhyphen{} это новая альтернатива PulseAudio,
которая призвана избавить от проблем pulse, уменьшить задержки звука и потребление памяти.

\begin{sphinxVerbatim}[commandchars=\\\{\}]
sudo pacman \PYGZhy{}S jack2 pipewire pipewire\PYGZhy{}alsa pavucontrol pipewire\PYGZhy{}pulse alsa\PYGZhy{}utils wireplumber
\end{sphinxVerbatim}

\index{lowlatency@\spxentry{lowlatency}}\index{audio@\spxentry{audio}}\index{alsa@\spxentry{alsa}}\ignorespaces 

\subsection{Простая ALSA}
\label{\detokenize{source/generic-system-acceleration:alsa}}\label{\detokenize{source/generic-system-acceleration:index-7}}\label{\detokenize{source/generic-system-acceleration:id5}}
\sphinxAtStartPar
ALSA \sphinxhyphen{} это тот самый звук (условно, на самом деле это звуковая подсистема ядра),
который идёт напрямую из ядра и является самым быстрым,
так как не вынужден проходить множество программных прослоек и микширование.

\begin{sphinxVerbatim}[commandchars=\\\{\}]
sudo pacman \PYGZhy{}S alsa alsa\PYGZhy{}utils alsa\PYGZhy{}firmware alsa\PYGZhy{}card\PYGZhy{}profiles alsa\PYGZhy{}plugins
\end{sphinxVerbatim}

\sphinxAtStartPar
Поэтому, если у вас нет потребности в микшировании каналов,
записи аудио через микрофон и вы не слушаете музыку через Bluetooth, то ALSA может вам подойти.Пакет \sphinxstyleemphasis{alsa\sphinxhyphen{}utils} также содержит консольный Микшер (настройка громкости), который вызывается командой alsamixer.

\sphinxAtStartPar
Вообще, выбор звукового сервера не такая уж сложная задача как вам может показаться,
достаточно взглянуть на следующую схему:

\noindent\sphinxincludegraphics{{generic-system-acceleration-2}.png}

\index{startup\sphinxhyphen{}acceleration@\spxentry{startup\sphinxhyphen{}acceleration}}\index{networkmanager@\spxentry{networkmanager}}\index{service@\spxentry{service}}\ignorespaces 

\section{Ускорение загрузки системы (Отключение NetworkManager\sphinxhyphen{}wait\sphinxhyphen{}online)}
\label{\detokenize{source/generic-system-acceleration:networkmanager-wait-online}}\label{\detokenize{source/generic-system-acceleration:startup-acceleration}}\label{\detokenize{source/generic-system-acceleration:index-8}}
\sphinxAtStartPar
В большинстве случаев для настройки интернет подключения вы, скорее всего, будете использовать NetworkManager,
т.к. он является в этом деле швейцарским ножом и поставляется по умолчанию.
Однако, если вы пропишите команду \sphinxstyleemphasis{systemd\sphinxhyphen{}analyze blame}, то узнаете, что он задерживает загрузку системы примерно на \textasciitilde{}4 секунды.
Чтобы это исправить выполните:

\begin{sphinxVerbatim}[commandchars=\\\{\}]
sudo systemctl mask NetworkManager\PYGZhy{}wait\PYGZhy{}online.service
\end{sphinxVerbatim}

\index{startup\sphinxhyphen{}acceleration@\spxentry{startup\sphinxhyphen{}acceleration}}\index{hdd@\spxentry{hdd}}\index{lz4@\spxentry{lz4}}\index{mkinitcpio@\spxentry{mkinitcpio}}\ignorespaces 

\subsection{Ускорение загрузки ядра на HDD накопителях (\sphinxstyleemphasis{Только для жестких дисков})}
\label{\detokenize{source/generic-system-acceleration:hdd}}\label{\detokenize{source/generic-system-acceleration:speed-up-hdd-startup}}\label{\detokenize{source/generic-system-acceleration:index-9}}
\sphinxAtStartPar
Убедитесь, что пакет \sphinxhref{https://archlinux.org/packages/core/x86\_64/lz4/}{lz4} установлен:

\begin{sphinxVerbatim}[commandchars=\\\{\}]
sudo pacman \PYGZhy{}S lz4
\end{sphinxVerbatim}

\sphinxAtStartPar
Отредактируйте файл::

\begin{sphinxVerbatim}[commandchars=\\\{\}]
sudo nano /etc/mkinitcpio.conf
\end{sphinxVerbatim}

\sphinxAtStartPar
Теперь выполните следующие действия:
\begin{itemize}
\item {} 
\sphinxAtStartPar
Добавьте \sphinxstyleemphasis{lz4 lz4\_compress} в массив \sphinxstyleemphasis{MODULES} (ограничен скобками)

\item {} 
\sphinxAtStartPar
Раскомментируйте или добавьте строку с надписью \sphinxstyleemphasis{COMPRESSION="lz4"}

\item {} 
\sphinxAtStartPar
Добавьте строку если её нет \sphinxhyphen{}  \sphinxstyleemphasis{COMPRESSION\_OPTIONS="\sphinxhyphen{}9"}

\item {} 
\sphinxAtStartPar
Добавите \sphinxstyleemphasis{shutdown} в массив \sphinxstyleemphasis{HOOKS} (ограничен скобками)

\end{itemize}

\sphinxAtStartPar
Это ускорит загрузку системы на слабых жёстких дисках благодаря более подходящему методу сжатия образов ядра.

\index{pacman@\spxentry{pacman}}\index{settings@\spxentry{settings}}\index{parallel\sphinxhyphen{}downloading@\spxentry{parallel\sphinxhyphen{}downloading}}\ignorespaces 

\section{Одновременная загрузка двух и более пакетов}
\label{\detokenize{source/generic-system-acceleration:parallel-downloading}}\label{\detokenize{source/generic-system-acceleration:index-10}}\label{\detokenize{source/generic-system-acceleration:id6}}
\sphinxAtStartPar
Начиная с шестой версии pacman поддерживает параллельную загрузку пакетов.
Чтобы её включить отредактируйте \sphinxstyleemphasis{/etc/pacman.conf}:

\begin{sphinxVerbatim}[commandchars=\\\{\}]
sudo nano /etc/pacman.conf \PYG{c+c1}{\PYGZsh{} Раскомментируйте строчку ниже}

\PYG{c+c1}{\PYGZsh{} Где 4 \PYGZhy{} количество пакетов для одновременной загрузки}
\PYG{n+nv}{ParallelDownloads} \PYG{o}{=} \PYG{l+m}{4}
\end{sphinxVerbatim}

\index{powerpill@\spxentry{powerpill}}\index{parallel\sphinxhyphen{}downloading@\spxentry{parallel\sphinxhyphen{}downloading}}\ignorespaces 

\subsection{Альтернативно можно использовать powerpill (Спасибо Zee Captain)}
\label{\detokenize{source/generic-system-acceleration:powerpill-zee-captain}}\label{\detokenize{source/generic-system-acceleration:powerpill}}\label{\detokenize{source/generic-system-acceleration:index-11}}
\begin{sphinxVerbatim}[commandchars=\\\{\}]
git clone https://aur.archlinux.org/powerpill.git
\PYG{n+nb}{cd} powerpill
makepkg \PYGZhy{}sric
\end{sphinxVerbatim}

\sphinxAtStartPar
После установки выполните обновление баз данных:

\begin{sphinxVerbatim}[commandchars=\\\{\}]
sudo powerpill \PYGZhy{}Syu
\end{sphinxVerbatim}


\section{Твики драйверов Mesa}
\label{\detokenize{source/generic-system-acceleration:id7}}
\index{amd@\spxentry{amd}}\index{sam@\spxentry{sam}}\index{bar@\spxentry{bar}}\ignorespaces 

\subsection{Форсирование использования AMD SAM \sphinxstyleemphasis{(Только для опытных пользователей)}.}
\label{\detokenize{source/generic-system-acceleration:amd-sam}}\label{\detokenize{source/generic-system-acceleration:force-amd-sam}}\label{\detokenize{source/generic-system-acceleration:index-12}}
\sphinxAtStartPar
AMD Smart Acess Memory (или Resizble Bar) — это технология которая позволяет процессору получить доступ сразу ко всей видеопамяти GPU,
а не по отдельности для каждого распаянного чипа создавая задержки. Несмотря на то,
что данная технология заявлена только для оборудования AMD и требует новейших комплектующих для обеспечения своей работы,
мы активируем технологию для видеокарты 10 летней давновсти ATI Radeon HD 7770 и сравним буст производительности в паре игр.

\begin{sphinxadmonition}{danger}{Опасно:}
\sphinxAtStartPar
Для включения данной технологии в настройках вашего BIOS (UEFI) должна быть включена опция \sphinxstyleemphasis{"Re\sphinxhyphen{}Size BAR Support"} и \sphinxstyleemphasis{"Above 4G Decoding"}.
Если таких параметров в вашем BIOS (UEFI) нет \sphinxhyphen{} скорее всего технология не поддерживается вашей материнской платой и не стоит даже пытаться её включить.
\end{sphinxadmonition}

\sphinxAtStartPar
Чтобы активировать SAM в Linux нужно отредактировать конфигурацию DRI, дописав в конфиг следующие строки:

\begin{sphinxVerbatim}[commandchars=\\\{\}]
nano \PYGZti{}/.drirc \PYG{c+c1}{\PYGZsh{} Редактируем конфигурационный файл}

\PYG{c+c1}{\PYGZsh{} Прописать строки ниже}

\PYGZlt{}?xml \PYG{n+nv}{version}\PYG{o}{=}\PYG{l+s+s2}{\PYGZdq{}1.0\PYGZdq{}} \PYG{n+nv}{standalone}\PYG{o}{=}\PYG{l+s+s2}{\PYGZdq{}yes\PYGZdq{}}?\PYGZgt{}
\PYGZlt{}driconf\PYGZgt{}
  \PYGZlt{}device\PYGZgt{}
    \PYGZlt{}application \PYG{n+nv}{name}\PYG{o}{=}\PYG{l+s+s2}{\PYGZdq{}Default\PYGZdq{}}\PYGZgt{}
      \PYGZlt{}option \PYG{n+nv}{name}\PYG{o}{=}\PYG{l+s+s2}{\PYGZdq{}radeonsi\PYGZus{}enable\PYGZus{}sam\PYGZdq{}} \PYG{n+nv}{value}\PYG{o}{=}\PYG{l+s+s2}{\PYGZdq{}true\PYGZdq{}} /\PYGZgt{}
    \PYGZlt{}/application\PYGZgt{}
  \PYGZlt{}/device\PYGZgt{}
\PYGZlt{}/driconf\PYGZgt{}
\end{sphinxVerbatim}

\sphinxAtStartPar
Альтернативно её можно активировать через глобальные переменные окружения:

\begin{sphinxVerbatim}[commandchars=\\\{\}]
sudo nano /etc/enviroment \PYG{c+c1}{\PYGZsh{} Редактируем конфигурационный файл}

\PYG{c+c1}{\PYGZsh{} Добавить следующие строки}
\PYG{n+nv}{radeonsi\PYGZus{}enable\PYGZus{}sam}\PYG{o}{=}\PYG{n+nb}{true}
\PYG{c+c1}{\PYGZsh{} Если используете драйвер RADV}
\PYG{n+nv}{RADV\PYGZus{}PERFTEST}\PYG{o}{=}sam
\end{sphinxVerbatim}

\sphinxAtStartPar
Проверить работу технологии можно через команду:

\begin{sphinxVerbatim}[commandchars=\\\{\}]
\PYG{n+nv}{AMD\PYGZus{}DEBUG}\PYG{o}{=}info glxinfo \PYG{p}{|} grep smart \PYG{c+c1}{\PYGZsh{} Должно быть smart\PYGZus{}access\PYGZus{}memory = 1}
\end{sphinxVerbatim}

\sphinxAtStartPar
\sphinxstylestrong{Пример тестирования технологии на видеокарте старого поколения (Windows)}

\sphinxAtStartPar
\sphinxurl{https://youtu.be/tZmPi9tfLbc}

\index{amd@\spxentry{amd}}\index{tweaks@\spxentry{tweaks}}\ignorespaces 

\subsection{Решение проблем работы графики Vega 11 (Спасибо @Vochatrak\sphinxhyphen{}az\sphinxhyphen{}ezm)}
\label{\detokenize{source/generic-system-acceleration:vega-11-vochatrak-az-ezm}}\label{\detokenize{source/generic-system-acceleration:bug-solution-for-vega}}\label{\detokenize{source/generic-system-acceleration:index-13}}
\sphinxAtStartPar
На оборудовании со встроенным видеоядром Vega 11 может встретиться баг драйвера, при котором возникают случайные зависания графики.
Проблема наиболее актуальна для \sphinxstyleemphasis{Ryzen 2XXXG} и чуть реже встречается на Ryzen серии \sphinxstyleemphasis{3XXXG}, но потенциально имеет место быть и на более
новых видеоядрах Vega.

\sphinxAtStartPar
Решается через добавление следующих параметров ядра:

\begin{sphinxVerbatim}[commandchars=\\\{\}]
\PYG{c+c1}{\PYGZsh{} Редактируем конфигурационный файл в зависимости от того, какой у вас загрузчик}
sudo nano /etc/default/grub

\PYG{c+c1}{\PYGZsh{} Параметры можно дописать к уже имеющимся}
\PYG{n+nv}{GRUB\PYGZus{}CMDLINE\PYGZus{}LINUX\PYGZus{}DEFAULT}\PYG{o}{=}\PYG{l+s+s2}{\PYGZdq{}mdgpu.gttsize=8192 amdgpu.lockup\PYGZus{}timeout=1000 amdgpu.gpu\PYGZus{}recovery=1 amdgpu.noretry=0 amdgpu.ppfeaturemask=0xfffd3fff amdgpu.deep\PYGZus{}color=1 systemd.unified\PYGZus{}cgroup\PYGZus{}hierarchy=true\PYGZdq{}}
\end{sphinxVerbatim}

\sphinxAtStartPar
На всякий случай можно дописать ещё одну переменную окружения:

\begin{sphinxVerbatim}[commandchars=\\\{\}]
\PYG{c+c1}{\PYGZsh{} Прописать строчку ниже}
sudo nano /etc/enviroment

\PYG{n+nv}{AMD\PYGZus{}DEBUG}\PYG{o}{=}nodcc
\end{sphinxVerbatim}

\sphinxAtStartPar
Для подробностей можете ознакомиться со следующими темами:

\sphinxAtStartPar
\sphinxurl{https://www.linux.org.ru/forum/linux-hardware/16312119}

\sphinxAtStartPar
\sphinxurl{https://www.linux.org.ru/forum/desktop/16257286}

\sphinxstepscope


\chapter{Экстра оптимизации}
\label{\detokenize{source/extra-optimizations:extra-optimizations}}\label{\detokenize{source/extra-optimizations:id1}}\label{\detokenize{source/extra-optimizations::doc}}
\index{cpu@\spxentry{cpu}}\index{cpupower@\spxentry{cpupower}}\index{governor@\spxentry{governor}}\index{performance@\spxentry{performance}}\ignorespaces 

\section{Перевод процессора из стандартного энергосбережения в режим производительности}
\label{\detokenize{source/extra-optimizations:maximum-cpu-performance}}\label{\detokenize{source/extra-optimizations:index-0}}\label{\detokenize{source/extra-optimizations:id2}}
\sphinxAtStartPar
По умолчанию ваш процессор динамически меняет свою частоту, что в принципе правильно и дает баланс между энергосбережением и производительностью.
Но если вы все таки хотите выжать все соки, то вы можете закрепить применение режима максимальной производительности для вашего процессора.
Это также поможет вам избегать "падений" частоты во время игры, которые могли вызывать микрофризы во время игры.

\sphinxAtStartPar
Закрепим режим максимальной производительности:

\begin{sphinxVerbatim}[commandchars=\\\{\}]
sudo pacman \PYGZhy{}S cpupower                       \PYG{c+c1}{\PYGZsh{} Установит менеджер управления частотой процессора}
sudo cpupower frequency\PYGZhy{}set \PYGZhy{}g performance    \PYG{c+c1}{\PYGZsh{} Выставляет максимальную  производительность до перезагрузки системы.}
\end{sphinxVerbatim}

\sphinxAtStartPar
\sphinxcode{\sphinxupquote{sudo nano /etc/default/cpupower}} \# Редактируем строчку ниже

\noindent\sphinxincludegraphics{{extra-optimizations-1}.png}

\sphinxAtStartPar
\sphinxstyleemphasis{governor=’performance’} \# Высокая производительность всегда!

\sphinxAtStartPar
\sphinxcode{\sphinxupquote{sudo systemctl enable cpupower}} \# Включить как постоянную службу которая установит вечный perfomance mode.

\index{cpupower@\spxentry{cpupower}}\index{gui@\spxentry{gui}}\index{frequencies@\spxentry{frequencies}}\index{governor@\spxentry{governor}}\index{performance@\spxentry{performance}}\ignorespaces 

\subsection{GUI для изменение частоты процессора (\sphinxstyleemphasis{Может не работать с Xanmod})}
\label{\detokenize{source/extra-optimizations:gui-xanmod}}\label{\detokenize{source/extra-optimizations:cpupower-gui}}\label{\detokenize{source/extra-optimizations:index-1}}
\noindent\sphinxincludegraphics{{extra-optimizations-2}.png}

\sphinxAtStartPar
\sphinxstylestrong{Установка}:

\begin{sphinxVerbatim}[commandchars=\\\{\}]
git clone https://aur.archlinux.org/cpupower\PYGZhy{}gui.git
\PYG{n+nb}{cd} cpupower\PYGZhy{}gui
makepkg \PYGZhy{}sric
\end{sphinxVerbatim}

\index{cpupower@\spxentry{cpupower}}\index{auto\sphinxhyphen{}cpufreq@\spxentry{auto\sphinxhyphen{}cpufreq}}\index{frequencies@\spxentry{frequencies}}\index{governor@\spxentry{governor}}\index{performance@\spxentry{performance}}\ignorespaces 

\subsection{Альтернатива \sphinxhyphen{} Auto\sphinxhyphen{}Cpufreq}
\label{\detokenize{source/extra-optimizations:auto-cpufreq}}\label{\detokenize{source/extra-optimizations:index-2}}\label{\detokenize{source/extra-optimizations:id3}}
\sphinxAtStartPar
\sphinxstylestrong{Установка}:

\begin{sphinxVerbatim}[commandchars=\\\{\}]
git clone https://aur.archlinux.org/auto\PYGZhy{}cpufreq\PYGZhy{}git.git  \PYG{c+c1}{\PYGZsh{} Скачиваем исходники}
\PYG{n+nb}{cd} auto\PYGZhy{}cpufreq\PYGZhy{}git                                       \PYG{c+c1}{\PYGZsh{} Переходим в директорию}
makepkg \PYGZhy{}sric                                             \PYG{c+c1}{\PYGZsh{} Сборка и установка}
systemctl \PYG{n+nb}{enable} auto\PYGZhy{}cpufreq                             \PYG{c+c1}{\PYGZsh{} Включает службу как постоянную}
systemctl start auto\PYGZhy{}cpufreq                              \PYG{c+c1}{\PYGZsh{} Запускает службу}
\end{sphinxVerbatim}

\begin{sphinxadmonition}{attention}{Внимание:}
\sphinxAtStartPar
Может конфликтовать со встроенным менеджером питания в GNOME 41+.
Убедитесь, что он у вас выключен:

\begin{sphinxVerbatim}[commandchars=\\\{\}]
sudo systemctl disable \PYGZhy{}\PYGZhy{}now power\PYGZhy{}profiles\PYGZhy{}daemon.service
\end{sphinxVerbatim}
\end{sphinxadmonition}

\index{hibernation@\spxentry{hibernation}}\index{suspend@\spxentry{suspend}}\index{polkit@\spxentry{polkit}}\ignorespaces 

\section{Отключение спящего режима и гибернации}
\label{\detokenize{source/extra-optimizations:disabling-hibernation-and-sleep}}\label{\detokenize{source/extra-optimizations:index-3}}\label{\detokenize{source/extra-optimizations:id4}}
\sphinxAtStartPar
\sphinxcode{\sphinxupquote{sudo pacman \sphinxhyphen{}S polkit}}  \# Установить для управления системными привилегиями.

\sphinxAtStartPar
\sphinxcode{\sphinxupquote{sudo nano /etc/polkit\sphinxhyphen{}1/rules.d/10\sphinxhyphen{}disable\sphinxhyphen{}suspend.rules}}  \# Убираем спящий режим и гибернацию (из меню и вообще).
Если такого файла нет, то создайте его. Файл должен выглядеть вот так:

\begin{sphinxVerbatim}[commandchars=\\\{\}]
polkit.addRule\PYG{o}{(}\PYG{k}{function}\PYG{o}{(}action, subject\PYG{o}{)} \PYG{o}{\PYGZob{}}
  \PYG{k}{if} \PYG{o}{(}action.id \PYG{o}{=}\PYG{o}{=} \PYG{l+s+s2}{\PYGZdq{}org.freedesktop.login1.suspend\PYGZdq{}} \PYG{o}{||}
      action.id \PYG{o}{=}\PYG{o}{=} \PYG{l+s+s2}{\PYGZdq{}org.freedesktop.login1.suspend\PYGZhy{}multiple\PYGZhy{}sessions\PYGZdq{}} \PYG{o}{||}
      action.id \PYG{o}{=}\PYG{o}{=} \PYG{l+s+s2}{\PYGZdq{}org.freedesktop.login1.hibernate\PYGZdq{}} \PYG{o}{||}
      action.id \PYG{o}{=}\PYG{o}{=} \PYG{l+s+s2}{\PYGZdq{}org.freedesktop.login1.hibernate\PYGZhy{}multiple\PYGZhy{}sessions\PYGZdq{}}\PYG{o}{)}
  \PYG{o}{\PYGZob{}}
      \PYG{k}{return} polkit.Result.NO\PYG{p}{;}
  \PYG{o}{\PYGZcb{}}
\PYG{o}{\PYGZcb{}}\PYG{o}{)}\PYG{p}{;}
\end{sphinxVerbatim}

\index{kernel@\spxentry{kernel}}\index{dumps@\spxentry{dumps}}\index{coredump@\spxentry{coredump}}\ignorespaces 

\section{Отключение дампов ядра (\sphinxstyleemphasis{Только для опытных пользователей})}
\label{\detokenize{source/extra-optimizations:disabling-kernel-dumps}}\label{\detokenize{source/extra-optimizations:index-4}}\label{\detokenize{source/extra-optimizations:id5}}
\sphinxAtStartPar
Отредактируйте \sphinxstyleemphasis{/etc/systemd/coredump.conf} в разделе \sphinxstyleemphasis{{[}Coredump{]}} раскомментируйте \sphinxstyleemphasis{Storage = external} и замените его на \sphinxstyleemphasis{Storage = none}.

\sphinxAtStartPar
Затем выполните следующую команду:

\sphinxAtStartPar
\sphinxcode{\sphinxupquote{sudo systemctl daemon\sphinxhyphen{}reload}}

\sphinxAtStartPar
Уже одно это действие отключает сохранение резервных копий, но они все еще находятся в памяти.
Если вы хотите полностью отключить дампы ядра, то измените \sphinxstyleemphasis{soft} на \sphinxstyleemphasis{\#* hard core 0} в \sphinxstyleemphasis{/etc/security/limits.conf}.

\index{swap@\spxentry{swap}}\index{swapfile@\spxentry{swapfile}}\ignorespaces 

\section{Отключение файла подкачки}
\label{\detokenize{source/extra-optimizations:disabling-swap}}\label{\detokenize{source/extra-optimizations:index-5}}\label{\detokenize{source/extra-optimizations:id6}}
\sphinxAtStartPar
Для лучшей игровой производительности следует использовать файловую систему Btrfs и не задействовать файл подкачки
(вместо него стоит использовать выше упомянутый zramswap, конечно при условии что у вас не слишком слабый процессор и оперативной памяти больше чем 4 ГБ),
а также без страха отключать фиксы уязвимостей, которые сильно урезают производительность процессора (о них написано в следующем разделе).

\begin{sphinxVerbatim}[commandchars=\\\{\}]
sudo swapoff /dev/sdxy  \PYG{c+c1}{\PYGZsh{} Вместо xy ваше название (Например sdb1).}
sudo swapoff \PYGZhy{}a         \PYG{c+c1}{\PYGZsh{} Отключает все swap\PYGZhy{}разделы/файлы}
sudo rm \PYGZhy{}f /swapfile    \PYG{c+c1}{\PYGZsh{} Удалить файл подкачки с диска}
sudo nano /etc/fstab    \PYG{c+c1}{\PYGZsh{} Уберите самую нижнюю строчку полностью.}
\end{sphinxVerbatim}

\sphinxstepscope


\chapter{Настройка параметров ядра}
\label{\detokenize{source/kernel-parameters:kernel-parameters}}\label{\detokenize{source/kernel-parameters:id1}}\label{\detokenize{source/kernel-parameters::doc}}
\index{bootloader@\spxentry{bootloader}}\index{patch\sphinxhyphen{}off@\spxentry{patch\sphinxhyphen{}off}}\index{grub@\spxentry{grub}}\index{grub\sphinxhyphen{}customizer@\spxentry{grub\sphinxhyphen{}customizer}}\index{mitigations\sphinxhyphen{}off@\spxentry{mitigations\sphinxhyphen{}off}}\ignorespaces 

\section{Обновление загрузчика и отключение ненужных заплаток}
\label{\detokenize{source/kernel-parameters:update-bootloader-parameters}}\label{\detokenize{source/kernel-parameters:index-0}}\label{\detokenize{source/kernel-parameters:id2}}
\sphinxAtStartPar
По умолчанию в ядре Linux включено довольно много исправлений безопасности, которые однако существенно снижают производительность процессора.
Вы можете их отключить через редактирование параметров загрузчика. Рассмотрим на примере GRUB:

\sphinxAtStartPar
\sphinxcode{\sphinxupquote{sudo nano /etc/default/grub}} \# Редактируем настройки вручную или через grub\sphinxhyphen{}customizer как на изображении:

\noindent\sphinxincludegraphics{{kernel-parameters-1}.png}

\begin{sphinxVerbatim}[commandchars=\\\{\}]
\PYG{n+nv}{GRUB\PYGZus{}CMDLINE\PYGZus{}LINUX\PYGZus{}DEFAULT}\PYG{o}{=}\PYG{l+s+s2}{\PYGZdq{}quiet splash rootfstype=btrfs lpj=3499912 raid=noautodetect elevator=noop mitigations=off preempt=none nowatchdog audit=0\PYGZdq{}}
\end{sphinxVerbatim}

\sphinxAtStartPar
\sphinxcode{\sphinxupquote{sudo grub\sphinxhyphen{}mkconfig \sphinxhyphen{}o /boot/grub/grub.cfg}}
\# Обновляем загрузчик, можно так же сделать через grub\sphinxhyphen{}customizer, добавить и прожать, затем сохранить на 2 и 1 вкладке.

\index{settings@\spxentry{settings}}\index{mitigations\sphinxhyphen{}off@\spxentry{mitigations\sphinxhyphen{}off}}\ignorespaces 

\subsection{Разъяснения}
\label{\detokenize{source/kernel-parameters:explanations}}\label{\detokenize{source/kernel-parameters:index-1}}\label{\detokenize{source/kernel-parameters:id3}}
\sphinxAtStartPar
\sphinxcode{\sphinxupquote{lpj=}} \sphinxhyphen{} Уникальный параметр для каждой системы. Его значение автоматически определяется во время загрузки, что довольно трудоемко, поэтому лучше задать вручную.
Определить ваше значение для lpj можно через следующую команду: \sphinxcode{\sphinxupquote{sudo dmesg | grep "lpj="}}

\sphinxAtStartPar
\sphinxcode{\sphinxupquote{mitigations=off}} \sphinxhyphen{} Непосредственно отключает все заплатки безопасности ядра (включая Spectre и Meltdown).
Подробнее об этом написано \sphinxhref{https://linuxreviews.org/HOWTO\_make\_Linux\_run\_blazing\_fast\_(again)\_on\_Intel\_CPUs}{здесь}.

\sphinxAtStartPar
\sphinxcode{\sphinxupquote{raid=noautodetect}} \sphinxhyphen{} Отключает проверку на RAID во время загрузки. Если вы его используете \sphinxhyphen{} \sphinxstylestrong{НЕ} прописывайте данный параметр.

\sphinxAtStartPar
\sphinxcode{\sphinxupquote{rootfstype=btrfs}} \sphinxhyphen{} Здесь указываем название файловой системы в которой у вас отформатирован корень.

\sphinxAtStartPar
\sphinxcode{\sphinxupquote{elevator=noop}} \sphinxhyphen{} Указывает для всех дисков планировщик ввода NONE. \sphinxstylestrong{Не использовать если у вас жесткий диск}.

\sphinxAtStartPar
\sphinxcode{\sphinxupquote{nowatchdog}} \sphinxhyphen{} Отключает сторожевые таймеры. Позволяет избавиться от заиканий в онлайн играх.

\sphinxstepscope


\chapter{Файловые системы}
\label{\detokenize{source/file-systems:file-systems}}\label{\detokenize{source/file-systems:id1}}\label{\detokenize{source/file-systems::doc}}
\index{btrfs@\spxentry{btrfs}}\index{ext4@\spxentry{ext4}}\index{zfs@\spxentry{zfs}}\ignorespaces 

\section{Нюансы выбора файловой системы и флагов монтирования}
\label{\detokenize{source/file-systems:file-system-selection}}\label{\detokenize{source/file-systems:index-0}}\label{\detokenize{source/file-systems:id2}}
\sphinxAtStartPar
В отличие от Windows, в Linux\sphinxhyphen{}подобных системах выбор файловой системы не ограничивается в обязательном порядке со стороны дистрибутива,
и может применяться исходя из личных предпочтений пользователя с оглядкой на поддержку со стороны ядра.

\sphinxAtStartPar
Вот краткое описание основных высокопроизводительных файловых систем:

\sphinxAtStartPar
\sphinxstylestrong{EXT4} \sphinxhyphen{} универсальный солдат, что подходит, как для SSD носителей, так и для HDD. Имеет самое большое распространение и документацию.
Обеспечивает высокие показатели чтения и записи.
Из минусов стоит отметить, что данная файловая система начинает проигрывать более новым представителям на рынке и требует ручного допиливания для SSD
(SATA, NVMe, PCI и т.п.).
Хорошо подходит для домашнего компьютера и файлопомойки, а также серверам которым необходима максимальная стабильность.

\sphinxAtStartPar
\sphinxstylestrong{BTRFS} \sphinxhyphen{} Т1000 из мира файловых систем.
Является наследником идей EXT2\sphinxhyphen{}3, и прекрасно подходит для SSD носителей,
ибо имеет модули автодетекта, что позволяет не сильно париться с настройкой TRIM и флагов монтирования.
Скорость чтения сопоставима, а иногда (Особенно при высоких нагрузках) превышают показатели EXT4.
Идеальный выбор для игровой/домашней системы на базе Linux.

\index{mount@\spxentry{mount}}\index{options@\spxentry{options}}\index{btrfs@\spxentry{btrfs}}\index{fstab@\spxentry{fstab}}\ignorespaces 

\subsection{Оптимальные флаги монтирования}
\label{\detokenize{source/file-systems:mount-options}}\label{\detokenize{source/file-systems:index-1}}\label{\detokenize{source/file-systems:id3}}
\sphinxAtStartPar
Вот оптимальные параметры для SSD носителей.
Описание каждого из них вы можете найти \sphinxhyphen{} \sphinxhref{https://zen.yandex.ru/media/id/5d8ac4740a451800acb6049f/linux-uskoriaem-sistemu-4-5e91d777378f6957923055b9}{здесь}.

\begin{sphinxVerbatim}[commandchars=\\\{\}]
rw,relatime,ssd,ssd\PYGZus{}spread,space\PYGZus{}cache\PYG{o}{=}v2,max\PYGZus{}inline\PYG{o}{=}\PYG{l+m}{256},commit\PYG{o}{=}\PYG{l+m}{600},nodatacow
\end{sphinxVerbatim}

\sphinxAtStartPar
Прежде всего, отметим, что вы можете изменить \sphinxstyleemphasis{relatime} на \sphinxstyleemphasis{noatime}
или \sphinxstyleemphasis{lazytime} \sphinxhyphen{} все три параметра отвечают за запоминание времени доступа к файлами и прочим связанным с ним атрибутами, что только портит отклик.

\sphinxAtStartPar
Параметр noatime полностью выключает данную функцию, что может привести к некоторым багам в приложениях зависимых от времени (например git),
но автор никогда не встречал данной проблемы.
Параметр \sphinxstyleemphasis{lazytime} успешно будет выполнять все функции времени, но выполнять их промежуточную запись в оперативной памяти,
что позволит избежать замедления без потери функционала, но говорят \sphinxstyleemphasis{lazytime} чудит, поэтому автор советует \sphinxstyleemphasis{noatime}.

\sphinxAtStartPar
Но если вы хотите минимум возможных проблем, то оставьте флаг \sphinxstyleemphasis{relatime}.

\sphinxAtStartPar
\sphinxstyleemphasis{space\_cache} можно заменить на \sphinxstyleemphasis{space\_cache=v2}, что тоже даст определенную прибавку производительности.

\sphinxAtStartPar
Прописывать их нужно в файл \sphinxcode{\sphinxupquote{/etc/fstab}} для корневого и домашнего разделов.
Некоторые из данных флагов будут применяться только для новых файлов.

\noindent\sphinxincludegraphics{{file-systems-1}.png}

\begin{sphinxadmonition}{attention}{Внимание:}
\sphinxAtStartPar
Параметр \sphinxcode{\sphinxupquote{commit=600}} может вызывать повышенное потребление оперативной памяти и портить данные на диске.
\end{sphinxadmonition}

\begin{sphinxadmonition}{attention}{Внимание:}
\sphinxAtStartPar
При использовании Btrfs для корневого раздела, обязательно установите пакет \sphinxhref{https://archlinux.org/packages/core/x86\_64/btrfs-progs/}{btrfs\sphinxhyphen{}progs}.
\end{sphinxadmonition}

\index{btrfs@\spxentry{btrfs}}\index{compression@\spxentry{compression}}\index{zstd@\spxentry{zstd}}\index{lzo@\spxentry{lzo}}\index{zib@\spxentry{zib}}\ignorespaces 

\section{Сжатие в файловой системе Btrfs}
\label{\detokenize{source/file-systems:btrfs}}\label{\detokenize{source/file-systems:btrfs-comperssion}}\label{\detokenize{source/file-systems:index-2}}
\sphinxAtStartPar
В файловой системе Btrfs есть возможность включения сжатия. Все записываемые файлы по возможности будут сжиматься и экономить пространство на носителе HDD или SSD.

\sphinxAtStartPar
Для SSD это может быть важно в связи с их ограниченным ресурсом на запись.

\sphinxAtStartPar
Согласно \sphinxhref{https://btrfs.wiki.kernel.org/index.php/Compression}{wiki Btrfs}, официально имеется 3 поддерживаемых алгоритма:
\begin{itemize}
\item {} 
\sphinxAtStartPar
zlib \sphinxhyphen{} высокая степень сжатия, но низкая скорость сжатия и распаковки

\item {} 
\sphinxAtStartPar
lzo \sphinxhyphen{} высокая скорость сжатия и распаковки, но наименьший уровень сжатия из представленных алгоритмов

\item {} 
\sphinxAtStartPar
zstd \sphinxhyphen{} степень сжатия сравнимая с zlib и более быстрые сжатие и распаковка, однако уступающие по скорости lzo

\end{itemize}

\sphinxAtStartPar
Для включения алгоритма сжатия в файловой системе необходимо:
\begin{enumerate}
\sphinxsetlistlabels{\arabic}{enumi}{enumii}{}{.}%
\item {} 
\sphinxAtStartPar
Убедиться в наличии необходимого алгоритма в системе или установить выбранный (zlib, lzo или zstd соответственно).

\item {} 
\sphinxAtStartPar
Отредактировать файл \sphinxcode{\sphinxupquote{etc/fstab}}, добавив для необходимого раздела или носителя следующий флаг монтирования:

\end{enumerate}

\begin{sphinxVerbatim}[commandchars=\\\{\}]
\PYG{n+nv}{compress}\PYG{o}{=}\PYG{l+s+s1}{\PYGZsq{}алгоритм\PYGZsq{}}:N
\end{sphinxVerbatim}

\sphinxAtStartPar
(Где \sphinxcode{\sphinxupquote{N}} \sphinxhyphen{} степень сжатия: для zlib \sphinxhyphen{} \sphinxcode{\sphinxupquote{N}} = 1,2,...9; для lzo \sphinxhyphen{} выбор уровня сжатия не предусмотрен,
поэтому \sphinxcode{\sphinxupquote{:N}} \sphinxhyphen{} не указываются; для zstd \sphinxhyphen{} \sphinxcode{\sphinxupquote{N}} =1,2,...15. Чем выше данный параметр, тем сильнее будут сжиматься данные, конечно при условии что это возможно,
но также будет повышена нагрузка на процессор, поскольку сжатие выполняется за счет его ресурсов.
Cогласно \sphinxhref{https://btrfs.wiki.kernel.org/index.php/Compression}{wiki Btrfs}, оптимальным значением \sphinxcode{\sphinxupquote{N}} по отношению \sphinxcode{\sphinxupquote{степень сжатия}} / \sphinxcode{\sphinxupquote{скорость}} считается \sphinxcode{\sphinxupquote{3}})

\sphinxAtStartPar
Например для zstd со степенью сжатия 3 запись будет выглядеть примерно следующим образом, если учесть приведенные выше флаги монтирования:

\begin{sphinxVerbatim}[commandchars=\\\{\}]
rw,relatime,compress\PYG{o}{=}zstd:3,ssd,ssd\PYGZus{}spread,space\PYGZus{}cache\PYG{o}{=}v2,max\PYGZus{}inline\PYG{o}{=}\PYG{l+m}{256},commit\PYG{o}{=}\PYG{l+m}{600}
\end{sphinxVerbatim}

\begin{sphinxadmonition}{attention}{Внимание:}
\sphinxAtStartPar
Сжатие файловой системы не работает вместе с флагом монтирования \sphinxcode{\sphinxupquote{nodatacow}}.
\end{sphinxadmonition}

\sphinxAtStartPar
После выставления данного флага монтирования новые файлы начнут сжиматься при записи на диск. Для сжатия уже имеющихся данных необходимо выполнить команду:

\begin{sphinxVerbatim}[commandchars=\\\{\}]
sudo btrfs filesystem defragment \PYGZhy{}calg /path
\end{sphinxVerbatim}

\sphinxAtStartPar
(Где \sphinxcode{\sphinxupquote{\sphinxhyphen{}calg}} \sphinxhyphen{} алгоритм (указывается как czlib, clzo или czstd в зависимости от выбранного алгоритма), \sphinxcode{\sphinxupquote{path}} \sphinxhyphen{} путь к разделу или папке)

\sphinxAtStartPar
Для сжатия уже существующих данных в папке или целого раздела необходимо указать ключ \sphinxcode{\sphinxupquote{\sphinxhyphen{}r}} перед \sphinxcode{\sphinxupquote{\sphinxhyphen{}calg}}:

\begin{sphinxVerbatim}[commandchars=\\\{\}]
sudo btrfs filesystem defragment \PYGZhy{}r \PYGZhy{}calg /path
\end{sphinxVerbatim}

\sphinxAtStartPar
(Где \sphinxcode{\sphinxupquote{path}} \sphinxhyphen{} путь к разделу или папке)

\begin{sphinxadmonition}{attention}{Внимание:}
\sphinxAtStartPar
Степень сжатия в данном случае не указывается!
\end{sphinxadmonition}

\index{comperssion@\spxentry{comperssion}}\index{zstd@\spxentry{zstd}}\index{test@\spxentry{test}}\index{compsize@\spxentry{compsize}}\ignorespaces 

\subsection{Определение эффективности сжатия}
\label{\detokenize{source/file-systems:efficiency-test}}\label{\detokenize{source/file-systems:index-3}}\label{\detokenize{source/file-systems:id6}}
\sphinxAtStartPar
Если вы хотите определить эффективность сжатия на вашем разделе/диске, то вам необходимо воспользоваться программой \sphinxhref{https://github.com/kilobyte/compsize}{compsize}.
Установить ее можно с помощью команды:

\begin{sphinxVerbatim}[commandchars=\\\{\}]
sudo pacman \PYGZhy{}S compsize
\end{sphinxVerbatim}

\sphinxAtStartPar
Для выполнения проверки на эффективность необходимо использовать команду:

\begin{sphinxVerbatim}[commandchars=\\\{\}]
sudo compsize /path
\end{sphinxVerbatim}

\sphinxAtStartPar
(Где \sphinxcode{\sphinxupquote{path}} \sphinxhyphen{} путь к разделу, папке или файлу)

\sphinxAtStartPar
Пример вывода команды:

\noindent\sphinxincludegraphics{{compsize}.png}

\sphinxAtStartPar
Пояснения:
\begin{itemize}
\item {} \begin{description}
\sphinxlineitem{Первый столбец:}\begin{itemize}
\item {} 
\sphinxAtStartPar
Строка \sphinxstyleemphasis{TOTAL} \sphinxhyphen{} итоговые данные, которые учитывают все сжатые и не сжатые файлы и разные алгоритмы (если такие имеются).

\item {} 
\sphinxAtStartPar
Строка \sphinxstyleemphasis{none} \sphinxhyphen{} данные, которые не были сжаты.

\item {} 
\sphinxAtStartPar
Далее отображаются все использованные алгоритмы (в данном случае \sphinxhyphen{} zstd).

\end{itemize}

\end{description}

\item {} 
\sphinxAtStartPar
Второй столбец показывает данные в процентах.

\item {} 
\sphinxAtStartPar
Третий столбец отображает фактически использованное место на диске/разделе.

\item {} 
\sphinxAtStartPar
Четвертый столбец показывает данные без сжатия.

\item {} 
\sphinxAtStartPar
Пятый \sphinxhyphen{} видимый размер файла, тот, который зачастую отображается в системе.

\end{itemize}

\index{compression@\spxentry{compression}}\index{zstd@\spxentry{zstd}}\index{test@\spxentry{test}}\index{phoronix\sphinxhyphen{}test\sphinxhyphen{}suite@\spxentry{phoronix\sphinxhyphen{}test\sphinxhyphen{}suite}}\ignorespaces 

\subsection{Скорость обработки алгоритма zstd на примере AMD Ryzen 7 3700X}
\label{\detokenize{source/file-systems:zstd-amd-ryzen-7-3700x}}\label{\detokenize{source/file-systems:zstd-compression-test}}\label{\detokenize{source/file-systems:index-4}}
\sphinxAtStartPar
Для сравнения степеней сжатия алгоритма zstd использовалась бенчмарк платформа \sphinxhref{https://github.com/phoronix-test-suite/phoronix-test-suite}{phoronix\sphinxhyphen{}test\sphinxhyphen{}suite}.
В данной программе, для проверки скорости сжатия и распаковки данных, доступно три степени \sphinxhyphen{} \sphinxcode{\sphinxupquote{3}}, \sphinxcode{\sphinxupquote{8}}, \sphinxcode{\sphinxupquote{19}}.
Для получения информации о падении скорости выполнения сжатия нам будет достаточно и первых двух, поскольку степень сжатия 19 на данный момент не поддерживается
(однако данные также приведены для ознакомления), и если обратить внимание на полученные данные, то это и не имеет особого смысла. Далее представлены результаты замеров:

\noindent\sphinxincludegraphics{{zstd_3}.png}

\noindent\sphinxincludegraphics{{zstd_8}.png}

\noindent\sphinxincludegraphics{{zstd_19}.png}

\sphinxAtStartPar
Как можно видеть из графиков, падение скорости при перехода от степени \sphinxcode{\sphinxupquote{3}} к степени \sphinxcode{\sphinxupquote{8}} сопровождается падением скорости сжатия более чем в \sphinxstylestrong{4,7} раз
(не говоря о более высоких степенях сжатия) и практически не изменяется при выполнении распаковки,
что может негативно сказаться на скорости установки программ и возможно в некоторых других ситуациях которые требует выполнения записи на диск.

\sphinxAtStartPar
Стоит отметить, что в случае выполнения установки игр с использованием степени сжатия \sphinxcode{\sphinxupquote{15}},
было замечено повышение нагрузки на процессор вплоть до 72\sphinxhyphen{}75\% в тех случаях, когда файлы поддавались сжатию.

\index{btrfs@\spxentry{btrfs}}\index{games@\spxentry{games}}\index{compression@\spxentry{compression}}\index{test@\spxentry{test}}\ignorespaces 

\subsection{Список протестированных игр на эффективность сжатия (Спасибо @dewdpol!)}
\label{\detokenize{source/file-systems:dewdpol}}\label{\detokenize{source/file-systems:comparison-table}}\label{\detokenize{source/file-systems:index-5}}
\sphinxAtStartPar
Далее представлен список протестированных игр на сжатие в файловой системе Btrfs.
Данные были получены с помощью программы compsize и являются округленными, поэтому информация может нести частично ознакомительный характер.


\begin{savenotes}\sphinxatlongtablestart\begin{longtable}[c]{|l|l|l|l|l|l|l|l|}
\hline
\sphinxstyletheadfamily 
\sphinxAtStartPar
№
&\sphinxstyletheadfamily 
\sphinxAtStartPar
Игра
&\sphinxstyletheadfamily 
\sphinxAtStartPar
Алгоритм
&\sphinxstyletheadfamily 
\sphinxAtStartPar
Уровень сжатия
&\sphinxstyletheadfamily 
\sphinxAtStartPar
Необходимое место (N)
&\sphinxstyletheadfamily 
\sphinxAtStartPar
Используемое место(U)
&\sphinxstyletheadfamily 
\sphinxAtStartPar
U/N
&\sphinxstyletheadfamily 
\sphinxAtStartPar
Экономия
\\
\hline
\endfirsthead

\multicolumn{8}{c}%
{\makebox[0pt]{\sphinxtablecontinued{\tablename\ \thetable{} \textendash{} продолжение с предыдущей страницы}}}\\
\hline
\sphinxstyletheadfamily 
\sphinxAtStartPar
№
&\sphinxstyletheadfamily 
\sphinxAtStartPar
Игра
&\sphinxstyletheadfamily 
\sphinxAtStartPar
Алгоритм
&\sphinxstyletheadfamily 
\sphinxAtStartPar
Уровень сжатия
&\sphinxstyletheadfamily 
\sphinxAtStartPar
Необходимое место (N)
&\sphinxstyletheadfamily 
\sphinxAtStartPar
Используемое место(U)
&\sphinxstyletheadfamily 
\sphinxAtStartPar
U/N
&\sphinxstyletheadfamily 
\sphinxAtStartPar
Экономия
\\
\hline
\endhead

\hline
\multicolumn{8}{r}{\makebox[0pt][r]{\sphinxtablecontinued{continues on next page}}}\\
\endfoot

\endlastfoot
\sphinxmultirow{2}{9}{%
\begin{varwidth}[t]{\sphinxcolwidth{1}{8}}
\sphinxAtStartPar
1
\par
\vskip-\baselineskip\vbox{\hbox{\strut}}\end{varwidth}%
}%
&\sphinxmultirow{2}{10}{%
\begin{varwidth}[t]{\sphinxcolwidth{1}{8}}
\sphinxAtStartPar
A Plague Tale: Innocence
\par
\vskip-\baselineskip\vbox{\hbox{\strut}}\end{varwidth}%
}%
&\sphinxmultirow{2}{11}{%
\begin{varwidth}[t]{\sphinxcolwidth{1}{8}}
\sphinxAtStartPar
zstd
\par
\vskip-\baselineskip\vbox{\hbox{\strut}}\end{varwidth}%
}%
&
\sphinxAtStartPar
3
&\sphinxmultirow{2}{13}{%
\begin{varwidth}[t]{\sphinxcolwidth{1}{8}}
\sphinxAtStartPar
41 GB
\par
\vskip-\baselineskip\vbox{\hbox{\strut}}\end{varwidth}%
}%
&\sphinxmultirow{2}{14}{%
\begin{varwidth}[t]{\sphinxcolwidth{1}{8}}
\sphinxAtStartPar
41 GB
\par
\vskip-\baselineskip\vbox{\hbox{\strut}}\end{varwidth}%
}%
&\sphinxmultirow{2}{15}{%
\begin{varwidth}[t]{\sphinxcolwidth{1}{8}}
\sphinxAtStartPar
99\%
\par
\vskip-\baselineskip\vbox{\hbox{\strut}}\end{varwidth}%
}%
&
\sphinxAtStartPar
182 MB
\\
\cline{4-4}\cline{8-8}\sphinxtablestrut{9}&\sphinxtablestrut{10}&\sphinxtablestrut{11}&
\sphinxAtStartPar
15
&\sphinxtablestrut{13}&\sphinxtablestrut{14}&\sphinxtablestrut{15}&
\sphinxAtStartPar
306 MB
\\
\hline\sphinxmultirow{2}{19}{%
\begin{varwidth}[t]{\sphinxcolwidth{1}{8}}
\sphinxAtStartPar
2
\par
\vskip-\baselineskip\vbox{\hbox{\strut}}\end{varwidth}%
}%
&\sphinxmultirow{2}{20}{%
\begin{varwidth}[t]{\sphinxcolwidth{1}{8}}
\sphinxAtStartPar
A Story About My Uncle
\par
\vskip-\baselineskip\vbox{\hbox{\strut}}\end{varwidth}%
}%
&\sphinxmultirow{2}{21}{%
\begin{varwidth}[t]{\sphinxcolwidth{1}{8}}
\sphinxAtStartPar
zstd
\par
\vskip-\baselineskip\vbox{\hbox{\strut}}\end{varwidth}%
}%
&
\sphinxAtStartPar
3
&\sphinxmultirow{2}{23}{%
\begin{varwidth}[t]{\sphinxcolwidth{1}{8}}
\sphinxAtStartPar
1,1 GB
\par
\vskip-\baselineskip\vbox{\hbox{\strut}}\end{varwidth}%
}%
&\sphinxmultirow{2}{24}{%
\begin{varwidth}[t]{\sphinxcolwidth{1}{8}}
\sphinxAtStartPar
1,1 GB
\par
\vskip-\baselineskip\vbox{\hbox{\strut}}\end{varwidth}%
}%
&
\sphinxAtStartPar
94\%
&
\sphinxAtStartPar
63 MB
\\
\cline{4-4}\cline{7-8}\sphinxtablestrut{19}&\sphinxtablestrut{20}&\sphinxtablestrut{21}&
\sphinxAtStartPar
15
&\sphinxtablestrut{23}&\sphinxtablestrut{24}&
\sphinxAtStartPar
93\%
&
\sphinxAtStartPar
74 MB
\\
\hline\sphinxmultirow{2}{30}{%
\begin{varwidth}[t]{\sphinxcolwidth{1}{8}}
\sphinxAtStartPar
3
\par
\vskip-\baselineskip\vbox{\hbox{\strut}}\end{varwidth}%
}%
&\sphinxmultirow{2}{31}{%
\begin{varwidth}[t]{\sphinxcolwidth{1}{8}}
\sphinxAtStartPar
Aegis Defenders
\par
\vskip-\baselineskip\vbox{\hbox{\strut}}\end{varwidth}%
}%
&\sphinxmultirow{2}{32}{%
\begin{varwidth}[t]{\sphinxcolwidth{1}{8}}
\sphinxAtStartPar
zstd
\par
\vskip-\baselineskip\vbox{\hbox{\strut}}\end{varwidth}%
}%
&
\sphinxAtStartPar
3
&\sphinxmultirow{2}{34}{%
\begin{varwidth}[t]{\sphinxcolwidth{1}{8}}
\sphinxAtStartPar
1,3 GB
\par
\vskip-\baselineskip\vbox{\hbox{\strut}}\end{varwidth}%
}%
&
\sphinxAtStartPar
240 MB
&
\sphinxAtStartPar
17\%
&
\sphinxAtStartPar
1,10 GB
\\
\cline{4-4}\cline{6-8}\sphinxtablestrut{30}&\sphinxtablestrut{31}&\sphinxtablestrut{32}&
\sphinxAtStartPar
15
&\sphinxtablestrut{34}&
\sphinxAtStartPar
230 MB
&
\sphinxAtStartPar
16\%
&
\sphinxAtStartPar
1,11 GB
\\
\hline\sphinxmultirow{2}{42}{%
\begin{varwidth}[t]{\sphinxcolwidth{1}{8}}
\sphinxAtStartPar
4
\par
\vskip-\baselineskip\vbox{\hbox{\strut}}\end{varwidth}%
}%
&\sphinxmultirow{2}{43}{%
\begin{varwidth}[t]{\sphinxcolwidth{1}{8}}
\sphinxAtStartPar
Among Us
\par
\vskip-\baselineskip\vbox{\hbox{\strut}}\end{varwidth}%
}%
&\sphinxmultirow{2}{44}{%
\begin{varwidth}[t]{\sphinxcolwidth{1}{8}}
\sphinxAtStartPar
zstd
\par
\vskip-\baselineskip\vbox{\hbox{\strut}}\end{varwidth}%
}%
&
\sphinxAtStartPar
3
&\sphinxmultirow{2}{46}{%
\begin{varwidth}[t]{\sphinxcolwidth{1}{8}}
\sphinxAtStartPar
429 MB
\par
\vskip-\baselineskip\vbox{\hbox{\strut}}\end{varwidth}%
}%
&
\sphinxAtStartPar
284 MB
&
\sphinxAtStartPar
66\%
&
\sphinxAtStartPar
145 MB
\\
\cline{4-4}\cline{6-8}\sphinxtablestrut{42}&\sphinxtablestrut{43}&\sphinxtablestrut{44}&
\sphinxAtStartPar
15
&\sphinxtablestrut{46}&
\sphinxAtStartPar
279 MB
&
\sphinxAtStartPar
65\%
&
\sphinxAtStartPar
150 MB
\\
\hline\sphinxmultirow{2}{54}{%
\begin{varwidth}[t]{\sphinxcolwidth{1}{8}}
\sphinxAtStartPar
5
\par
\vskip-\baselineskip\vbox{\hbox{\strut}}\end{varwidth}%
}%
&\sphinxmultirow{2}{55}{%
\begin{varwidth}[t]{\sphinxcolwidth{1}{8}}
\sphinxAtStartPar
Aragami
\par
\vskip-\baselineskip\vbox{\hbox{\strut}}\end{varwidth}%
}%
&\sphinxmultirow{2}{56}{%
\begin{varwidth}[t]{\sphinxcolwidth{1}{8}}
\sphinxAtStartPar
zstd
\par
\vskip-\baselineskip\vbox{\hbox{\strut}}\end{varwidth}%
}%
&
\sphinxAtStartPar
3
&\sphinxmultirow{2}{58}{%
\begin{varwidth}[t]{\sphinxcolwidth{1}{8}}
\sphinxAtStartPar
7,6 GB
\par
\vskip-\baselineskip\vbox{\hbox{\strut}}\end{varwidth}%
}%
&
\sphinxAtStartPar
5,4 GB
&
\sphinxAtStartPar
71\%
&
\sphinxAtStartPar
2,20 GB
\\
\cline{4-4}\cline{6-8}\sphinxtablestrut{54}&\sphinxtablestrut{55}&\sphinxtablestrut{56}&
\sphinxAtStartPar
15
&\sphinxtablestrut{58}&
\sphinxAtStartPar
5,3 GB
&
\sphinxAtStartPar
69\%
&
\sphinxAtStartPar
2,27 GB
\\
\hline\sphinxmultirow{2}{66}{%
\begin{varwidth}[t]{\sphinxcolwidth{1}{8}}
\sphinxAtStartPar
6
\par
\vskip-\baselineskip\vbox{\hbox{\strut}}\end{varwidth}%
}%
&\sphinxmultirow{2}{67}{%
\begin{varwidth}[t]{\sphinxcolwidth{1}{8}}
\sphinxAtStartPar
Armello
\par
\vskip-\baselineskip\vbox{\hbox{\strut}}\end{varwidth}%
}%
&\sphinxmultirow{2}{68}{%
\begin{varwidth}[t]{\sphinxcolwidth{1}{8}}
\sphinxAtStartPar
zstd
\par
\vskip-\baselineskip\vbox{\hbox{\strut}}\end{varwidth}%
}%
&
\sphinxAtStartPar
3
&\sphinxmultirow{2}{70}{%
\begin{varwidth}[t]{\sphinxcolwidth{1}{8}}
\sphinxAtStartPar
1,6 GB
\par
\vskip-\baselineskip\vbox{\hbox{\strut}}\end{varwidth}%
}%
&\sphinxmultirow{2}{71}{%
\begin{varwidth}[t]{\sphinxcolwidth{1}{8}}
\sphinxAtStartPar
1,5 GB
\par
\vskip-\baselineskip\vbox{\hbox{\strut}}\end{varwidth}%
}%
&
\sphinxAtStartPar
95\%
&
\sphinxAtStartPar
73 MB
\\
\cline{4-4}\cline{7-8}\sphinxtablestrut{66}&\sphinxtablestrut{67}&\sphinxtablestrut{68}&
\sphinxAtStartPar
15
&\sphinxtablestrut{70}&\sphinxtablestrut{71}&
\sphinxAtStartPar
94\%
&
\sphinxAtStartPar
83 MB
\\
\hline\sphinxmultirow{2}{77}{%
\begin{varwidth}[t]{\sphinxcolwidth{1}{8}}
\sphinxAtStartPar
7
\par
\vskip-\baselineskip\vbox{\hbox{\strut}}\end{varwidth}%
}%
&\sphinxmultirow{2}{78}{%
\begin{varwidth}[t]{\sphinxcolwidth{1}{8}}
\sphinxAtStartPar
Bastion
\par
\vskip-\baselineskip\vbox{\hbox{\strut}}\end{varwidth}%
}%
&\sphinxmultirow{2}{79}{%
\begin{varwidth}[t]{\sphinxcolwidth{1}{8}}
\sphinxAtStartPar
zstd
\par
\vskip-\baselineskip\vbox{\hbox{\strut}}\end{varwidth}%
}%
&
\sphinxAtStartPar
3
&\sphinxmultirow{2}{81}{%
\begin{varwidth}[t]{\sphinxcolwidth{1}{8}}
\sphinxAtStartPar
1,1 GB
\par
\vskip-\baselineskip\vbox{\hbox{\strut}}\end{varwidth}%
}%
&
\sphinxAtStartPar
1,1 GB
&
\sphinxAtStartPar
94\%
&
\sphinxAtStartPar
67 MB
\\
\cline{4-4}\cline{6-8}\sphinxtablestrut{77}&\sphinxtablestrut{78}&\sphinxtablestrut{79}&
\sphinxAtStartPar
15
&\sphinxtablestrut{81}&
\sphinxAtStartPar
1,0 GB
&
\sphinxAtStartPar
93\%
&
\sphinxAtStartPar
81 MB
\\
\hline\sphinxmultirow{2}{89}{%
\begin{varwidth}[t]{\sphinxcolwidth{1}{8}}
\sphinxAtStartPar
8
\par
\vskip-\baselineskip\vbox{\hbox{\strut}}\end{varwidth}%
}%
&\sphinxmultirow{2}{90}{%
\begin{varwidth}[t]{\sphinxcolwidth{1}{8}}
\sphinxAtStartPar
BattleBlock Theater
\par
\vskip-\baselineskip\vbox{\hbox{\strut}}\end{varwidth}%
}%
&\sphinxmultirow{2}{91}{%
\begin{varwidth}[t]{\sphinxcolwidth{1}{8}}
\sphinxAtStartPar
zstd
\par
\vskip-\baselineskip\vbox{\hbox{\strut}}\end{varwidth}%
}%
&
\sphinxAtStartPar
3
&\sphinxmultirow{2}{93}{%
\begin{varwidth}[t]{\sphinxcolwidth{1}{8}}
\sphinxAtStartPar
1,8 GB
\par
\vskip-\baselineskip\vbox{\hbox{\strut}}\end{varwidth}%
}%
&\sphinxmultirow{2}{94}{%
\begin{varwidth}[t]{\sphinxcolwidth{1}{8}}
\sphinxAtStartPar
1,7 GB
\par
\vskip-\baselineskip\vbox{\hbox{\strut}}\end{varwidth}%
}%
&\sphinxmultirow{2}{95}{%
\begin{varwidth}[t]{\sphinxcolwidth{1}{8}}
\sphinxAtStartPar
93\%
\par
\vskip-\baselineskip\vbox{\hbox{\strut}}\end{varwidth}%
}%
&
\sphinxAtStartPar
117,8 MB
\\
\cline{4-4}\cline{8-8}\sphinxtablestrut{89}&\sphinxtablestrut{90}&\sphinxtablestrut{91}&
\sphinxAtStartPar
15
&\sphinxtablestrut{93}&\sphinxtablestrut{94}&\sphinxtablestrut{95}&
\sphinxAtStartPar
118,7 MB
\\
\hline\sphinxmultirow{2}{99}{%
\begin{varwidth}[t]{\sphinxcolwidth{1}{8}}
\sphinxAtStartPar
9
\par
\vskip-\baselineskip\vbox{\hbox{\strut}}\end{varwidth}%
}%
&\sphinxmultirow{2}{100}{%
\begin{varwidth}[t]{\sphinxcolwidth{1}{8}}
\sphinxAtStartPar
Beholder
\par
\vskip-\baselineskip\vbox{\hbox{\strut}}\end{varwidth}%
}%
&\sphinxmultirow{2}{101}{%
\begin{varwidth}[t]{\sphinxcolwidth{1}{8}}
\sphinxAtStartPar
zstd
\par
\vskip-\baselineskip\vbox{\hbox{\strut}}\end{varwidth}%
}%
&
\sphinxAtStartPar
3
&\sphinxmultirow{2}{103}{%
\begin{varwidth}[t]{\sphinxcolwidth{1}{8}}
\sphinxAtStartPar
1,9 GB
\par
\vskip-\baselineskip\vbox{\hbox{\strut}}\end{varwidth}%
}%
&
\sphinxAtStartPar
1,0 GB
&
\sphinxAtStartPar
55\%
&
\sphinxAtStartPar
0,77 GB
\\
\cline{4-4}\cline{6-8}\sphinxtablestrut{99}&\sphinxtablestrut{100}&\sphinxtablestrut{101}&
\sphinxAtStartPar
15
&\sphinxtablestrut{103}&
\sphinxAtStartPar
1,1 GB
&
\sphinxAtStartPar
58\%
&
\sphinxAtStartPar
0,82 GB
\\
\hline\sphinxmultirow{2}{111}{%
\begin{varwidth}[t]{\sphinxcolwidth{1}{8}}
\sphinxAtStartPar
10
\par
\vskip-\baselineskip\vbox{\hbox{\strut}}\end{varwidth}%
}%
&\sphinxmultirow{2}{112}{%
\begin{varwidth}[t]{\sphinxcolwidth{1}{8}}
\sphinxAtStartPar
Beholder 2
\par
\vskip-\baselineskip\vbox{\hbox{\strut}}\end{varwidth}%
}%
&\sphinxmultirow{2}{113}{%
\begin{varwidth}[t]{\sphinxcolwidth{1}{8}}
\sphinxAtStartPar
zstd
\par
\vskip-\baselineskip\vbox{\hbox{\strut}}\end{varwidth}%
}%
&
\sphinxAtStartPar
3
&\sphinxmultirow{2}{115}{%
\begin{varwidth}[t]{\sphinxcolwidth{1}{8}}
\sphinxAtStartPar
2,5 GB
\par
\vskip-\baselineskip\vbox{\hbox{\strut}}\end{varwidth}%
}%
&
\sphinxAtStartPar
2,2 GB
&
\sphinxAtStartPar
85\%
&
\sphinxAtStartPar
385 MB
\\
\cline{4-4}\cline{6-8}\sphinxtablestrut{111}&\sphinxtablestrut{112}&\sphinxtablestrut{113}&
\sphinxAtStartPar
15
&\sphinxtablestrut{115}&
\sphinxAtStartPar
2,1 GB
&
\sphinxAtStartPar
81\%
&
\sphinxAtStartPar
483 MB
\\
\hline\sphinxmultirow{2}{123}{%
\begin{varwidth}[t]{\sphinxcolwidth{1}{8}}
\sphinxAtStartPar
11
\par
\vskip-\baselineskip\vbox{\hbox{\strut}}\end{varwidth}%
}%
&\sphinxmultirow{2}{124}{%
\begin{varwidth}[t]{\sphinxcolwidth{1}{8}}
\sphinxAtStartPar
Blasphemous
\par
\vskip-\baselineskip\vbox{\hbox{\strut}}\end{varwidth}%
}%
&\sphinxmultirow{2}{125}{%
\begin{varwidth}[t]{\sphinxcolwidth{1}{8}}
\sphinxAtStartPar
zstd
\par
\vskip-\baselineskip\vbox{\hbox{\strut}}\end{varwidth}%
}%
&
\sphinxAtStartPar
3
&\sphinxmultirow{2}{127}{%
\begin{varwidth}[t]{\sphinxcolwidth{1}{8}}
\sphinxAtStartPar
854 MB
\par
\vskip-\baselineskip\vbox{\hbox{\strut}}\end{varwidth}%
}%
&
\sphinxAtStartPar
805 MB
&
\sphinxAtStartPar
94\%
&
\sphinxAtStartPar
48 MB
\\
\cline{4-4}\cline{6-8}\sphinxtablestrut{123}&\sphinxtablestrut{124}&\sphinxtablestrut{125}&
\sphinxAtStartPar
15
&\sphinxtablestrut{127}&
\sphinxAtStartPar
802 MB
&
\sphinxAtStartPar
93\%
&
\sphinxAtStartPar
51 MB
\\
\hline\sphinxmultirow{2}{135}{%
\begin{varwidth}[t]{\sphinxcolwidth{1}{8}}
\sphinxAtStartPar
12
\par
\vskip-\baselineskip\vbox{\hbox{\strut}}\end{varwidth}%
}%
&\sphinxmultirow{2}{136}{%
\begin{varwidth}[t]{\sphinxcolwidth{1}{8}}
\sphinxAtStartPar
Blue Fire
\par
\vskip-\baselineskip\vbox{\hbox{\strut}}\end{varwidth}%
}%
&\sphinxmultirow{2}{137}{%
\begin{varwidth}[t]{\sphinxcolwidth{1}{8}}
\sphinxAtStartPar
zstd
\par
\vskip-\baselineskip\vbox{\hbox{\strut}}\end{varwidth}%
}%
&
\sphinxAtStartPar
3
&\sphinxmultirow{2}{139}{%
\begin{varwidth}[t]{\sphinxcolwidth{1}{8}}
\sphinxAtStartPar
6,0 GB
\par
\vskip-\baselineskip\vbox{\hbox{\strut}}\end{varwidth}%
}%
&
\sphinxAtStartPar
4,9 GB
&
\sphinxAtStartPar
81\%
&
\sphinxAtStartPar
1,10 GB
\\
\cline{4-4}\cline{6-8}\sphinxtablestrut{135}&\sphinxtablestrut{136}&\sphinxtablestrut{137}&
\sphinxAtStartPar
15
&\sphinxtablestrut{139}&
\sphinxAtStartPar
4,7 GB
&
\sphinxAtStartPar
77\%
&
\sphinxAtStartPar
1,30 GB
\\
\hline\sphinxmultirow{2}{147}{%
\begin{varwidth}[t]{\sphinxcolwidth{1}{8}}
\sphinxAtStartPar
13
\par
\vskip-\baselineskip\vbox{\hbox{\strut}}\end{varwidth}%
}%
&\sphinxmultirow{2}{148}{%
\begin{varwidth}[t]{\sphinxcolwidth{1}{8}}
\sphinxAtStartPar
Brothers \sphinxhyphen{} A Tale of Two Sons
\par
\vskip-\baselineskip\vbox{\hbox{\strut}}\end{varwidth}%
}%
&\sphinxmultirow{2}{149}{%
\begin{varwidth}[t]{\sphinxcolwidth{1}{8}}
\sphinxAtStartPar
zstd
\par
\vskip-\baselineskip\vbox{\hbox{\strut}}\end{varwidth}%
}%
&
\sphinxAtStartPar
3
&\sphinxmultirow{2}{151}{%
\begin{varwidth}[t]{\sphinxcolwidth{1}{8}}
\sphinxAtStartPar
1,2 GB
\par
\vskip-\baselineskip\vbox{\hbox{\strut}}\end{varwidth}%
}%
&\sphinxmultirow{2}{152}{%
\begin{varwidth}[t]{\sphinxcolwidth{1}{8}}
\sphinxAtStartPar
1,1 GB
\par
\vskip-\baselineskip\vbox{\hbox{\strut}}\end{varwidth}%
}%
&\sphinxmultirow{2}{153}{%
\begin{varwidth}[t]{\sphinxcolwidth{1}{8}}
\sphinxAtStartPar
95\%
\par
\vskip-\baselineskip\vbox{\hbox{\strut}}\end{varwidth}%
}%
&
\sphinxAtStartPar
53 MB
\\
\cline{4-4}\cline{8-8}\sphinxtablestrut{147}&\sphinxtablestrut{148}&\sphinxtablestrut{149}&
\sphinxAtStartPar
15
&\sphinxtablestrut{151}&\sphinxtablestrut{152}&\sphinxtablestrut{153}&
\sphinxAtStartPar
52 MB
\\
\hline\sphinxmultirow{2}{157}{%
\begin{varwidth}[t]{\sphinxcolwidth{1}{8}}
\sphinxAtStartPar
14
\par
\vskip-\baselineskip\vbox{\hbox{\strut}}\end{varwidth}%
}%
&\sphinxmultirow{2}{158}{%
\begin{varwidth}[t]{\sphinxcolwidth{1}{8}}
\sphinxAtStartPar
Castle Crashers
\par
\vskip-\baselineskip\vbox{\hbox{\strut}}\end{varwidth}%
}%
&\sphinxmultirow{2}{159}{%
\begin{varwidth}[t]{\sphinxcolwidth{1}{8}}
\sphinxAtStartPar
zstd
\par
\vskip-\baselineskip\vbox{\hbox{\strut}}\end{varwidth}%
}%
&
\sphinxAtStartPar
3
&\sphinxmultirow{2}{161}{%
\begin{varwidth}[t]{\sphinxcolwidth{1}{8}}
\sphinxAtStartPar
199 MB
\par
\vskip-\baselineskip\vbox{\hbox{\strut}}\end{varwidth}%
}%
&\sphinxmultirow{2}{162}{%
\begin{varwidth}[t]{\sphinxcolwidth{1}{8}}
\sphinxAtStartPar
183 MB
\par
\vskip-\baselineskip\vbox{\hbox{\strut}}\end{varwidth}%
}%
&
\sphinxAtStartPar
92\%
&
\sphinxAtStartPar
15,4 MB
\\
\cline{4-4}\cline{7-8}\sphinxtablestrut{157}&\sphinxtablestrut{158}&\sphinxtablestrut{159}&
\sphinxAtStartPar
15
&\sphinxtablestrut{161}&\sphinxtablestrut{162}&
\sphinxAtStartPar
91\%
&
\sphinxAtStartPar
15,8 MB
\\
\hline\sphinxmultirow{2}{168}{%
\begin{varwidth}[t]{\sphinxcolwidth{1}{8}}
\sphinxAtStartPar
15
\par
\vskip-\baselineskip\vbox{\hbox{\strut}}\end{varwidth}%
}%
&\sphinxmultirow{2}{169}{%
\begin{varwidth}[t]{\sphinxcolwidth{1}{8}}
\sphinxAtStartPar
Celeste
\par
\vskip-\baselineskip\vbox{\hbox{\strut}}\end{varwidth}%
}%
&\sphinxmultirow{2}{170}{%
\begin{varwidth}[t]{\sphinxcolwidth{1}{8}}
\sphinxAtStartPar
zstd
\par
\vskip-\baselineskip\vbox{\hbox{\strut}}\end{varwidth}%
}%
&
\sphinxAtStartPar
3
&\sphinxmultirow{2}{172}{%
\begin{varwidth}[t]{\sphinxcolwidth{1}{8}}
\sphinxAtStartPar
1,1 GB
\par
\vskip-\baselineskip\vbox{\hbox{\strut}}\end{varwidth}%
}%
&
\sphinxAtStartPar
897 MB
&
\sphinxAtStartPar
78\%
&
\sphinxAtStartPar
251 MB
\\
\cline{4-4}\cline{6-8}\sphinxtablestrut{168}&\sphinxtablestrut{169}&\sphinxtablestrut{170}&
\sphinxAtStartPar
15
&\sphinxtablestrut{172}&
\sphinxAtStartPar
871 MB
&
\sphinxAtStartPar
75\%
&
\sphinxAtStartPar
277 MB
\\
\hline\sphinxmultirow{2}{180}{%
\begin{varwidth}[t]{\sphinxcolwidth{1}{8}}
\sphinxAtStartPar
16
\par
\vskip-\baselineskip\vbox{\hbox{\strut}}\end{varwidth}%
}%
&\sphinxmultirow{2}{181}{%
\begin{varwidth}[t]{\sphinxcolwidth{1}{8}}
\sphinxAtStartPar
Child of light
\par
\vskip-\baselineskip\vbox{\hbox{\strut}}\end{varwidth}%
}%
&\sphinxmultirow{2}{182}{%
\begin{varwidth}[t]{\sphinxcolwidth{1}{8}}
\sphinxAtStartPar
zstd
\par
\vskip-\baselineskip\vbox{\hbox{\strut}}\end{varwidth}%
}%
&
\sphinxAtStartPar
3
&\sphinxmultirow{2}{184}{%
\begin{varwidth}[t]{\sphinxcolwidth{1}{8}}
\sphinxAtStartPar
2,3 GB
\par
\vskip-\baselineskip\vbox{\hbox{\strut}}\end{varwidth}%
}%
&\sphinxmultirow{2}{185}{%
\begin{varwidth}[t]{\sphinxcolwidth{1}{8}}
\sphinxAtStartPar
2,3 GB
\par
\vskip-\baselineskip\vbox{\hbox{\strut}}\end{varwidth}%
}%
&\sphinxmultirow{2}{186}{%
\begin{varwidth}[t]{\sphinxcolwidth{1}{8}}
\sphinxAtStartPar
99\%
\par
\vskip-\baselineskip\vbox{\hbox{\strut}}\end{varwidth}%
}%
&
\sphinxAtStartPar
15 MB
\\
\cline{4-4}\cline{8-8}\sphinxtablestrut{180}&\sphinxtablestrut{181}&\sphinxtablestrut{182}&
\sphinxAtStartPar
15
&\sphinxtablestrut{184}&\sphinxtablestrut{185}&\sphinxtablestrut{186}&
\sphinxAtStartPar
9,5 MB
\\
\hline\sphinxmultirow{2}{190}{%
\begin{varwidth}[t]{\sphinxcolwidth{1}{8}}
\sphinxAtStartPar
17
\par
\vskip-\baselineskip\vbox{\hbox{\strut}}\end{varwidth}%
}%
&\sphinxmultirow{2}{191}{%
\begin{varwidth}[t]{\sphinxcolwidth{1}{8}}
\sphinxAtStartPar
Children of Morta
\par
\vskip-\baselineskip\vbox{\hbox{\strut}}\end{varwidth}%
}%
&\sphinxmultirow{2}{192}{%
\begin{varwidth}[t]{\sphinxcolwidth{1}{8}}
\sphinxAtStartPar
zstd
\par
\vskip-\baselineskip\vbox{\hbox{\strut}}\end{varwidth}%
}%
&
\sphinxAtStartPar
3
&\sphinxmultirow{2}{194}{%
\begin{varwidth}[t]{\sphinxcolwidth{1}{8}}
\sphinxAtStartPar
1,6 GB
\par
\vskip-\baselineskip\vbox{\hbox{\strut}}\end{varwidth}%
}%
&\sphinxmultirow{2}{195}{%
\begin{varwidth}[t]{\sphinxcolwidth{1}{8}}
\sphinxAtStartPar
1,5 GB
\par
\vskip-\baselineskip\vbox{\hbox{\strut}}\end{varwidth}%
}%
&\sphinxmultirow{2}{196}{%
\begin{varwidth}[t]{\sphinxcolwidth{1}{8}}
\sphinxAtStartPar
94\%
\par
\vskip-\baselineskip\vbox{\hbox{\strut}}\end{varwidth}%
}%
&
\sphinxAtStartPar
87 MB
\\
\cline{4-4}\cline{8-8}\sphinxtablestrut{190}&\sphinxtablestrut{191}&\sphinxtablestrut{192}&
\sphinxAtStartPar
15
&\sphinxtablestrut{194}&\sphinxtablestrut{195}&\sphinxtablestrut{196}&
\sphinxAtStartPar
92 MB
\\
\hline\sphinxmultirow{2}{200}{%
\begin{varwidth}[t]{\sphinxcolwidth{1}{8}}
\sphinxAtStartPar
18
\par
\vskip-\baselineskip\vbox{\hbox{\strut}}\end{varwidth}%
}%
&\sphinxmultirow{2}{201}{%
\begin{varwidth}[t]{\sphinxcolwidth{1}{8}}
\sphinxAtStartPar
CODE VEIN
\par
\vskip-\baselineskip\vbox{\hbox{\strut}}\end{varwidth}%
}%
&\sphinxmultirow{2}{202}{%
\begin{varwidth}[t]{\sphinxcolwidth{1}{8}}
\sphinxAtStartPar
zstd
\par
\vskip-\baselineskip\vbox{\hbox{\strut}}\end{varwidth}%
}%
&
\sphinxAtStartPar
3
&\sphinxmultirow{2}{204}{%
\begin{varwidth}[t]{\sphinxcolwidth{1}{8}}
\sphinxAtStartPar
35 GB
\par
\vskip-\baselineskip\vbox{\hbox{\strut}}\end{varwidth}%
}%
&\sphinxmultirow{2}{205}{%
\begin{varwidth}[t]{\sphinxcolwidth{1}{8}}
\sphinxAtStartPar
35 GB
\par
\vskip-\baselineskip\vbox{\hbox{\strut}}\end{varwidth}%
}%
&\sphinxmultirow{2}{206}{%
\begin{varwidth}[t]{\sphinxcolwidth{1}{8}}
\sphinxAtStartPar
99\%
\par
\vskip-\baselineskip\vbox{\hbox{\strut}}\end{varwidth}%
}%
&
\sphinxAtStartPar
75 MB
\\
\cline{4-4}\cline{8-8}\sphinxtablestrut{200}&\sphinxtablestrut{201}&\sphinxtablestrut{202}&
\sphinxAtStartPar
15
&\sphinxtablestrut{204}&\sphinxtablestrut{205}&\sphinxtablestrut{206}&
\sphinxAtStartPar
124 MB
\\
\hline\sphinxmultirow{2}{210}{%
\begin{varwidth}[t]{\sphinxcolwidth{1}{8}}
\sphinxAtStartPar
19
\par
\vskip-\baselineskip\vbox{\hbox{\strut}}\end{varwidth}%
}%
&\sphinxmultirow{2}{211}{%
\begin{varwidth}[t]{\sphinxcolwidth{1}{8}}
\sphinxAtStartPar
Cortex Command
\par
\vskip-\baselineskip\vbox{\hbox{\strut}}\end{varwidth}%
}%
&\sphinxmultirow{2}{212}{%
\begin{varwidth}[t]{\sphinxcolwidth{1}{8}}
\sphinxAtStartPar
zstd
\par
\vskip-\baselineskip\vbox{\hbox{\strut}}\end{varwidth}%
}%
&
\sphinxAtStartPar
3
&\sphinxmultirow{2}{214}{%
\begin{varwidth}[t]{\sphinxcolwidth{1}{8}}
\sphinxAtStartPar
97 MB
\par
\vskip-\baselineskip\vbox{\hbox{\strut}}\end{varwidth}%
}%
&
\sphinxAtStartPar
65 MB
&
\sphinxAtStartPar
67\%
&
\sphinxAtStartPar
32 MB
\\
\cline{4-4}\cline{6-8}\sphinxtablestrut{210}&\sphinxtablestrut{211}&\sphinxtablestrut{212}&
\sphinxAtStartPar
15
&\sphinxtablestrut{214}&
\sphinxAtStartPar
64 MB
&
\sphinxAtStartPar
66\%
&
\sphinxAtStartPar
33 MB
\\
\hline\sphinxmultirow{2}{222}{%
\begin{varwidth}[t]{\sphinxcolwidth{1}{8}}
\sphinxAtStartPar
20
\par
\vskip-\baselineskip\vbox{\hbox{\strut}}\end{varwidth}%
}%
&\sphinxmultirow{2}{223}{%
\begin{varwidth}[t]{\sphinxcolwidth{1}{8}}
\sphinxAtStartPar
Cuphead
\par
\vskip-\baselineskip\vbox{\hbox{\strut}}\end{varwidth}%
}%
&\sphinxmultirow{2}{224}{%
\begin{varwidth}[t]{\sphinxcolwidth{1}{8}}
\sphinxAtStartPar
zstd
\par
\vskip-\baselineskip\vbox{\hbox{\strut}}\end{varwidth}%
}%
&
\sphinxAtStartPar
3
&\sphinxmultirow{2}{226}{%
\begin{varwidth}[t]{\sphinxcolwidth{1}{8}}
\sphinxAtStartPar
3,6 GB
\par
\vskip-\baselineskip\vbox{\hbox{\strut}}\end{varwidth}%
}%
&\sphinxmultirow{2}{227}{%
\begin{varwidth}[t]{\sphinxcolwidth{1}{8}}
\sphinxAtStartPar
3,3 GB
\par
\vskip-\baselineskip\vbox{\hbox{\strut}}\end{varwidth}%
}%
&\sphinxmultirow{2}{228}{%
\begin{varwidth}[t]{\sphinxcolwidth{1}{8}}
\sphinxAtStartPar
93\%
\par
\vskip-\baselineskip\vbox{\hbox{\strut}}\end{varwidth}%
}%
&
\sphinxAtStartPar
223 MB
\\
\cline{4-4}\cline{8-8}\sphinxtablestrut{222}&\sphinxtablestrut{223}&\sphinxtablestrut{224}&
\sphinxAtStartPar
15
&\sphinxtablestrut{226}&\sphinxtablestrut{227}&\sphinxtablestrut{228}&
\sphinxAtStartPar
233 MB
\\
\hline\sphinxmultirow{2}{232}{%
\begin{varwidth}[t]{\sphinxcolwidth{1}{8}}
\sphinxAtStartPar
21
\par
\vskip-\baselineskip\vbox{\hbox{\strut}}\end{varwidth}%
}%
&\sphinxmultirow{2}{233}{%
\begin{varwidth}[t]{\sphinxcolwidth{1}{8}}
\sphinxAtStartPar
Curse of Dead Gods
\par
\vskip-\baselineskip\vbox{\hbox{\strut}}\end{varwidth}%
}%
&\sphinxmultirow{2}{234}{%
\begin{varwidth}[t]{\sphinxcolwidth{1}{8}}
\sphinxAtStartPar
zsrd
\par
\vskip-\baselineskip\vbox{\hbox{\strut}}\end{varwidth}%
}%
&
\sphinxAtStartPar
3
&\sphinxmultirow{2}{236}{%
\begin{varwidth}[t]{\sphinxcolwidth{1}{8}}
\sphinxAtStartPar
2,7 GB
\par
\vskip-\baselineskip\vbox{\hbox{\strut}}\end{varwidth}%
}%
&\sphinxmultirow{2}{237}{%
\begin{varwidth}[t]{\sphinxcolwidth{1}{8}}
\sphinxAtStartPar
1,4 GB
\par
\vskip-\baselineskip\vbox{\hbox{\strut}}\end{varwidth}%
}%
&
\sphinxAtStartPar
53\%
&
\sphinxAtStartPar
1,25 GB
\\
\cline{4-4}\cline{7-8}\sphinxtablestrut{232}&\sphinxtablestrut{233}&\sphinxtablestrut{234}&
\sphinxAtStartPar
15
&\sphinxtablestrut{236}&\sphinxtablestrut{237}&
\sphinxAtStartPar
51\%
&
\sphinxAtStartPar
1,29 GB
\\
\hline\sphinxmultirow{2}{243}{%
\begin{varwidth}[t]{\sphinxcolwidth{1}{8}}
\sphinxAtStartPar
22
\par
\vskip-\baselineskip\vbox{\hbox{\strut}}\end{varwidth}%
}%
&\sphinxmultirow{2}{244}{%
\begin{varwidth}[t]{\sphinxcolwidth{1}{8}}
\sphinxAtStartPar
D\sphinxhyphen{}Corp
\par
\vskip-\baselineskip\vbox{\hbox{\strut}}\end{varwidth}%
}%
&\sphinxmultirow{2}{245}{%
\begin{varwidth}[t]{\sphinxcolwidth{1}{8}}
\sphinxAtStartPar
zstd
\par
\vskip-\baselineskip\vbox{\hbox{\strut}}\end{varwidth}%
}%
&
\sphinxAtStartPar
3
&\sphinxmultirow{2}{247}{%
\begin{varwidth}[t]{\sphinxcolwidth{1}{8}}
\sphinxAtStartPar
1,2 GB
\par
\vskip-\baselineskip\vbox{\hbox{\strut}}\end{varwidth}%
}%
&
\sphinxAtStartPar
720 MB
&
\sphinxAtStartPar
57\%
&
\sphinxAtStartPar
525 MB
\\
\cline{4-4}\cline{6-8}\sphinxtablestrut{243}&\sphinxtablestrut{244}&\sphinxtablestrut{245}&
\sphinxAtStartPar
15
&\sphinxtablestrut{247}&
\sphinxAtStartPar
697 MB
&
\sphinxAtStartPar
55\%
&
\sphinxAtStartPar
549 MB
\\
\hline\sphinxmultirow{2}{255}{%
\begin{varwidth}[t]{\sphinxcolwidth{1}{8}}
\sphinxAtStartPar
23
\par
\vskip-\baselineskip\vbox{\hbox{\strut}}\end{varwidth}%
}%
&\sphinxmultirow{2}{256}{%
\begin{varwidth}[t]{\sphinxcolwidth{1}{8}}
\sphinxAtStartPar
Dark Souls: Prepare To Die Edition
\par
\vskip-\baselineskip\vbox{\hbox{\strut}}\end{varwidth}%
}%
&\sphinxmultirow{2}{257}{%
\begin{varwidth}[t]{\sphinxcolwidth{1}{8}}
\sphinxAtStartPar
zstd
\par
\vskip-\baselineskip\vbox{\hbox{\strut}}\end{varwidth}%
}%
&
\sphinxAtStartPar
3
&\sphinxmultirow{2}{259}{%
\begin{varwidth}[t]{\sphinxcolwidth{1}{8}}
\sphinxAtStartPar
3,7 GB
\par
\vskip-\baselineskip\vbox{\hbox{\strut}}\end{varwidth}%
}%
&\sphinxmultirow{2}{260}{%
\begin{varwidth}[t]{\sphinxcolwidth{1}{8}}
\sphinxAtStartPar
3,7 GB
\par
\vskip-\baselineskip\vbox{\hbox{\strut}}\end{varwidth}%
}%
&\sphinxmultirow{2}{261}{%
\begin{varwidth}[t]{\sphinxcolwidth{1}{8}}
\sphinxAtStartPar
99\%
\par
\vskip-\baselineskip\vbox{\hbox{\strut}}\end{varwidth}%
}%
&
\sphinxAtStartPar
1,57 MB
\\
\cline{4-4}\cline{8-8}\sphinxtablestrut{255}&\sphinxtablestrut{256}&\sphinxtablestrut{257}&
\sphinxAtStartPar
15
&\sphinxtablestrut{259}&\sphinxtablestrut{260}&\sphinxtablestrut{261}&
\sphinxAtStartPar
1,61 MB
\\
\hline\sphinxmultirow{2}{265}{%
\begin{varwidth}[t]{\sphinxcolwidth{1}{8}}
\sphinxAtStartPar
24
\par
\vskip-\baselineskip\vbox{\hbox{\strut}}\end{varwidth}%
}%
&\sphinxmultirow{2}{266}{%
\begin{varwidth}[t]{\sphinxcolwidth{1}{8}}
\sphinxAtStartPar
Dark Souls III
\par
\vskip-\baselineskip\vbox{\hbox{\strut}}\end{varwidth}%
}%
&\sphinxmultirow{2}{267}{%
\begin{varwidth}[t]{\sphinxcolwidth{1}{8}}
\sphinxAtStartPar
zstd
\par
\vskip-\baselineskip\vbox{\hbox{\strut}}\end{varwidth}%
}%
&
\sphinxAtStartPar
3
&\sphinxmultirow{2}{269}{%
\begin{varwidth}[t]{\sphinxcolwidth{1}{8}}
\sphinxAtStartPar
24 GB
\par
\vskip-\baselineskip\vbox{\hbox{\strut}}\end{varwidth}%
}%
&\sphinxmultirow{2}{270}{%
\begin{varwidth}[t]{\sphinxcolwidth{1}{8}}
\sphinxAtStartPar
24 GB
\par
\vskip-\baselineskip\vbox{\hbox{\strut}}\end{varwidth}%
}%
&\sphinxmultirow{2}{271}{%
\begin{varwidth}[t]{\sphinxcolwidth{1}{8}}
\sphinxAtStartPar
99\%
\par
\vskip-\baselineskip\vbox{\hbox{\strut}}\end{varwidth}%
}%
&
\sphinxAtStartPar
0,53 MB
\\
\cline{4-4}\cline{8-8}\sphinxtablestrut{265}&\sphinxtablestrut{266}&\sphinxtablestrut{267}&
\sphinxAtStartPar
15
&\sphinxtablestrut{269}&\sphinxtablestrut{270}&\sphinxtablestrut{271}&
\sphinxAtStartPar
0,60 MB
\\
\hline\sphinxmultirow{2}{275}{%
\begin{varwidth}[t]{\sphinxcolwidth{1}{8}}
\sphinxAtStartPar
25
\par
\vskip-\baselineskip\vbox{\hbox{\strut}}\end{varwidth}%
}%
&\sphinxmultirow{2}{276}{%
\begin{varwidth}[t]{\sphinxcolwidth{1}{8}}
\sphinxAtStartPar
Darkest Dungeon
\par
\vskip-\baselineskip\vbox{\hbox{\strut}}\end{varwidth}%
}%
&\sphinxmultirow{2}{277}{%
\begin{varwidth}[t]{\sphinxcolwidth{1}{8}}
\sphinxAtStartPar
zstd
\par
\vskip-\baselineskip\vbox{\hbox{\strut}}\end{varwidth}%
}%
&
\sphinxAtStartPar
3
&\sphinxmultirow{2}{279}{%
\begin{varwidth}[t]{\sphinxcolwidth{1}{8}}
\sphinxAtStartPar
3,2 GB
\par
\vskip-\baselineskip\vbox{\hbox{\strut}}\end{varwidth}%
}%
&\sphinxmultirow{2}{280}{%
\begin{varwidth}[t]{\sphinxcolwidth{1}{8}}
\sphinxAtStartPar
2,8 GB
\par
\vskip-\baselineskip\vbox{\hbox{\strut}}\end{varwidth}%
}%
&
\sphinxAtStartPar
88\%
&
\sphinxAtStartPar
394 MB
\\
\cline{4-4}\cline{7-8}\sphinxtablestrut{275}&\sphinxtablestrut{276}&\sphinxtablestrut{277}&
\sphinxAtStartPar
15
&\sphinxtablestrut{279}&\sphinxtablestrut{280}&
\sphinxAtStartPar
87\%
&
\sphinxAtStartPar
410 MB
\\
\hline\sphinxmultirow{2}{286}{%
\begin{varwidth}[t]{\sphinxcolwidth{1}{8}}
\sphinxAtStartPar
26
\par
\vskip-\baselineskip\vbox{\hbox{\strut}}\end{varwidth}%
}%
&\sphinxmultirow{2}{287}{%
\begin{varwidth}[t]{\sphinxcolwidth{1}{8}}
\sphinxAtStartPar
Darkestville Catle
\par
\vskip-\baselineskip\vbox{\hbox{\strut}}\end{varwidth}%
}%
&\sphinxmultirow{2}{288}{%
\begin{varwidth}[t]{\sphinxcolwidth{1}{8}}
\sphinxAtStartPar
zstd
\par
\vskip-\baselineskip\vbox{\hbox{\strut}}\end{varwidth}%
}%
&
\sphinxAtStartPar
3
&\sphinxmultirow{2}{290}{%
\begin{varwidth}[t]{\sphinxcolwidth{1}{8}}
\sphinxAtStartPar
1,7 GB
\par
\vskip-\baselineskip\vbox{\hbox{\strut}}\end{varwidth}%
}%
&
\sphinxAtStartPar
798 MB
&
\sphinxAtStartPar
40\%
&
\sphinxAtStartPar
0,99 GB
\\
\cline{4-4}\cline{6-8}\sphinxtablestrut{286}&\sphinxtablestrut{287}&\sphinxtablestrut{288}&
\sphinxAtStartPar
15
&\sphinxtablestrut{290}&
\sphinxAtStartPar
682 MB
&
\sphinxAtStartPar
38\%
&
\sphinxAtStartPar
1,02 GB
\\
\hline\sphinxmultirow{2}{298}{%
\begin{varwidth}[t]{\sphinxcolwidth{1}{8}}
\sphinxAtStartPar
27
\par
\vskip-\baselineskip\vbox{\hbox{\strut}}\end{varwidth}%
}%
&\sphinxmultirow{2}{299}{%
\begin{varwidth}[t]{\sphinxcolwidth{1}{8}}
\sphinxAtStartPar
Darksiders III
\par
\vskip-\baselineskip\vbox{\hbox{\strut}}\end{varwidth}%
}%
&\sphinxmultirow{2}{300}{%
\begin{varwidth}[t]{\sphinxcolwidth{1}{8}}
\sphinxAtStartPar
zstd
\par
\vskip-\baselineskip\vbox{\hbox{\strut}}\end{varwidth}%
}%
&
\sphinxAtStartPar
3
&\sphinxmultirow{2}{302}{%
\begin{varwidth}[t]{\sphinxcolwidth{1}{8}}
\sphinxAtStartPar
24 GB
\par
\vskip-\baselineskip\vbox{\hbox{\strut}}\end{varwidth}%
}%
&\sphinxmultirow{2}{303}{%
\begin{varwidth}[t]{\sphinxcolwidth{1}{8}}
\sphinxAtStartPar
24 GB
\par
\vskip-\baselineskip\vbox{\hbox{\strut}}\end{varwidth}%
}%
&\sphinxmultirow{2}{304}{%
\begin{varwidth}[t]{\sphinxcolwidth{1}{8}}
\sphinxAtStartPar
99\%
\par
\vskip-\baselineskip\vbox{\hbox{\strut}}\end{varwidth}%
}%
&
\sphinxAtStartPar
22 MB
\\
\cline{4-4}\cline{8-8}\sphinxtablestrut{298}&\sphinxtablestrut{299}&\sphinxtablestrut{300}&
\sphinxAtStartPar
15
&\sphinxtablestrut{302}&\sphinxtablestrut{303}&\sphinxtablestrut{304}&
\sphinxAtStartPar
30 MB
\\
\hline\sphinxmultirow{2}{308}{%
\begin{varwidth}[t]{\sphinxcolwidth{1}{8}}
\sphinxAtStartPar
28
\par
\vskip-\baselineskip\vbox{\hbox{\strut}}\end{varwidth}%
}%
&\sphinxmultirow{2}{309}{%
\begin{varwidth}[t]{\sphinxcolwidth{1}{8}}
\sphinxAtStartPar
Dead Cells
\par
\vskip-\baselineskip\vbox{\hbox{\strut}}\end{varwidth}%
}%
&\sphinxmultirow{2}{310}{%
\begin{varwidth}[t]{\sphinxcolwidth{1}{8}}
\sphinxAtStartPar
zstd
\par
\vskip-\baselineskip\vbox{\hbox{\strut}}\end{varwidth}%
}%
&
\sphinxAtStartPar
3
&\sphinxmultirow{2}{312}{%
\begin{varwidth}[t]{\sphinxcolwidth{1}{8}}
\sphinxAtStartPar
1,1 GB
\par
\vskip-\baselineskip\vbox{\hbox{\strut}}\end{varwidth}%
}%
&
\sphinxAtStartPar
1,1 GB
&\sphinxmultirow{2}{314}{%
\begin{varwidth}[t]{\sphinxcolwidth{1}{8}}
\sphinxAtStartPar
97\%
\par
\vskip-\baselineskip\vbox{\hbox{\strut}}\end{varwidth}%
}%
&
\sphinxAtStartPar
24 MB
\\
\cline{4-4}\cline{6-6}\cline{8-8}\sphinxtablestrut{308}&\sphinxtablestrut{309}&\sphinxtablestrut{310}&
\sphinxAtStartPar
15
&\sphinxtablestrut{312}&
\sphinxAtStartPar
1,0 GB
&\sphinxtablestrut{314}&
\sphinxAtStartPar
31 MB
\\
\hline\sphinxmultirow{2}{319}{%
\begin{varwidth}[t]{\sphinxcolwidth{1}{8}}
\sphinxAtStartPar
29
\par
\vskip-\baselineskip\vbox{\hbox{\strut}}\end{varwidth}%
}%
&\sphinxmultirow{2}{320}{%
\begin{varwidth}[t]{\sphinxcolwidth{1}{8}}
\sphinxAtStartPar
Death's Door
\par
\vskip-\baselineskip\vbox{\hbox{\strut}}\end{varwidth}%
}%
&\sphinxmultirow{2}{321}{%
\begin{varwidth}[t]{\sphinxcolwidth{1}{8}}
\sphinxAtStartPar
zstd
\par
\vskip-\baselineskip\vbox{\hbox{\strut}}\end{varwidth}%
}%
&
\sphinxAtStartPar
3
&\sphinxmultirow{2}{323}{%
\begin{varwidth}[t]{\sphinxcolwidth{1}{8}}
\sphinxAtStartPar
3,6 GB
\par
\vskip-\baselineskip\vbox{\hbox{\strut}}\end{varwidth}%
}%
&\sphinxmultirow{2}{324}{%
\begin{varwidth}[t]{\sphinxcolwidth{1}{8}}
\sphinxAtStartPar
2,1 GB
\par
\vskip-\baselineskip\vbox{\hbox{\strut}}\end{varwidth}%
}%
&
\sphinxAtStartPar
58\%
&
\sphinxAtStartPar
1,48 GB
\\
\cline{4-4}\cline{7-8}\sphinxtablestrut{319}&\sphinxtablestrut{320}&\sphinxtablestrut{321}&
\sphinxAtStartPar
15
&\sphinxtablestrut{323}&\sphinxtablestrut{324}&
\sphinxAtStartPar
57\%
&
\sphinxAtStartPar
1,54 GB
\\
\hline\sphinxmultirow{2}{330}{%
\begin{varwidth}[t]{\sphinxcolwidth{1}{8}}
\sphinxAtStartPar
30
\par
\vskip-\baselineskip\vbox{\hbox{\strut}}\end{varwidth}%
}%
&\sphinxmultirow{2}{331}{%
\begin{varwidth}[t]{\sphinxcolwidth{1}{8}}
\sphinxAtStartPar
Death's Gambit: Afterlife
\par
\vskip-\baselineskip\vbox{\hbox{\strut}}\end{varwidth}%
}%
&\sphinxmultirow{2}{332}{%
\begin{varwidth}[t]{\sphinxcolwidth{1}{8}}
\sphinxAtStartPar
zstd
\par
\vskip-\baselineskip\vbox{\hbox{\strut}}\end{varwidth}%
}%
&
\sphinxAtStartPar
3
&\sphinxmultirow{2}{334}{%
\begin{varwidth}[t]{\sphinxcolwidth{1}{8}}
\sphinxAtStartPar
1 GB
\par
\vskip-\baselineskip\vbox{\hbox{\strut}}\end{varwidth}%
}%
&
\sphinxAtStartPar
729 MB
&
\sphinxAtStartPar
66\%
&
\sphinxAtStartPar
367 MB
\\
\cline{4-4}\cline{6-8}\sphinxtablestrut{330}&\sphinxtablestrut{331}&\sphinxtablestrut{332}&
\sphinxAtStartPar
15
&\sphinxtablestrut{334}&
\sphinxAtStartPar
720 MB
&
\sphinxAtStartPar
65\%
&
\sphinxAtStartPar
376 MB
\\
\hline\sphinxmultirow{2}{342}{%
\begin{varwidth}[t]{\sphinxcolwidth{1}{8}}
\sphinxAtStartPar
31
\par
\vskip-\baselineskip\vbox{\hbox{\strut}}\end{varwidth}%
}%
&\sphinxmultirow{2}{343}{%
\begin{varwidth}[t]{\sphinxcolwidth{1}{8}}
\sphinxAtStartPar
Deponia: The Complete Journey
\par
\vskip-\baselineskip\vbox{\hbox{\strut}}\end{varwidth}%
}%
&\sphinxmultirow{2}{344}{%
\begin{varwidth}[t]{\sphinxcolwidth{1}{8}}
\sphinxAtStartPar
zstd
\par
\vskip-\baselineskip\vbox{\hbox{\strut}}\end{varwidth}%
}%
&
\sphinxAtStartPar
3
&\sphinxmultirow{2}{346}{%
\begin{varwidth}[t]{\sphinxcolwidth{1}{8}}
\sphinxAtStartPar
9,5 GB
\par
\vskip-\baselineskip\vbox{\hbox{\strut}}\end{varwidth}%
}%
&\sphinxmultirow{2}{347}{%
\begin{varwidth}[t]{\sphinxcolwidth{1}{8}}
\sphinxAtStartPar
9,5 GB
\par
\vskip-\baselineskip\vbox{\hbox{\strut}}\end{varwidth}%
}%
&\sphinxmultirow{2}{348}{%
\begin{varwidth}[t]{\sphinxcolwidth{1}{8}}
\sphinxAtStartPar
99\%
\par
\vskip-\baselineskip\vbox{\hbox{\strut}}\end{varwidth}%
}%
&
\sphinxAtStartPar
24,2 MB
\\
\cline{4-4}\cline{8-8}\sphinxtablestrut{342}&\sphinxtablestrut{343}&\sphinxtablestrut{344}&
\sphinxAtStartPar
15
&\sphinxtablestrut{346}&\sphinxtablestrut{347}&\sphinxtablestrut{348}&
\sphinxAtStartPar
25,6 MB
\\
\hline\sphinxmultirow{2}{352}{%
\begin{varwidth}[t]{\sphinxcolwidth{1}{8}}
\sphinxAtStartPar
32
\par
\vskip-\baselineskip\vbox{\hbox{\strut}}\end{varwidth}%
}%
&\sphinxmultirow{2}{353}{%
\begin{varwidth}[t]{\sphinxcolwidth{1}{8}}
\sphinxAtStartPar
Devil May Cry 5
\par
\vskip-\baselineskip\vbox{\hbox{\strut}}\end{varwidth}%
}%
&\sphinxmultirow{2}{354}{%
\begin{varwidth}[t]{\sphinxcolwidth{1}{8}}
\sphinxAtStartPar
zstd
\par
\vskip-\baselineskip\vbox{\hbox{\strut}}\end{varwidth}%
}%
&
\sphinxAtStartPar
3
&\sphinxmultirow{2}{356}{%
\begin{varwidth}[t]{\sphinxcolwidth{1}{8}}
\sphinxAtStartPar
33 GB
\par
\vskip-\baselineskip\vbox{\hbox{\strut}}\end{varwidth}%
}%
&\sphinxmultirow{2}{357}{%
\begin{varwidth}[t]{\sphinxcolwidth{1}{8}}
\sphinxAtStartPar
33 GB
\par
\vskip-\baselineskip\vbox{\hbox{\strut}}\end{varwidth}%
}%
&\sphinxmultirow{2}{358}{%
\begin{varwidth}[t]{\sphinxcolwidth{1}{8}}
\sphinxAtStartPar
99\%
\par
\vskip-\baselineskip\vbox{\hbox{\strut}}\end{varwidth}%
}%
&
\sphinxAtStartPar
82 MB
\\
\cline{4-4}\cline{8-8}\sphinxtablestrut{352}&\sphinxtablestrut{353}&\sphinxtablestrut{354}&
\sphinxAtStartPar
15
&\sphinxtablestrut{356}&\sphinxtablestrut{357}&\sphinxtablestrut{358}&
\sphinxAtStartPar
86 MB
\\
\hline\sphinxmultirow{2}{362}{%
\begin{varwidth}[t]{\sphinxcolwidth{1}{8}}
\sphinxAtStartPar
33
\par
\vskip-\baselineskip\vbox{\hbox{\strut}}\end{varwidth}%
}%
&\sphinxmultirow{2}{363}{%
\begin{varwidth}[t]{\sphinxcolwidth{1}{8}}
\sphinxAtStartPar
Disco Elysium
\par
\vskip-\baselineskip\vbox{\hbox{\strut}}\end{varwidth}%
}%
&\sphinxmultirow{2}{364}{%
\begin{varwidth}[t]{\sphinxcolwidth{1}{8}}
\sphinxAtStartPar
zstd
\par
\vskip-\baselineskip\vbox{\hbox{\strut}}\end{varwidth}%
}%
&
\sphinxAtStartPar
3
&\sphinxmultirow{2}{366}{%
\begin{varwidth}[t]{\sphinxcolwidth{1}{8}}
\sphinxAtStartPar
9,5 GB
\par
\vskip-\baselineskip\vbox{\hbox{\strut}}\end{varwidth}%
}%
&\sphinxmultirow{2}{367}{%
\begin{varwidth}[t]{\sphinxcolwidth{1}{8}}
\sphinxAtStartPar
9,1 GB
\par
\vskip-\baselineskip\vbox{\hbox{\strut}}\end{varwidth}%
}%
&
\sphinxAtStartPar
96\%
&
\sphinxAtStartPar
305 MB
\\
\cline{4-4}\cline{7-8}\sphinxtablestrut{362}&\sphinxtablestrut{363}&\sphinxtablestrut{364}&
\sphinxAtStartPar
15
&\sphinxtablestrut{366}&\sphinxtablestrut{367}&
\sphinxAtStartPar
95\%
&
\sphinxAtStartPar
391 MB
\\
\hline\sphinxmultirow{2}{373}{%
\begin{varwidth}[t]{\sphinxcolwidth{1}{8}}
\sphinxAtStartPar
34
\par
\vskip-\baselineskip\vbox{\hbox{\strut}}\end{varwidth}%
}%
&\sphinxmultirow{2}{374}{%
\begin{varwidth}[t]{\sphinxcolwidth{1}{8}}
\sphinxAtStartPar
Don't Starve Together
\par
\vskip-\baselineskip\vbox{\hbox{\strut}}\end{varwidth}%
}%
&\sphinxmultirow{2}{375}{%
\begin{varwidth}[t]{\sphinxcolwidth{1}{8}}
\sphinxAtStartPar
zstd
\par
\vskip-\baselineskip\vbox{\hbox{\strut}}\end{varwidth}%
}%
&
\sphinxAtStartPar
3
&\sphinxmultirow{2}{377}{%
\begin{varwidth}[t]{\sphinxcolwidth{1}{8}}
\sphinxAtStartPar
2,5 GB
\par
\vskip-\baselineskip\vbox{\hbox{\strut}}\end{varwidth}%
}%
&\sphinxmultirow{2}{378}{%
\begin{varwidth}[t]{\sphinxcolwidth{1}{8}}
\sphinxAtStartPar
1,8 GB
\par
\vskip-\baselineskip\vbox{\hbox{\strut}}\end{varwidth}%
}%
&
\sphinxAtStartPar
74\%
&
\sphinxAtStartPar
651 MB
\\
\cline{4-4}\cline{7-8}\sphinxtablestrut{373}&\sphinxtablestrut{374}&\sphinxtablestrut{375}&
\sphinxAtStartPar
15
&\sphinxtablestrut{377}&\sphinxtablestrut{378}&
\sphinxAtStartPar
73\%
&
\sphinxAtStartPar
679 MB
\\
\hline\sphinxmultirow{2}{384}{%
\begin{varwidth}[t]{\sphinxcolwidth{1}{8}}
\sphinxAtStartPar
35
\par
\vskip-\baselineskip\vbox{\hbox{\strut}}\end{varwidth}%
}%
&\sphinxmultirow{2}{385}{%
\begin{varwidth}[t]{\sphinxcolwidth{1}{8}}
\sphinxAtStartPar
Eldest Souls
\par
\vskip-\baselineskip\vbox{\hbox{\strut}}\end{varwidth}%
}%
&\sphinxmultirow{2}{386}{%
\begin{varwidth}[t]{\sphinxcolwidth{1}{8}}
\sphinxAtStartPar
zstd
\par
\vskip-\baselineskip\vbox{\hbox{\strut}}\end{varwidth}%
}%
&
\sphinxAtStartPar
3
&\sphinxmultirow{2}{388}{%
\begin{varwidth}[t]{\sphinxcolwidth{1}{8}}
\sphinxAtStartPar
1,0 GB
\par
\vskip-\baselineskip\vbox{\hbox{\strut}}\end{varwidth}%
}%
&
\sphinxAtStartPar
720 MB
&
\sphinxAtStartPar
69\%
&
\sphinxAtStartPar
314 MB
\\
\cline{4-4}\cline{6-8}\sphinxtablestrut{384}&\sphinxtablestrut{385}&\sphinxtablestrut{386}&
\sphinxAtStartPar
15
&\sphinxtablestrut{388}&
\sphinxAtStartPar
708 MB
&
\sphinxAtStartPar
68\%
&
\sphinxAtStartPar
326 MB
\\
\hline\sphinxmultirow{2}{396}{%
\begin{varwidth}[t]{\sphinxcolwidth{1}{8}}
\sphinxAtStartPar
36
\par
\vskip-\baselineskip\vbox{\hbox{\strut}}\end{varwidth}%
}%
&\sphinxmultirow{2}{397}{%
\begin{varwidth}[t]{\sphinxcolwidth{1}{8}}
\sphinxAtStartPar
Evergate
\par
\vskip-\baselineskip\vbox{\hbox{\strut}}\end{varwidth}%
}%
&\sphinxmultirow{2}{398}{%
\begin{varwidth}[t]{\sphinxcolwidth{1}{8}}
\sphinxAtStartPar
zstd
\par
\vskip-\baselineskip\vbox{\hbox{\strut}}\end{varwidth}%
}%
&
\sphinxAtStartPar
3
&\sphinxmultirow{2}{400}{%
\begin{varwidth}[t]{\sphinxcolwidth{1}{8}}
\sphinxAtStartPar
2,9 GB
\par
\vskip-\baselineskip\vbox{\hbox{\strut}}\end{varwidth}%
}%
&\sphinxmultirow{2}{401}{%
\begin{varwidth}[t]{\sphinxcolwidth{1}{8}}
\sphinxAtStartPar
1,9 GB
\par
\vskip-\baselineskip\vbox{\hbox{\strut}}\end{varwidth}%
}%
&
\sphinxAtStartPar
64\%
&
\sphinxAtStartPar
1,01 GB
\\
\cline{4-4}\cline{7-8}\sphinxtablestrut{396}&\sphinxtablestrut{397}&\sphinxtablestrut{398}&
\sphinxAtStartPar
15
&\sphinxtablestrut{400}&\sphinxtablestrut{401}&
\sphinxAtStartPar
63\%
&
\sphinxAtStartPar
1,03 GB
\\
\hline\sphinxmultirow{2}{407}{%
\begin{varwidth}[t]{\sphinxcolwidth{1}{8}}
\sphinxAtStartPar
37
\par
\vskip-\baselineskip\vbox{\hbox{\strut}}\end{varwidth}%
}%
&\sphinxmultirow{2}{408}{%
\begin{varwidth}[t]{\sphinxcolwidth{1}{8}}
\sphinxAtStartPar
Frostpunk
\par
\vskip-\baselineskip\vbox{\hbox{\strut}}\end{varwidth}%
}%
&\sphinxmultirow{2}{409}{%
\begin{varwidth}[t]{\sphinxcolwidth{1}{8}}
\sphinxAtStartPar
zstd
\par
\vskip-\baselineskip\vbox{\hbox{\strut}}\end{varwidth}%
}%
&
\sphinxAtStartPar
3
&\sphinxmultirow{2}{411}{%
\begin{varwidth}[t]{\sphinxcolwidth{1}{8}}
\sphinxAtStartPar
8,9 GB
\par
\vskip-\baselineskip\vbox{\hbox{\strut}}\end{varwidth}%
}%
&\sphinxmultirow{2}{412}{%
\begin{varwidth}[t]{\sphinxcolwidth{1}{8}}
\sphinxAtStartPar
8,9 GB
\par
\vskip-\baselineskip\vbox{\hbox{\strut}}\end{varwidth}%
}%
&\sphinxmultirow{2}{413}{%
\begin{varwidth}[t]{\sphinxcolwidth{1}{8}}
\sphinxAtStartPar
99\%
\par
\vskip-\baselineskip\vbox{\hbox{\strut}}\end{varwidth}%
}%
&
\sphinxAtStartPar
24 MB
\\
\cline{4-4}\cline{8-8}\sphinxtablestrut{407}&\sphinxtablestrut{408}&\sphinxtablestrut{409}&
\sphinxAtStartPar
15
&\sphinxtablestrut{411}&\sphinxtablestrut{412}&\sphinxtablestrut{413}&
\sphinxAtStartPar
25,2 MB
\\
\hline\sphinxmultirow{2}{417}{%
\begin{varwidth}[t]{\sphinxcolwidth{1}{8}}
\sphinxAtStartPar
38
\par
\vskip-\baselineskip\vbox{\hbox{\strut}}\end{varwidth}%
}%
&\sphinxmultirow{2}{418}{%
\begin{varwidth}[t]{\sphinxcolwidth{1}{8}}
\sphinxAtStartPar
Furi
\par
\vskip-\baselineskip\vbox{\hbox{\strut}}\end{varwidth}%
}%
&\sphinxmultirow{2}{419}{%
\begin{varwidth}[t]{\sphinxcolwidth{1}{8}}
\sphinxAtStartPar
zstd
\par
\vskip-\baselineskip\vbox{\hbox{\strut}}\end{varwidth}%
}%
&
\sphinxAtStartPar
3
&\sphinxmultirow{2}{421}{%
\begin{varwidth}[t]{\sphinxcolwidth{1}{8}}
\sphinxAtStartPar
4,3 GB
\par
\vskip-\baselineskip\vbox{\hbox{\strut}}\end{varwidth}%
}%
&\sphinxmultirow{2}{422}{%
\begin{varwidth}[t]{\sphinxcolwidth{1}{8}}
\sphinxAtStartPar
2,7 GB
\par
\vskip-\baselineskip\vbox{\hbox{\strut}}\end{varwidth}%
}%
&
\sphinxAtStartPar
62\%
&
\sphinxAtStartPar
1,53 GB
\\
\cline{4-4}\cline{7-8}\sphinxtablestrut{417}&\sphinxtablestrut{418}&\sphinxtablestrut{419}&
\sphinxAtStartPar
15
&\sphinxtablestrut{421}&\sphinxtablestrut{422}&
\sphinxAtStartPar
63\%
&
\sphinxAtStartPar
1,52 GB
\\
\hline\sphinxmultirow{2}{428}{%
\begin{varwidth}[t]{\sphinxcolwidth{1}{8}}
\sphinxAtStartPar
39
\par
\vskip-\baselineskip\vbox{\hbox{\strut}}\end{varwidth}%
}%
&\sphinxmultirow{2}{429}{%
\begin{varwidth}[t]{\sphinxcolwidth{1}{8}}
\sphinxAtStartPar
Gato Roboto
\par
\vskip-\baselineskip\vbox{\hbox{\strut}}\end{varwidth}%
}%
&\sphinxmultirow{2}{430}{%
\begin{varwidth}[t]{\sphinxcolwidth{1}{8}}
\sphinxAtStartPar
zstd
\par
\vskip-\baselineskip\vbox{\hbox{\strut}}\end{varwidth}%
}%
&
\sphinxAtStartPar
3
&\sphinxmultirow{2}{432}{%
\begin{varwidth}[t]{\sphinxcolwidth{1}{8}}
\sphinxAtStartPar
440 MB
\par
\vskip-\baselineskip\vbox{\hbox{\strut}}\end{varwidth}%
}%
&
\sphinxAtStartPar
415 MB
&\sphinxmultirow{2}{434}{%
\begin{varwidth}[t]{\sphinxcolwidth{1}{8}}
\sphinxAtStartPar
94\%
\par
\vskip-\baselineskip\vbox{\hbox{\strut}}\end{varwidth}%
}%
&
\sphinxAtStartPar
25,5 MB
\\
\cline{4-4}\cline{6-6}\cline{8-8}\sphinxtablestrut{428}&\sphinxtablestrut{429}&\sphinxtablestrut{430}&
\sphinxAtStartPar
15
&\sphinxtablestrut{432}&
\sphinxAtStartPar
414 MB
&\sphinxtablestrut{434}&
\sphinxAtStartPar
26,1 MB
\\
\hline\sphinxmultirow{2}{439}{%
\begin{varwidth}[t]{\sphinxcolwidth{1}{8}}
\sphinxAtStartPar
40
\par
\vskip-\baselineskip\vbox{\hbox{\strut}}\end{varwidth}%
}%
&\sphinxmultirow{2}{440}{%
\begin{varwidth}[t]{\sphinxcolwidth{1}{8}}
\sphinxAtStartPar
Gears Tactics
\par
\vskip-\baselineskip\vbox{\hbox{\strut}}\end{varwidth}%
}%
&\sphinxmultirow{2}{441}{%
\begin{varwidth}[t]{\sphinxcolwidth{1}{8}}
\sphinxAtStartPar
zstd
\par
\vskip-\baselineskip\vbox{\hbox{\strut}}\end{varwidth}%
}%
&
\sphinxAtStartPar
3
&\sphinxmultirow{2}{443}{%
\begin{varwidth}[t]{\sphinxcolwidth{1}{8}}
\sphinxAtStartPar
29 GB
\par
\vskip-\baselineskip\vbox{\hbox{\strut}}\end{varwidth}%
}%
&\sphinxmultirow{2}{444}{%
\begin{varwidth}[t]{\sphinxcolwidth{1}{8}}
\sphinxAtStartPar
29 GB
\par
\vskip-\baselineskip\vbox{\hbox{\strut}}\end{varwidth}%
}%
&\sphinxmultirow{2}{445}{%
\begin{varwidth}[t]{\sphinxcolwidth{1}{8}}
\sphinxAtStartPar
99\%
\par
\vskip-\baselineskip\vbox{\hbox{\strut}}\end{varwidth}%
}%
&
\sphinxAtStartPar
66 MB
\\
\cline{4-4}\cline{8-8}\sphinxtablestrut{439}&\sphinxtablestrut{440}&\sphinxtablestrut{441}&
\sphinxAtStartPar
15
&\sphinxtablestrut{443}&\sphinxtablestrut{444}&\sphinxtablestrut{445}&
\sphinxAtStartPar
97 MB
\\
\hline\sphinxmultirow{2}{449}{%
\begin{varwidth}[t]{\sphinxcolwidth{1}{8}}
\sphinxAtStartPar
41
\par
\vskip-\baselineskip\vbox{\hbox{\strut}}\end{varwidth}%
}%
&\sphinxmultirow{2}{450}{%
\begin{varwidth}[t]{\sphinxcolwidth{1}{8}}
\sphinxAtStartPar
Ghost of a Tale
\par
\vskip-\baselineskip\vbox{\hbox{\strut}}\end{varwidth}%
}%
&\sphinxmultirow{2}{451}{%
\begin{varwidth}[t]{\sphinxcolwidth{1}{8}}
\sphinxAtStartPar
zstd
\par
\vskip-\baselineskip\vbox{\hbox{\strut}}\end{varwidth}%
}%
&
\sphinxAtStartPar
3
&\sphinxmultirow{2}{453}{%
\begin{varwidth}[t]{\sphinxcolwidth{1}{8}}
\sphinxAtStartPar
4,7 GB
\par
\vskip-\baselineskip\vbox{\hbox{\strut}}\end{varwidth}%
}%
&\sphinxmultirow{2}{454}{%
\begin{varwidth}[t]{\sphinxcolwidth{1}{8}}
\sphinxAtStartPar
3,7 GB
\par
\vskip-\baselineskip\vbox{\hbox{\strut}}\end{varwidth}%
}%
&\sphinxmultirow{2}{455}{%
\begin{varwidth}[t]{\sphinxcolwidth{1}{8}}
\sphinxAtStartPar
79\%
\par
\vskip-\baselineskip\vbox{\hbox{\strut}}\end{varwidth}%
}%
&
\sphinxAtStartPar
0,90 GB
\\
\cline{4-4}\cline{8-8}\sphinxtablestrut{449}&\sphinxtablestrut{450}&\sphinxtablestrut{451}&
\sphinxAtStartPar
15
&\sphinxtablestrut{453}&\sphinxtablestrut{454}&\sphinxtablestrut{455}&
\sphinxAtStartPar
0,94 GB
\\
\hline\sphinxmultirow{2}{459}{%
\begin{varwidth}[t]{\sphinxcolwidth{1}{8}}
\sphinxAtStartPar
42
\par
\vskip-\baselineskip\vbox{\hbox{\strut}}\end{varwidth}%
}%
&\sphinxmultirow{2}{460}{%
\begin{varwidth}[t]{\sphinxcolwidth{1}{8}}
\sphinxAtStartPar
Ghostrunner
\par
\vskip-\baselineskip\vbox{\hbox{\strut}}\end{varwidth}%
}%
&\sphinxmultirow{2}{461}{%
\begin{varwidth}[t]{\sphinxcolwidth{1}{8}}
\sphinxAtStartPar
zstd
\par
\vskip-\baselineskip\vbox{\hbox{\strut}}\end{varwidth}%
}%
&
\sphinxAtStartPar
3
&\sphinxmultirow{2}{463}{%
\begin{varwidth}[t]{\sphinxcolwidth{1}{8}}
\sphinxAtStartPar
24 GB
\par
\vskip-\baselineskip\vbox{\hbox{\strut}}\end{varwidth}%
}%
&\sphinxmultirow{2}{464}{%
\begin{varwidth}[t]{\sphinxcolwidth{1}{8}}
\sphinxAtStartPar
20 GB
\par
\vskip-\baselineskip\vbox{\hbox{\strut}}\end{varwidth}%
}%
&\sphinxmultirow{2}{465}{%
\begin{varwidth}[t]{\sphinxcolwidth{1}{8}}
\sphinxAtStartPar
84\%
\par
\vskip-\baselineskip\vbox{\hbox{\strut}}\end{varwidth}%
}%
&\sphinxmultirow{2}{466}{%
\begin{varwidth}[t]{\sphinxcolwidth{1}{8}}
\sphinxAtStartPar
3,7 GB
\par
\vskip-\baselineskip\vbox{\hbox{\strut}}\end{varwidth}%
}%
\\
\cline{4-4}\sphinxtablestrut{459}&\sphinxtablestrut{460}&\sphinxtablestrut{461}&
\sphinxAtStartPar
15
&\sphinxtablestrut{463}&\sphinxtablestrut{464}&\sphinxtablestrut{465}&\sphinxtablestrut{466}\\
\hline\sphinxmultirow{2}{468}{%
\begin{varwidth}[t]{\sphinxcolwidth{1}{8}}
\sphinxAtStartPar
43
\par
\vskip-\baselineskip\vbox{\hbox{\strut}}\end{varwidth}%
}%
&\sphinxmultirow{2}{469}{%
\begin{varwidth}[t]{\sphinxcolwidth{1}{8}}
\sphinxAtStartPar
Gibbous \sphinxhyphen{} a Cthulhu Adventure
\par
\vskip-\baselineskip\vbox{\hbox{\strut}}\end{varwidth}%
}%
&\sphinxmultirow{2}{470}{%
\begin{varwidth}[t]{\sphinxcolwidth{1}{8}}
\sphinxAtStartPar
zstd
\par
\vskip-\baselineskip\vbox{\hbox{\strut}}\end{varwidth}%
}%
&
\sphinxAtStartPar
3
&\sphinxmultirow{2}{472}{%
\begin{varwidth}[t]{\sphinxcolwidth{1}{8}}
\sphinxAtStartPar
9,0 GB
\par
\vskip-\baselineskip\vbox{\hbox{\strut}}\end{varwidth}%
}%
&
\sphinxAtStartPar
4,2 GB
&
\sphinxAtStartPar
47\%
&
\sphinxAtStartPar
4,76\%
\\
\cline{4-4}\cline{6-8}\sphinxtablestrut{468}&\sphinxtablestrut{469}&\sphinxtablestrut{470}&
\sphinxAtStartPar
15
&\sphinxtablestrut{472}&
\sphinxAtStartPar
4,1 GB
&
\sphinxAtStartPar
46\%
&
\sphinxAtStartPar
4,87 GB
\\
\hline\sphinxmultirow{2}{480}{%
\begin{varwidth}[t]{\sphinxcolwidth{1}{8}}
\sphinxAtStartPar
44
\par
\vskip-\baselineskip\vbox{\hbox{\strut}}\end{varwidth}%
}%
&\sphinxmultirow{2}{481}{%
\begin{varwidth}[t]{\sphinxcolwidth{1}{8}}
\sphinxAtStartPar
Gris
\par
\vskip-\baselineskip\vbox{\hbox{\strut}}\end{varwidth}%
}%
&\sphinxmultirow{2}{482}{%
\begin{varwidth}[t]{\sphinxcolwidth{1}{8}}
\sphinxAtStartPar
zstd
\par
\vskip-\baselineskip\vbox{\hbox{\strut}}\end{varwidth}%
}%
&
\sphinxAtStartPar
3
&\sphinxmultirow{2}{484}{%
\begin{varwidth}[t]{\sphinxcolwidth{1}{8}}
\sphinxAtStartPar
3,2 GB
\par
\vskip-\baselineskip\vbox{\hbox{\strut}}\end{varwidth}%
}%
&\sphinxmultirow{2}{485}{%
\begin{varwidth}[t]{\sphinxcolwidth{1}{8}}
\sphinxAtStartPar
1,5 GB
\par
\vskip-\baselineskip\vbox{\hbox{\strut}}\end{varwidth}%
}%
&
\sphinxAtStartPar
47\%
&
\sphinxAtStartPar
1,70 GB
\\
\cline{4-4}\cline{7-8}\sphinxtablestrut{480}&\sphinxtablestrut{481}&\sphinxtablestrut{482}&
\sphinxAtStartPar
15
&\sphinxtablestrut{484}&\sphinxtablestrut{485}&
\sphinxAtStartPar
46\%
&
\sphinxAtStartPar
1,73 GB
\\
\hline\sphinxmultirow{2}{491}{%
\begin{varwidth}[t]{\sphinxcolwidth{1}{8}}
\sphinxAtStartPar
45
\par
\vskip-\baselineskip\vbox{\hbox{\strut}}\end{varwidth}%
}%
&\sphinxmultirow{2}{492}{%
\begin{varwidth}[t]{\sphinxcolwidth{1}{8}}
\sphinxAtStartPar
Hades
\par
\vskip-\baselineskip\vbox{\hbox{\strut}}\end{varwidth}%
}%
&\sphinxmultirow{2}{493}{%
\begin{varwidth}[t]{\sphinxcolwidth{1}{8}}
\sphinxAtStartPar
zstd
\par
\vskip-\baselineskip\vbox{\hbox{\strut}}\end{varwidth}%
}%
&
\sphinxAtStartPar
3
&\sphinxmultirow{2}{495}{%
\begin{varwidth}[t]{\sphinxcolwidth{1}{8}}
\sphinxAtStartPar
11 GB
\par
\vskip-\baselineskip\vbox{\hbox{\strut}}\end{varwidth}%
}%
&\sphinxmultirow{2}{496}{%
\begin{varwidth}[t]{\sphinxcolwidth{1}{8}}
\sphinxAtStartPar
10 GB
\par
\vskip-\baselineskip\vbox{\hbox{\strut}}\end{varwidth}%
}%
&\sphinxmultirow{2}{497}{%
\begin{varwidth}[t]{\sphinxcolwidth{1}{8}}
\sphinxAtStartPar
95\%
\par
\vskip-\baselineskip\vbox{\hbox{\strut}}\end{varwidth}%
}%
&
\sphinxAtStartPar
480 MB
\\
\cline{4-4}\cline{8-8}\sphinxtablestrut{491}&\sphinxtablestrut{492}&\sphinxtablestrut{493}&
\sphinxAtStartPar
15
&\sphinxtablestrut{495}&\sphinxtablestrut{496}&\sphinxtablestrut{497}&
\sphinxAtStartPar
498 MB
\\
\hline\sphinxmultirow{2}{501}{%
\begin{varwidth}[t]{\sphinxcolwidth{1}{8}}
\sphinxAtStartPar
46
\par
\vskip-\baselineskip\vbox{\hbox{\strut}}\end{varwidth}%
}%
&\sphinxmultirow{2}{502}{%
\begin{varwidth}[t]{\sphinxcolwidth{1}{8}}
\sphinxAtStartPar
Hand of Fate
\par
\vskip-\baselineskip\vbox{\hbox{\strut}}\end{varwidth}%
}%
&\sphinxmultirow{2}{503}{%
\begin{varwidth}[t]{\sphinxcolwidth{1}{8}}
\sphinxAtStartPar
zstd
\par
\vskip-\baselineskip\vbox{\hbox{\strut}}\end{varwidth}%
}%
&
\sphinxAtStartPar
3
&\sphinxmultirow{2}{505}{%
\begin{varwidth}[t]{\sphinxcolwidth{1}{8}}
\sphinxAtStartPar
2,5 GB
\par
\vskip-\baselineskip\vbox{\hbox{\strut}}\end{varwidth}%
}%
&\sphinxmultirow{2}{506}{%
\begin{varwidth}[t]{\sphinxcolwidth{1}{8}}
\sphinxAtStartPar
2,2 GB
\par
\vskip-\baselineskip\vbox{\hbox{\strut}}\end{varwidth}%
}%
&
\sphinxAtStartPar
90\%
&
\sphinxAtStartPar
255 MB
\\
\cline{4-4}\cline{7-8}\sphinxtablestrut{501}&\sphinxtablestrut{502}&\sphinxtablestrut{503}&
\sphinxAtStartPar
15
&\sphinxtablestrut{505}&\sphinxtablestrut{506}&
\sphinxAtStartPar
89\%
&
\sphinxAtStartPar
287 MB
\\
\hline\sphinxmultirow{2}{512}{%
\begin{varwidth}[t]{\sphinxcolwidth{1}{8}}
\sphinxAtStartPar
47
\par
\vskip-\baselineskip\vbox{\hbox{\strut}}\end{varwidth}%
}%
&\sphinxmultirow{2}{513}{%
\begin{varwidth}[t]{\sphinxcolwidth{1}{8}}
\sphinxAtStartPar
Hand of Fate 2
\par
\vskip-\baselineskip\vbox{\hbox{\strut}}\end{varwidth}%
}%
&\sphinxmultirow{2}{514}{%
\begin{varwidth}[t]{\sphinxcolwidth{1}{8}}
\sphinxAtStartPar
zstd
\par
\vskip-\baselineskip\vbox{\hbox{\strut}}\end{varwidth}%
}%
&
\sphinxAtStartPar
3
&\sphinxmultirow{2}{516}{%
\begin{varwidth}[t]{\sphinxcolwidth{1}{8}}
\sphinxAtStartPar
4,1 GB
\par
\vskip-\baselineskip\vbox{\hbox{\strut}}\end{varwidth}%
}%
&\sphinxmultirow{2}{517}{%
\begin{varwidth}[t]{\sphinxcolwidth{1}{8}}
\sphinxAtStartPar
4,1 GB
\par
\vskip-\baselineskip\vbox{\hbox{\strut}}\end{varwidth}%
}%
&\sphinxmultirow{2}{518}{%
\begin{varwidth}[t]{\sphinxcolwidth{1}{8}}
\sphinxAtStartPar
99\%
\par
\vskip-\baselineskip\vbox{\hbox{\strut}}\end{varwidth}%
}%
&
\sphinxAtStartPar
35 MB
\\
\cline{4-4}\cline{8-8}\sphinxtablestrut{512}&\sphinxtablestrut{513}&\sphinxtablestrut{514}&
\sphinxAtStartPar
15
&\sphinxtablestrut{516}&\sphinxtablestrut{517}&\sphinxtablestrut{518}&
\sphinxAtStartPar
38 MB
\\
\hline\sphinxmultirow{2}{522}{%
\begin{varwidth}[t]{\sphinxcolwidth{1}{8}}
\sphinxAtStartPar
48
\par
\vskip-\baselineskip\vbox{\hbox{\strut}}\end{varwidth}%
}%
&\sphinxmultirow{2}{523}{%
\begin{varwidth}[t]{\sphinxcolwidth{1}{8}}
\sphinxAtStartPar
Hellblade: Sanua's Sacrifice
\par
\vskip-\baselineskip\vbox{\hbox{\strut}}\end{varwidth}%
}%
&\sphinxmultirow{2}{524}{%
\begin{varwidth}[t]{\sphinxcolwidth{1}{8}}
\sphinxAtStartPar
zstd
\par
\vskip-\baselineskip\vbox{\hbox{\strut}}\end{varwidth}%
}%
&
\sphinxAtStartPar
3
&\sphinxmultirow{2}{526}{%
\begin{varwidth}[t]{\sphinxcolwidth{1}{8}}
\sphinxAtStartPar
18 GB
\par
\vskip-\baselineskip\vbox{\hbox{\strut}}\end{varwidth}%
}%
&
\sphinxAtStartPar
16 GB
&
\sphinxAtStartPar
87\%
&
\sphinxAtStartPar
2,3 GB
\\
\cline{4-4}\cline{6-8}\sphinxtablestrut{522}&\sphinxtablestrut{523}&\sphinxtablestrut{524}&
\sphinxAtStartPar
15
&\sphinxtablestrut{526}&
\sphinxAtStartPar
18 GB
&
\sphinxAtStartPar
96\%
&
\sphinxAtStartPar
693 MB
\\
\hline\sphinxmultirow{2}{534}{%
\begin{varwidth}[t]{\sphinxcolwidth{1}{8}}
\sphinxAtStartPar
49
\par
\vskip-\baselineskip\vbox{\hbox{\strut}}\end{varwidth}%
}%
&\sphinxmultirow{2}{535}{%
\begin{varwidth}[t]{\sphinxcolwidth{1}{8}}
\sphinxAtStartPar
Helldivers
\par
\vskip-\baselineskip\vbox{\hbox{\strut}}\end{varwidth}%
}%
&\sphinxmultirow{2}{536}{%
\begin{varwidth}[t]{\sphinxcolwidth{1}{8}}
\sphinxAtStartPar
zstd
\par
\vskip-\baselineskip\vbox{\hbox{\strut}}\end{varwidth}%
}%
&
\sphinxAtStartPar
3
&\sphinxmultirow{2}{538}{%
\begin{varwidth}[t]{\sphinxcolwidth{1}{8}}
\sphinxAtStartPar
6,4 GB
\par
\vskip-\baselineskip\vbox{\hbox{\strut}}\end{varwidth}%
}%
&\sphinxmultirow{2}{539}{%
\begin{varwidth}[t]{\sphinxcolwidth{1}{8}}
\sphinxAtStartPar
6,4 GB
\par
\vskip-\baselineskip\vbox{\hbox{\strut}}\end{varwidth}%
}%
&\sphinxmultirow{2}{540}{%
\begin{varwidth}[t]{\sphinxcolwidth{1}{8}}
\sphinxAtStartPar
99\%
\par
\vskip-\baselineskip\vbox{\hbox{\strut}}\end{varwidth}%
}%
&
\sphinxAtStartPar
25 MB
\\
\cline{4-4}\cline{8-8}\sphinxtablestrut{534}&\sphinxtablestrut{535}&\sphinxtablestrut{536}&
\sphinxAtStartPar
15
&\sphinxtablestrut{538}&\sphinxtablestrut{539}&\sphinxtablestrut{540}&
\sphinxAtStartPar
27 MB
\\
\hline\sphinxmultirow{2}{544}{%
\begin{varwidth}[t]{\sphinxcolwidth{1}{8}}
\sphinxAtStartPar
50
\par
\vskip-\baselineskip\vbox{\hbox{\strut}}\end{varwidth}%
}%
&\sphinxmultirow{2}{545}{%
\begin{varwidth}[t]{\sphinxcolwidth{1}{8}}
\sphinxAtStartPar
Hob
\par
\vskip-\baselineskip\vbox{\hbox{\strut}}\end{varwidth}%
}%
&\sphinxmultirow{2}{546}{%
\begin{varwidth}[t]{\sphinxcolwidth{1}{8}}
\sphinxAtStartPar
zstd
\par
\vskip-\baselineskip\vbox{\hbox{\strut}}\end{varwidth}%
}%
&
\sphinxAtStartPar
3
&\sphinxmultirow{2}{548}{%
\begin{varwidth}[t]{\sphinxcolwidth{1}{8}}
\sphinxAtStartPar
2,4 GB
\par
\vskip-\baselineskip\vbox{\hbox{\strut}}\end{varwidth}%
}%
&
\sphinxAtStartPar
2,2 GB
&
\sphinxAtStartPar
90\%
&
\sphinxAtStartPar
230 MB
\\
\cline{4-4}\cline{6-8}\sphinxtablestrut{544}&\sphinxtablestrut{545}&\sphinxtablestrut{546}&
\sphinxAtStartPar
15
&\sphinxtablestrut{548}&
\sphinxAtStartPar
2,1 GB
&
\sphinxAtStartPar
89\%
&
\sphinxAtStartPar
250 MB
\\
\hline\sphinxmultirow{2}{556}{%
\begin{varwidth}[t]{\sphinxcolwidth{1}{8}}
\sphinxAtStartPar
51
\par
\vskip-\baselineskip\vbox{\hbox{\strut}}\end{varwidth}%
}%
&\sphinxmultirow{2}{557}{%
\begin{varwidth}[t]{\sphinxcolwidth{1}{8}}
\sphinxAtStartPar
Hollow Knight
\par
\vskip-\baselineskip\vbox{\hbox{\strut}}\end{varwidth}%
}%
&\sphinxmultirow{2}{558}{%
\begin{varwidth}[t]{\sphinxcolwidth{1}{8}}
\sphinxAtStartPar
zstd
\par
\vskip-\baselineskip\vbox{\hbox{\strut}}\end{varwidth}%
}%
&
\sphinxAtStartPar
3
&\sphinxmultirow{2}{560}{%
\begin{varwidth}[t]{\sphinxcolwidth{1}{8}}
\sphinxAtStartPar
7,5 GB
\par
\vskip-\baselineskip\vbox{\hbox{\strut}}\end{varwidth}%
}%
&
\sphinxAtStartPar
1,5 GB
&
\sphinxAtStartPar
20\%
&
\sphinxAtStartPar
5,87 GB
\\
\cline{4-4}\cline{6-8}\sphinxtablestrut{556}&\sphinxtablestrut{557}&\sphinxtablestrut{558}&
\sphinxAtStartPar
15
&\sphinxtablestrut{560}&
\sphinxAtStartPar
1,4 GB
&
\sphinxAtStartPar
19\%
&
\sphinxAtStartPar
5,98 GB
\\
\hline\sphinxmultirow{2}{568}{%
\begin{varwidth}[t]{\sphinxcolwidth{1}{8}}
\sphinxAtStartPar
52
\par
\vskip-\baselineskip\vbox{\hbox{\strut}}\end{varwidth}%
}%
&\sphinxmultirow{2}{569}{%
\begin{varwidth}[t]{\sphinxcolwidth{1}{8}}
\sphinxAtStartPar
Inmost
\par
\vskip-\baselineskip\vbox{\hbox{\strut}}\end{varwidth}%
}%
&\sphinxmultirow{2}{570}{%
\begin{varwidth}[t]{\sphinxcolwidth{1}{8}}
\sphinxAtStartPar
zstd
\par
\vskip-\baselineskip\vbox{\hbox{\strut}}\end{varwidth}%
}%
&
\sphinxAtStartPar
3
&\sphinxmultirow{2}{572}{%
\begin{varwidth}[t]{\sphinxcolwidth{1}{8}}
\sphinxAtStartPar
1,3 GB
\par
\vskip-\baselineskip\vbox{\hbox{\strut}}\end{varwidth}%
}%
&
\sphinxAtStartPar
649 MB
&\sphinxmultirow{2}{574}{%
\begin{varwidth}[t]{\sphinxcolwidth{1}{8}}
\sphinxAtStartPar
47\%
\par
\vskip-\baselineskip\vbox{\hbox{\strut}}\end{varwidth}%
}%
&
\sphinxAtStartPar
709 MB
\\
\cline{4-4}\cline{6-6}\cline{8-8}\sphinxtablestrut{568}&\sphinxtablestrut{569}&\sphinxtablestrut{570}&
\sphinxAtStartPar
15
&\sphinxtablestrut{572}&
\sphinxAtStartPar
638 MB
&\sphinxtablestrut{574}&
\sphinxAtStartPar
720 MB
\\
\hline\sphinxmultirow{2}{579}{%
\begin{varwidth}[t]{\sphinxcolwidth{1}{8}}
\sphinxAtStartPar
53
\par
\vskip-\baselineskip\vbox{\hbox{\strut}}\end{varwidth}%
}%
&\sphinxmultirow{2}{580}{%
\begin{varwidth}[t]{\sphinxcolwidth{1}{8}}
\sphinxAtStartPar
Jotun
\par
\vskip-\baselineskip\vbox{\hbox{\strut}}\end{varwidth}%
}%
&\sphinxmultirow{2}{581}{%
\begin{varwidth}[t]{\sphinxcolwidth{1}{8}}
\sphinxAtStartPar
zstd
\par
\vskip-\baselineskip\vbox{\hbox{\strut}}\end{varwidth}%
}%
&
\sphinxAtStartPar
3
&\sphinxmultirow{2}{583}{%
\begin{varwidth}[t]{\sphinxcolwidth{1}{8}}
\sphinxAtStartPar
3,8 GB
\par
\vskip-\baselineskip\vbox{\hbox{\strut}}\end{varwidth}%
}%
&\sphinxmultirow{2}{584}{%
\begin{varwidth}[t]{\sphinxcolwidth{1}{8}}
\sphinxAtStartPar
1,8 GB
\par
\vskip-\baselineskip\vbox{\hbox{\strut}}\end{varwidth}%
}%
&
\sphinxAtStartPar
48\%
&
\sphinxAtStartPar
1,91 GB
\\
\cline{4-4}\cline{7-8}\sphinxtablestrut{579}&\sphinxtablestrut{580}&\sphinxtablestrut{581}&
\sphinxAtStartPar
15
&\sphinxtablestrut{583}&\sphinxtablestrut{584}&
\sphinxAtStartPar
49\%
&
\sphinxAtStartPar
1,84 GB
\\
\hline\sphinxmultirow{2}{590}{%
\begin{varwidth}[t]{\sphinxcolwidth{1}{8}}
\sphinxAtStartPar
54
\par
\vskip-\baselineskip\vbox{\hbox{\strut}}\end{varwidth}%
}%
&\sphinxmultirow{2}{591}{%
\begin{varwidth}[t]{\sphinxcolwidth{1}{8}}
\sphinxAtStartPar
Journey
\par
\vskip-\baselineskip\vbox{\hbox{\strut}}\end{varwidth}%
}%
&\sphinxmultirow{2}{592}{%
\begin{varwidth}[t]{\sphinxcolwidth{1}{8}}
\sphinxAtStartPar
zstd
\par
\vskip-\baselineskip\vbox{\hbox{\strut}}\end{varwidth}%
}%
&
\sphinxAtStartPar
3
&\sphinxmultirow{2}{594}{%
\begin{varwidth}[t]{\sphinxcolwidth{1}{8}}
\sphinxAtStartPar
3,3 GB
\par
\vskip-\baselineskip\vbox{\hbox{\strut}}\end{varwidth}%
}%
&
\sphinxAtStartPar
1,8 GB
&
\sphinxAtStartPar
55\%
&
\sphinxAtStartPar
1,49 GB
\\
\cline{4-4}\cline{6-8}\sphinxtablestrut{590}&\sphinxtablestrut{591}&\sphinxtablestrut{592}&
\sphinxAtStartPar
15
&\sphinxtablestrut{594}&
\sphinxAtStartPar
1,9 GB
&
\sphinxAtStartPar
56\%
&
\sphinxAtStartPar
1,44 GB
\\
\hline\sphinxmultirow{2}{602}{%
\begin{varwidth}[t]{\sphinxcolwidth{1}{8}}
\sphinxAtStartPar
55
\par
\vskip-\baselineskip\vbox{\hbox{\strut}}\end{varwidth}%
}%
&\sphinxmultirow{2}{603}{%
\begin{varwidth}[t]{\sphinxcolwidth{1}{8}}
\sphinxAtStartPar
Katana ZERO
\par
\vskip-\baselineskip\vbox{\hbox{\strut}}\end{varwidth}%
}%
&\sphinxmultirow{2}{604}{%
\begin{varwidth}[t]{\sphinxcolwidth{1}{8}}
\sphinxAtStartPar
zstd
\par
\vskip-\baselineskip\vbox{\hbox{\strut}}\end{varwidth}%
}%
&
\sphinxAtStartPar
3
&\sphinxmultirow{2}{606}{%
\begin{varwidth}[t]{\sphinxcolwidth{1}{8}}
\sphinxAtStartPar
216 MB
\par
\vskip-\baselineskip\vbox{\hbox{\strut}}\end{varwidth}%
}%
&
\sphinxAtStartPar
178 MB
&
\sphinxAtStartPar
82\%
&
\sphinxAtStartPar
38 MB
\\
\cline{4-4}\cline{6-8}\sphinxtablestrut{602}&\sphinxtablestrut{603}&\sphinxtablestrut{604}&
\sphinxAtStartPar
15
&\sphinxtablestrut{606}&
\sphinxAtStartPar
177 MB
&
\sphinxAtStartPar
81\%
&
\sphinxAtStartPar
39 MB
\\
\hline\sphinxmultirow{2}{614}{%
\begin{varwidth}[t]{\sphinxcolwidth{1}{8}}
\sphinxAtStartPar
56
\par
\vskip-\baselineskip\vbox{\hbox{\strut}}\end{varwidth}%
}%
&\sphinxmultirow{2}{615}{%
\begin{varwidth}[t]{\sphinxcolwidth{1}{8}}
\sphinxAtStartPar
Kate
\par
\vskip-\baselineskip\vbox{\hbox{\strut}}\end{varwidth}%
}%
&\sphinxmultirow{2}{616}{%
\begin{varwidth}[t]{\sphinxcolwidth{1}{8}}
\sphinxAtStartPar
zstd
\par
\vskip-\baselineskip\vbox{\hbox{\strut}}\end{varwidth}%
}%
&
\sphinxAtStartPar
3
&\sphinxmultirow{2}{618}{%
\begin{varwidth}[t]{\sphinxcolwidth{1}{8}}
\sphinxAtStartPar
254 MB
\par
\vskip-\baselineskip\vbox{\hbox{\strut}}\end{varwidth}%
}%
&
\sphinxAtStartPar
104 MB
&
\sphinxAtStartPar
40\%
&
\sphinxAtStartPar
151 MB
\\
\cline{4-4}\cline{6-8}\sphinxtablestrut{614}&\sphinxtablestrut{615}&\sphinxtablestrut{616}&
\sphinxAtStartPar
15
&\sphinxtablestrut{618}&
\sphinxAtStartPar
100 MB
&
\sphinxAtStartPar
39\%
&
\sphinxAtStartPar
155 MB
\\
\hline\sphinxmultirow{2}{626}{%
\begin{varwidth}[t]{\sphinxcolwidth{1}{8}}
\sphinxAtStartPar
57
\par
\vskip-\baselineskip\vbox{\hbox{\strut}}\end{varwidth}%
}%
&\sphinxmultirow{2}{627}{%
\begin{varwidth}[t]{\sphinxcolwidth{1}{8}}
\sphinxAtStartPar
Limbo
\par
\vskip-\baselineskip\vbox{\hbox{\strut}}\end{varwidth}%
}%
&\sphinxmultirow{2}{628}{%
\begin{varwidth}[t]{\sphinxcolwidth{1}{8}}
\sphinxAtStartPar
zstd
\par
\vskip-\baselineskip\vbox{\hbox{\strut}}\end{varwidth}%
}%
&
\sphinxAtStartPar
3
&\sphinxmultirow{2}{630}{%
\begin{varwidth}[t]{\sphinxcolwidth{1}{8}}
\sphinxAtStartPar
98 MB
\par
\vskip-\baselineskip\vbox{\hbox{\strut}}\end{varwidth}%
}%
&\sphinxmultirow{2}{631}{%
\begin{varwidth}[t]{\sphinxcolwidth{1}{8}}
\sphinxAtStartPar
97 MB
\par
\vskip-\baselineskip\vbox{\hbox{\strut}}\end{varwidth}%
}%
&\sphinxmultirow{2}{632}{%
\begin{varwidth}[t]{\sphinxcolwidth{1}{8}}
\sphinxAtStartPar
98\%
\par
\vskip-\baselineskip\vbox{\hbox{\strut}}\end{varwidth}%
}%
&
\sphinxAtStartPar
1,7 MB
\\
\cline{4-4}\cline{8-8}\sphinxtablestrut{626}&\sphinxtablestrut{627}&\sphinxtablestrut{628}&
\sphinxAtStartPar
15
&\sphinxtablestrut{630}&\sphinxtablestrut{631}&\sphinxtablestrut{632}&
\sphinxAtStartPar
1,8 MB
\\
\hline\sphinxmultirow{2}{636}{%
\begin{varwidth}[t]{\sphinxcolwidth{1}{8}}
\sphinxAtStartPar
58
\par
\vskip-\baselineskip\vbox{\hbox{\strut}}\end{varwidth}%
}%
&\sphinxmultirow{2}{637}{%
\begin{varwidth}[t]{\sphinxcolwidth{1}{8}}
\sphinxAtStartPar
Little Nightmare
\par
\vskip-\baselineskip\vbox{\hbox{\strut}}\end{varwidth}%
}%
&\sphinxmultirow{2}{638}{%
\begin{varwidth}[t]{\sphinxcolwidth{1}{8}}
\sphinxAtStartPar
zstd
\par
\vskip-\baselineskip\vbox{\hbox{\strut}}\end{varwidth}%
}%
&
\sphinxAtStartPar
3
&\sphinxmultirow{2}{640}{%
\begin{varwidth}[t]{\sphinxcolwidth{1}{8}}
\sphinxAtStartPar
8,9 GB
\par
\vskip-\baselineskip\vbox{\hbox{\strut}}\end{varwidth}%
}%
&
\sphinxAtStartPar
5,8 GB
&
\sphinxAtStartPar
65\%
&
\sphinxAtStartPar
3,1 GB
\\
\cline{4-4}\cline{6-8}\sphinxtablestrut{636}&\sphinxtablestrut{637}&\sphinxtablestrut{638}&
\sphinxAtStartPar
15
&\sphinxtablestrut{640}&
\sphinxAtStartPar
4,8 GB
&
\sphinxAtStartPar
54\%
&
\sphinxAtStartPar
4,1 GB
\\
\hline\sphinxmultirow{2}{648}{%
\begin{varwidth}[t]{\sphinxcolwidth{1}{8}}
\sphinxAtStartPar
59
\par
\vskip-\baselineskip\vbox{\hbox{\strut}}\end{varwidth}%
}%
&\sphinxmultirow{2}{649}{%
\begin{varwidth}[t]{\sphinxcolwidth{1}{8}}
\sphinxAtStartPar
Loop Hero
\par
\vskip-\baselineskip\vbox{\hbox{\strut}}\end{varwidth}%
}%
&\sphinxmultirow{2}{650}{%
\begin{varwidth}[t]{\sphinxcolwidth{1}{8}}
\sphinxAtStartPar
zstd
\par
\vskip-\baselineskip\vbox{\hbox{\strut}}\end{varwidth}%
}%
&
\sphinxAtStartPar
3
&\sphinxmultirow{2}{652}{%
\begin{varwidth}[t]{\sphinxcolwidth{1}{8}}
\sphinxAtStartPar
140 MB
\par
\vskip-\baselineskip\vbox{\hbox{\strut}}\end{varwidth}%
}%
&
\sphinxAtStartPar
116 MB
&
\sphinxAtStartPar
83\%
&
\sphinxAtStartPar
22,8 MB
\\
\cline{4-4}\cline{6-8}\sphinxtablestrut{648}&\sphinxtablestrut{649}&\sphinxtablestrut{650}&
\sphinxAtStartPar
15
&\sphinxtablestrut{652}&
\sphinxAtStartPar
115 MB
&
\sphinxAtStartPar
82\%
&
\sphinxAtStartPar
23,9 MB
\\
\hline\sphinxmultirow{2}{660}{%
\begin{varwidth}[t]{\sphinxcolwidth{1}{8}}
\sphinxAtStartPar
60
\par
\vskip-\baselineskip\vbox{\hbox{\strut}}\end{varwidth}%
}%
&\sphinxmultirow{2}{661}{%
\begin{varwidth}[t]{\sphinxcolwidth{1}{8}}
\sphinxAtStartPar
Magicka
\par
\vskip-\baselineskip\vbox{\hbox{\strut}}\end{varwidth}%
}%
&\sphinxmultirow{2}{662}{%
\begin{varwidth}[t]{\sphinxcolwidth{1}{8}}
\sphinxAtStartPar
zstd
\par
\vskip-\baselineskip\vbox{\hbox{\strut}}\end{varwidth}%
}%
&
\sphinxAtStartPar
3
&\sphinxmultirow{2}{664}{%
\begin{varwidth}[t]{\sphinxcolwidth{1}{8}}
\sphinxAtStartPar
1,6 GB
\par
\vskip-\baselineskip\vbox{\hbox{\strut}}\end{varwidth}%
}%
&\sphinxmultirow{2}{665}{%
\begin{varwidth}[t]{\sphinxcolwidth{1}{8}}
\sphinxAtStartPar
1,6 GB
\par
\vskip-\baselineskip\vbox{\hbox{\strut}}\end{varwidth}%
}%
&
\sphinxAtStartPar
96\%
&
\sphinxAtStartPar
68 MB
\\
\cline{4-4}\cline{7-8}\sphinxtablestrut{660}&\sphinxtablestrut{661}&\sphinxtablestrut{662}&
\sphinxAtStartPar
15
&\sphinxtablestrut{664}&\sphinxtablestrut{665}&
\sphinxAtStartPar
95\%
&
\sphinxAtStartPar
71 MB
\\
\hline\sphinxmultirow{2}{671}{%
\begin{varwidth}[t]{\sphinxcolwidth{1}{8}}
\sphinxAtStartPar
61
\par
\vskip-\baselineskip\vbox{\hbox{\strut}}\end{varwidth}%
}%
&\sphinxmultirow{2}{672}{%
\begin{varwidth}[t]{\sphinxcolwidth{1}{8}}
\sphinxAtStartPar
Magicka 2
\par
\vskip-\baselineskip\vbox{\hbox{\strut}}\end{varwidth}%
}%
&\sphinxmultirow{2}{673}{%
\begin{varwidth}[t]{\sphinxcolwidth{1}{8}}
\sphinxAtStartPar
zstd
\par
\vskip-\baselineskip\vbox{\hbox{\strut}}\end{varwidth}%
}%
&
\sphinxAtStartPar
3
&\sphinxmultirow{2}{675}{%
\begin{varwidth}[t]{\sphinxcolwidth{1}{8}}
\sphinxAtStartPar
2,9 GB
\par
\vskip-\baselineskip\vbox{\hbox{\strut}}\end{varwidth}%
}%
&&\sphinxmultirow{2}{677}{%
\begin{varwidth}[t]{\sphinxcolwidth{1}{8}}
\sphinxAtStartPar
99\%
\par
\vskip-\baselineskip\vbox{\hbox{\strut}}\end{varwidth}%
}%
&
\sphinxAtStartPar
8,1 MB
\\
\cline{4-4}\cline{6-6}\cline{8-8}\sphinxtablestrut{671}&\sphinxtablestrut{672}&\sphinxtablestrut{673}&
\sphinxAtStartPar
15
&\sphinxtablestrut{675}&
\sphinxAtStartPar
2,9 GB
&\sphinxtablestrut{677}&
\sphinxAtStartPar
8,7 MB
\\
\hline\sphinxmultirow{2}{682}{%
\begin{varwidth}[t]{\sphinxcolwidth{1}{8}}
\sphinxAtStartPar
62
\par
\vskip-\baselineskip\vbox{\hbox{\strut}}\end{varwidth}%
}%
&\sphinxmultirow{2}{683}{%
\begin{varwidth}[t]{\sphinxcolwidth{1}{8}}
\sphinxAtStartPar
Mark of the Ninja: Remastered
\par
\vskip-\baselineskip\vbox{\hbox{\strut}}\end{varwidth}%
}%
&\sphinxmultirow{2}{684}{%
\begin{varwidth}[t]{\sphinxcolwidth{1}{8}}
\sphinxAtStartPar
zstd
\par
\vskip-\baselineskip\vbox{\hbox{\strut}}\end{varwidth}%
}%
&
\sphinxAtStartPar
3
&\sphinxmultirow{2}{686}{%
\begin{varwidth}[t]{\sphinxcolwidth{1}{8}}
\sphinxAtStartPar
7,5 GB
\par
\vskip-\baselineskip\vbox{\hbox{\strut}}\end{varwidth}%
}%
&\sphinxmultirow{2}{687}{%
\begin{varwidth}[t]{\sphinxcolwidth{1}{8}}
\sphinxAtStartPar
6,9 GB
\par
\vskip-\baselineskip\vbox{\hbox{\strut}}\end{varwidth}%
}%
&\sphinxmultirow{2}{688}{%
\begin{varwidth}[t]{\sphinxcolwidth{1}{8}}
\sphinxAtStartPar
92\%
\par
\vskip-\baselineskip\vbox{\hbox{\strut}}\end{varwidth}%
}%
&
\sphinxAtStartPar
564 MB
\\
\cline{4-4}\cline{8-8}\sphinxtablestrut{682}&\sphinxtablestrut{683}&\sphinxtablestrut{684}&
\sphinxAtStartPar
15
&\sphinxtablestrut{686}&\sphinxtablestrut{687}&\sphinxtablestrut{688}&
\sphinxAtStartPar
591 MB
\\
\hline\sphinxmultirow{2}{692}{%
\begin{varwidth}[t]{\sphinxcolwidth{1}{8}}
\sphinxAtStartPar
63
\par
\vskip-\baselineskip\vbox{\hbox{\strut}}\end{varwidth}%
}%
&\sphinxmultirow{2}{693}{%
\begin{varwidth}[t]{\sphinxcolwidth{1}{8}}
\sphinxAtStartPar
Master of Anima
\par
\vskip-\baselineskip\vbox{\hbox{\strut}}\end{varwidth}%
}%
&\sphinxmultirow{2}{694}{%
\begin{varwidth}[t]{\sphinxcolwidth{1}{8}}
\sphinxAtStartPar
zstd
\par
\vskip-\baselineskip\vbox{\hbox{\strut}}\end{varwidth}%
}%
&
\sphinxAtStartPar
3
&\sphinxmultirow{2}{696}{%
\begin{varwidth}[t]{\sphinxcolwidth{1}{8}}
\sphinxAtStartPar
1,5 GB
\par
\vskip-\baselineskip\vbox{\hbox{\strut}}\end{varwidth}%
}%
&\sphinxmultirow{2}{697}{%
\begin{varwidth}[t]{\sphinxcolwidth{1}{8}}
\sphinxAtStartPar
1,2 GB
\par
\vskip-\baselineskip\vbox{\hbox{\strut}}\end{varwidth}%
}%
&
\sphinxAtStartPar
81\%
&
\sphinxAtStartPar
292 MB
\\
\cline{4-4}\cline{7-8}\sphinxtablestrut{692}&\sphinxtablestrut{693}&\sphinxtablestrut{694}&
\sphinxAtStartPar
15
&\sphinxtablestrut{696}&\sphinxtablestrut{697}&
\sphinxAtStartPar
80\%
&
\sphinxAtStartPar
308 MB
\\
\hline\sphinxmultirow{2}{703}{%
\begin{varwidth}[t]{\sphinxcolwidth{1}{8}}
\sphinxAtStartPar
64
\par
\vskip-\baselineskip\vbox{\hbox{\strut}}\end{varwidth}%
}%
&\sphinxmultirow{2}{704}{%
\begin{varwidth}[t]{\sphinxcolwidth{1}{8}}
\sphinxAtStartPar
METAL GEAR RISING: REVENGEANCE
\par
\vskip-\baselineskip\vbox{\hbox{\strut}}\end{varwidth}%
}%
&\sphinxmultirow{2}{705}{%
\begin{varwidth}[t]{\sphinxcolwidth{1}{8}}
\sphinxAtStartPar
zstd
\par
\vskip-\baselineskip\vbox{\hbox{\strut}}\end{varwidth}%
}%
&
\sphinxAtStartPar
3
&\sphinxmultirow{2}{707}{%
\begin{varwidth}[t]{\sphinxcolwidth{1}{8}}
\sphinxAtStartPar
24 GB
\par
\vskip-\baselineskip\vbox{\hbox{\strut}}\end{varwidth}%
}%
&\sphinxmultirow{2}{708}{%
\begin{varwidth}[t]{\sphinxcolwidth{1}{8}}
\sphinxAtStartPar
24 GB
\par
\vskip-\baselineskip\vbox{\hbox{\strut}}\end{varwidth}%
}%
&\sphinxmultirow{2}{709}{%
\begin{varwidth}[t]{\sphinxcolwidth{1}{8}}
\sphinxAtStartPar
99\%
\par
\vskip-\baselineskip\vbox{\hbox{\strut}}\end{varwidth}%
}%
&
\sphinxAtStartPar
17,8 MB
\\
\cline{4-4}\cline{8-8}\sphinxtablestrut{703}&\sphinxtablestrut{704}&\sphinxtablestrut{705}&
\sphinxAtStartPar
15
&\sphinxtablestrut{707}&\sphinxtablestrut{708}&\sphinxtablestrut{709}&
\sphinxAtStartPar
19,4 MB
\\
\hline\sphinxmultirow{2}{713}{%
\begin{varwidth}[t]{\sphinxcolwidth{1}{8}}
\sphinxAtStartPar
65
\par
\vskip-\baselineskip\vbox{\hbox{\strut}}\end{varwidth}%
}%
&\sphinxmultirow{2}{714}{%
\begin{varwidth}[t]{\sphinxcolwidth{1}{8}}
\sphinxAtStartPar
Moonlighter
\par
\vskip-\baselineskip\vbox{\hbox{\strut}}\end{varwidth}%
}%
&\sphinxmultirow{2}{715}{%
\begin{varwidth}[t]{\sphinxcolwidth{1}{8}}
\sphinxAtStartPar
zstd
\par
\vskip-\baselineskip\vbox{\hbox{\strut}}\end{varwidth}%
}%
&
\sphinxAtStartPar
3
&\sphinxmultirow{2}{717}{%
\begin{varwidth}[t]{\sphinxcolwidth{1}{8}}
\sphinxAtStartPar
1,1 GB
\par
\vskip-\baselineskip\vbox{\hbox{\strut}}\end{varwidth}%
}%
&
\sphinxAtStartPar
577 MB
&\sphinxmultirow{2}{719}{%
\begin{varwidth}[t]{\sphinxcolwidth{1}{8}}
\sphinxAtStartPar
48\%
\par
\vskip-\baselineskip\vbox{\hbox{\strut}}\end{varwidth}%
}%
&
\sphinxAtStartPar
608 MB
\\
\cline{4-4}\cline{6-6}\cline{8-8}\sphinxtablestrut{713}&\sphinxtablestrut{714}&\sphinxtablestrut{715}&
\sphinxAtStartPar
15
&\sphinxtablestrut{717}&
\sphinxAtStartPar
572 MB
&\sphinxtablestrut{719}&
\sphinxAtStartPar
613 MB
\\
\hline\sphinxmultirow{2}{724}{%
\begin{varwidth}[t]{\sphinxcolwidth{1}{8}}
\sphinxAtStartPar
66
\par
\vskip-\baselineskip\vbox{\hbox{\strut}}\end{varwidth}%
}%
&\sphinxmultirow{2}{725}{%
\begin{varwidth}[t]{\sphinxcolwidth{1}{8}}
\sphinxAtStartPar
Move or Die
\par
\vskip-\baselineskip\vbox{\hbox{\strut}}\end{varwidth}%
}%
&\sphinxmultirow{2}{726}{%
\begin{varwidth}[t]{\sphinxcolwidth{1}{8}}
\sphinxAtStartPar
zstd
\par
\vskip-\baselineskip\vbox{\hbox{\strut}}\end{varwidth}%
}%
&
\sphinxAtStartPar
3
&\sphinxmultirow{2}{728}{%
\begin{varwidth}[t]{\sphinxcolwidth{1}{8}}
\sphinxAtStartPar
666 MB
\par
\vskip-\baselineskip\vbox{\hbox{\strut}}\end{varwidth}%
}%
&
\sphinxAtStartPar
572 MB
&\sphinxmultirow{2}{730}{%
\begin{varwidth}[t]{\sphinxcolwidth{1}{8}}
\sphinxAtStartPar
85\%
\par
\vskip-\baselineskip\vbox{\hbox{\strut}}\end{varwidth}%
}%
&
\sphinxAtStartPar
94 MB
\\
\cline{4-4}\cline{6-6}\cline{8-8}\sphinxtablestrut{724}&\sphinxtablestrut{725}&\sphinxtablestrut{726}&
\sphinxAtStartPar
15
&\sphinxtablestrut{728}&
\sphinxAtStartPar
567 MB
&\sphinxtablestrut{730}&
\sphinxAtStartPar
99 MB
\\
\hline\sphinxmultirow{2}{735}{%
\begin{varwidth}[t]{\sphinxcolwidth{1}{8}}
\sphinxAtStartPar
67
\par
\vskip-\baselineskip\vbox{\hbox{\strut}}\end{varwidth}%
}%
&\sphinxmultirow{2}{736}{%
\begin{varwidth}[t]{\sphinxcolwidth{1}{8}}
\sphinxAtStartPar
My Friend Pedro
\par
\vskip-\baselineskip\vbox{\hbox{\strut}}\end{varwidth}%
}%
&\sphinxmultirow{2}{737}{%
\begin{varwidth}[t]{\sphinxcolwidth{1}{8}}
\sphinxAtStartPar
zstd
\par
\vskip-\baselineskip\vbox{\hbox{\strut}}\end{varwidth}%
}%
&
\sphinxAtStartPar
3
&\sphinxmultirow{2}{739}{%
\begin{varwidth}[t]{\sphinxcolwidth{1}{8}}
\sphinxAtStartPar
3,5 GB
\par
\vskip-\baselineskip\vbox{\hbox{\strut}}\end{varwidth}%
}%
&\sphinxmultirow{2}{740}{%
\begin{varwidth}[t]{\sphinxcolwidth{1}{8}}
\sphinxAtStartPar
2,9 GB
\par
\vskip-\baselineskip\vbox{\hbox{\strut}}\end{varwidth}%
}%
&
\sphinxAtStartPar
82\%
&
\sphinxAtStartPar
637 MB
\\
\cline{4-4}\cline{7-8}\sphinxtablestrut{735}&\sphinxtablestrut{736}&\sphinxtablestrut{737}&
\sphinxAtStartPar
15
&\sphinxtablestrut{739}&\sphinxtablestrut{740}&
\sphinxAtStartPar
81\%
&
\sphinxAtStartPar
666 MB
\\
\hline\sphinxmultirow{2}{746}{%
\begin{varwidth}[t]{\sphinxcolwidth{1}{8}}
\sphinxAtStartPar
68
\par
\vskip-\baselineskip\vbox{\hbox{\strut}}\end{varwidth}%
}%
&\sphinxmultirow{2}{747}{%
\begin{varwidth}[t]{\sphinxcolwidth{1}{8}}
\sphinxAtStartPar
Nier:Automata
\par
\vskip-\baselineskip\vbox{\hbox{\strut}}\end{varwidth}%
}%
&\sphinxmultirow{2}{748}{%
\begin{varwidth}[t]{\sphinxcolwidth{1}{8}}
\sphinxAtStartPar
zstd
\par
\vskip-\baselineskip\vbox{\hbox{\strut}}\end{varwidth}%
}%
&
\sphinxAtStartPar
3
&\sphinxmultirow{2}{750}{%
\begin{varwidth}[t]{\sphinxcolwidth{1}{8}}
\sphinxAtStartPar
40 GB
\par
\vskip-\baselineskip\vbox{\hbox{\strut}}\end{varwidth}%
}%
&\sphinxmultirow{2}{751}{%
\begin{varwidth}[t]{\sphinxcolwidth{1}{8}}
\sphinxAtStartPar
37 GB
\par
\vskip-\baselineskip\vbox{\hbox{\strut}}\end{varwidth}%
}%
&\sphinxmultirow{2}{752}{%
\begin{varwidth}[t]{\sphinxcolwidth{1}{8}}
\sphinxAtStartPar
91\%
\par
\vskip-\baselineskip\vbox{\hbox{\strut}}\end{varwidth}%
}%
&
\sphinxAtStartPar
3,5 GB
\\
\cline{4-4}\cline{8-8}\sphinxtablestrut{746}&\sphinxtablestrut{747}&\sphinxtablestrut{748}&
\sphinxAtStartPar
15
&\sphinxtablestrut{750}&\sphinxtablestrut{751}&\sphinxtablestrut{752}&
\sphinxAtStartPar
3,3 GB
\\
\hline\sphinxmultirow{2}{756}{%
\begin{varwidth}[t]{\sphinxcolwidth{1}{8}}
\sphinxAtStartPar
69
\par
\vskip-\baselineskip\vbox{\hbox{\strut}}\end{varwidth}%
}%
&\sphinxmultirow{2}{757}{%
\begin{varwidth}[t]{\sphinxcolwidth{1}{8}}
\sphinxAtStartPar
Nine Parchments
\par
\vskip-\baselineskip\vbox{\hbox{\strut}}\end{varwidth}%
}%
&\sphinxmultirow{2}{758}{%
\begin{varwidth}[t]{\sphinxcolwidth{1}{8}}
\sphinxAtStartPar
zstd
\par
\vskip-\baselineskip\vbox{\hbox{\strut}}\end{varwidth}%
}%
&
\sphinxAtStartPar
3
&\sphinxmultirow{2}{760}{%
\begin{varwidth}[t]{\sphinxcolwidth{1}{8}}
\sphinxAtStartPar
5,7 GB
\par
\vskip-\baselineskip\vbox{\hbox{\strut}}\end{varwidth}%
}%
&\sphinxmultirow{2}{761}{%
\begin{varwidth}[t]{\sphinxcolwidth{1}{8}}
\sphinxAtStartPar
5,7 GB
\par
\vskip-\baselineskip\vbox{\hbox{\strut}}\end{varwidth}%
}%
&\sphinxmultirow{2}{762}{%
\begin{varwidth}[t]{\sphinxcolwidth{1}{8}}
\sphinxAtStartPar
98\%
\par
\vskip-\baselineskip\vbox{\hbox{\strut}}\end{varwidth}%
}%
&
\sphinxAtStartPar
68 MB
\\
\cline{4-4}\cline{8-8}\sphinxtablestrut{756}&\sphinxtablestrut{757}&\sphinxtablestrut{758}&
\sphinxAtStartPar
15
&\sphinxtablestrut{760}&\sphinxtablestrut{761}&\sphinxtablestrut{762}&
\sphinxAtStartPar
78 MB
\\
\hline\sphinxmultirow{2}{766}{%
\begin{varwidth}[t]{\sphinxcolwidth{1}{8}}
\sphinxAtStartPar
70
\par
\vskip-\baselineskip\vbox{\hbox{\strut}}\end{varwidth}%
}%
&\sphinxmultirow{2}{767}{%
\begin{varwidth}[t]{\sphinxcolwidth{1}{8}}
\sphinxAtStartPar
Ori and the Blind Forest: Definitive Edition
\par
\vskip-\baselineskip\vbox{\hbox{\strut}}\end{varwidth}%
}%
&\sphinxmultirow{2}{768}{%
\begin{varwidth}[t]{\sphinxcolwidth{1}{8}}
\sphinxAtStartPar
zstd
\par
\vskip-\baselineskip\vbox{\hbox{\strut}}\end{varwidth}%
}%
&
\sphinxAtStartPar
3
&\sphinxmultirow{2}{770}{%
\begin{varwidth}[t]{\sphinxcolwidth{1}{8}}
\sphinxAtStartPar
10 GB
\par
\vskip-\baselineskip\vbox{\hbox{\strut}}\end{varwidth}%
}%
&
\sphinxAtStartPar
4,9 GB
&
\sphinxAtStartPar
48\%
&
\sphinxAtStartPar
5,3 GB
\\
\cline{4-4}\cline{6-8}\sphinxtablestrut{766}&\sphinxtablestrut{767}&\sphinxtablestrut{768}&
\sphinxAtStartPar
15
&\sphinxtablestrut{770}&
\sphinxAtStartPar
4,7 GB
&
\sphinxAtStartPar
46\%
&
\sphinxAtStartPar
5,5 GB
\\
\hline\sphinxmultirow{2}{778}{%
\begin{varwidth}[t]{\sphinxcolwidth{1}{8}}
\sphinxAtStartPar
71
\par
\vskip-\baselineskip\vbox{\hbox{\strut}}\end{varwidth}%
}%
&\sphinxmultirow{2}{779}{%
\begin{varwidth}[t]{\sphinxcolwidth{1}{8}}
\sphinxAtStartPar
Ori and the Will of the Wisps
\par
\vskip-\baselineskip\vbox{\hbox{\strut}}\end{varwidth}%
}%
&\sphinxmultirow{2}{780}{%
\begin{varwidth}[t]{\sphinxcolwidth{1}{8}}
\sphinxAtStartPar
zstd
\par
\vskip-\baselineskip\vbox{\hbox{\strut}}\end{varwidth}%
}%
&
\sphinxAtStartPar
3
&\sphinxmultirow{2}{782}{%
\begin{varwidth}[t]{\sphinxcolwidth{1}{8}}
\sphinxAtStartPar
11 GB
\par
\vskip-\baselineskip\vbox{\hbox{\strut}}\end{varwidth}%
}%
&
\sphinxAtStartPar
5,5 GB
&
\sphinxAtStartPar
48\%
&
\sphinxAtStartPar
5,8 GB
\\
\cline{4-4}\cline{6-8}\sphinxtablestrut{778}&\sphinxtablestrut{779}&\sphinxtablestrut{780}&
\sphinxAtStartPar
15
&\sphinxtablestrut{782}&
\sphinxAtStartPar
5,3 GB
&
\sphinxAtStartPar
46\%
&
\sphinxAtStartPar
6,1 GB
\\
\hline\sphinxmultirow{2}{790}{%
\begin{varwidth}[t]{\sphinxcolwidth{1}{8}}
\sphinxAtStartPar
72
\par
\vskip-\baselineskip\vbox{\hbox{\strut}}\end{varwidth}%
}%
&\sphinxmultirow{2}{791}{%
\begin{varwidth}[t]{\sphinxcolwidth{1}{8}}
\sphinxAtStartPar
Othercide
\par
\vskip-\baselineskip\vbox{\hbox{\strut}}\end{varwidth}%
}%
&\sphinxmultirow{2}{792}{%
\begin{varwidth}[t]{\sphinxcolwidth{1}{8}}
\sphinxAtStartPar
zstd
\par
\vskip-\baselineskip\vbox{\hbox{\strut}}\end{varwidth}%
}%
&
\sphinxAtStartPar
3
&\sphinxmultirow{2}{794}{%
\begin{varwidth}[t]{\sphinxcolwidth{1}{8}}
\sphinxAtStartPar
6,0 GB
\par
\vskip-\baselineskip\vbox{\hbox{\strut}}\end{varwidth}%
}%
&\sphinxmultirow{2}{795}{%
\begin{varwidth}[t]{\sphinxcolwidth{1}{8}}
\sphinxAtStartPar
5,9 GB
\par
\vskip-\baselineskip\vbox{\hbox{\strut}}\end{varwidth}%
}%
&\sphinxmultirow{2}{796}{%
\begin{varwidth}[t]{\sphinxcolwidth{1}{8}}
\sphinxAtStartPar
98\%
\par
\vskip-\baselineskip\vbox{\hbox{\strut}}\end{varwidth}%
}%
&
\sphinxAtStartPar
94 MB
\\
\cline{4-4}\cline{8-8}\sphinxtablestrut{790}&\sphinxtablestrut{791}&\sphinxtablestrut{792}&
\sphinxAtStartPar
15
&\sphinxtablestrut{794}&\sphinxtablestrut{795}&\sphinxtablestrut{796}&
\sphinxAtStartPar
113 MB
\\
\hline\sphinxmultirow{2}{800}{%
\begin{varwidth}[t]{\sphinxcolwidth{1}{8}}
\sphinxAtStartPar
73
\par
\vskip-\baselineskip\vbox{\hbox{\strut}}\end{varwidth}%
}%
&\sphinxmultirow{2}{801}{%
\begin{varwidth}[t]{\sphinxcolwidth{1}{8}}
\sphinxAtStartPar
Out of Line
\par
\vskip-\baselineskip\vbox{\hbox{\strut}}\end{varwidth}%
}%
&\sphinxmultirow{2}{802}{%
\begin{varwidth}[t]{\sphinxcolwidth{1}{8}}
\sphinxAtStartPar
zstd
\par
\vskip-\baselineskip\vbox{\hbox{\strut}}\end{varwidth}%
}%
&
\sphinxAtStartPar
3
&\sphinxmultirow{2}{804}{%
\begin{varwidth}[t]{\sphinxcolwidth{1}{8}}
\sphinxAtStartPar
1,3 GB
\par
\vskip-\baselineskip\vbox{\hbox{\strut}}\end{varwidth}%
}%
&
\sphinxAtStartPar
497 MB
&
\sphinxAtStartPar
37\%
&
\sphinxAtStartPar
836 MB
\\
\cline{4-4}\cline{6-8}\sphinxtablestrut{800}&\sphinxtablestrut{801}&\sphinxtablestrut{802}&
\sphinxAtStartPar
15
&\sphinxtablestrut{804}&
\sphinxAtStartPar
476 MB
&
\sphinxAtStartPar
35\%
&
\sphinxAtStartPar
857 MB
\\
\hline\sphinxmultirow{2}{812}{%
\begin{varwidth}[t]{\sphinxcolwidth{1}{8}}
\sphinxAtStartPar
74
\par
\vskip-\baselineskip\vbox{\hbox{\strut}}\end{varwidth}%
}%
&\sphinxmultirow{2}{813}{%
\begin{varwidth}[t]{\sphinxcolwidth{1}{8}}
\sphinxAtStartPar
Outland
\par
\vskip-\baselineskip\vbox{\hbox{\strut}}\end{varwidth}%
}%
&\sphinxmultirow{2}{814}{%
\begin{varwidth}[t]{\sphinxcolwidth{1}{8}}
\sphinxAtStartPar
zstd
\par
\vskip-\baselineskip\vbox{\hbox{\strut}}\end{varwidth}%
}%
&
\sphinxAtStartPar
3
&\sphinxmultirow{2}{816}{%
\begin{varwidth}[t]{\sphinxcolwidth{1}{8}}
\sphinxAtStartPar
675 MB
\par
\vskip-\baselineskip\vbox{\hbox{\strut}}\end{varwidth}%
}%
&
\sphinxAtStartPar
593 MB
&\sphinxmultirow{2}{818}{%
\begin{varwidth}[t]{\sphinxcolwidth{1}{8}}
\sphinxAtStartPar
87\%
\par
\vskip-\baselineskip\vbox{\hbox{\strut}}\end{varwidth}%
}%
&
\sphinxAtStartPar
82 MB
\\
\cline{4-4}\cline{6-6}\cline{8-8}\sphinxtablestrut{812}&\sphinxtablestrut{813}&\sphinxtablestrut{814}&
\sphinxAtStartPar
15
&\sphinxtablestrut{816}&
\sphinxAtStartPar
589 MB
&\sphinxtablestrut{818}&
\sphinxAtStartPar
86 MB
\\
\hline\sphinxmultirow{2}{823}{%
\begin{varwidth}[t]{\sphinxcolwidth{1}{8}}
\sphinxAtStartPar
75
\par
\vskip-\baselineskip\vbox{\hbox{\strut}}\end{varwidth}%
}%
&\sphinxmultirow{2}{824}{%
\begin{varwidth}[t]{\sphinxcolwidth{1}{8}}
\sphinxAtStartPar
Overcooked! 2
\par
\vskip-\baselineskip\vbox{\hbox{\strut}}\end{varwidth}%
}%
&\sphinxmultirow{2}{825}{%
\begin{varwidth}[t]{\sphinxcolwidth{1}{8}}
\sphinxAtStartPar
zstd
\par
\vskip-\baselineskip\vbox{\hbox{\strut}}\end{varwidth}%
}%
&
\sphinxAtStartPar
3
&\sphinxmultirow{2}{827}{%
\begin{varwidth}[t]{\sphinxcolwidth{1}{8}}
\sphinxAtStartPar
7,9 GB
\par
\vskip-\baselineskip\vbox{\hbox{\strut}}\end{varwidth}%
}%
&\sphinxmultirow{2}{828}{%
\begin{varwidth}[t]{\sphinxcolwidth{1}{8}}
\sphinxAtStartPar
7,7 GB
\par
\vskip-\baselineskip\vbox{\hbox{\strut}}\end{varwidth}%
}%
&
\sphinxAtStartPar
98\%
&
\sphinxAtStartPar
161 MB
\\
\cline{4-4}\cline{7-8}\sphinxtablestrut{823}&\sphinxtablestrut{824}&\sphinxtablestrut{825}&
\sphinxAtStartPar
15
&\sphinxtablestrut{827}&\sphinxtablestrut{828}&
\sphinxAtStartPar
97\%
&
\sphinxAtStartPar
169 MB
\\
\hline\sphinxmultirow{2}{834}{%
\begin{varwidth}[t]{\sphinxcolwidth{1}{8}}
\sphinxAtStartPar
76
\par
\vskip-\baselineskip\vbox{\hbox{\strut}}\end{varwidth}%
}%
&\sphinxmultirow{2}{835}{%
\begin{varwidth}[t]{\sphinxcolwidth{1}{8}}
\sphinxAtStartPar
Papers, Please
\par
\vskip-\baselineskip\vbox{\hbox{\strut}}\end{varwidth}%
}%
&\sphinxmultirow{2}{836}{%
\begin{varwidth}[t]{\sphinxcolwidth{1}{8}}
\sphinxAtStartPar
zstd
\par
\vskip-\baselineskip\vbox{\hbox{\strut}}\end{varwidth}%
}%
&
\sphinxAtStartPar
3
&\sphinxmultirow{2}{838}{%
\begin{varwidth}[t]{\sphinxcolwidth{1}{8}}
\sphinxAtStartPar
58 MB
\par
\vskip-\baselineskip\vbox{\hbox{\strut}}\end{varwidth}%
}%
&
\sphinxAtStartPar
45 MB
&
\sphinxAtStartPar
77\%
&
\sphinxAtStartPar
13 MB
\\
\cline{4-4}\cline{6-8}\sphinxtablestrut{834}&\sphinxtablestrut{835}&\sphinxtablestrut{836}&
\sphinxAtStartPar
15
&\sphinxtablestrut{838}&
\sphinxAtStartPar
44 MB
&
\sphinxAtStartPar
76\%
&
\sphinxAtStartPar
13,6 MB
\\
\hline\sphinxmultirow{2}{846}{%
\begin{varwidth}[t]{\sphinxcolwidth{1}{8}}
\sphinxAtStartPar
77
\par
\vskip-\baselineskip\vbox{\hbox{\strut}}\end{varwidth}%
}%
&\sphinxmultirow{2}{847}{%
\begin{varwidth}[t]{\sphinxcolwidth{1}{8}}
\sphinxAtStartPar
Path of Exile
\par
\vskip-\baselineskip\vbox{\hbox{\strut}}\end{varwidth}%
}%
&\sphinxmultirow{2}{848}{%
\begin{varwidth}[t]{\sphinxcolwidth{1}{8}}
\sphinxAtStartPar
zstd
\par
\vskip-\baselineskip\vbox{\hbox{\strut}}\end{varwidth}%
}%
&
\sphinxAtStartPar
3
&\sphinxmultirow{2}{850}{%
\begin{varwidth}[t]{\sphinxcolwidth{1}{8}}
\sphinxAtStartPar
27 GB
\par
\vskip-\baselineskip\vbox{\hbox{\strut}}\end{varwidth}%
}%
&\sphinxmultirow{2}{851}{%
\begin{varwidth}[t]{\sphinxcolwidth{1}{8}}
\sphinxAtStartPar
27 GB
\par
\vskip-\baselineskip\vbox{\hbox{\strut}}\end{varwidth}%
}%
&\sphinxmultirow{2}{852}{%
\begin{varwidth}[t]{\sphinxcolwidth{1}{8}}
\sphinxAtStartPar
99\%
\par
\vskip-\baselineskip\vbox{\hbox{\strut}}\end{varwidth}%
}%
&
\sphinxAtStartPar
27 MB
\\
\cline{4-4}\cline{8-8}\sphinxtablestrut{846}&\sphinxtablestrut{847}&\sphinxtablestrut{848}&
\sphinxAtStartPar
15
&\sphinxtablestrut{850}&\sphinxtablestrut{851}&\sphinxtablestrut{852}&
\sphinxAtStartPar
29 MB
\\
\hline\sphinxmultirow{2}{856}{%
\begin{varwidth}[t]{\sphinxcolwidth{1}{8}}
\sphinxAtStartPar
78
\par
\vskip-\baselineskip\vbox{\hbox{\strut}}\end{varwidth}%
}%
&\sphinxmultirow{2}{857}{%
\begin{varwidth}[t]{\sphinxcolwidth{1}{8}}
\sphinxAtStartPar
Peace, Death!
\par
\vskip-\baselineskip\vbox{\hbox{\strut}}\end{varwidth}%
}%
&\sphinxmultirow{2}{858}{%
\begin{varwidth}[t]{\sphinxcolwidth{1}{8}}
\sphinxAtStartPar
zstd
\par
\vskip-\baselineskip\vbox{\hbox{\strut}}\end{varwidth}%
}%
&
\sphinxAtStartPar
3
&\sphinxmultirow{2}{860}{%
\begin{varwidth}[t]{\sphinxcolwidth{1}{8}}
\sphinxAtStartPar
83 MB
\par
\vskip-\baselineskip\vbox{\hbox{\strut}}\end{varwidth}%
}%
&\sphinxmultirow{2}{861}{%
\begin{varwidth}[t]{\sphinxcolwidth{1}{8}}
\sphinxAtStartPar
76 MB
\par
\vskip-\baselineskip\vbox{\hbox{\strut}}\end{varwidth}%
}%
&\sphinxmultirow{2}{862}{%
\begin{varwidth}[t]{\sphinxcolwidth{1}{8}}
\sphinxAtStartPar
91\%
\par
\vskip-\baselineskip\vbox{\hbox{\strut}}\end{varwidth}%
}%
&
\sphinxAtStartPar
7,2 MB
\\
\cline{4-4}\cline{8-8}\sphinxtablestrut{856}&\sphinxtablestrut{857}&\sphinxtablestrut{858}&
\sphinxAtStartPar
15
&\sphinxtablestrut{860}&\sphinxtablestrut{861}&\sphinxtablestrut{862}&
\sphinxAtStartPar
7,5 MB
\\
\hline\sphinxmultirow{2}{866}{%
\begin{varwidth}[t]{\sphinxcolwidth{1}{8}}
\sphinxAtStartPar
79
\par
\vskip-\baselineskip\vbox{\hbox{\strut}}\end{varwidth}%
}%
&\sphinxmultirow{2}{867}{%
\begin{varwidth}[t]{\sphinxcolwidth{1}{8}}
\sphinxAtStartPar
Peace, Death! 2
\par
\vskip-\baselineskip\vbox{\hbox{\strut}}\end{varwidth}%
}%
&\sphinxmultirow{2}{868}{%
\begin{varwidth}[t]{\sphinxcolwidth{1}{8}}
\sphinxAtStartPar
zstd
\par
\vskip-\baselineskip\vbox{\hbox{\strut}}\end{varwidth}%
}%
&
\sphinxAtStartPar
3
&\sphinxmultirow{2}{870}{%
\begin{varwidth}[t]{\sphinxcolwidth{1}{8}}
\sphinxAtStartPar
34 MB
\par
\vskip-\baselineskip\vbox{\hbox{\strut}}\end{varwidth}%
}%
&\sphinxmultirow{2}{871}{%
\begin{varwidth}[t]{\sphinxcolwidth{1}{8}}
\sphinxAtStartPar
26 MB
\par
\vskip-\baselineskip\vbox{\hbox{\strut}}\end{varwidth}%
}%
&\sphinxmultirow{2}{872}{%
\begin{varwidth}[t]{\sphinxcolwidth{1}{8}}
\sphinxAtStartPar
78\%
\par
\vskip-\baselineskip\vbox{\hbox{\strut}}\end{varwidth}%
}%
&
\sphinxAtStartPar
7,04 MB
\\
\cline{4-4}\cline{8-8}\sphinxtablestrut{866}&\sphinxtablestrut{867}&\sphinxtablestrut{868}&
\sphinxAtStartPar
15
&\sphinxtablestrut{870}&\sphinxtablestrut{871}&\sphinxtablestrut{872}&
\sphinxAtStartPar
7,51 MB
\\
\hline\sphinxmultirow{2}{876}{%
\begin{varwidth}[t]{\sphinxcolwidth{1}{8}}
\sphinxAtStartPar
80
\par
\vskip-\baselineskip\vbox{\hbox{\strut}}\end{varwidth}%
}%
&\sphinxmultirow{2}{877}{%
\begin{varwidth}[t]{\sphinxcolwidth{1}{8}}
\sphinxAtStartPar
Pummel Party
\par
\vskip-\baselineskip\vbox{\hbox{\strut}}\end{varwidth}%
}%
&\sphinxmultirow{2}{878}{%
\begin{varwidth}[t]{\sphinxcolwidth{1}{8}}
\sphinxAtStartPar
zstd
\par
\vskip-\baselineskip\vbox{\hbox{\strut}}\end{varwidth}%
}%
&
\sphinxAtStartPar
3
&\sphinxmultirow{2}{880}{%
\begin{varwidth}[t]{\sphinxcolwidth{1}{8}}
\sphinxAtStartPar
2,1 GB
\par
\vskip-\baselineskip\vbox{\hbox{\strut}}\end{varwidth}%
}%
&\sphinxmultirow{2}{881}{%
\begin{varwidth}[t]{\sphinxcolwidth{1}{8}}
\sphinxAtStartPar
1,4 GB
\par
\vskip-\baselineskip\vbox{\hbox{\strut}}\end{varwidth}%
}%
&
\sphinxAtStartPar
67\%
&
\sphinxAtStartPar
712 MB
\\
\cline{4-4}\cline{7-8}\sphinxtablestrut{876}&\sphinxtablestrut{877}&\sphinxtablestrut{878}&
\sphinxAtStartPar
15
&\sphinxtablestrut{880}&\sphinxtablestrut{881}&
\sphinxAtStartPar
66\%
&
\sphinxAtStartPar
723 MB
\\
\hline\sphinxmultirow{2}{887}{%
\begin{varwidth}[t]{\sphinxcolwidth{1}{8}}
\sphinxAtStartPar
81
\par
\vskip-\baselineskip\vbox{\hbox{\strut}}\end{varwidth}%
}%
&\sphinxmultirow{2}{888}{%
\begin{varwidth}[t]{\sphinxcolwidth{1}{8}}
\sphinxAtStartPar
Remember Me
\par
\vskip-\baselineskip\vbox{\hbox{\strut}}\end{varwidth}%
}%
&\sphinxmultirow{2}{889}{%
\begin{varwidth}[t]{\sphinxcolwidth{1}{8}}
\sphinxAtStartPar
zstd
\par
\vskip-\baselineskip\vbox{\hbox{\strut}}\end{varwidth}%
}%
&
\sphinxAtStartPar
3
&\sphinxmultirow{2}{891}{%
\begin{varwidth}[t]{\sphinxcolwidth{1}{8}}
\sphinxAtStartPar
6,7 GB
\par
\vskip-\baselineskip\vbox{\hbox{\strut}}\end{varwidth}%
}%
&\sphinxmultirow{2}{892}{%
\begin{varwidth}[t]{\sphinxcolwidth{1}{8}}
\sphinxAtStartPar
6,6 GB
\par
\vskip-\baselineskip\vbox{\hbox{\strut}}\end{varwidth}%
}%
&\sphinxmultirow{2}{893}{%
\begin{varwidth}[t]{\sphinxcolwidth{1}{8}}
\sphinxAtStartPar
99\%
\par
\vskip-\baselineskip\vbox{\hbox{\strut}}\end{varwidth}%
}%
&
\sphinxAtStartPar
57 MB
\\
\cline{4-4}\cline{8-8}\sphinxtablestrut{887}&\sphinxtablestrut{888}&\sphinxtablestrut{889}&
\sphinxAtStartPar
15
&\sphinxtablestrut{891}&\sphinxtablestrut{892}&\sphinxtablestrut{893}&
\sphinxAtStartPar
58 MB
\\
\hline\sphinxmultirow{2}{897}{%
\begin{varwidth}[t]{\sphinxcolwidth{1}{8}}
\sphinxAtStartPar
82
\par
\vskip-\baselineskip\vbox{\hbox{\strut}}\end{varwidth}%
}%
&\sphinxmultirow{2}{898}{%
\begin{varwidth}[t]{\sphinxcolwidth{1}{8}}
\sphinxAtStartPar
Rocket League
\par
\vskip-\baselineskip\vbox{\hbox{\strut}}\end{varwidth}%
}%
&\sphinxmultirow{2}{899}{%
\begin{varwidth}[t]{\sphinxcolwidth{1}{8}}
\sphinxAtStartPar
zstd
\par
\vskip-\baselineskip\vbox{\hbox{\strut}}\end{varwidth}%
}%
&
\sphinxAtStartPar
3
&\sphinxmultirow{2}{901}{%
\begin{varwidth}[t]{\sphinxcolwidth{1}{8}}
\sphinxAtStartPar
18 GB
\par
\vskip-\baselineskip\vbox{\hbox{\strut}}\end{varwidth}%
}%
&\sphinxmultirow{2}{902}{%
\begin{varwidth}[t]{\sphinxcolwidth{1}{8}}
\sphinxAtStartPar
18 GB
\par
\vskip-\baselineskip\vbox{\hbox{\strut}}\end{varwidth}%
}%
&\sphinxmultirow{2}{903}{%
\begin{varwidth}[t]{\sphinxcolwidth{1}{8}}
\sphinxAtStartPar
99\%
\par
\vskip-\baselineskip\vbox{\hbox{\strut}}\end{varwidth}%
}%
&
\sphinxAtStartPar
20 MB
\\
\cline{4-4}\cline{8-8}\sphinxtablestrut{897}&\sphinxtablestrut{898}&\sphinxtablestrut{899}&
\sphinxAtStartPar
15
&\sphinxtablestrut{901}&\sphinxtablestrut{902}&\sphinxtablestrut{903}&
\sphinxAtStartPar
46 MB
\\
\hline\sphinxmultirow{2}{907}{%
\begin{varwidth}[t]{\sphinxcolwidth{1}{8}}
\sphinxAtStartPar
83
\par
\vskip-\baselineskip\vbox{\hbox{\strut}}\end{varwidth}%
}%
&\sphinxmultirow{2}{908}{%
\begin{varwidth}[t]{\sphinxcolwidth{1}{8}}
\sphinxAtStartPar
RUINER
\par
\vskip-\baselineskip\vbox{\hbox{\strut}}\end{varwidth}%
}%
&\sphinxmultirow{2}{909}{%
\begin{varwidth}[t]{\sphinxcolwidth{1}{8}}
\sphinxAtStartPar
zstd
\par
\vskip-\baselineskip\vbox{\hbox{\strut}}\end{varwidth}%
}%
&
\sphinxAtStartPar
3
&\sphinxmultirow{2}{911}{%
\begin{varwidth}[t]{\sphinxcolwidth{1}{8}}
\sphinxAtStartPar
10 GB
\par
\vskip-\baselineskip\vbox{\hbox{\strut}}\end{varwidth}%
}%
&\sphinxmultirow{2}{912}{%
\begin{varwidth}[t]{\sphinxcolwidth{1}{8}}
\sphinxAtStartPar
10 GB
\par
\vskip-\baselineskip\vbox{\hbox{\strut}}\end{varwidth}%
}%
&\sphinxmultirow{2}{913}{%
\begin{varwidth}[t]{\sphinxcolwidth{1}{8}}
\sphinxAtStartPar
99\%
\par
\vskip-\baselineskip\vbox{\hbox{\strut}}\end{varwidth}%
}%
&
\sphinxAtStartPar
50 MB
\\
\cline{4-4}\cline{8-8}\sphinxtablestrut{907}&\sphinxtablestrut{908}&\sphinxtablestrut{909}&
\sphinxAtStartPar
15
&\sphinxtablestrut{911}&\sphinxtablestrut{912}&\sphinxtablestrut{913}&
\sphinxAtStartPar
77 MB
\\
\hline\sphinxmultirow{2}{917}{%
\begin{varwidth}[t]{\sphinxcolwidth{1}{8}}
\sphinxAtStartPar
84
\par
\vskip-\baselineskip\vbox{\hbox{\strut}}\end{varwidth}%
}%
&\sphinxmultirow{2}{918}{%
\begin{varwidth}[t]{\sphinxcolwidth{1}{8}}
\sphinxAtStartPar
Salt and Sanctuary
\par
\vskip-\baselineskip\vbox{\hbox{\strut}}\end{varwidth}%
}%
&\sphinxmultirow{2}{919}{%
\begin{varwidth}[t]{\sphinxcolwidth{1}{8}}
\sphinxAtStartPar
zstd
\par
\vskip-\baselineskip\vbox{\hbox{\strut}}\end{varwidth}%
}%
&
\sphinxAtStartPar
3
&\sphinxmultirow{2}{921}{%
\begin{varwidth}[t]{\sphinxcolwidth{1}{8}}
\sphinxAtStartPar
563 MB
\par
\vskip-\baselineskip\vbox{\hbox{\strut}}\end{varwidth}%
}%
&\sphinxmultirow{2}{922}{%
\begin{varwidth}[t]{\sphinxcolwidth{1}{8}}
\sphinxAtStartPar
540 MB
\par
\vskip-\baselineskip\vbox{\hbox{\strut}}\end{varwidth}%
}%
&\sphinxmultirow{2}{923}{%
\begin{varwidth}[t]{\sphinxcolwidth{1}{8}}
\sphinxAtStartPar
95\%
\par
\vskip-\baselineskip\vbox{\hbox{\strut}}\end{varwidth}%
}%
&
\sphinxAtStartPar
23 MB
\\
\cline{4-4}\cline{8-8}\sphinxtablestrut{917}&\sphinxtablestrut{918}&\sphinxtablestrut{919}&
\sphinxAtStartPar
15
&\sphinxtablestrut{921}&\sphinxtablestrut{922}&\sphinxtablestrut{923}&
\sphinxAtStartPar
24 MB
\\
\hline\sphinxmultirow{2}{927}{%
\begin{varwidth}[t]{\sphinxcolwidth{1}{8}}
\sphinxAtStartPar
85
\par
\vskip-\baselineskip\vbox{\hbox{\strut}}\end{varwidth}%
}%
&\sphinxmultirow{2}{928}{%
\begin{varwidth}[t]{\sphinxcolwidth{1}{8}}
\sphinxAtStartPar
Samorost 1
\par
\vskip-\baselineskip\vbox{\hbox{\strut}}\end{varwidth}%
}%
&\sphinxmultirow{2}{929}{%
\begin{varwidth}[t]{\sphinxcolwidth{1}{8}}
\sphinxAtStartPar
zstd
\par
\vskip-\baselineskip\vbox{\hbox{\strut}}\end{varwidth}%
}%
&
\sphinxAtStartPar
3
&\sphinxmultirow{2}{931}{%
\begin{varwidth}[t]{\sphinxcolwidth{1}{8}}
\sphinxAtStartPar
68 MB
\par
\vskip-\baselineskip\vbox{\hbox{\strut}}\end{varwidth}%
}%
&\sphinxmultirow{2}{932}{%
\begin{varwidth}[t]{\sphinxcolwidth{1}{8}}
\sphinxAtStartPar
68 MB
\par
\vskip-\baselineskip\vbox{\hbox{\strut}}\end{varwidth}%
}%
&\sphinxmultirow{2}{933}{%
\begin{varwidth}[t]{\sphinxcolwidth{1}{8}}
\sphinxAtStartPar
99\%
\par
\vskip-\baselineskip\vbox{\hbox{\strut}}\end{varwidth}%
}%
&
\sphinxAtStartPar
19 KB
\\
\cline{4-4}\cline{8-8}\sphinxtablestrut{927}&\sphinxtablestrut{928}&\sphinxtablestrut{929}&
\sphinxAtStartPar
15
&\sphinxtablestrut{931}&\sphinxtablestrut{932}&\sphinxtablestrut{933}&
\sphinxAtStartPar
23 KB
\\
\hline\sphinxmultirow{2}{937}{%
\begin{varwidth}[t]{\sphinxcolwidth{1}{8}}
\sphinxAtStartPar
86
\par
\vskip-\baselineskip\vbox{\hbox{\strut}}\end{varwidth}%
}%
&\sphinxmultirow{2}{938}{%
\begin{varwidth}[t]{\sphinxcolwidth{1}{8}}
\sphinxAtStartPar
Samorost 2
\par
\vskip-\baselineskip\vbox{\hbox{\strut}}\end{varwidth}%
}%
&\sphinxmultirow{2}{939}{%
\begin{varwidth}[t]{\sphinxcolwidth{1}{8}}
\sphinxAtStartPar
zstd
\par
\vskip-\baselineskip\vbox{\hbox{\strut}}\end{varwidth}%
}%
&
\sphinxAtStartPar
3
&\sphinxmultirow{2}{941}{%
\begin{varwidth}[t]{\sphinxcolwidth{1}{8}}
\sphinxAtStartPar
141 MB
\par
\vskip-\baselineskip\vbox{\hbox{\strut}}\end{varwidth}%
}%
&
\sphinxAtStartPar
141 MB
&
\sphinxAtStartPar
99\%
&
\sphinxAtStartPar
1,22 MB
\\
\cline{4-4}\cline{6-8}\sphinxtablestrut{937}&\sphinxtablestrut{938}&\sphinxtablestrut{939}&
\sphinxAtStartPar
15
&\sphinxtablestrut{941}&
\sphinxAtStartPar
140 MB
&
\sphinxAtStartPar
98\%
&
\sphinxAtStartPar
1,33 MB
\\
\hline\sphinxmultirow{2}{949}{%
\begin{varwidth}[t]{\sphinxcolwidth{1}{8}}
\sphinxAtStartPar
87
\par
\vskip-\baselineskip\vbox{\hbox{\strut}}\end{varwidth}%
}%
&\sphinxmultirow{2}{950}{%
\begin{varwidth}[t]{\sphinxcolwidth{1}{8}}
\sphinxAtStartPar
Samorost 3
\par
\vskip-\baselineskip\vbox{\hbox{\strut}}\end{varwidth}%
}%
&\sphinxmultirow{2}{951}{%
\begin{varwidth}[t]{\sphinxcolwidth{1}{8}}
\sphinxAtStartPar
zstd
\par
\vskip-\baselineskip\vbox{\hbox{\strut}}\end{varwidth}%
}%
&
\sphinxAtStartPar
3
&\sphinxmultirow{2}{953}{%
\begin{varwidth}[t]{\sphinxcolwidth{1}{8}}
\sphinxAtStartPar
1,1 GB
\par
\vskip-\baselineskip\vbox{\hbox{\strut}}\end{varwidth}%
}%
&\sphinxmultirow{2}{954}{%
\begin{varwidth}[t]{\sphinxcolwidth{1}{8}}
\sphinxAtStartPar
1,0 GB
\par
\vskip-\baselineskip\vbox{\hbox{\strut}}\end{varwidth}%
}%
&
\sphinxAtStartPar
99\%
&
\sphinxAtStartPar
9,5 MB
\\
\cline{4-4}\cline{7-8}\sphinxtablestrut{949}&\sphinxtablestrut{950}&\sphinxtablestrut{951}&
\sphinxAtStartPar
15
&\sphinxtablestrut{953}&\sphinxtablestrut{954}&
\sphinxAtStartPar
96\%
&
\sphinxAtStartPar
43 MB
\\
\hline\sphinxmultirow{2}{960}{%
\begin{varwidth}[t]{\sphinxcolwidth{1}{8}}
\sphinxAtStartPar
88
\par
\vskip-\baselineskip\vbox{\hbox{\strut}}\end{varwidth}%
}%
&\sphinxmultirow{2}{961}{%
\begin{varwidth}[t]{\sphinxcolwidth{1}{8}}
\sphinxAtStartPar
Sekiro: Shadow Die Twice
\par
\vskip-\baselineskip\vbox{\hbox{\strut}}\end{varwidth}%
}%
&\sphinxmultirow{2}{962}{%
\begin{varwidth}[t]{\sphinxcolwidth{1}{8}}
\sphinxAtStartPar
zstd
\par
\vskip-\baselineskip\vbox{\hbox{\strut}}\end{varwidth}%
}%
&
\sphinxAtStartPar
3
&\sphinxmultirow{2}{964}{%
\begin{varwidth}[t]{\sphinxcolwidth{1}{8}}
\sphinxAtStartPar
13 GB
\par
\vskip-\baselineskip\vbox{\hbox{\strut}}\end{varwidth}%
}%
&\sphinxmultirow{2}{965}{%
\begin{varwidth}[t]{\sphinxcolwidth{1}{8}}
\sphinxAtStartPar
13 GB
\par
\vskip-\baselineskip\vbox{\hbox{\strut}}\end{varwidth}%
}%
&\sphinxmultirow{2}{966}{%
\begin{varwidth}[t]{\sphinxcolwidth{1}{8}}
\sphinxAtStartPar
99\%
\par
\vskip-\baselineskip\vbox{\hbox{\strut}}\end{varwidth}%
}%
&
\sphinxAtStartPar
1,5 MB
\\
\cline{4-4}\cline{8-8}\sphinxtablestrut{960}&\sphinxtablestrut{961}&\sphinxtablestrut{962}&
\sphinxAtStartPar
15
&\sphinxtablestrut{964}&\sphinxtablestrut{965}&\sphinxtablestrut{966}&
\sphinxAtStartPar
1,6 MB
\\
\hline\sphinxmultirow{2}{970}{%
\begin{varwidth}[t]{\sphinxcolwidth{1}{8}}
\sphinxAtStartPar
89
\par
\vskip-\baselineskip\vbox{\hbox{\strut}}\end{varwidth}%
}%
&\sphinxmultirow{2}{971}{%
\begin{varwidth}[t]{\sphinxcolwidth{1}{8}}
\sphinxAtStartPar
Severed Steel
\par
\vskip-\baselineskip\vbox{\hbox{\strut}}\end{varwidth}%
}%
&\sphinxmultirow{2}{972}{%
\begin{varwidth}[t]{\sphinxcolwidth{1}{8}}
\sphinxAtStartPar
zstd
\par
\vskip-\baselineskip\vbox{\hbox{\strut}}\end{varwidth}%
}%
&
\sphinxAtStartPar
3
&\sphinxmultirow{2}{974}{%
\begin{varwidth}[t]{\sphinxcolwidth{1}{8}}
\sphinxAtStartPar
4,0 GB
\par
\vskip-\baselineskip\vbox{\hbox{\strut}}\end{varwidth}%
}%
&\sphinxmultirow{2}{975}{%
\begin{varwidth}[t]{\sphinxcolwidth{1}{8}}
\sphinxAtStartPar
2,7 GB
\par
\vskip-\baselineskip\vbox{\hbox{\strut}}\end{varwidth}%
}%
&
\sphinxAtStartPar
68\%
&
\sphinxAtStartPar
1,22 GB
\\
\cline{4-4}\cline{7-8}\sphinxtablestrut{970}&\sphinxtablestrut{971}&\sphinxtablestrut{972}&
\sphinxAtStartPar
15
&\sphinxtablestrut{974}&\sphinxtablestrut{975}&
\sphinxAtStartPar
67\%
&
\sphinxAtStartPar
1,26 GB
\\
\hline\sphinxmultirow{2}{981}{%
\begin{varwidth}[t]{\sphinxcolwidth{1}{8}}
\sphinxAtStartPar
90
\par
\vskip-\baselineskip\vbox{\hbox{\strut}}\end{varwidth}%
}%
&\sphinxmultirow{2}{982}{%
\begin{varwidth}[t]{\sphinxcolwidth{1}{8}}
\sphinxAtStartPar
Shadow Tactics: Blades of the Shogun
\par
\vskip-\baselineskip\vbox{\hbox{\strut}}\end{varwidth}%
}%
&\sphinxmultirow{2}{983}{%
\begin{varwidth}[t]{\sphinxcolwidth{1}{8}}
\sphinxAtStartPar
zstd
\par
\vskip-\baselineskip\vbox{\hbox{\strut}}\end{varwidth}%
}%
&
\sphinxAtStartPar
3
&\sphinxmultirow{2}{985}{%
\begin{varwidth}[t]{\sphinxcolwidth{1}{8}}
\sphinxAtStartPar
7,3 GB
\par
\vskip-\baselineskip\vbox{\hbox{\strut}}\end{varwidth}%
}%
&
\sphinxAtStartPar
5,0 GB
&
\sphinxAtStartPar
69\%
&
\sphinxAtStartPar
2,2 GB
\\
\cline{4-4}\cline{6-8}\sphinxtablestrut{981}&\sphinxtablestrut{982}&\sphinxtablestrut{983}&
\sphinxAtStartPar
15
&\sphinxtablestrut{985}&
\sphinxAtStartPar
4,8 GB
&
\sphinxAtStartPar
66\%
&
\sphinxAtStartPar
2,5 GB
\\
\hline\sphinxmultirow{2}{993}{%
\begin{varwidth}[t]{\sphinxcolwidth{1}{8}}
\sphinxAtStartPar
91
\par
\vskip-\baselineskip\vbox{\hbox{\strut}}\end{varwidth}%
}%
&\sphinxmultirow{2}{994}{%
\begin{varwidth}[t]{\sphinxcolwidth{1}{8}}
\sphinxAtStartPar
Shadowrun Returns
\par
\vskip-\baselineskip\vbox{\hbox{\strut}}\end{varwidth}%
}%
&\sphinxmultirow{2}{995}{%
\begin{varwidth}[t]{\sphinxcolwidth{1}{8}}
\sphinxAtStartPar
zstd
\par
\vskip-\baselineskip\vbox{\hbox{\strut}}\end{varwidth}%
}%
&
\sphinxAtStartPar
3
&\sphinxmultirow{2}{997}{%
\begin{varwidth}[t]{\sphinxcolwidth{1}{8}}
\sphinxAtStartPar
2,8 GB
\par
\vskip-\baselineskip\vbox{\hbox{\strut}}\end{varwidth}%
}%
&
\sphinxAtStartPar
1,1 GB
&
\sphinxAtStartPar
39\%
&
\sphinxAtStartPar
1,68 GB
\\
\cline{4-4}\cline{6-8}\sphinxtablestrut{993}&\sphinxtablestrut{994}&\sphinxtablestrut{995}&
\sphinxAtStartPar
15
&\sphinxtablestrut{997}&
\sphinxAtStartPar
1,0 GB
&
\sphinxAtStartPar
37\%
&
\sphinxAtStartPar
1,74 GB
\\
\hline\sphinxmultirow{2}{1005}{%
\begin{varwidth}[t]{\sphinxcolwidth{1}{8}}
\sphinxAtStartPar
92
\par
\vskip-\baselineskip\vbox{\hbox{\strut}}\end{varwidth}%
}%
&\sphinxmultirow{2}{1006}{%
\begin{varwidth}[t]{\sphinxcolwidth{1}{8}}
\sphinxAtStartPar
Shattered \sphinxhyphen{} Tale of the Forgotten King
\par
\vskip-\baselineskip\vbox{\hbox{\strut}}\end{varwidth}%
}%
&\sphinxmultirow{2}{1007}{%
\begin{varwidth}[t]{\sphinxcolwidth{1}{8}}
\sphinxAtStartPar
zstd
\par
\vskip-\baselineskip\vbox{\hbox{\strut}}\end{varwidth}%
}%
&
\sphinxAtStartPar
3
&\sphinxmultirow{2}{1009}{%
\begin{varwidth}[t]{\sphinxcolwidth{1}{8}}
\sphinxAtStartPar
6,3 GB
\par
\vskip-\baselineskip\vbox{\hbox{\strut}}\end{varwidth}%
}%
&\sphinxmultirow{2}{1010}{%
\begin{varwidth}[t]{\sphinxcolwidth{1}{8}}
\sphinxAtStartPar
6,3 GB
\par
\vskip-\baselineskip\vbox{\hbox{\strut}}\end{varwidth}%
}%
&\sphinxmultirow{2}{1011}{%
\begin{varwidth}[t]{\sphinxcolwidth{1}{8}}
\sphinxAtStartPar
99\%
\par
\vskip-\baselineskip\vbox{\hbox{\strut}}\end{varwidth}%
}%
&
\sphinxAtStartPar
14,7 MB
\\
\cline{4-4}\cline{8-8}\sphinxtablestrut{1005}&\sphinxtablestrut{1006}&\sphinxtablestrut{1007}&
\sphinxAtStartPar
15
&\sphinxtablestrut{1009}&\sphinxtablestrut{1010}&\sphinxtablestrut{1011}&
\sphinxAtStartPar
15,7 MB
\\
\hline\sphinxmultirow{2}{1015}{%
\begin{varwidth}[t]{\sphinxcolwidth{1}{8}}
\sphinxAtStartPar
93
\par
\vskip-\baselineskip\vbox{\hbox{\strut}}\end{varwidth}%
}%
&\sphinxmultirow{2}{1016}{%
\begin{varwidth}[t]{\sphinxcolwidth{1}{8}}
\sphinxAtStartPar
Shiro
\par
\vskip-\baselineskip\vbox{\hbox{\strut}}\end{varwidth}%
}%
&\sphinxmultirow{2}{1017}{%
\begin{varwidth}[t]{\sphinxcolwidth{1}{8}}
\sphinxAtStartPar
zstd
\par
\vskip-\baselineskip\vbox{\hbox{\strut}}\end{varwidth}%
}%
&
\sphinxAtStartPar
3
&\sphinxmultirow{2}{1019}{%
\begin{varwidth}[t]{\sphinxcolwidth{1}{8}}
\sphinxAtStartPar
80 MB
\par
\vskip-\baselineskip\vbox{\hbox{\strut}}\end{varwidth}%
}%
&
\sphinxAtStartPar
74 MB
&\sphinxmultirow{2}{1021}{%
\begin{varwidth}[t]{\sphinxcolwidth{1}{8}}
\sphinxAtStartPar
91\%
\par
\vskip-\baselineskip\vbox{\hbox{\strut}}\end{varwidth}%
}%
&
\sphinxAtStartPar
6,5 MB
\\
\cline{4-4}\cline{6-6}\cline{8-8}\sphinxtablestrut{1015}&\sphinxtablestrut{1016}&\sphinxtablestrut{1017}&
\sphinxAtStartPar
15
&\sphinxtablestrut{1019}&
\sphinxAtStartPar
73 MB
&\sphinxtablestrut{1021}&
\sphinxAtStartPar
6,7 MB
\\
\hline\sphinxmultirow{2}{1026}{%
\begin{varwidth}[t]{\sphinxcolwidth{1}{8}}
\sphinxAtStartPar
94
\par
\vskip-\baselineskip\vbox{\hbox{\strut}}\end{varwidth}%
}%
&\sphinxmultirow{2}{1027}{%
\begin{varwidth}[t]{\sphinxcolwidth{1}{8}}
\sphinxAtStartPar
Skul: The Hero Slayer
\par
\vskip-\baselineskip\vbox{\hbox{\strut}}\end{varwidth}%
}%
&\sphinxmultirow{2}{1028}{%
\begin{varwidth}[t]{\sphinxcolwidth{1}{8}}
\sphinxAtStartPar
zstd
\par
\vskip-\baselineskip\vbox{\hbox{\strut}}\end{varwidth}%
}%
&
\sphinxAtStartPar
3
&\sphinxmultirow{2}{1030}{%
\begin{varwidth}[t]{\sphinxcolwidth{1}{8}}
\sphinxAtStartPar
1016 MB
\par
\vskip-\baselineskip\vbox{\hbox{\strut}}\end{varwidth}%
}%
&
\sphinxAtStartPar
1001 MB
&
\sphinxAtStartPar
98\%
&
\sphinxAtStartPar
14,5 MB
\\
\cline{4-4}\cline{6-8}\sphinxtablestrut{1026}&\sphinxtablestrut{1027}&\sphinxtablestrut{1028}&
\sphinxAtStartPar
15
&\sphinxtablestrut{1030}&
\sphinxAtStartPar
987 MB
&
\sphinxAtStartPar
97\%
&
\sphinxAtStartPar
29 MB
\\
\hline\sphinxmultirow{2}{1038}{%
\begin{varwidth}[t]{\sphinxcolwidth{1}{8}}
\sphinxAtStartPar
95
\par
\vskip-\baselineskip\vbox{\hbox{\strut}}\end{varwidth}%
}%
&\sphinxmultirow{2}{1039}{%
\begin{varwidth}[t]{\sphinxcolwidth{1}{8}}
\sphinxAtStartPar
SpeedRunners
\par
\vskip-\baselineskip\vbox{\hbox{\strut}}\end{varwidth}%
}%
&\sphinxmultirow{2}{1040}{%
\begin{varwidth}[t]{\sphinxcolwidth{1}{8}}
\sphinxAtStartPar
zstd
\par
\vskip-\baselineskip\vbox{\hbox{\strut}}\end{varwidth}%
}%
&
\sphinxAtStartPar
3
&\sphinxmultirow{2}{1042}{%
\begin{varwidth}[t]{\sphinxcolwidth{1}{8}}
\sphinxAtStartPar
662 MB
\par
\vskip-\baselineskip\vbox{\hbox{\strut}}\end{varwidth}%
}%
&
\sphinxAtStartPar
651 MB
&\sphinxmultirow{2}{1044}{%
\begin{varwidth}[t]{\sphinxcolwidth{1}{8}}
\sphinxAtStartPar
98\%
\par
\vskip-\baselineskip\vbox{\hbox{\strut}}\end{varwidth}%
}%
&
\sphinxAtStartPar
11 MB
\\
\cline{4-4}\cline{6-6}\cline{8-8}\sphinxtablestrut{1038}&\sphinxtablestrut{1039}&\sphinxtablestrut{1040}&
\sphinxAtStartPar
15
&\sphinxtablestrut{1042}&
\sphinxAtStartPar
650 MB
&\sphinxtablestrut{1044}&
\sphinxAtStartPar
12 MB
\\
\hline\sphinxmultirow{2}{1049}{%
\begin{varwidth}[t]{\sphinxcolwidth{1}{8}}
\sphinxAtStartPar
96
\par
\vskip-\baselineskip\vbox{\hbox{\strut}}\end{varwidth}%
}%
&\sphinxmultirow{2}{1050}{%
\begin{varwidth}[t]{\sphinxcolwidth{1}{8}}
\sphinxAtStartPar
Spiritfarer: Farewell
\par
\vskip-\baselineskip\vbox{\hbox{\strut}}\end{varwidth}%
}%
&\sphinxmultirow{2}{1051}{%
\begin{varwidth}[t]{\sphinxcolwidth{1}{8}}
\sphinxAtStartPar
zstd
\par
\vskip-\baselineskip\vbox{\hbox{\strut}}\end{varwidth}%
}%
&
\sphinxAtStartPar
3
&\sphinxmultirow{2}{1053}{%
\begin{varwidth}[t]{\sphinxcolwidth{1}{8}}
\sphinxAtStartPar
6,0 GB
\par
\vskip-\baselineskip\vbox{\hbox{\strut}}\end{varwidth}%
}%
&\sphinxmultirow{2}{1054}{%
\begin{varwidth}[t]{\sphinxcolwidth{1}{8}}
\sphinxAtStartPar
2,3 GB
\par
\vskip-\baselineskip\vbox{\hbox{\strut}}\end{varwidth}%
}%
&
\sphinxAtStartPar
38\%
&
\sphinxAtStartPar
3,60 GB
\\
\cline{4-4}\cline{7-8}\sphinxtablestrut{1049}&\sphinxtablestrut{1050}&\sphinxtablestrut{1051}&
\sphinxAtStartPar
15
&\sphinxtablestrut{1053}&\sphinxtablestrut{1054}&
\sphinxAtStartPar
39\%
&
\sphinxAtStartPar
3,58 GB
\\
\hline\sphinxmultirow{2}{1060}{%
\begin{varwidth}[t]{\sphinxcolwidth{1}{8}}
\sphinxAtStartPar
97
\par
\vskip-\baselineskip\vbox{\hbox{\strut}}\end{varwidth}%
}%
&\sphinxmultirow{2}{1061}{%
\begin{varwidth}[t]{\sphinxcolwidth{1}{8}}
\sphinxAtStartPar
Stoneshard: Prologue
\par
\vskip-\baselineskip\vbox{\hbox{\strut}}\end{varwidth}%
}%
&\sphinxmultirow{2}{1062}{%
\begin{varwidth}[t]{\sphinxcolwidth{1}{8}}
\sphinxAtStartPar
zstd
\par
\vskip-\baselineskip\vbox{\hbox{\strut}}\end{varwidth}%
}%
&
\sphinxAtStartPar
3
&\sphinxmultirow{2}{1064}{%
\begin{varwidth}[t]{\sphinxcolwidth{1}{8}}
\sphinxAtStartPar
289 MB
\par
\vskip-\baselineskip\vbox{\hbox{\strut}}\end{varwidth}%
}%
&
\sphinxAtStartPar
261 MB
&
\sphinxAtStartPar
90\%
&
\sphinxAtStartPar
27,2 MB
\\
\cline{4-4}\cline{6-8}\sphinxtablestrut{1060}&\sphinxtablestrut{1061}&\sphinxtablestrut{1062}&
\sphinxAtStartPar
15
&\sphinxtablestrut{1064}&
\sphinxAtStartPar
260 MB
&
\sphinxAtStartPar
89\%
&
\sphinxAtStartPar
28,4 MB
\\
\hline\sphinxmultirow{2}{1072}{%
\begin{varwidth}[t]{\sphinxcolwidth{1}{8}}
\sphinxAtStartPar
98
\par
\vskip-\baselineskip\vbox{\hbox{\strut}}\end{varwidth}%
}%
&\sphinxmultirow{2}{1073}{%
\begin{varwidth}[t]{\sphinxcolwidth{1}{8}}
\sphinxAtStartPar
Stories: The Path of Destinies
\par
\vskip-\baselineskip\vbox{\hbox{\strut}}\end{varwidth}%
}%
&\sphinxmultirow{2}{1074}{%
\begin{varwidth}[t]{\sphinxcolwidth{1}{8}}
\sphinxAtStartPar
zstd
\par
\vskip-\baselineskip\vbox{\hbox{\strut}}\end{varwidth}%
}%
&
\sphinxAtStartPar
3
&\sphinxmultirow{2}{1076}{%
\begin{varwidth}[t]{\sphinxcolwidth{1}{8}}
\sphinxAtStartPar
1,6 GB
\par
\vskip-\baselineskip\vbox{\hbox{\strut}}\end{varwidth}%
}%
&\sphinxmultirow{2}{1077}{%
\begin{varwidth}[t]{\sphinxcolwidth{1}{8}}
\sphinxAtStartPar
1,6 GB
\par
\vskip-\baselineskip\vbox{\hbox{\strut}}\end{varwidth}%
}%
&\sphinxmultirow{2}{1078}{%
\begin{varwidth}[t]{\sphinxcolwidth{1}{8}}
\sphinxAtStartPar
99\%
\par
\vskip-\baselineskip\vbox{\hbox{\strut}}\end{varwidth}%
}%
&
\sphinxAtStartPar
13,8 MB
\\
\cline{4-4}\cline{8-8}\sphinxtablestrut{1072}&\sphinxtablestrut{1073}&\sphinxtablestrut{1074}&
\sphinxAtStartPar
15
&\sphinxtablestrut{1076}&\sphinxtablestrut{1077}&\sphinxtablestrut{1078}&
\sphinxAtStartPar
14,8 MB
\\
\hline\sphinxmultirow{2}{1082}{%
\begin{varwidth}[t]{\sphinxcolwidth{1}{8}}
\sphinxAtStartPar
99
\par
\vskip-\baselineskip\vbox{\hbox{\strut}}\end{varwidth}%
}%
&\sphinxmultirow{2}{1083}{%
\begin{varwidth}[t]{\sphinxcolwidth{1}{8}}
\sphinxAtStartPar
Styx: Master of Shadow
\par
\vskip-\baselineskip\vbox{\hbox{\strut}}\end{varwidth}%
}%
&\sphinxmultirow{2}{1084}{%
\begin{varwidth}[t]{\sphinxcolwidth{1}{8}}
\sphinxAtStartPar
zstd
\par
\vskip-\baselineskip\vbox{\hbox{\strut}}\end{varwidth}%
}%
&
\sphinxAtStartPar
3
&\sphinxmultirow{2}{1086}{%
\begin{varwidth}[t]{\sphinxcolwidth{1}{8}}
\sphinxAtStartPar
6,7 GB
\par
\vskip-\baselineskip\vbox{\hbox{\strut}}\end{varwidth}%
}%
&\sphinxmultirow{2}{1087}{%
\begin{varwidth}[t]{\sphinxcolwidth{1}{8}}
\sphinxAtStartPar
6,6 GB
\par
\vskip-\baselineskip\vbox{\hbox{\strut}}\end{varwidth}%
}%
&\sphinxmultirow{2}{1088}{%
\begin{varwidth}[t]{\sphinxcolwidth{1}{8}}
\sphinxAtStartPar
98\%
\par
\vskip-\baselineskip\vbox{\hbox{\strut}}\end{varwidth}%
}%
&
\sphinxAtStartPar
108 MB
\\
\cline{4-4}\cline{8-8}\sphinxtablestrut{1082}&\sphinxtablestrut{1083}&\sphinxtablestrut{1084}&
\sphinxAtStartPar
15
&\sphinxtablestrut{1086}&\sphinxtablestrut{1087}&\sphinxtablestrut{1088}&
\sphinxAtStartPar
114 MB
\\
\hline\sphinxmultirow{2}{1092}{%
\begin{varwidth}[t]{\sphinxcolwidth{1}{8}}
\sphinxAtStartPar
100
\par
\vskip-\baselineskip\vbox{\hbox{\strut}}\end{varwidth}%
}%
&\sphinxmultirow{2}{1093}{%
\begin{varwidth}[t]{\sphinxcolwidth{1}{8}}
\sphinxAtStartPar
Styx: Shards of Darkness
\par
\vskip-\baselineskip\vbox{\hbox{\strut}}\end{varwidth}%
}%
&\sphinxmultirow{2}{1094}{%
\begin{varwidth}[t]{\sphinxcolwidth{1}{8}}
\sphinxAtStartPar
zstd
\par
\vskip-\baselineskip\vbox{\hbox{\strut}}\end{varwidth}%
}%
&
\sphinxAtStartPar
3
&\sphinxmultirow{2}{1096}{%
\begin{varwidth}[t]{\sphinxcolwidth{1}{8}}
\sphinxAtStartPar
10 GB
\par
\vskip-\baselineskip\vbox{\hbox{\strut}}\end{varwidth}%
}%
&\sphinxmultirow{2}{1097}{%
\begin{varwidth}[t]{\sphinxcolwidth{1}{8}}
\sphinxAtStartPar
10 GB
\par
\vskip-\baselineskip\vbox{\hbox{\strut}}\end{varwidth}%
}%
&\sphinxmultirow{2}{1098}{%
\begin{varwidth}[t]{\sphinxcolwidth{1}{8}}
\sphinxAtStartPar
99\%
\par
\vskip-\baselineskip\vbox{\hbox{\strut}}\end{varwidth}%
}%
&
\sphinxAtStartPar
17,1 MB
\\
\cline{4-4}\cline{8-8}\sphinxtablestrut{1092}&\sphinxtablestrut{1093}&\sphinxtablestrut{1094}&
\sphinxAtStartPar
15
&\sphinxtablestrut{1096}&\sphinxtablestrut{1097}&\sphinxtablestrut{1098}&
\sphinxAtStartPar
22,9 MB
\\
\hline\sphinxmultirow{2}{1102}{%
\begin{varwidth}[t]{\sphinxcolwidth{1}{8}}
\sphinxAtStartPar
101
\par
\vskip-\baselineskip\vbox{\hbox{\strut}}\end{varwidth}%
}%
&\sphinxmultirow{2}{1103}{%
\begin{varwidth}[t]{\sphinxcolwidth{1}{8}}
\sphinxAtStartPar
Sundered: Eldritch Edition
\par
\vskip-\baselineskip\vbox{\hbox{\strut}}\end{varwidth}%
}%
&\sphinxmultirow{2}{1104}{%
\begin{varwidth}[t]{\sphinxcolwidth{1}{8}}
\sphinxAtStartPar
zstd
\par
\vskip-\baselineskip\vbox{\hbox{\strut}}\end{varwidth}%
}%
&
\sphinxAtStartPar
3
&\sphinxmultirow{2}{1106}{%
\begin{varwidth}[t]{\sphinxcolwidth{1}{8}}
\sphinxAtStartPar
2,2 GB
\par
\vskip-\baselineskip\vbox{\hbox{\strut}}\end{varwidth}%
}%
&
\sphinxAtStartPar
1,7 GB
&
\sphinxAtStartPar
75\%
&
\sphinxAtStartPar
584 MB
\\
\cline{4-4}\cline{6-8}\sphinxtablestrut{1102}&\sphinxtablestrut{1103}&\sphinxtablestrut{1104}&
\sphinxAtStartPar
15
&\sphinxtablestrut{1106}&
\sphinxAtStartPar
1,5 GB
&
\sphinxAtStartPar
69\%
&
\sphinxAtStartPar
719 MB
\\
\hline\sphinxmultirow{2}{1114}{%
\begin{varwidth}[t]{\sphinxcolwidth{1}{8}}
\sphinxAtStartPar
102
\par
\vskip-\baselineskip\vbox{\hbox{\strut}}\end{varwidth}%
}%
&\sphinxmultirow{2}{1115}{%
\begin{varwidth}[t]{\sphinxcolwidth{1}{8}}
\sphinxAtStartPar
SYNTHETIK
\par
\vskip-\baselineskip\vbox{\hbox{\strut}}\end{varwidth}%
}%
&\sphinxmultirow{2}{1116}{%
\begin{varwidth}[t]{\sphinxcolwidth{1}{8}}
\sphinxAtStartPar
zstd
\par
\vskip-\baselineskip\vbox{\hbox{\strut}}\end{varwidth}%
}%
&
\sphinxAtStartPar
3
&\sphinxmultirow{2}{1118}{%
\begin{varwidth}[t]{\sphinxcolwidth{1}{8}}
\sphinxAtStartPar
599 MB
\par
\vskip-\baselineskip\vbox{\hbox{\strut}}\end{varwidth}%
}%
&
\sphinxAtStartPar
518 MB
&\sphinxmultirow{2}{1120}{%
\begin{varwidth}[t]{\sphinxcolwidth{1}{8}}
\sphinxAtStartPar
86\%
\par
\vskip-\baselineskip\vbox{\hbox{\strut}}\end{varwidth}%
}%
&
\sphinxAtStartPar
81 MB
\\
\cline{4-4}\cline{6-6}\cline{8-8}\sphinxtablestrut{1114}&\sphinxtablestrut{1115}&\sphinxtablestrut{1116}&
\sphinxAtStartPar
15
&\sphinxtablestrut{1118}&
\sphinxAtStartPar
516 MB
&\sphinxtablestrut{1120}&
\sphinxAtStartPar
83 MB
\\
\hline\sphinxmultirow{2}{1125}{%
\begin{varwidth}[t]{\sphinxcolwidth{1}{8}}
\sphinxAtStartPar
103
\par
\vskip-\baselineskip\vbox{\hbox{\strut}}\end{varwidth}%
}%
&\sphinxmultirow{2}{1126}{%
\begin{varwidth}[t]{\sphinxcolwidth{1}{8}}
\sphinxAtStartPar
Tabletop Simulator
\par
\vskip-\baselineskip\vbox{\hbox{\strut}}\end{varwidth}%
}%
&\sphinxmultirow{2}{1127}{%
\begin{varwidth}[t]{\sphinxcolwidth{1}{8}}
\sphinxAtStartPar
zstd
\par
\vskip-\baselineskip\vbox{\hbox{\strut}}\end{varwidth}%
}%
&
\sphinxAtStartPar
3
&\sphinxmultirow{2}{1129}{%
\begin{varwidth}[t]{\sphinxcolwidth{1}{8}}
\sphinxAtStartPar
2,7 GB
\par
\vskip-\baselineskip\vbox{\hbox{\strut}}\end{varwidth}%
}%
&\sphinxmultirow{2}{1130}{%
\begin{varwidth}[t]{\sphinxcolwidth{1}{8}}
\sphinxAtStartPar
1,7GB
\par
\vskip-\baselineskip\vbox{\hbox{\strut}}\end{varwidth}%
}%
&
\sphinxAtStartPar
65\%
&
\sphinxAtStartPar
0,91 GB
\\
\cline{4-4}\cline{7-8}\sphinxtablestrut{1125}&\sphinxtablestrut{1126}&\sphinxtablestrut{1127}&
\sphinxAtStartPar
15
&\sphinxtablestrut{1129}&\sphinxtablestrut{1130}&
\sphinxAtStartPar
63\%
&
\sphinxAtStartPar
0,95 GB
\\
\hline\sphinxmultirow{2}{1136}{%
\begin{varwidth}[t]{\sphinxcolwidth{1}{8}}
\sphinxAtStartPar
104
\par
\vskip-\baselineskip\vbox{\hbox{\strut}}\end{varwidth}%
}%
&\sphinxmultirow{2}{1137}{%
\begin{varwidth}[t]{\sphinxcolwidth{1}{8}}
\sphinxAtStartPar
The Escapists 2
\par
\vskip-\baselineskip\vbox{\hbox{\strut}}\end{varwidth}%
}%
&\sphinxmultirow{2}{1138}{%
\begin{varwidth}[t]{\sphinxcolwidth{1}{8}}
\sphinxAtStartPar
zstd
\par
\vskip-\baselineskip\vbox{\hbox{\strut}}\end{varwidth}%
}%
&
\sphinxAtStartPar
3
&\sphinxmultirow{2}{1140}{%
\begin{varwidth}[t]{\sphinxcolwidth{1}{8}}
\sphinxAtStartPar
2,4 GB
\par
\vskip-\baselineskip\vbox{\hbox{\strut}}\end{varwidth}%
}%
&\sphinxmultirow{2}{1141}{%
\begin{varwidth}[t]{\sphinxcolwidth{1}{8}}
\sphinxAtStartPar
1,7 GB
\par
\vskip-\baselineskip\vbox{\hbox{\strut}}\end{varwidth}%
}%
&\sphinxmultirow{2}{1142}{%
\begin{varwidth}[t]{\sphinxcolwidth{1}{8}}
\sphinxAtStartPar
71\%
\par
\vskip-\baselineskip\vbox{\hbox{\strut}}\end{varwidth}%
}%
&
\sphinxAtStartPar
710 MB
\\
\cline{4-4}\cline{8-8}\sphinxtablestrut{1136}&\sphinxtablestrut{1137}&\sphinxtablestrut{1138}&
\sphinxAtStartPar
15
&\sphinxtablestrut{1140}&\sphinxtablestrut{1141}&\sphinxtablestrut{1142}&
\sphinxAtStartPar
717 MB
\\
\hline\sphinxmultirow{2}{1146}{%
\begin{varwidth}[t]{\sphinxcolwidth{1}{8}}
\sphinxAtStartPar
105
\par
\vskip-\baselineskip\vbox{\hbox{\strut}}\end{varwidth}%
}%
&\sphinxmultirow{2}{1147}{%
\begin{varwidth}[t]{\sphinxcolwidth{1}{8}}
\sphinxAtStartPar
The Life and Suffering of Sir Brante
\par
\vskip-\baselineskip\vbox{\hbox{\strut}}\end{varwidth}%
}%
&\sphinxmultirow{2}{1148}{%
\begin{varwidth}[t]{\sphinxcolwidth{1}{8}}
\sphinxAtStartPar
zstd
\par
\vskip-\baselineskip\vbox{\hbox{\strut}}\end{varwidth}%
}%
&
\sphinxAtStartPar
3
&\sphinxmultirow{2}{1150}{%
\begin{varwidth}[t]{\sphinxcolwidth{1}{8}}
\sphinxAtStartPar
2,7 GB
\par
\vskip-\baselineskip\vbox{\hbox{\strut}}\end{varwidth}%
}%
&\sphinxmultirow{2}{1151}{%
\begin{varwidth}[t]{\sphinxcolwidth{1}{8}}
\sphinxAtStartPar
1,1 GB
\par
\vskip-\baselineskip\vbox{\hbox{\strut}}\end{varwidth}%
}%
&
\sphinxAtStartPar
42\%
&
\sphinxAtStartPar
1,52 GB
\\
\cline{4-4}\cline{7-8}\sphinxtablestrut{1146}&\sphinxtablestrut{1147}&\sphinxtablestrut{1148}&
\sphinxAtStartPar
15
&\sphinxtablestrut{1150}&\sphinxtablestrut{1151}&
\sphinxAtStartPar
43\%
&
\sphinxAtStartPar
1,48 GB
\\
\hline\sphinxmultirow{2}{1157}{%
\begin{varwidth}[t]{\sphinxcolwidth{1}{8}}
\sphinxAtStartPar
106
\par
\vskip-\baselineskip\vbox{\hbox{\strut}}\end{varwidth}%
}%
&\sphinxmultirow{2}{1158}{%
\begin{varwidth}[t]{\sphinxcolwidth{1}{8}}
\sphinxAtStartPar
The Cave
\par
\vskip-\baselineskip\vbox{\hbox{\strut}}\end{varwidth}%
}%
&\sphinxmultirow{2}{1159}{%
\begin{varwidth}[t]{\sphinxcolwidth{1}{8}}
\sphinxAtStartPar
zstd
\par
\vskip-\baselineskip\vbox{\hbox{\strut}}\end{varwidth}%
}%
&
\sphinxAtStartPar
3
&\sphinxmultirow{2}{1161}{%
\begin{varwidth}[t]{\sphinxcolwidth{1}{8}}
\sphinxAtStartPar
1,1 GB
\par
\vskip-\baselineskip\vbox{\hbox{\strut}}\end{varwidth}%
}%
&\sphinxmultirow{2}{1162}{%
\begin{varwidth}[t]{\sphinxcolwidth{1}{8}}
\sphinxAtStartPar
1,1 GB
\par
\vskip-\baselineskip\vbox{\hbox{\strut}}\end{varwidth}%
}%
&\sphinxmultirow{2}{1163}{%
\begin{varwidth}[t]{\sphinxcolwidth{1}{8}}
\sphinxAtStartPar
98\%
\par
\vskip-\baselineskip\vbox{\hbox{\strut}}\end{varwidth}%
}%
&
\sphinxAtStartPar
22 MB
\\
\cline{4-4}\cline{8-8}\sphinxtablestrut{1157}&\sphinxtablestrut{1158}&\sphinxtablestrut{1159}&
\sphinxAtStartPar
15
&\sphinxtablestrut{1161}&\sphinxtablestrut{1162}&\sphinxtablestrut{1163}&
\sphinxAtStartPar
24 MB
\\
\hline\sphinxmultirow{2}{1167}{%
\begin{varwidth}[t]{\sphinxcolwidth{1}{8}}
\sphinxAtStartPar
107
\par
\vskip-\baselineskip\vbox{\hbox{\strut}}\end{varwidth}%
}%
&\sphinxmultirow{2}{1168}{%
\begin{varwidth}[t]{\sphinxcolwidth{1}{8}}
\sphinxAtStartPar
The Red Solstice
\par
\vskip-\baselineskip\vbox{\hbox{\strut}}\end{varwidth}%
}%
&\sphinxmultirow{2}{1169}{%
\begin{varwidth}[t]{\sphinxcolwidth{1}{8}}
\sphinxAtStartPar
zstd
\par
\vskip-\baselineskip\vbox{\hbox{\strut}}\end{varwidth}%
}%
&
\sphinxAtStartPar
3
&\sphinxmultirow{2}{1171}{%
\begin{varwidth}[t]{\sphinxcolwidth{1}{8}}
\sphinxAtStartPar
2,7 GB
\par
\vskip-\baselineskip\vbox{\hbox{\strut}}\end{varwidth}%
}%
&\sphinxmultirow{2}{1172}{%
\begin{varwidth}[t]{\sphinxcolwidth{1}{8}}
\sphinxAtStartPar
1,4 GB
\par
\vskip-\baselineskip\vbox{\hbox{\strut}}\end{varwidth}%
}%
&
\sphinxAtStartPar
52\%
&
\sphinxAtStartPar
1,31 GB
\\
\cline{4-4}\cline{7-8}\sphinxtablestrut{1167}&\sphinxtablestrut{1168}&\sphinxtablestrut{1169}&
\sphinxAtStartPar
15
&\sphinxtablestrut{1171}&\sphinxtablestrut{1172}&
\sphinxAtStartPar
51\%
&
\sphinxAtStartPar
1,34 GB
\\
\hline\sphinxmultirow{2}{1178}{%
\begin{varwidth}[t]{\sphinxcolwidth{1}{8}}
\sphinxAtStartPar
108
\par
\vskip-\baselineskip\vbox{\hbox{\strut}}\end{varwidth}%
}%
&\sphinxmultirow{2}{1179}{%
\begin{varwidth}[t]{\sphinxcolwidth{1}{8}}
\sphinxAtStartPar
They Always Run
\par
\vskip-\baselineskip\vbox{\hbox{\strut}}\end{varwidth}%
}%
&\sphinxmultirow{2}{1180}{%
\begin{varwidth}[t]{\sphinxcolwidth{1}{8}}
\sphinxAtStartPar
zstd
\par
\vskip-\baselineskip\vbox{\hbox{\strut}}\end{varwidth}%
}%
&
\sphinxAtStartPar
3
&\sphinxmultirow{2}{1182}{%
\begin{varwidth}[t]{\sphinxcolwidth{1}{8}}
\sphinxAtStartPar
10 GB
\par
\vskip-\baselineskip\vbox{\hbox{\strut}}\end{varwidth}%
}%
&
\sphinxAtStartPar
4,2 GB
&
\sphinxAtStartPar
39\%
&
\sphinxAtStartPar
6,6 GB
\\
\cline{4-4}\cline{6-8}\sphinxtablestrut{1178}&\sphinxtablestrut{1179}&\sphinxtablestrut{1180}&
\sphinxAtStartPar
15
&\sphinxtablestrut{1182}&
\sphinxAtStartPar
3,8 GB
&
\sphinxAtStartPar
34\%
&
\sphinxAtStartPar
7,1 GB
\\
\hline\sphinxmultirow{2}{1190}{%
\begin{varwidth}[t]{\sphinxcolwidth{1}{8}}
\sphinxAtStartPar
109
\par
\vskip-\baselineskip\vbox{\hbox{\strut}}\end{varwidth}%
}%
&\sphinxmultirow{2}{1191}{%
\begin{varwidth}[t]{\sphinxcolwidth{1}{8}}
\sphinxAtStartPar
This War of Mine
\par
\vskip-\baselineskip\vbox{\hbox{\strut}}\end{varwidth}%
}%
&\sphinxmultirow{2}{1192}{%
\begin{varwidth}[t]{\sphinxcolwidth{1}{8}}
\sphinxAtStartPar
zstd
\par
\vskip-\baselineskip\vbox{\hbox{\strut}}\end{varwidth}%
}%
&
\sphinxAtStartPar
3
&\sphinxmultirow{2}{1194}{%
\begin{varwidth}[t]{\sphinxcolwidth{1}{8}}
\sphinxAtStartPar
2,6 GB
\par
\vskip-\baselineskip\vbox{\hbox{\strut}}\end{varwidth}%
}%
&\sphinxmultirow{2}{1195}{%
\begin{varwidth}[t]{\sphinxcolwidth{1}{8}}
\sphinxAtStartPar
2,5 GB
\par
\vskip-\baselineskip\vbox{\hbox{\strut}}\end{varwidth}%
}%
&\sphinxmultirow{2}{1196}{%
\begin{varwidth}[t]{\sphinxcolwidth{1}{8}}
\sphinxAtStartPar
98\%
\par
\vskip-\baselineskip\vbox{\hbox{\strut}}\end{varwidth}%
}%
&
\sphinxAtStartPar
34 MB
\\
\cline{4-4}\cline{8-8}\sphinxtablestrut{1190}&\sphinxtablestrut{1191}&\sphinxtablestrut{1192}&
\sphinxAtStartPar
15
&\sphinxtablestrut{1194}&\sphinxtablestrut{1195}&\sphinxtablestrut{1196}&
\sphinxAtStartPar
36 MB
\\
\hline\sphinxmultirow{2}{1200}{%
\begin{varwidth}[t]{\sphinxcolwidth{1}{8}}
\sphinxAtStartPar
110
\par
\vskip-\baselineskip\vbox{\hbox{\strut}}\end{varwidth}%
}%
&\sphinxmultirow{2}{1201}{%
\begin{varwidth}[t]{\sphinxcolwidth{1}{8}}
\sphinxAtStartPar
Titan Souls
\par
\vskip-\baselineskip\vbox{\hbox{\strut}}\end{varwidth}%
}%
&\sphinxmultirow{2}{1202}{%
\begin{varwidth}[t]{\sphinxcolwidth{1}{8}}
\sphinxAtStartPar
zstd
\par
\vskip-\baselineskip\vbox{\hbox{\strut}}\end{varwidth}%
}%
&
\sphinxAtStartPar
3
&\sphinxmultirow{2}{1204}{%
\begin{varwidth}[t]{\sphinxcolwidth{1}{8}}
\sphinxAtStartPar
206 MB
\par
\vskip-\baselineskip\vbox{\hbox{\strut}}\end{varwidth}%
}%
&
\sphinxAtStartPar
183 MB
&\sphinxmultirow{2}{1206}{%
\begin{varwidth}[t]{\sphinxcolwidth{1}{8}}
\sphinxAtStartPar
88\%
\par
\vskip-\baselineskip\vbox{\hbox{\strut}}\end{varwidth}%
}%
&
\sphinxAtStartPar
21,9 MB
\\
\cline{4-4}\cline{6-6}\cline{8-8}\sphinxtablestrut{1200}&\sphinxtablestrut{1201}&\sphinxtablestrut{1202}&
\sphinxAtStartPar
15
&\sphinxtablestrut{1204}&
\sphinxAtStartPar
182 MB
&\sphinxtablestrut{1206}&
\sphinxAtStartPar
22,5 MB
\\
\hline\sphinxmultirow{2}{1211}{%
\begin{varwidth}[t]{\sphinxcolwidth{1}{8}}
\sphinxAtStartPar
111
\par
\vskip-\baselineskip\vbox{\hbox{\strut}}\end{varwidth}%
}%
&\sphinxmultirow{2}{1212}{%
\begin{varwidth}[t]{\sphinxcolwidth{1}{8}}
\sphinxAtStartPar
Transistor
\par
\vskip-\baselineskip\vbox{\hbox{\strut}}\end{varwidth}%
}%
&\sphinxmultirow{2}{1213}{%
\begin{varwidth}[t]{\sphinxcolwidth{1}{8}}
\sphinxAtStartPar
zstd
\par
\vskip-\baselineskip\vbox{\hbox{\strut}}\end{varwidth}%
}%
&
\sphinxAtStartPar
3
&\sphinxmultirow{2}{1215}{%
\begin{varwidth}[t]{\sphinxcolwidth{1}{8}}
\sphinxAtStartPar
3,0 GB
\par
\vskip-\baselineskip\vbox{\hbox{\strut}}\end{varwidth}%
}%
&\sphinxmultirow{2}{1216}{%
\begin{varwidth}[t]{\sphinxcolwidth{1}{8}}
\sphinxAtStartPar
2,7 GB
\par
\vskip-\baselineskip\vbox{\hbox{\strut}}\end{varwidth}%
}%
&
\sphinxAtStartPar
88\%
&
\sphinxAtStartPar
364 MB
\\
\cline{4-4}\cline{7-8}\sphinxtablestrut{1211}&\sphinxtablestrut{1212}&\sphinxtablestrut{1213}&
\sphinxAtStartPar
15
&\sphinxtablestrut{1215}&\sphinxtablestrut{1216}&
\sphinxAtStartPar
87\%
&
\sphinxAtStartPar
384 MB
\\
\hline\sphinxmultirow{2}{1222}{%
\begin{varwidth}[t]{\sphinxcolwidth{1}{8}}
\sphinxAtStartPar
112
\par
\vskip-\baselineskip\vbox{\hbox{\strut}}\end{varwidth}%
}%
&\sphinxmultirow{2}{1223}{%
\begin{varwidth}[t]{\sphinxcolwidth{1}{8}}
\sphinxAtStartPar
Trine
\par
\vskip-\baselineskip\vbox{\hbox{\strut}}\end{varwidth}%
}%
&\sphinxmultirow{2}{1224}{%
\begin{varwidth}[t]{\sphinxcolwidth{1}{8}}
\sphinxAtStartPar
zstd
\par
\vskip-\baselineskip\vbox{\hbox{\strut}}\end{varwidth}%
}%
&
\sphinxAtStartPar
3
&\sphinxmultirow{2}{1226}{%
\begin{varwidth}[t]{\sphinxcolwidth{1}{8}}
\sphinxAtStartPar
1,3 GB
\par
\vskip-\baselineskip\vbox{\hbox{\strut}}\end{varwidth}%
}%
&\sphinxmultirow{2}{1227}{%
\begin{varwidth}[t]{\sphinxcolwidth{1}{8}}
\sphinxAtStartPar
1,3 GB
\par
\vskip-\baselineskip\vbox{\hbox{\strut}}\end{varwidth}%
}%
&
\sphinxAtStartPar
97\%
&
\sphinxAtStartPar
41 MB
\\
\cline{4-4}\cline{7-8}\sphinxtablestrut{1222}&\sphinxtablestrut{1223}&\sphinxtablestrut{1224}&
\sphinxAtStartPar
15
&\sphinxtablestrut{1226}&\sphinxtablestrut{1227}&
\sphinxAtStartPar
96\%
&
\sphinxAtStartPar
44 MB
\\
\hline\sphinxmultirow{2}{1233}{%
\begin{varwidth}[t]{\sphinxcolwidth{1}{8}}
\sphinxAtStartPar
113
\par
\vskip-\baselineskip\vbox{\hbox{\strut}}\end{varwidth}%
}%
&\sphinxmultirow{2}{1234}{%
\begin{varwidth}[t]{\sphinxcolwidth{1}{8}}
\sphinxAtStartPar
Undertale
\par
\vskip-\baselineskip\vbox{\hbox{\strut}}\end{varwidth}%
}%
&\sphinxmultirow{2}{1235}{%
\begin{varwidth}[t]{\sphinxcolwidth{1}{8}}
\sphinxAtStartPar
zstd
\par
\vskip-\baselineskip\vbox{\hbox{\strut}}\end{varwidth}%
}%
&
\sphinxAtStartPar
3
&\sphinxmultirow{2}{1237}{%
\begin{varwidth}[t]{\sphinxcolwidth{1}{8}}
\sphinxAtStartPar
155 MB
\par
\vskip-\baselineskip\vbox{\hbox{\strut}}\end{varwidth}%
}%
&
\sphinxAtStartPar
141 MB
&\sphinxmultirow{2}{1239}{%
\begin{varwidth}[t]{\sphinxcolwidth{1}{8}}
\sphinxAtStartPar
90\%
\par
\vskip-\baselineskip\vbox{\hbox{\strut}}\end{varwidth}%
}%
&
\sphinxAtStartPar
14,2 MB
\\
\cline{4-4}\cline{6-6}\cline{8-8}\sphinxtablestrut{1233}&\sphinxtablestrut{1234}&\sphinxtablestrut{1235}&
\sphinxAtStartPar
15
&\sphinxtablestrut{1237}&
\sphinxAtStartPar
140 MB
&\sphinxtablestrut{1239}&
\sphinxAtStartPar
14,9 MB
\\
\hline\sphinxmultirow{2}{1244}{%
\begin{varwidth}[t]{\sphinxcolwidth{1}{8}}
\sphinxAtStartPar
114
\par
\vskip-\baselineskip\vbox{\hbox{\strut}}\end{varwidth}%
}%
&\sphinxmultirow{2}{1245}{%
\begin{varwidth}[t]{\sphinxcolwidth{1}{8}}
\sphinxAtStartPar
Valiant Hearts: The Great War
\par
\vskip-\baselineskip\vbox{\hbox{\strut}}\end{varwidth}%
}%
&\sphinxmultirow{2}{1246}{%
\begin{varwidth}[t]{\sphinxcolwidth{1}{8}}
\sphinxAtStartPar
zstd
\par
\vskip-\baselineskip\vbox{\hbox{\strut}}\end{varwidth}%
}%
&
\sphinxAtStartPar
3
&\sphinxmultirow{2}{1248}{%
\begin{varwidth}[t]{\sphinxcolwidth{1}{8}}
\sphinxAtStartPar
1,2 GB
\par
\vskip-\baselineskip\vbox{\hbox{\strut}}\end{varwidth}%
}%
&\sphinxmultirow{2}{1249}{%
\begin{varwidth}[t]{\sphinxcolwidth{1}{8}}
\sphinxAtStartPar
1,1 GB
\par
\vskip-\baselineskip\vbox{\hbox{\strut}}\end{varwidth}%
}%
&\sphinxmultirow{2}{1250}{%
\begin{varwidth}[t]{\sphinxcolwidth{1}{8}}
\sphinxAtStartPar
99\%
\par
\vskip-\baselineskip\vbox{\hbox{\strut}}\end{varwidth}%
}%
&
\sphinxAtStartPar
9,8 MB
\\
\cline{4-4}\cline{8-8}\sphinxtablestrut{1244}&\sphinxtablestrut{1245}&\sphinxtablestrut{1246}&
\sphinxAtStartPar
15
&\sphinxtablestrut{1248}&\sphinxtablestrut{1249}&\sphinxtablestrut{1250}&
\sphinxAtStartPar
10,2 MB
\\
\hline\sphinxmultirow{2}{1254}{%
\begin{varwidth}[t]{\sphinxcolwidth{1}{8}}
\sphinxAtStartPar
115
\par
\vskip-\baselineskip\vbox{\hbox{\strut}}\end{varwidth}%
}%
&\sphinxmultirow{2}{1255}{%
\begin{varwidth}[t]{\sphinxcolwidth{1}{8}}
\sphinxAtStartPar
Vanquish
\par
\vskip-\baselineskip\vbox{\hbox{\strut}}\end{varwidth}%
}%
&\sphinxmultirow{2}{1256}{%
\begin{varwidth}[t]{\sphinxcolwidth{1}{8}}
\sphinxAtStartPar
zstd
\par
\vskip-\baselineskip\vbox{\hbox{\strut}}\end{varwidth}%
}%
&
\sphinxAtStartPar
3
&\sphinxmultirow{2}{1258}{%
\begin{varwidth}[t]{\sphinxcolwidth{1}{8}}
\sphinxAtStartPar
18 GB
\par
\vskip-\baselineskip\vbox{\hbox{\strut}}\end{varwidth}%
}%
&\sphinxmultirow{2}{1259}{%
\begin{varwidth}[t]{\sphinxcolwidth{1}{8}}
\sphinxAtStartPar
18 GB
\par
\vskip-\baselineskip\vbox{\hbox{\strut}}\end{varwidth}%
}%
&\sphinxmultirow{2}{1260}{%
\begin{varwidth}[t]{\sphinxcolwidth{1}{8}}
\sphinxAtStartPar
99\%
\par
\vskip-\baselineskip\vbox{\hbox{\strut}}\end{varwidth}%
}%
&
\sphinxAtStartPar
7,7 MB
\\
\cline{4-4}\cline{8-8}\sphinxtablestrut{1254}&\sphinxtablestrut{1255}&\sphinxtablestrut{1256}&
\sphinxAtStartPar
15
&\sphinxtablestrut{1258}&\sphinxtablestrut{1259}&\sphinxtablestrut{1260}&
\sphinxAtStartPar
12,3 MB
\\
\hline\sphinxmultirow{2}{1264}{%
\begin{varwidth}[t]{\sphinxcolwidth{1}{8}}
\sphinxAtStartPar
116
\par
\vskip-\baselineskip\vbox{\hbox{\strut}}\end{varwidth}%
}%
&\sphinxmultirow{2}{1265}{%
\begin{varwidth}[t]{\sphinxcolwidth{1}{8}}
\sphinxAtStartPar
Vesper
\par
\vskip-\baselineskip\vbox{\hbox{\strut}}\end{varwidth}%
}%
&\sphinxmultirow{2}{1266}{%
\begin{varwidth}[t]{\sphinxcolwidth{1}{8}}
\sphinxAtStartPar
zstd
\par
\vskip-\baselineskip\vbox{\hbox{\strut}}\end{varwidth}%
}%
&
\sphinxAtStartPar
3
&\sphinxmultirow{2}{1268}{%
\begin{varwidth}[t]{\sphinxcolwidth{1}{8}}
\sphinxAtStartPar
2,8 GB
\par
\vskip-\baselineskip\vbox{\hbox{\strut}}\end{varwidth}%
}%
&
\sphinxAtStartPar
998 NB
&
\sphinxAtStartPar
34\%
&
\sphinxAtStartPar
1,88 GB
\\
\cline{4-4}\cline{6-8}\sphinxtablestrut{1264}&\sphinxtablestrut{1265}&\sphinxtablestrut{1266}&
\sphinxAtStartPar
15
&\sphinxtablestrut{1268}&
\sphinxAtStartPar
964 MB
&
\sphinxAtStartPar
32\%
&
\sphinxAtStartPar
1,92 GB
\\
\hline\sphinxmultirow{2}{1276}{%
\begin{varwidth}[t]{\sphinxcolwidth{1}{8}}
\sphinxAtStartPar
117
\par
\vskip-\baselineskip\vbox{\hbox{\strut}}\end{varwidth}%
}%
&\sphinxmultirow{2}{1277}{%
\begin{varwidth}[t]{\sphinxcolwidth{1}{8}}
\sphinxAtStartPar
Void Bastards
\par
\vskip-\baselineskip\vbox{\hbox{\strut}}\end{varwidth}%
}%
&\sphinxmultirow{2}{1278}{%
\begin{varwidth}[t]{\sphinxcolwidth{1}{8}}
\sphinxAtStartPar
zstd
\par
\vskip-\baselineskip\vbox{\hbox{\strut}}\end{varwidth}%
}%
&
\sphinxAtStartPar
3
&\sphinxmultirow{2}{1280}{%
\begin{varwidth}[t]{\sphinxcolwidth{1}{8}}
\sphinxAtStartPar
5,7 GB
\par
\vskip-\baselineskip\vbox{\hbox{\strut}}\end{varwidth}%
}%
&\sphinxmultirow{2}{1281}{%
\begin{varwidth}[t]{\sphinxcolwidth{1}{8}}
\sphinxAtStartPar
2,3 GB
\par
\vskip-\baselineskip\vbox{\hbox{\strut}}\end{varwidth}%
}%
&
\sphinxAtStartPar
40\%
&
\sphinxAtStartPar
3,30 GB
\\
\cline{4-4}\cline{7-8}\sphinxtablestrut{1276}&\sphinxtablestrut{1277}&\sphinxtablestrut{1278}&
\sphinxAtStartPar
15
&\sphinxtablestrut{1280}&\sphinxtablestrut{1281}&
\sphinxAtStartPar
41\%
&
\sphinxAtStartPar
3,28 GB
\\
\hline\sphinxmultirow{2}{1287}{%
\begin{varwidth}[t]{\sphinxcolwidth{1}{8}}
\sphinxAtStartPar
118
\par
\vskip-\baselineskip\vbox{\hbox{\strut}}\end{varwidth}%
}%
&\sphinxmultirow{2}{1288}{%
\begin{varwidth}[t]{\sphinxcolwidth{1}{8}}
\sphinxAtStartPar
Wasteland 2: Director's Cut
\par
\vskip-\baselineskip\vbox{\hbox{\strut}}\end{varwidth}%
}%
&\sphinxmultirow{2}{1289}{%
\begin{varwidth}[t]{\sphinxcolwidth{1}{8}}
\sphinxAtStartPar
zstd
\par
\vskip-\baselineskip\vbox{\hbox{\strut}}\end{varwidth}%
}%
&
\sphinxAtStartPar
3
&\sphinxmultirow{2}{1291}{%
\begin{varwidth}[t]{\sphinxcolwidth{1}{8}}
\sphinxAtStartPar
14 GB
\par
\vskip-\baselineskip\vbox{\hbox{\strut}}\end{varwidth}%
}%
&\sphinxmultirow{2}{1292}{%
\begin{varwidth}[t]{\sphinxcolwidth{1}{8}}
\sphinxAtStartPar
13 GB
\par
\vskip-\baselineskip\vbox{\hbox{\strut}}\end{varwidth}%
}%
&\sphinxmultirow{2}{1293}{%
\begin{varwidth}[t]{\sphinxcolwidth{1}{8}}
\sphinxAtStartPar
91\%
\par
\vskip-\baselineskip\vbox{\hbox{\strut}}\end{varwidth}%
}%
&
\sphinxAtStartPar
1,24 GB
\\
\cline{4-4}\cline{8-8}\sphinxtablestrut{1287}&\sphinxtablestrut{1288}&\sphinxtablestrut{1289}&
\sphinxAtStartPar
15
&\sphinxtablestrut{1291}&\sphinxtablestrut{1292}&\sphinxtablestrut{1293}&
\sphinxAtStartPar
1.10 GB
\\
\hline\sphinxmultirow{2}{1297}{%
\begin{varwidth}[t]{\sphinxcolwidth{1}{8}}
\sphinxAtStartPar
119
\par
\vskip-\baselineskip\vbox{\hbox{\strut}}\end{varwidth}%
}%
&\sphinxmultirow{2}{1298}{%
\begin{varwidth}[t]{\sphinxcolwidth{1}{8}}
\sphinxAtStartPar
Wasteland 3
\par
\vskip-\baselineskip\vbox{\hbox{\strut}}\end{varwidth}%
}%
&\sphinxmultirow{2}{1299}{%
\begin{varwidth}[t]{\sphinxcolwidth{1}{8}}
\sphinxAtStartPar
zstd
\par
\vskip-\baselineskip\vbox{\hbox{\strut}}\end{varwidth}%
}%
&
\sphinxAtStartPar
3
&\sphinxmultirow{2}{1301}{%
\begin{varwidth}[t]{\sphinxcolwidth{1}{8}}
\sphinxAtStartPar
26 GB
\par
\vskip-\baselineskip\vbox{\hbox{\strut}}\end{varwidth}%
}%
&
\sphinxAtStartPar
24 GB
&
\sphinxAtStartPar
91\%
&
\sphinxAtStartPar
2,11 GB
\\
\cline{4-4}\cline{6-8}\sphinxtablestrut{1297}&\sphinxtablestrut{1298}&\sphinxtablestrut{1299}&
\sphinxAtStartPar
15
&\sphinxtablestrut{1301}&
\sphinxAtStartPar
23 GB
&
\sphinxAtStartPar
89\%
&
\sphinxAtStartPar
2,71 GB
\\
\hline\sphinxmultirow{2}{1309}{%
\begin{varwidth}[t]{\sphinxcolwidth{1}{8}}
\sphinxAtStartPar
120
\par
\vskip-\baselineskip\vbox{\hbox{\strut}}\end{varwidth}%
}%
&\sphinxmultirow{2}{1310}{%
\begin{varwidth}[t]{\sphinxcolwidth{1}{8}}
\sphinxAtStartPar
Witch It
\par
\vskip-\baselineskip\vbox{\hbox{\strut}}\end{varwidth}%
}%
&\sphinxmultirow{2}{1311}{%
\begin{varwidth}[t]{\sphinxcolwidth{1}{8}}
\sphinxAtStartPar
zsta
\par
\vskip-\baselineskip\vbox{\hbox{\strut}}\end{varwidth}%
}%
&
\sphinxAtStartPar
3
&\sphinxmultirow{2}{1313}{%
\begin{varwidth}[t]{\sphinxcolwidth{1}{8}}
\sphinxAtStartPar
4,2 GB
\par
\vskip-\baselineskip\vbox{\hbox{\strut}}\end{varwidth}%
}%
&\sphinxmultirow{2}{1314}{%
\begin{varwidth}[t]{\sphinxcolwidth{1}{8}}
\sphinxAtStartPar
4,1 GB
\par
\vskip-\baselineskip\vbox{\hbox{\strut}}\end{varwidth}%
}%
&
\sphinxAtStartPar
98\%
&
\sphinxAtStartPar
85 MB
\\
\cline{4-4}\cline{7-8}\sphinxtablestrut{1309}&\sphinxtablestrut{1310}&\sphinxtablestrut{1311}&
\sphinxAtStartPar
15
&\sphinxtablestrut{1313}&\sphinxtablestrut{1314}&
\sphinxAtStartPar
97\%
&
\sphinxAtStartPar
95 MB
\\
\hline\sphinxmultirow{2}{1320}{%
\begin{varwidth}[t]{\sphinxcolwidth{1}{8}}
\sphinxAtStartPar
121
\par
\vskip-\baselineskip\vbox{\hbox{\strut}}\end{varwidth}%
}%
&\sphinxmultirow{2}{1321}{%
\begin{varwidth}[t]{\sphinxcolwidth{1}{8}}
\sphinxAtStartPar
Wizard of Legend
\par
\vskip-\baselineskip\vbox{\hbox{\strut}}\end{varwidth}%
}%
&\sphinxmultirow{2}{1322}{%
\begin{varwidth}[t]{\sphinxcolwidth{1}{8}}
\sphinxAtStartPar
zstd
\par
\vskip-\baselineskip\vbox{\hbox{\strut}}\end{varwidth}%
}%
&
\sphinxAtStartPar
3
&\sphinxmultirow{2}{1324}{%
\begin{varwidth}[t]{\sphinxcolwidth{1}{8}}
\sphinxAtStartPar
786 MB
\par
\vskip-\baselineskip\vbox{\hbox{\strut}}\end{varwidth}%
}%
&
\sphinxAtStartPar
475 MB
&
\sphinxAtStartPar
60\%
&
\sphinxAtStartPar
312 MB
\\
\cline{4-4}\cline{6-8}\sphinxtablestrut{1320}&\sphinxtablestrut{1321}&\sphinxtablestrut{1322}&
\sphinxAtStartPar
15
&\sphinxtablestrut{1324}&
\sphinxAtStartPar
468 MB
&
\sphinxAtStartPar
59\%
&
\sphinxAtStartPar
318 MB
\\
\hline&&&&&&&\\
\hline\sphinxmultirow{2}{1340}{%
\begin{varwidth}[t]{\sphinxcolwidth{1}{8}}
\par
\vskip-\baselineskip\vbox{\hbox{\strut}}\end{varwidth}%
}%
&\sphinxmultirow{2}{1341}{%
\begin{varwidth}[t]{\sphinxcolwidth{1}{8}}
\sphinxAtStartPar
Итого
\par
\vskip-\baselineskip\vbox{\hbox{\strut}}\end{varwidth}%
}%
&\sphinxmultirow{2}{1342}{%
\begin{varwidth}[t]{\sphinxcolwidth{1}{8}}
\sphinxAtStartPar
zstd
\par
\vskip-\baselineskip\vbox{\hbox{\strut}}\end{varwidth}%
}%
&
\sphinxAtStartPar
3
&\sphinxmultirow{2}{1344}{%
\begin{varwidth}[t]{\sphinxcolwidth{1}{8}}
\sphinxAtStartPar
761 GB
\par
\vskip-\baselineskip\vbox{\hbox{\strut}}\end{varwidth}%
}%
&
\sphinxAtStartPar
666 GB
&\sphinxmultirow{2}{1346}{%
\begin{varwidth}[t]{\sphinxcolwidth{1}{8}}
\sphinxAtStartPar
87\%
\par
\vskip-\baselineskip\vbox{\hbox{\strut}}\end{varwidth}%
}%
&
\sphinxAtStartPar
94 GB
\\
\cline{4-4}\cline{6-6}\cline{8-8}\sphinxtablestrut{1340}&\sphinxtablestrut{1341}&\sphinxtablestrut{1342}&
\sphinxAtStartPar
15
&\sphinxtablestrut{1344}&
\sphinxAtStartPar
664 GB
&\sphinxtablestrut{1346}&
\sphinxAtStartPar
97 GB
\\
\hline&&&&&&&\\
\hline\sphinxmultirow{2}{1359}{%
\begin{varwidth}[t]{\sphinxcolwidth{1}{8}}
\par
\vskip-\baselineskip\vbox{\hbox{\strut}}\end{varwidth}%
}%
&\sphinxmultirow{2}{1360}{%
\begin{varwidth}[t]{\sphinxcolwidth{1}{8}}
\sphinxAtStartPar
Кэш шейдеров представленных здесь игр в Steam
\par
\vskip-\baselineskip\vbox{\hbox{\strut}}\end{varwidth}%
}%
&\sphinxmultirow{2}{1361}{%
\begin{varwidth}[t]{\sphinxcolwidth{1}{8}}
\sphinxAtStartPar
zstd
\par
\vskip-\baselineskip\vbox{\hbox{\strut}}\end{varwidth}%
}%
&
\sphinxAtStartPar
3
&\sphinxmultirow{2}{1363}{%
\begin{varwidth}[t]{\sphinxcolwidth{1}{8}}
\sphinxAtStartPar
26 GB
\par
\vskip-\baselineskip\vbox{\hbox{\strut}}\end{varwidth}%
}%
&\sphinxmultirow{2}{1364}{%
\begin{varwidth}[t]{\sphinxcolwidth{1}{8}}
\sphinxAtStartPar
25 GB
\par
\vskip-\baselineskip\vbox{\hbox{\strut}}\end{varwidth}%
}%
&\sphinxmultirow{2}{1365}{%
\begin{varwidth}[t]{\sphinxcolwidth{1}{8}}
\sphinxAtStartPar
99\%
\par
\vskip-\baselineskip\vbox{\hbox{\strut}}\end{varwidth}%
}%
&
\sphinxAtStartPar
217 MB
\\
\cline{4-4}\cline{8-8}\sphinxtablestrut{1359}&\sphinxtablestrut{1360}&\sphinxtablestrut{1361}&
\sphinxAtStartPar
15
&\sphinxtablestrut{1363}&\sphinxtablestrut{1364}&\sphinxtablestrut{1365}&
\sphinxAtStartPar
218 MB
\\
\hline
\end{longtable}\sphinxatlongtableend\end{savenotes}

\sphinxAtStartPar
Примечания:
\begin{itemize}
\item {} 
\sphinxAtStartPar
По возможности данный список будет расширяться новыми играми и другими алгоритмами сжатия.

\item {} 
\sphinxAtStartPar
U/N \sphinxhyphen{} выраженное в процентах соотношение количества фактически занятого места к необходимому,
т.е. если от 100\% отнять U/N можно получить процент сэкономленного места на диске. Из чего следует, что чем меньше данный показатель, тем лучше.

\item {} 
\sphinxAtStartPar
Экономия рассчитывалась вручную с округлением в меньшую сторону. Другими словами, если получалось 1,3087... GB, то записывалось как 1,30 GB.

\end{itemize}

\index{btrfs@\spxentry{btrfs}}\index{compression@\spxentry{compression}}\index{test@\spxentry{test}}\index{results@\spxentry{results}}\ignorespaces 

\subsubsection{Промежуточные результаты}
\label{\detokenize{source/file-systems:intermediate-results}}\label{\detokenize{source/file-systems:index-6}}\label{\detokenize{source/file-systems:id7}}\begin{itemize}
\item {} 
\sphinxAtStartPar
\sphinxstylestrong{64} игр из представленных \sphinxstylestrong{121} \sphinxhyphen{} практически не сжимаются, экономия места достигает всего 0\sphinxhyphen{}10\%.

\item {} 
\sphinxAtStartPar
\sphinxstylestrong{33} игр из представленных \sphinxstylestrong{121} \sphinxhyphen{} сжимаются с низкой эффективностью, экономия места составляет 11\sphinxhyphen{}40\%.

\item {} 
\sphinxAtStartPar
\sphinxstylestrong{22} игры из представленных \sphinxstylestrong{121} \sphinxhyphen{} сжимаются со средней эффективностью, экономия места составляет 41\sphinxhyphen{}70\%.

\item {} 
\sphinxAtStartPar
\sphinxstylestrong{2} игры из представленных \sphinxstylestrong{121} \sphinxhyphen{} сжимаются хорошо, экономия места составляет 71\sphinxhyphen{}90\%.

\item {} 
\sphinxAtStartPar
Кэш шейдеров, который собирается и хранится на диске в Steam (при включении данной функции) сжимается незначительно \sphinxhyphen{} менее 1\% экономии.

\item {} 
\sphinxAtStartPar
С учетом разницы в экономии места порядка \sphinxstylestrong{3 GB} между максимальной степенью сжатия \sphinxcode{\sphinxupquote{15}} и рекомендуемой для Btrfs \sphinxhyphen{} \sphinxcode{\sphinxupquote{3}},
и значительного падения скорости выполнения сжатия, можно отметить, что использование степени сжатия выше \sphinxcode{\sphinxupquote{3}} выглядит крайне сомнительно.

\end{itemize}

\sphinxstepscope


\chapter{Кастомные ядра}
\label{\detokenize{source/custom-kernels:custom-kernels}}\label{\detokenize{source/custom-kernels:id1}}\label{\detokenize{source/custom-kernels::doc}}
\sphinxAtStartPar
Прежде чем мы начнем, хотелось бы прояснить такой вопрос: "А зачем вообще нужны эти кастомные ядра?".
Чтобы дать ответ на данный вопрос, стоит понимать, что ядро Linux является определенным универсальным стандартом в мире
операционных систем, которое одинаково подходит как для домашнего ПК, ноутбука, телефона, так и для сервера, роутера, микро\sphinxhyphen{}контроллера.
То есть, ядро по умолчанию является швейцарским ножом, позволяющим применять себя в разных задачах,
но при этом не быть наилучшим в чём\sphinxhyphen{}то конкретном. Нет, это не значит что на ванильном ядре вы не сможете запустить какую\sphinxhyphen{}то игру скажем через Wine или Proton,
но такой опыт не будет лучшим, т.к. за такую многопрофильность ядру приходится платить меньшей производительностью в определенных задачах.
Кастомные же ядро подразумевает определенную заточенность под что\sphinxhyphen{}то конкретное,
делая упор на что\sphinxhyphen{}то одно. В нашем случае это производительность и игры, а также улучшение опыта использования Linux на домашнем
ПК или ноутбуке. В этом нам и помогут нижеперечисленные ядра вместе с их правильной настройкой.

\sphinxAtStartPar
И ещё просим вас заранее установить стабильное linux\sphinxhyphen{}lts ядро.
В случае возникновения проблем вы всегда сможете откатиться на эту версию ядра.

\sphinxAtStartPar
Проверка ядра используемого в данный момент осуществляется следующей командой: \sphinxcode{\sphinxupquote{uname \sphinxhyphen{}r}}.

\index{kernel@\spxentry{kernel}}\ignorespaces 

\section{Выбор ядра}
\label{\detokenize{source/custom-kernels:kernel-choose}}\label{\detokenize{source/custom-kernels:index-0}}\label{\detokenize{source/custom-kernels:id2}}
\sphinxAtStartPar
Хотелось бы отметить, что универсального рецепта по сборке "лучшего ядра" не существует,
и каждый выбирает то, что конкретно для него лучше работает. Поэтому мы рекомендуем вам
установить и попробовать каждое ядро из предложенных ниже на своем железе, проводя тесты
в любимых играх/востребованных задачах.

\index{kernel@\spxentry{kernel}}\index{zen@\spxentry{zen}}\index{native\sphinxhyphen{}compilation@\spxentry{native\sphinxhyphen{}compilation}}\ignorespaces 

\subsection{linux\sphinxhyphen{}zen}
\label{\detokenize{source/custom-kernels:linux-zen}}\label{\detokenize{source/custom-kernels:index-1}}\label{\detokenize{source/custom-kernels:id3}}
\sphinxAtStartPar
Zen ядро \sphinxhyphen{} это плод коллективных усилий по объединению патчей улучшающих опыт домашнего использования
Linux, но при этом работающих стабильно и не ломающих совместимость с теми или иными вещами.

\sphinxAtStartPar
Это отличный выбор для неискушенного пользователя, что не ставит задачи в получении максимальной
производительности и покорении максимальной планки FPS. Рекомендуется установить всем, кто
не хочет сильно париться с компиляцией и настройкой других ядер.
Ядро можно установить из репозиториев, ибо оно имеет официальную поддержку дистрибутивом.

\sphinxAtStartPar
Из основных улучшений:
\begin{itemize}
\item {} 
\sphinxAtStartPar
Улучшено поведение планировщика CFS для значительного снижения задержек в несколько раз.

\item {} 
\sphinxAtStartPar
Задействован эффективный алгоритм сжатия LZ4 для файла подкачки через zswap.

\item {} 
\sphinxAtStartPar
Используется планировщик ввода/ввода (IO) BFQ, имеющий более низкие задержки
и бОльшую пропускную способность (Подробнее \sphinxhyphen{} \sphinxhref{https://www.kernel.org/doc/html/latest/block/bfq-iosched.html\#when-may-bfq-be-useful}{тут}).

\item {} 
\sphinxAtStartPar
Ядро приспособлено к сохранению качества отклика системы при высокой нагрузке.

\end{itemize}

\sphinxAtStartPar
\sphinxstylestrong{I. Установка}

\begin{sphinxVerbatim}[commandchars=\\\{\}]
sudo pacman \PYGZhy{}S linux\PYGZhy{}zen linux\PYGZhy{}zen\PYGZhy{}headers

\PYG{c+c1}{\PYGZsh{} Если у вас не GRUB \PYGZhy{} используйте команду обновления вашего загрузчика}
sudo grub\PYGZhy{}mkconfig \PYGZhy{}o /boot/grub/grub.cfg
\end{sphinxVerbatim}

\sphinxAtStartPar
\sphinxstylestrong{II. Установка (нативная компиляция)}

\begin{sphinxVerbatim}[commandchars=\\\{\}]
git clone https://aur.archlinux.org/linux\PYGZhy{}zen\PYGZhy{}git.git
\PYG{n+nb}{cd} linux\PYGZhy{}zen\PYGZhy{}git
makepkg \PYGZhy{}sric
sudo grub\PYGZhy{}mkconfig \PYGZhy{}o /boot/grub/grub.cfg
\end{sphinxVerbatim}

\index{kernel@\spxentry{kernel}}\index{liquorix@\spxentry{liquorix}}\index{lqx@\spxentry{lqx}}\index{native\sphinxhyphen{}compilation@\spxentry{native\sphinxhyphen{}compilation}}\ignorespaces 

\subsection{liquorix}
\label{\detokenize{source/custom-kernels:liquorix}}\label{\detokenize{source/custom-kernels:linux-lqx}}\label{\detokenize{source/custom-kernels:index-2}}
\sphinxAtStartPar
linux\sphinxhyphen{}lqx \sphinxhyphen{} является по сути тем же Zen, но заточенным под Debian системы.
Несмотря на это, авторы ARU считают его лучшим ядром на Arch Linux, по крайне мере для процессоров Intel.
При нативной компиляции идеально подходит для игр в связке с wine\sphinxhyphen{}tkg.

\sphinxAtStartPar
Содержит множество изменений не вошедших в Zen (в виду разных причин):
\begin{itemize}
\item {} 
\sphinxAtStartPar
Планировщик PDS по умолчанию (На выбор предоставляются также BMQ и CFS), имеющий лучшие результаты в повседневных задачах.

\item {} 
\sphinxAtStartPar
Алгоритм TCP BBR2 обеспечивающий более высокую скорость работы и максимизирует пропускную способность

\item {} 
\sphinxAtStartPar
Multigenerational LRU для сохранения качества отклика в условиях нехватки оперативной памяти.

\item {} 
\sphinxAtStartPar
1000Hz тактовая частота по умолчанию для более точного планирования с минимумом зависаний.

\item {} 
\sphinxAtStartPar
Предпочтительно жёсткое вытеснение процессов из очереди. Это гарантирует вам то, что не один процесс не сможет
выжрать все процессорное время (дать системе зависнуть).

\item {} 
\sphinxAtStartPar
Содержит минимальное количество отладочных функций.

\item {} 
\sphinxAtStartPar
Другие внутренние изменения ядра.

\end{itemize}

\sphinxAtStartPar
При этом наследует все изменения из Zen.

\sphinxAtStartPar
\sphinxstylestrong{I. Установка (бинарные пакеты)}

\begin{sphinxVerbatim}[commandchars=\\\{\}]
sudo pacman\PYGZhy{}key \PYGZhy{}\PYGZhy{}keyserver hkps://keyserver.ubuntu.com \PYGZhy{}\PYGZhy{}recv\PYGZhy{}keys 9AE4078033F8024D
sudo pacman\PYGZhy{}key \PYGZhy{}\PYGZhy{}lsign\PYGZhy{}key 9AE4078033F8024D      \PYG{c+c1}{\PYGZsh{} Добавляем GPG ключ}
sudo nano /etc/pacman.conf

\PYG{c+c1}{\PYGZsh{} Прописываем эти две строчки в файл}
\PYG{o}{[}liquorix\PYG{o}{]}
\PYG{n+nv}{Server} \PYG{o}{=} https://liquorix.net/archlinux/\PYG{n+nv}{\PYGZdl{}repo}/\PYG{n+nv}{\PYGZdl{}arch}
\end{sphinxVerbatim}

\noindent\sphinxincludegraphics{{custom-kernels-16}.png}

\begin{sphinxVerbatim}[commandchars=\\\{\}]
sudo pacman \PYGZhy{}Suuyy
sudo pacman \PYGZhy{}S linux\PYGZhy{}lqx linux\PYGZhy{}lqx\PYGZhy{}headers
sudo grub\PYGZhy{}mkconfig \PYGZhy{}o /boot/grub/grub.cfg
\end{sphinxVerbatim}

\sphinxAtStartPar
Вариант установки I рекомендуется если не хотите компилировать,
но тогда производительность будет хуже чем у аналогичного скомпилированного ядра.

\sphinxAtStartPar
\sphinxstylestrong{II. Установка и настройка}

\sphinxAtStartPar
В этом случае мы настроим ядро и выполним его нативную\sphinxhyphen{}компиляцию.
Тонкая насторйка ядра позволит дать ещё больше производительности и может
ускорить сам процесс сборки.

\begin{sphinxVerbatim}[commandchars=\\\{\}]
git clone https://aur.archlinux.org/linux\PYGZhy{}lqx.git                 \PYG{c+c1}{\PYGZsh{} Скачивание исходников.}
\PYG{n+nb}{cd} linux\PYGZhy{}lqx                                                      \PYG{c+c1}{\PYGZsh{} Переход в linux\PYGZhy{}lqx}
gpg \PYGZhy{}\PYGZhy{}keyserver keyserver.ubuntu.com \PYGZhy{}\PYGZhy{}recv\PYGZhy{}keys 38DBBDC86092693E \PYG{c+c1}{\PYGZsh{} GPG ключ}
sed \PYGZhy{}i \PYG{l+s+s1}{\PYGZsq{}s/\PYGZus{}makenconfig=/\PYGZus{}makenconfig=y/\PYGZsq{}} PKGBUILD                 \PYG{c+c1}{\PYGZsh{} Включаем ручную настройку}
makepkg \PYGZhy{}sric
\end{sphinxVerbatim}

\sphinxAtStartPar
После некоторого времени с началом компиляции перед вами предстанет окно с настройкой ядра.
Подробные инструкции и рекомендации по настройке вы найдете в следующем разделе.

\noindent\sphinxincludegraphics{{lqx-menunconfig}.png}

\index{kernel@\spxentry{kernel}}\index{xanmod@\spxentry{xanmod}}\index{native\sphinxhyphen{}compilation@\spxentry{native\sphinxhyphen{}compilation}}\ignorespaces 

\subsection{Xanmod}
\label{\detokenize{source/custom-kernels:xanmod}}\label{\detokenize{source/custom-kernels:linux-xanmod}}\label{\detokenize{source/custom-kernels:index-3}}
\sphinxAtStartPar
Альтернатива liquorix, так же нацеленная на оптимизацию под игрушки и повышение плавности работы системы.
Новомодное ядро, которое включает в себе часть уже описанных выше изменений из zen/lqx. Помимо прочего имеет:
\begin{itemize}
\item {} 
\sphinxAtStartPar
Улучшенный метод обработки TCP пакетов (BBRv2)

\item {} 
\sphinxAtStartPar
Частично включает в себя патчи от Clear Linux (так же как и linux\sphinxhyphen{}lqx)

\item {} 
\sphinxAtStartPar
WineSync, альтернатива Fsync, ещё одна реализация синхронизации NT примитов для Wine, но вынесенная в качестве отдельного модуля.
Для остальных ядер может быть установлена через AUR пакет \sphinxhref{https://aur.archlinux.org/packages/winesync-dkms}{winesync\sphinxhyphen{}dkms}.

\item {} 
\sphinxAtStartPar
Высокая пропускная способность устройств ввода/вывода.

\item {} 
\sphinxAtStartPar
Улучшения систем кэширования и управления памятью.

\end{itemize}

\sphinxAtStartPar
Полный список включаемых в него патчей вы можете найти здесь: \sphinxurl{https://github.com/xanmod/linux-patches}

\begin{sphinxadmonition}{attention}{Внимание:}
\sphinxAtStartPar
Не рекомендуется обладателям процессоров Intel, т.к. возможно все ещё имеет нерешенную проблему со сбросом частот процессора от данного производителя (\sphinxurl{https://forum.xanmod.org/thread-3800.html})
\end{sphinxadmonition}

\sphinxAtStartPar
\sphinxstylestrong{I. Установка (компиляция)}:

\begin{sphinxVerbatim}[commandchars=\\\{\}]
git clone https://aur.archlinux.org/linux\PYGZhy{}xanmod.git                    \PYG{c+c1}{\PYGZsh{} Скачивание исходников.}
\PYG{n+nb}{cd} linux\PYGZhy{}xanmod                                                         \PYG{c+c1}{\PYGZsh{} Переход в linux\PYGZhy{}xanmod}
gpg \PYGZhy{}\PYGZhy{}keyserver hkp://keyserver.ubuntu.com \PYGZhy{}\PYGZhy{}recv\PYGZhy{}keys 38DBBDC86092693E \PYG{c+c1}{\PYGZsh{} GPG ключ}
\PYG{n+nb}{export} \PYG{n+nv}{\PYGZus{}makenconfig}\PYG{o}{=}y \PYG{n+nv}{\PYGZus{}use\PYGZus{}numa}\PYG{o}{=}n \PYG{n+nv}{use\PYGZus{}tracers}\PYG{o}{=}n \PYG{n+nv}{\PYGZus{}compiler}\PYG{o}{=}clang         \PYG{c+c1}{\PYGZsh{} Включаем ручную настройку}
makepkg \PYGZhy{}sric                                                           \PYG{c+c1}{\PYGZsh{} Сборка и установка}
\end{sphinxVerbatim}

\sphinxAtStartPar
После некоторого времени с начала сборки у вас должно появится окно с ручной настройкой ядра.
Этот процесс мы подробнее рассмотрим в следующей главе.

\noindent\sphinxincludegraphics{{xanmod-menunconfig}.png}

\index{kernel@\spxentry{kernel}}\index{linux\sphinxhyphen{}tkg@\spxentry{linux\sphinxhyphen{}tkg}}\index{native\sphinxhyphen{}compilation@\spxentry{native\sphinxhyphen{}compilation}}\ignorespaces 

\subsection{linux\sphinxhyphen{}tkg}
\label{\detokenize{source/custom-kernels:linux-tkg}}\label{\detokenize{source/custom-kernels:index-4}}\label{\detokenize{source/custom-kernels:id5}}
\sphinxAtStartPar
Является альтернативой всем трем ядрам выше,
что предоставляет возможность собрать ядро с набором множества патчей на улучшение производительности в игрушках (Futex2, Zenify).
Предоставляет выбор в сборке ядра с разными планировщиками.
Грубо говоря, это ядро сборная солянка из всех остальных ядер с большим набором патчей.

\sphinxAtStartPar
\sphinxstylestrong{I. Установка и настройка}:

\begin{sphinxVerbatim}[commandchars=\\\{\}]
git clone https://github.com/Frogging\PYGZhy{}Family/linux\PYGZhy{}tkg.git
\PYG{n+nb}{cd} linux\PYGZhy{}tkg
\end{sphinxVerbatim}

\sphinxAtStartPar
Есть две возможности предварительной настройки linux\sphinxhyphen{}tkg: либо через редактирование файла \sphinxstyleemphasis{customization.cfg},
либо через терминал по ходу процесса установки.
Мы выбираем первое и отредактируем \sphinxstyleemphasis{customization.cfg}:

\begin{sphinxVerbatim}[commandchars=\\\{\}]
nano customization.cfg
\end{sphinxVerbatim}

\sphinxAtStartPar
Итак, настройка здесь достаточно обширная поэтому мы будем останавливаться только на интересующих нас настройках:

\sphinxAtStartPar
\sphinxcode{\sphinxupquote{\_version="5.17"}} \sphinxhyphen{} Здесь выбираем версию ядра которую мы хотим установить.
Выбирайте самую последнюю из доступных.

\sphinxAtStartPar
\sphinxcode{\sphinxupquote{\_modprobeddb="false"}} \sphinxhyphen{} Опция отвечающая за сборку мини\sphinxhyphen{}ядра.
Подробнее о нем вы можете узнать в соответствующем разделе.
Если хотите собрать мини\sphinxhyphen{}ядро \sphinxhyphen{} пишите \sphinxstyleemphasis{"true"}, если нет \sphinxhyphen{} \sphinxstyleemphasis{"false"}.

\sphinxAtStartPar
\sphinxcode{\sphinxupquote{\_menuconfig="2"}} \sphinxhyphen{} Выбор настройки ядра через menuconfg/xconfig/nconfig.
Рекомендуется выбрать \sphinxstyleemphasis{"2"} чтобы перед сборкой можно было выполнить непосредственную настройку ядра через menunconfig как мы уже делали ранее с liquorix.

\sphinxAtStartPar
\sphinxcode{\sphinxupquote{\_cpusched="pds"}} \sphinxhyphen{} Выбор CPU планировщика ядра.
К выбору предоставляется довольно много планировщиков, но мы советуем обратить ваше внимание только на некоторых из них:
"pds",  "bmq", "cacule", "cfs" (дефолтный для ванильного ядра).
По некоторым данным, PDS дает больше FPS, а CacULE дает лучшие задержки по времени кадра (плавность).
Однако все слишком ситуативно чтобы выбрать из них лучшего, в каких\sphinxhyphen{}то играх/задачах будет выигрывать PDS, а в каких\sphinxhyphen{}то CaCULE и так далее.

\sphinxAtStartPar
Рекомендуется попробовать PDS или CacULE.

\sphinxAtStartPar
\sphinxcode{\sphinxupquote{\_rr\_interval="default"}} \sphinxhyphen{} Задает продолжительность удержания двумя задачами одинакового приоритета.
Рекомендуемое значение слишком зависит от выбранного планировщика, поэтому лучше всего задавайте \sphinxstyleemphasis{"default"}.

\sphinxAtStartPar
\sphinxcode{\sphinxupquote{\_default\_cpu\_gov="performance"}} \sphinxhyphen{} Выбирает режим по умолчанию в котором будет масштабироваться частота процессора.
Рекомендуется \sphinxstyleemphasis{"performance"} чтобы процессор по умолчанию работал в режиме высокой производительности.

\sphinxAtStartPar
\sphinxcode{\sphinxupquote{\_aggressive\_ondemand="false"}} \sphinxhyphen{} Задает агрессивное применение динамического управления частотой процессора по необходимости в выполняемой задаче,
обеспечивая тем самым энергоэффективность.
Но т.к. выше мы уже закрепили режим масштабирования "performance", то мы можем отключить этот параметр.
Однако пользователи ноутбуков могут оставить этот параметр включенным.

\sphinxAtStartPar
\sphinxcode{\sphinxupquote{\_disable\_acpi\_cpufreq="true"}} \sphinxhyphen{} Отключает универсальный acpi\_freq драйвер масштабирования частоты процессора в угоду фирменному драйверу Intel/AMD процессоров
что имеют лучшую производительность по сравнению с acpi\_freq.
Выбирайте значение по собственному усмотрению со знанием своего CPU.

\sphinxAtStartPar
\sphinxcode{\sphinxupquote{\_sched\_yield\_type="0"}} \sphinxhyphen{} Настраивает выполнение освобождения процесса от потребления процессорного времени путем его переноса в конец очереди выполнения процессов.
Рекомендуемое значение для лучшей производительности \sphinxhyphen{} \sphinxstyleemphasis{"0"}, т.е. не осуществлять перенос в конец очереди для освобождения процесса.

\sphinxAtStartPar
\sphinxcode{\sphinxupquote{\_tickless="0"}} \sphinxhyphen{} Рекомендуется выбирать статические тики таймера ядра.

\sphinxAtStartPar
\sphinxcode{\sphinxupquote{\_timer\_freq="1000"}} \sphinxhyphen{} Задает частоту таймера.
Рекомендуется 1000 для лучшей отзывчивости системы на домашнем ПК или ноутбуке.

\sphinxAtStartPar
\sphinxcode{\sphinxupquote{\_fsync="true"}} \sphinxhyphen{} Задействует поддержку ядром замены Esync от компании Valve \sphinxhyphen{} Fsync.
Обязательно к включению (\sphinxstyleemphasis{"true"}) для лучшей производительности в играх.

\sphinxAtStartPar
\sphinxcode{\sphinxupquote{\_futex2="true"}} \sphinxhyphen{} Осуществляет использование нового, экспериментального futex2 вызова что может дать лучшую производительность для игрушек запускаемых через Wine/Proton. Для обычных ядер поддержка Futex2 включена начиная с версии 5.16+.

\sphinxAtStartPar
\sphinxcode{\sphinxupquote{\_winesync="false"}} \sphinxhyphen{} Ещё одна замена esync, но уже от разработчиков Wine.

\sphinxAtStartPar
\sphinxcode{\sphinxupquote{\_zenify="true"}} \sphinxhyphen{} Применяет твики Zen и Liquorix для улучшения производительности в играх.
Настоятельно рекомендуется к включению.

\sphinxAtStartPar
\sphinxcode{\sphinxupquote{\_complierlevel="1"}} \sphinxhyphen{} Задает степень оптимизации ядра во время сборки.
Лучше всего выбирать \sphinxstyleemphasis{"1"}, т.е. сборку с \sphinxhyphen{}O2 флагом (высокая производительность).

\sphinxAtStartPar
\sphinxcode{\sphinxupquote{\_processor\_opt="native\_intel"}} \sphinxhyphen{} С учетом какой архитектуры процессора собирать ядро.
Настоятельно рекомендуется указать здесь либо архитектуру непосредственно вашего процессора (К примеру: "skylake"),
либо фирму производитель, где для Intel это \sphinxhyphen{} \sphinxstyleemphasis{"native\_intel"}, для AMD \sphinxhyphen{} \sphinxstyleemphasis{"native\_amd"}.

\sphinxAtStartPar
\sphinxcode{\sphinxupquote{\_ftracedisable="true"}} \sphinxhyphen{} Отключает лишние трекеры для отладки ядра.

\sphinxAtStartPar
\sphinxcode{\sphinxupquote{\_acs\_override="true"}} \sphinxhyphen{} Включает патч на разделение сгруппированных PCI устройств в IOMMU, которые могут понадобиться вам отдельно.
По умолчанию есть в linux\sphinxhyphen{}zen и linux\sphinxhyphen{}lqx.
Подробнее читайте \sphinxhyphen{} \sphinxhref{https://wiki.archlinux.org/title/PCI\_passthrough\_via\_OVMF\#Bypassing\_the\_IOMMU\_groups\_.28ACS\_override\_patch.29}{здесь}.
Советуем включить если в будущем вы хотите выполнить операцию проброса вашей видеокарты в виртуальную машину.

\sphinxAtStartPar
Вот и все. Остальные настройки \sphinxstyleemphasis{customization.cfg} вы можете выбрать по собственному предпочтению.
После того как мы закончили с настройкой, можно перейти непосредственно к сборке и установке ядра::

\begin{sphinxVerbatim}[commandchars=\\\{\}]
makepkg \PYGZhy{}sric \PYG{c+c1}{\PYGZsh{} Сборка и установка linux\PYGZhy{}tkg}
\end{sphinxVerbatim}

\index{kernel@\spxentry{kernel}}\index{linux\sphinxhyphen{}cachyos@\spxentry{linux\sphinxhyphen{}cachyos}}\index{native\sphinxhyphen{}compilation@\spxentry{native\sphinxhyphen{}compilation}}\ignorespaces 

\subsection{linux\sphinxhyphen{}cachyos}
\label{\detokenize{source/custom-kernels:linux-cachyos}}\label{\detokenize{source/custom-kernels:index-5}}\label{\detokenize{source/custom-kernels:id7}}
\sphinxAtStartPar
\sphinxhref{https://github.com/CachyOS/linux-cachyos}{linux\sphinxhyphen{}cachyos} \sphinxhyphen{} добротная альтернатива
всем остальным ядрам, также нацеленная на максимальную производительность вашей системы.
По субъективным ощущениям автора работает лучше чем Xanmod и TKG. Предлагает на выбор множество
планировщиков CPU. Сочетает в себе патчи которые уже были описаны для других ядер. А именно:
\begin{itemize}
\item {} 
\sphinxAtStartPar
Улучшенный планировщик ввода/вывода BFQ

\item {} 
\sphinxAtStartPar
Набор патчей LRU для сохранения качества отклика системы в условиях нехватки оперативной памяти.

\item {} 
\sphinxAtStartPar
Содержит новейшие исправления для Btrfs/Zstd

\item {} 
\sphinxAtStartPar
Заточен для сборки через LLVM/Clang (более подробно это описывается в последующем разделе)

\item {} 
\sphinxAtStartPar
Алгоритм для обработки сетевых пакетов BBRv2

\item {} 
\sphinxAtStartPar
Модули для поддержки эмуляции Android (Anbox)

\item {} 
\sphinxAtStartPar
Набор патчей от Clear Linux

\item {} 
\sphinxAtStartPar
И некоторые собственные настройки для ядра

\end{itemize}

\sphinxAtStartPar
Отдельным плюсом является быстрая обновляемость и оперативные исправления ошибок,
чем к сожалению не всегда может похвастаться linux\sphinxhyphen{}tkg.

\sphinxAtStartPar
\sphinxstylestrong{Установка I.}

\sphinxAtStartPar
А вот тут не все так просто, ибо прежде чем мы начнем, стоит оговориться,
что у этого ядра есть вариации с пятью разными планировщиками. Это: CFS,
BMQ, PDS, TT и BORE (есть ещё другие, но они менее активно сопровождаемые).
Автор рекомендует остановиться на BORE и PDS, как на наиболее проверенных
решениях. Но вы можете попробовать и другие варианты. Далее я буду выполнять
команды для установки ядра с BORE, но соответственно вы можете писать вместо bore
любой другой.

\begin{sphinxVerbatim}[commandchars=\\\{\}]
git clone https://github.com/CachyOS/linux\PYGZhy{}cachyos.git  \PYG{c+c1}{\PYGZsh{} Скачиваем исходники}
\PYG{n+nb}{cd} linux\PYGZhy{}cachyos\PYGZhy{}bore \PYG{c+c1}{\PYGZsh{} Если хотите использовать PDS, то соответственно пишите cd linux\PYGZhy{}cachyos\PYGZhy{}pds по аналогии}
sed \PYGZhy{}i \PYG{l+s+s1}{\PYGZsq{}s/\PYGZus{}use\PYGZus{}llvm\PYGZus{}lto=/\PYGZus{}use\PYGZus{}llvm\PYGZus{}lto=full/\PYGZsq{}} PKGBUILD \PYG{c+c1}{\PYGZsh{} Включаем сборку через Clang. Подробнее об этом в последующем разделе}
makepkg \PYGZhy{}sric
\end{sphinxVerbatim}

\sphinxAtStartPar
Данное ядро немного умнее других, поэтому определяет архитектуру вашего процессора
и автоматически указывает компилятору собирать себя именно под неё. Т.е. нативная
компиляция здесь есть по умолчанию, так что в принципе вы можете не сильно заморачиваться
с настройкой ядра или вовсе пропустить данный шаг. Но все таки, если у вас есть собственные
предпочтения относительно определенных параметров вашего ядра, то вы всегда можете включить ручную настройку
через menuconfig использую опцию PKGBUILD: \sphinxcode{\sphinxupquote{sed \sphinxhyphen{}i 's/\_makenconfig=/\_makenconfig=y/' PKGBUILD}}
(подобная команда введена для удобства, вы можете сделать то же самое через любой удобный вам
текстовый редактор, отредактировав файл PKGBUILD).

\sphinxAtStartPar
\sphinxstylestrong{Установка II (бинарные пакеты)}

\sphinxAtStartPar
Бинарную версию ядра можно получить либо через подключение стороннего репозитория,
либо скачав уже готовый пакет опять с того же репозитория, но не подключая его.
Со вторым всё просто, переходите на данный сайт: \sphinxurl{https://mirror.cachyos.org/repo/x86\_64/cachyos/} и ищите
версию ядра которая вам по вкусу. Потом устанавливаете через \sphinxcode{\sphinxupquote{sudo pacman \sphinxhyphen{}U}} (в конце пишете путь до скаченного файла).

\sphinxAtStartPar
Первый вариант также позволяет получать последние обновления, поэтому он предпочтительней:

\begin{sphinxVerbatim}[commandchars=\\\{\}]
sudo pacman\PYGZhy{}key \PYGZhy{}\PYGZhy{}recv\PYGZhy{}keys F3B607488DB35A47 \PYGZhy{}\PYGZhy{}keyserver keyserver.ubuntu.com
sudo pacman\PYGZhy{}key \PYGZhy{}\PYGZhy{}lsign\PYGZhy{}key F3B607488DB35A47
sudo pacman \PYGZhy{}U \PYG{l+s+s1}{\PYGZsq{}https://mirror.cachyos.org/repo/x86\PYGZus{}64/cachyos/cachyos\PYGZhy{}keyring\PYGZhy{}2\PYGZhy{}1\PYGZhy{}any.pkg.tar.zst\PYGZsq{}} \PYG{l+s+s1}{\PYGZsq{}https://mirror.cachyos.org/repo/x86\PYGZus{}64/cachyos/cachyos\PYGZhy{}mirrorlist\PYGZhy{}10\PYGZhy{}1\PYGZhy{}any.pkg.tar.zst\PYGZsq{}} \PYG{l+s+s1}{\PYGZsq{}https://mirror.cachyos.org/repo/x86\PYGZus{}64/cachyos/cachyos\PYGZhy{}v3\PYGZhy{}mirrorlist\PYGZhy{}10\PYGZhy{}1\PYGZhy{}any.pkg.tar.zst\PYGZsq{}}
\end{sphinxVerbatim}

\sphinxAtStartPar
Стоит учитывать, что у данного репозитория есть развилка по архитектурам. То есть он одновременно
поддерживает и x86\_64, и x86\_64v3. В чем разница? В том, что x86\_64v3 чуть более оптимизирован
для современных процессоров и использует инструкции, которые нельзя применить к обычной x86\_64 в угоду
совместимости.

\sphinxAtStartPar
Поэтому сначала проверим, поддерживает ли ваш процессора архитектуру x86\_64v3:

\begin{sphinxVerbatim}[commandchars=\\\{\}]
/lib/ld\PYGZhy{}linux\PYGZhy{}x86\PYGZhy{}64.so.2 \PYGZhy{}\PYGZhy{}help \PYG{p}{|} grep \PYG{l+s+s2}{\PYGZdq{}x86\PYGZhy{}64\PYGZhy{}v3 (supported, searched)\PYGZdq{}}
\end{sphinxVerbatim}

\sphinxAtStartPar
Если вывод команды НЕ пустой, то ваш процессор поддерживает x86\_64v3.

\sphinxAtStartPar
Пропишем репозиторий в /etc/pacman.conf:

\begin{sphinxVerbatim}[commandchars=\\\{\}]
sudo nano /etc/pacman.conf
\end{sphinxVerbatim}

\sphinxAtStartPar
Теперь, если у вас ЕСТЬ поддержка x86\_64v3, то пишем следующее:

\begin{sphinxVerbatim}[commandchars=\\\{\}]
\PYG{c+c1}{\PYGZsh{} Находим данную строку и редактируем:}
\PYG{n+nv}{Architecture} \PYG{o}{=} x86\PYGZus{}64 x86\PYGZus{}64\PYGZus{}v3

\PYG{c+c1}{\PYGZsh{} Спускаемся в самый низ файла и пишем:}
\PYG{o}{[}cachyos\PYGZhy{}v3\PYG{o}{]}
\PYG{n+nv}{Include} \PYG{o}{=} /etc/pacman.d/cachyos\PYGZhy{}v3\PYGZhy{}mirrorlist
\end{sphinxVerbatim}

\sphinxAtStartPar
Если же нет, то:

\begin{sphinxVerbatim}[commandchars=\\\{\}]
\PYG{c+c1}{\PYGZsh{} Спускаемся в самый низ файла и пишем:}
\PYG{o}{[}cachyos\PYG{o}{]}
\PYG{n+nv}{Include} \PYG{o}{=} /etc/pacman.d/cachyos\PYGZhy{}mirrorlist
\end{sphinxVerbatim}

\sphinxAtStartPar
После этого выполните обновление системы и вы сможете установить бинарное ядро:

\begin{sphinxVerbatim}[commandchars=\\\{\}]
sudo pacman \PYGZhy{}Syu
\end{sphinxVerbatim}

\sphinxAtStartPar
После этого тоже ставим пакет в соответствии с желаемым планировщиком: \sphinxcode{\sphinxupquote{sudo pacman \sphinxhyphen{}S linux\sphinxhyphen{}cachyos}}.
Или \sphinxcode{\sphinxupquote{sudo pacman \sphinxhyphen{}S linux\sphinxhyphen{}cachyos\sphinxhyphen{}bore}}. И так далее.

\index{kernel@\spxentry{kernel}}\index{configure@\spxentry{configure}}\ignorespaces 

\section{Настройка ядра}
\label{\detokenize{source/custom-kernels:manual-kernel-configuration}}\label{\detokenize{source/custom-kernels:index-6}}\label{\detokenize{source/custom-kernels:id8}}
\sphinxAtStartPar
При нативной компиляции ядра обязательным этапом является его настройка.
Хотя и заботливые сопровождающие кастомных ядер обычно уже заранее выполняют
всю работу за вас, есть пара моментов на которых стоит остановиться.

\sphinxAtStartPar
После начала компиляции через некоторое время перед вами должно появится меню настройки ядра.
Перемещение между пунктами в нем осуществляется стрелками на клавиатуре, переход в
следующий раздел через клавишу \sphinxstyleemphasis{Enter}, а выход из него через \sphinxstyleemphasis{Esc}.

\sphinxAtStartPar
Далее необходимо следовать графической инструкции.

\sphinxAtStartPar
\sphinxstylestrong{1.} Для начала выберем одну из важнейших настроек. Это выбор архитектуры процессора под которую будет компилироваться ядро.
По умолчанию выбрана \sphinxstyleemphasis{Generic}, то есть такое ядро будет собираться под абстрактную x86 архитектуру и при этом будет совместимо
с любым процессором, хоть AMD, хоть Intel. Главным же преимуществом именно нативной компиляции любого ПО состоит в задействовании
максимума производительности конкретно под вашу архитектуру процессора, с использованием всех доступных ему инструкций. А в случае
с ядром это особенно важно, ибо ядро это сердце операционной системы, и если его нативно собрать под себя, то мы получаем существенный
прирост и отличный отклик системы. Поэтому начиная с главного окна настройки перейдите в раздел \sphinxstyleemphasis{"Processor type and features"} и затем
в \sphinxstyleemphasis{"Processor family"}. Здесь выберите либо \sphinxstyleemphasis{"Intel\sphinxhyphen{}native optimizations"} если у вас процессор Intel, либо \sphinxstyleemphasis{"AMD\sphinxhyphen{}native optimizations"} если
у вас процессор AMD, как это показано на скриншотах ниже.

\sphinxAtStartPar
\sphinxstylestrong{1.1}

\noindent\sphinxincludegraphics{{processor-type-and-features-entry}.png}

\sphinxAtStartPar
\sphinxstylestrong{1.2}

\noindent\sphinxincludegraphics{{processor-family}.png}

\sphinxAtStartPar
\sphinxstylestrong{1.3}

\noindent\sphinxincludegraphics{{processor-family-choice}.png}

\sphinxAtStartPar
(\sphinxstylestrong{Важно}: автор выбрал здесь Intel\sphinxhyphen{}native, но \sphinxstylestrong{если у вас процессор от AMD выбирайте только AMD\sphinxhyphen{}native} )

\sphinxAtStartPar
\sphinxstylestrong{2.} Изменим поведение таймера ядра.
Дело в том, что ядро может осуществлять прерывания для перепланирования задач процессора либо статически через частоту N (один тик), либо динамически.
Динамический таймер работает только тогда, когда процессор находится в работе, т. е. когда процессор простаивает
таймер прерываний останавливает свою работу (из\sphinxhyphen{}за ненадобности). Существует также и вариант динамического таймера
когда тики не происходят даже тогда, когда процессор чем\sphinxhyphen{}то занят.

\sphinxAtStartPar
Собственно выбор этих трех вариантов и дан нам на скриншотах ниже, где:
\begin{itemize}
\item {} 
\sphinxAtStartPar
Periodic timer ticks \sphinxhyphen{} осуществление тика статически через частоту N

\item {} 
\sphinxAtStartPar
Idle dynticks system \sphinxhyphen{} прерывания через частоту тика N только тогда, когда процессор чем\sphinxhyphen{}то занят.

\item {} 
\sphinxAtStartPar
Full dynticks system \sphinxhyphen{} прерывания через частоту тика N, но не всегда, даже если процессор чем\sphinxhyphen{}то занят.

\end{itemize}

\sphinxAtStartPar
\sphinxstylestrong{Что из этого выбрать?}

\sphinxAtStartPar
По нашему мнению динамические тики лучше всего выбирать тем людям, которые хотят уменьшить энергопотребление системы.
В том числе всем пользователям ноутбуков/нетбуков посвящается. Обратите внимание, что \sphinxstyleemphasis{Full dynticks system} может одновременно
несколько ухудшить или улучшить производительность в зависимости от железа, но даёт реальные преимущества в экономии энергии.

\sphinxAtStartPar
Однако, если у вас домашний стационарный ПК или вам просто все равно на энергопотребление \sphinxhyphen{} лучше выбрать статические тики (\sphinxstyleemphasis{"Pereodic timer ticks"}),
ибо они потенциально дают более предсказуемый метод планирования прерываний.
Это значит, что теоретически у вас не будет тратиться время на выход таймера из "сна".
И на практике оказывается что периодические тики дают лучшую производительность в играх и мультимедиа.

\sphinxAtStartPar
\sphinxstylestrong{2.1}

\noindent\sphinxincludegraphics{{general-menu}.png}

\sphinxAtStartPar
\sphinxstylestrong{2.2}

\noindent\sphinxincludegraphics{{timer-subsystem-1}.png}

\sphinxAtStartPar
\sphinxstylestrong{2.3}

\noindent\sphinxincludegraphics{{timer-subsystem-2}.png}

\sphinxAtStartPar
\sphinxstylestrong{2.4}

\noindent\sphinxincludegraphics{{timer-subsystem-3}.png}

\sphinxAtStartPar
\sphinxstylestrong{3.} Просим вас удостовериться в значениях частоты вашего таймера.
Это как раз то самое N через которое происходит тик таймера и последующее за ним прерывание.
Рекомендуемое значение для домашнего ПК/Ноутбука это 1000.
Однако если вы имеете многоядерный процессор (12 и более потоков) или какой\sphinxhyphen{}нибудь серверный Intel Xeon,
то вы можете попробовать установить частоту ниже 1000.

\sphinxAtStartPar
\sphinxstylestrong{3.1}

\noindent\sphinxincludegraphics{{processor-type-and-features-entry}.png}

\sphinxAtStartPar
\sphinxstylestrong{3.2}

\noindent\sphinxincludegraphics{{timer-freqency}.png}

\sphinxAtStartPar
\sphinxstylestrong{3.3}

\noindent\sphinxincludegraphics{{timer-freqency-choice}.png}

\sphinxAtStartPar
\sphinxstylestrong{4.} Рекомендуем вам отключать отладочные функции ядра. Они тоже имеют свои накладные расходы и в большинстве случаев
вы ими пользоваться никогда не будете, а на крайний случай у вас должно быть установлено ядро linux\sphinxhyphen{}lts как запасной аэродром.
Для их отключения из главного меню перейдите в \sphinxstyleemphasis{"Kernel Hacking"} и сделайте там все так, как показано на скриншотах:

\begin{sphinxadmonition}{note}{Примечание:}
\sphinxAtStartPar
Обращаем ваше внимание на то, что на некоторых ядрах не все возможные отладочные параметры могут быть отключены.
Например Xanmod не позволяет отключить ряд параметров из списка ниже. Но вы можете ими пренебречь.
\end{sphinxadmonition}

\sphinxAtStartPar
\sphinxstylestrong{4.1}

\noindent\sphinxincludegraphics{{kernel-hacking}.png}

\sphinxAtStartPar
\sphinxstylestrong{4.2}

\noindent\sphinxincludegraphics{{kernel-debugging}.png}

\sphinxAtStartPar
\sphinxstylestrong{5.} Обладателям видеокарт NVIDIA советуем отключить поддержку фирменного фреймбуфера, как бы странно это не звучало.
Это позволит вам избежать проблемы конфликта фреймбуфера ядра и фреймбуфера бинарного драйвера NVIDIA. Сделайте это
как показано ниже:

\sphinxAtStartPar
\sphinxstylestrong{5.1}

\noindent\sphinxincludegraphics{{kernel-device-drivers}.png}

\sphinxAtStartPar
\sphinxstylestrong{5.2}

\noindent\sphinxincludegraphics{{kernel-graphics-support}.png}

\sphinxAtStartPar
\sphinxstylestrong{5.3}

\noindent\sphinxincludegraphics{{kernel-fb-devices-choice}.png}

\sphinxAtStartPar
\sphinxstylestrong{5.4}

\noindent\sphinxincludegraphics{{kernel-fb-devices}.png}

\sphinxAtStartPar
\sphinxstylestrong{5.5}

\noindent\sphinxincludegraphics{{kernel-nvidia-fb}.png}

\index{kernel@\spxentry{kernel}}\index{clang@\spxentry{clang}}\index{lto native\sphinxhyphen{}compilation@\spxentry{lto native\sphinxhyphen{}compilation}}\ignorespaces 

\section{Сборка ядра с помощью Clang + LTO}
\label{\detokenize{source/custom-kernels:clang-lto}}\label{\detokenize{source/custom-kernels:kernel-with-clang-lto}}\label{\detokenize{source/custom-kernels:index-7}}
\sphinxAtStartPar
В разделе \sphinxhref{https://ventureoo.github.io/ARU/source/generic-system-acceleration.html\#clang}{"Общее ускорение системы"}
мы уже говорили о преимуществах сборки пакетов при помощи компилятора Clang вместе с LTO оптимизациями.
Но ядро требует отдельного рассмотрения, ибо те параметры которые мы указали ранее в makepkg.conf не работают для сборки ядра,
и потому по прежнему будут применяться компиляторы GCC.

\sphinxAtStartPar
Чтобы активировать сборку ядра через Clang нужно:
\begin{itemize}
\item {} 
\sphinxAtStartPar
Для ядра linux\sphinxhyphen{}xanmod экспортировать данную переменную окружения перед выполнением команды сборки: \sphinxcode{\sphinxupquote{export \_compiler=clang}}

\item {} 
\sphinxAtStartPar
Для ядра linux\sphinxhyphen{}tkg в конфигурационном файле \sphinxstyleemphasis{customization.cfg} включить параметр \sphinxstyleemphasis{\_compiler="llvm"}
(В том же файле можно настроить применение LTO оптимизаций через параметр \sphinxstyleemphasis{\_lto\_mode}. О режимах LTO читайте далее).

\item {} 
\sphinxAtStartPar
Для всех остальных ядер, устанавливаемых из AUR (в том числе linux\sphinxhyphen{}lqx), нужно просто экспортировать переменные окружения \sphinxstyleemphasis{LLVM=1} и \sphinxstyleemphasis{LLVM\_IAS=1} перед командой сборки:

\begin{sphinxVerbatim}[commandchars=\\\{\}]
\PYG{n+nb}{export} \PYG{n+nv}{LLVM}\PYG{o}{=}\PYG{l+m}{1} \PYG{n+nv}{LLVM\PYGZus{}IAS}\PYG{o}{=}\PYG{l+m}{1} \PYG{c+c1}{\PYGZsh{} Без переменной LLVM\PYGZus{}IAS станет невозможным применение LTO оптимизаций}
makepkg \PYGZhy{}sric            \PYG{c+c1}{\PYGZsh{} Сборка и установка желаемого ядра}
\end{sphinxVerbatim}

\end{itemize}

\sphinxAtStartPar
Теперь перейдем к выбору режима LTO.
Для этого на этапе конфигурации вашего ядра зайдите в \sphinxstyleemphasis{"General architecture\sphinxhyphen{}dependent options"} \sphinxhyphen{}>
\sphinxstyleemphasis{"Link Time Optimization (LTO)"} как показано на изображениях:
\begin{enumerate}
\sphinxsetlistlabels{\arabic}{enumi}{enumii}{}{.}%
\item {} 
\end{enumerate}

\noindent\sphinxincludegraphics{{custom-kernels-17}.png}
\begin{enumerate}
\sphinxsetlistlabels{\arabic}{enumi}{enumii}{}{.}%
\setcounter{enumi}{1}
\item {} 
\end{enumerate}

\noindent\sphinxincludegraphics{{custom-kernels-18}.png}
\begin{enumerate}
\sphinxsetlistlabels{\arabic}{enumi}{enumii}{}{.}%
\setcounter{enumi}{2}
\item {} 
\end{enumerate}

\noindent\sphinxincludegraphics{{custom-kernels-19}.png}

\sphinxAtStartPar
На последнем изображении показано окно выбора режима применения LTO оптимизаций.
Этих режимов всего два:
\begin{enumerate}
\sphinxsetlistlabels{\arabic}{enumi}{enumii}{}{.}%
\item {} 
\sphinxAtStartPar
Полный (Full): использует один поток для линковки, во время сборки медленный и использует больше памяти,
но теоретически имеет больший прирост производительности в работе уже готового ядра.

\item {} 
\sphinxAtStartPar
Тонкий (Thin): работает в несколько потоков, во время сборки быстрее и использует меньше памяти, но может иметь более низкую производительность в итоге чем \sphinxstyleemphasis{Полный (Full)} режим.

\end{enumerate}

\sphinxAtStartPar
Мы рекомендуем использовать \sphinxstyleemphasis{"Полный (Full)"} режим чтобы получить в итоге лучшую производительность.

\begin{sphinxadmonition}{attention}{Внимание:}
\sphinxAtStartPar
Сборка ядра через Clang работает только с версией ядра 5.12 и выше!
\end{sphinxadmonition}

\sphinxAtStartPar
Больше подробностей по теме вы можете найти в данной статье:

\sphinxAtStartPar
\sphinxurl{https://habr.com/ru/company/ruvds/blog/561286/}

\sphinxstepscope


\chapter{Wine / Linux Gaming}
\label{\detokenize{source/linux-gaming:wine-linux-gaming}}\label{\detokenize{source/linux-gaming:linux-gaming}}\label{\detokenize{source/linux-gaming::doc}}
\index{wine@\spxentry{wine}}\index{wine\sphinxhyphen{}builds@\spxentry{wine\sphinxhyphen{}builds}}\index{gaming@\spxentry{gaming}}\ignorespaces 

\section{Основные составляющие}
\label{\detokenize{source/linux-gaming:main-components}}\label{\detokenize{source/linux-gaming:index-0}}\label{\detokenize{source/linux-gaming:id1}}
\sphinxAtStartPar
Переходя к запуску Windows\sphinxhyphen{}игр на Linux\sphinxhyphen{}системах, сто́ит иметь в виду, что никаких эмуляторов Windows на Linux не существует,
и весь запуск осуществляется с помощью сторонней реализации Windows API — Wine/Proton,
а также средств ретрансляции команд DirectX в доступные графические API на Linux (Vulkan, OpenGL) с помощью DXVK или иных ретранслятора кода.

\index{about@\spxentry{about}}\index{wine@\spxentry{wine}}\index{gaming@\spxentry{gaming}}\ignorespaces 

\subsection{Что такое Wine?}
\label{\detokenize{source/linux-gaming:wine}}\label{\detokenize{source/linux-gaming:about-wine}}\label{\detokenize{source/linux-gaming:index-1}}
\sphinxAtStartPar
Wine \sphinxhyphen{} слой совместимости для запуска Windows\sphinxhyphen{}приложений (в том числе игр) из под Linux (Unix\sphinxhyphen{}подобных систем).
Благодаря нему вы по факту сможете поиграть в большинство игр из вашей библиотеки Steam/GOG/Epic Games Store.
Исключением разве что являются игры с встроенными анти\sphinxhyphen{}чит системами, хотя благодаря усилиям Valve,
в ближайшем будущем, вероятно, это уже не будет являться такой большой проблемой.
Конечно, все не так гладко как хотелось бы, ведь для запуска и обеспечения работоспособности многих программ/игр придется ещё изрядно повозиться с его настройкой,
однако сама такая возможность в принципе является незаменимой для Linux пользователей, в частности геймеров.

\index{wine\sphinxhyphen{}builds@\spxentry{wine\sphinxhyphen{}builds}}\index{gaming@\spxentry{gaming}}\ignorespaces 

\subsection{Сборки Wine}
\label{\detokenize{source/linux-gaming:wine-builds}}\label{\detokenize{source/linux-gaming:index-2}}\label{\detokenize{source/linux-gaming:id2}}
\sphinxAtStartPar
Существуют различные сборки Wine.
Подобный зоопарк появился ввиду накопления большого количества различных патчей (сторонних изменений)
которые по какой\sphinxhyphen{}то причине не могут быть приняты в основную ветку разработки Wine.
Кроме того, стоит понимать что, как и в случае с ядрами, обычный Wine это прежде всего свободная реализация Windows API,
которая подразумевает запуск любых Windows приложений. При этом он не заточен конкретно под игры или любой другой софт.
Именно поэтому в том числе и появились такие вещи как Proton от компании Valve,
\sphinxstyleemphasis{являющийся по сути тем же Wine}, но с упором именно на игровую составляющую, исправляющий
многие проблемы обычного Wine связанные с играми.

\sphinxAtStartPar
На текущий момент есть две официальных сборки Wine которые поддерживаются непосредственно разработчиками:
\begin{itemize}
\item {} 
\sphinxAtStartPar
wine \sphinxhyphen{} обычная, стабильная версия, содержащая только проверенные изменения от разработчиков, и которая условно универсальна
для любых приложений.

\item {} 
\sphinxAtStartPar
wine\sphinxhyphen{}staging \sphinxhyphen{} содержащая те изменение которые пока не могут попасть в обычную версию, но которые могут помочь
исправить определенные баги и улучшить работу конкретных программ и частей Wine.

\end{itemize}

\sphinxAtStartPar
Но существуют также много альтернативных сборок основанных на Wine\sphinxhyphen{}staging с упором именно на игры, о них написано далее.

\index{installation@\spxentry{installation}}\index{wine\sphinxhyphen{}staging@\spxentry{wine\sphinxhyphen{}staging}}\index{gaming@\spxentry{gaming}}\index{dependencies@\spxentry{dependencies}}\ignorespaces 

\subsubsection{Установка wine\sphinxhyphen{}staging вместе с зависимостями}
\label{\detokenize{source/linux-gaming:wine-staging}}\label{\detokenize{source/linux-gaming:index-3}}\label{\detokenize{source/linux-gaming:id3}}
\sphinxAtStartPar
Бинарные версии ПО всегда доступны в репозиториях и очень удобны, но они не могут обеспечить достойный уровень производительности.
Для начала советуюм поставить wine\sphinxhyphen{}staging вместе со всеми зависимостями, а уже затем собрать wine\sphinxhyphen{}tkg.

\begin{sphinxVerbatim}[commandchars=\\\{\}]
sudo pacman \PYGZhy{}S wine\PYGZhy{}staging winetricks wine\PYGZhy{}mono giflib lib32\PYGZhy{}giflib libpng lib32\PYGZhy{}libpng libldap lib32\PYGZhy{}libldap gnutls lib32\PYGZhy{}gnutls mpg123 lib32\PYGZhy{}mpg123 openal lib32\PYGZhy{}openal v4l\PYGZhy{}utils lib32\PYGZhy{}v4l\PYGZhy{}utils libpulse lib32\PYGZhy{}libpulse libgpg\PYGZhy{}error lib32\PYGZhy{}libgpg\PYGZhy{}error alsa\PYGZhy{}plugins lib32\PYGZhy{}alsa\PYGZhy{}plugins alsa\PYGZhy{}lib lib32\PYGZhy{}alsa\PYGZhy{}lib libjpeg\PYGZhy{}turbo lib32\PYGZhy{}libjpeg\PYGZhy{}turbo sqlite lib32\PYGZhy{}sqlite libxcomposite lib32\PYGZhy{}libxcomposite libxinerama lib32\PYGZhy{}libgcrypt libgcrypt lib32\PYGZhy{}libxinerama ncurses lib32\PYGZhy{}ncurses opencl\PYGZhy{}icd\PYGZhy{}loader lib32\PYGZhy{}opencl\PYGZhy{}icd\PYGZhy{}loader libxslt lib32\PYGZhy{}libxslt libva lib32\PYGZhy{}libva gtk3 lib32\PYGZhy{}gtk3 gst\PYGZhy{}plugins\PYGZhy{}base\PYGZhy{}libs lib32\PYGZhy{}gst\PYGZhy{}plugins\PYGZhy{}base\PYGZhy{}libs vulkan\PYGZhy{}icd\PYGZhy{}loader lib32\PYGZhy{}vulkan\PYGZhy{}icd\PYGZhy{}loader
\end{sphinxVerbatim}

\index{installation@\spxentry{installation}}\index{wine@\spxentry{wine}}\index{wine\sphinxhyphen{}builds@\spxentry{wine\sphinxhyphen{}builds}}\ignorespaces 

\subsection{Альтернативные сборки Wine}
\label{\detokenize{source/linux-gaming:alternative-wine-builds}}\label{\detokenize{source/linux-gaming:index-4}}\label{\detokenize{source/linux-gaming:id4}}
\sphinxAtStartPar
По умолчанию обычные сборки Wine недостаточно хорошо заточены для комфортной игры ввиду их универсальности,
т.к. это все таки свободная реализация WinAPI в Linux и она не обязана использоваться только для запуска игр из под Windows в Linux.
Но существуют также альтернативные сборки Wine, с большим количеством различных патчей и улучшений, нацеленных в основном как раз на игры.

\index{installation@\spxentry{installation}}\index{wine@\spxentry{wine}}\index{wine\sphinxhyphen{}builds@\spxentry{wine\sphinxhyphen{}builds}}\index{wine\sphinxhyphen{}tkg@\spxentry{wine\sphinxhyphen{}tkg}}\index{native\sphinxhyphen{}compilation@\spxentry{native\sphinxhyphen{}compilation}}\ignorespaces 

\subsubsection{WINE\sphinxhyphen{}TKG}
\label{\detokenize{source/linux-gaming:wine-tkg}}\label{\detokenize{source/linux-gaming:wine-tkg-git}}\label{\detokenize{source/linux-gaming:index-5}}
\sphinxAtStartPar
\sphinxhref{https://github.com/Frogging-Family/wine-tkg-git}{WINE\sphinxhyphen{}TKG} \sphinxhyphen{} это, наверное, лучшая сборка Wine для опытных пользователей которые хотят улучшить свой опыт игры под линуксом.
Преимуществом данной сборки перед другими является огромное количество вложенных в неё патчей из разных источников (В том числе портированных из Proton).
Поэтому мы настоятельно рекомендуем её к установке если вы хотите получить не только больше производительности, но и совместимости с различными Windows играми.

\sphinxAtStartPar
Установку wine\sphinxhyphen{}tkg можно выполнить двумя способами:
\begin{enumerate}
\sphinxsetlistlabels{\Roman}{enumi}{enumii}{}{.}%
\item {} 
\sphinxAtStartPar
Установить из его PKGBUILD как мы это делал ранее с другими программами.

\item {} 
\sphinxAtStartPar
Собрать его полностью вручную из исходников.

\end{enumerate}

\sphinxAtStartPar
Мы выберем первый вариант установки, т.к. он самый простой и надежный.

\sphinxAtStartPar
Второй вариант вы можете осуществить по желанию, особенно если у вас дистрибутив отличный от Arch Linux.

\sphinxAtStartPar
\sphinxstylestrong{I. Установка}

\begin{sphinxVerbatim}[commandchars=\\\{\}]
git clone https://github.com/Frogging\PYGZhy{}Family/wine\PYGZhy{}tkg\PYGZhy{}git.git
\PYG{n+nb}{cd} wine\PYGZhy{}tkg\PYGZhy{}git/wine\PYGZhy{}tkg\PYGZhy{}git
\end{sphinxVerbatim}

\sphinxAtStartPar
По аналогии с linux\sphinxhyphen{}tkg, wine\sphinxhyphen{}tkg предоставляет возможность предварительно настроить себя перед установкой
на применение различных патчей и твиков через редактирование файла \sphinxstyleemphasis{customization.cfg}:

\begin{sphinxVerbatim}[commandchars=\\\{\}]
nano customization.cfg
\end{sphinxVerbatim}

\sphinxAtStartPar
Здесь нас интересует не так много настроек.
По сути можете оставлять все значения по умолчанию, кроме следующих параметров:

\sphinxAtStartPar
\sphinxcode{\sphinxupquote{\_use\_esync="true"}} \sphinxhyphen{} Включает поддержку esync что оптимизирует работу wineserver.
Активируется через переменную окружения \sphinxstyleemphasis{WINEESYNC=1}.

\sphinxAtStartPar
\sphinxcode{\sphinxupquote{\_use\_fsync="true"}} \sphinxhyphen{} Включает поддержку fsync, альтернативу esync которую можно задействовать через переменную окружения \sphinxstyleemphasis{WINEFSYNC=1}.
Оба параметра обязательны к включению для повышения производительности.

\sphinxAtStartPar
Подробное сравнение Esync и Fsync можно посмотреть в данном видео.

\sphinxAtStartPar
\sphinxurl{https://www.youtube.com/watch?v=-nlNJguG5\_0\&t=18s}

\sphinxAtStartPar
\sphinxcode{\sphinxupquote{\_launch\_with\_dedicated\_gpu="false"}} \sphinxhyphen{} Активирует запуск приложений через дискретный графический процессор на ноутбуках с PRIME.
Работает только с открытыми драйверами (Mesa), поэтому выбирайте сами нужно оно вам или нет.

\sphinxAtStartPar
\sphinxcode{\sphinxupquote{\_update\_winevulkan="true"}} \sphinxhyphen{} Включает свежие обновления библиотеки winevulkan. Обязательно оставляйте включенным.

\sphinxAtStartPar
\sphinxcode{\sphinxupquote{\_FS\_bypass\_compositor="true"}} \sphinxhyphen{} Задействует обход композитора приложениями запускаемыми через Wine.
Очень полезная и нужная опция для исправления проблем задержек и заиканий в играх,
в случае когда системный композитор пытается лишний раз осуществить композитинг над окном с игрой запущенной через Wine.
Обязательно включаем.

\sphinxAtStartPar
\sphinxcode{\sphinxupquote{\_proton\_fs\_hack="true"}} \sphinxhyphen{} Включает еще один очень нужный патч.
Вносит исправление  с помощью которого изменяя разрешение игры в полноэкранном режиме у вас не будет изменяться разрешение вашего рабочего стола. Включаем.

\sphinxAtStartPar
\sphinxcode{\sphinxupquote{\_msvcrt\_nativebuiltin="true"}} \sphinxhyphen{} Осуществляет нативную сборку mscvrt.dll. Лишним точно не будет, поэтому включаем.

\sphinxAtStartPar
\sphinxcode{\sphinxupquote{\_win10\_default="false"}} \sphinxhyphen{} Устанавливает в качестве версии по умолчанию Windows 10 в Wine.
Не рекомендуется к включению в виду того, что это может задействовать vkd3d в некоторых играх работающих на DirectX 12,
что однако ведет к ухудшению производительности по сравнению с DXVK при возможности запустить игру с DirectX 11.

\sphinxAtStartPar
\sphinxcode{\sphinxupquote{\_protonify="true"}} \sphinxhyphen{} Задействует множественные заплатки и патчи для Wine портированные из Proton.
По нашему мнению это маст хев, т.к. они содержат в себе множественные исправления для различных игр и оптимизаций к ним.
Настоятельно рекомендуется к включению.

\begin{sphinxadmonition}{attention}{Внимание:}
\sphinxAtStartPar
По умолчанию wine\sphinxhyphen{}tkg не использует нативные флаги которые вы указывали ранее в \sphinxstyleemphasis{/etc/makepkg.conf}.
Их нужно указать вручную отредактировав \sphinxstyleemphasis{wine\sphinxhyphen{}tkg\sphinxhyphen{}profiles/advanced\sphinxhyphen{}customization.cfg}:

\begin{sphinxVerbatim}[commandchars=\\\{\}]
nano wine\PYGZhy{}tkg\PYGZhy{}profiles/advanced\PYGZhy{}customization.cfg \PYG{c+c1}{\PYGZsh{} Отредактируйте строчки ниже}

\PYG{n+nv}{\PYGZus{}GCC\PYGZus{}FLAGS}\PYG{o}{=}\PYG{l+s+s2}{\PYGZdq{}\PYGZhy{}O2 \PYGZhy{}ftree\PYGZhy{}vectorize \PYGZhy{}march=native\PYGZdq{}}

\PYG{n+nv}{\PYGZus{}CROSS\PYGZus{}FLAGS}\PYG{o}{=}\PYG{l+s+s2}{\PYGZdq{}\PYGZhy{}O2 \PYGZhy{}ftree\PYGZhy{}vectorize \PYGZhy{}march=native\PYGZdq{}}
\end{sphinxVerbatim}
\end{sphinxadmonition}

\sphinxAtStartPar
На этом все, остальные настройки оставьте по умолчанию.

\sphinxAtStartPar
Теперь можно перейти к самой сборке и установке wine\sphinxhyphen{}tkg: \sphinxcode{\sphinxupquote{makepkg \sphinxhyphen{}sric}}

\sphinxAtStartPar
\sphinxstylestrong{II. Ручная установка}

\sphinxAtStartPar
Подробно описывать ручную сборку здесь мы не будем.
Поэтому лучше всего посмотрите видео версию, где это наглядно показано (7 минута 23 секунда):

\sphinxAtStartPar
\sphinxurl{https://www.youtube.com/watch?v=W1e6\_3dPlHk}

\index{installation@\spxentry{installation}}\index{wine@\spxentry{wine}}\index{wine\sphinxhyphen{}builds@\spxentry{wine\sphinxhyphen{}builds}}\index{wine\sphinxhyphen{}tkg@\spxentry{wine\sphinxhyphen{}tkg}}\index{native\sphinxhyphen{}compilation@\spxentry{native\sphinxhyphen{}compilation}}\index{userpatches@\spxentry{userpatches}}\ignorespaces 

\subsubsection{\sphinxstyleemphasis{wine\sphinxhyphen{}tkg\sphinxhyphen{}userpatches}}
\label{\detokenize{source/linux-gaming:wine-tkg-userpatches}}\label{\detokenize{source/linux-gaming:index-6}}\label{\detokenize{source/linux-gaming:id6}}
\sphinxAtStartPar
Это  дополнение к wine\sphinxhyphen{}tkg.
По сути это коллекция пользовательских патчей для улучшения производительности и совместности.
Среди них: улучшения работы с памятью, интерфейсом GDI, качества отклика клавиатуры через системные вызовы Futex,
повышение приоритета процессов Wine по умолчанию, и другие низкоуровневые изменения от сторонних разработчиков.

\sphinxAtStartPar
\sphinxstylestrong{Установка}:

\begin{sphinxVerbatim}[commandchars=\\\{\}]
git clone https://github.com/openglfreak/wine\PYGZhy{}tkg\PYGZhy{}userpatches
\PYG{n+nb}{cd} \PYGZti{}/wine\PYGZhy{}tkg\PYGZhy{}git/wine\PYGZhy{}tkg\PYGZhy{}git

nano wine\PYGZhy{}tkg\PYGZhy{}profiles/advanced\PYGZhy{}customization.cfg \PYG{c+c1}{\PYGZsh{} Отредактируйте строчку ниже}

\PYG{n+nv}{\PYGZus{}EXT\PYGZus{}CONFIG\PYGZus{}PATH}\PYG{o}{=}\PYG{l+s+s2}{\PYGZdq{}\PYGZti{}/wine\PYGZhy{}tkg\PYGZhy{}userpatches/wine\PYGZhy{}tkg.cfg\PYGZdq{}}
\end{sphinxVerbatim}

\sphinxAtStartPar
Пересоберите wine\sphinxhyphen{}tkg по инструкции выше.

\sphinxAtStartPar
Никакой дополнительной настройки (редактирования \sphinxstyleemphasis{customization.cfg}) при этом не требуется.

\index{installation@\spxentry{installation}}\index{wine@\spxentry{wine}}\index{gaming@\spxentry{gaming}}\index{native\sphinxhyphen{}compilation@\spxentry{native\sphinxhyphen{}compilation}}\ignorespaces 

\subsubsection{WINE\sphinxhyphen{}GE}
\label{\detokenize{source/linux-gaming:wine-ge}}\label{\detokenize{source/linux-gaming:wine-ge-custom}}\label{\detokenize{source/linux-gaming:index-7}}
\sphinxAtStartPar
Альтернативная сборка Wine, которая содержит самые последние патчи из Proton.
По сути он аналогичен Proton\sphinxhyphen{}GE (о нем далее), но используется для игр запускаемых вне Steam.

\sphinxAtStartPar
\sphinxstylestrong{I. Установка (компиляция)}

\begin{sphinxVerbatim}[commandchars=\\\{\}]
git clone https://aur.archlinux.org/wine\PYGZhy{}ge\PYGZhy{}custom.git
\PYG{n+nb}{cd} wine\PYGZhy{}ge\PYGZhy{}custom
sed \PYGZhy{}i \PYG{l+s+s1}{\PYGZsq{}s/\PYGZhy{}O3 \PYGZhy{}march=nocona \PYGZhy{}mtune=core\PYGZhy{}avx2 \PYGZhy{}pipe/\PYGZhy{}O2 \PYGZhy{}march=native \PYGZhy{}mtune=native \PYGZhy{}pipe\PYGZdq{}/\PYGZsq{}} PKGBUILD  \PYG{c+c1}{\PYGZsh{} Нативные флаги}
makepkg \PYGZhy{}sric
\end{sphinxVerbatim}

\sphinxAtStartPar
\sphinxstylestrong{II. Установка (Lutris, бинарник)}

\sphinxAtStartPar
В Lutris уже есть готовые сборки Wine\sphinxhyphen{}GE под названием lutris\sphinxhyphen{}ge.
Если вы не хотите долго париться с ручной компиляцией, то
можете использовать их (производительность при этом будет ниже
чем у вручную собранного WINE\sphinxhyphen{}GE под ваш процессор):

\noindent\sphinxincludegraphics{{lutris-wine-ge}.png}

\sphinxAtStartPar
И затем выберите его для нужной вам игры:

\noindent\sphinxincludegraphics{{lutris-wine-ge-choose}.png}

\index{installation@\spxentry{installation}}\index{proton@\spxentry{proton}}\index{gaming@\spxentry{gaming}}\index{native\sphinxhyphen{}compilation@\spxentry{native\sphinxhyphen{}compilation}}\ignorespaces 

\subsubsection{Proton\sphinxhyphen{}GE\sphinxhyphen{}Custom}
\label{\detokenize{source/linux-gaming:proton-ge-custom}}\label{\detokenize{source/linux-gaming:index-8}}\label{\detokenize{source/linux-gaming:id7}}
\sphinxAtStartPar
Proton\sphinxhyphen{}GE\sphinxhyphen{}Custom это форк проекта Proton для запуска Windows\sphinxhyphen{}игр с дополнительными патчами и оптимизациями не вошедшими в основную ветку Proton,
а также улучшение совместимости с некоторыми играми (например, Warframe).
Позволяет играть во многие проекты которые не заводятся с обычным Wine или Proton.

\sphinxAtStartPar
\sphinxstylestrong{I. Установка (бинарная версия):}:

\begin{sphinxVerbatim}[commandchars=\\\{\}]
git clone https://aur.archlinux.org/proton\PYGZhy{}ge\PYGZhy{}custom\PYGZhy{}bin
\PYG{n+nb}{cd} proton\PYGZhy{}ge\PYGZhy{}custom\PYGZhy{}bin
makepkg \PYGZhy{}sric
\end{sphinxVerbatim}

\sphinxAtStartPar
\sphinxstylestrong{II. Установка (компиляция, имеет много зависимостей):}:

\begin{sphinxVerbatim}[commandchars=\\\{\}]
git clone https://aur.archlinux.org/proton\PYGZhy{}ge\PYGZhy{}custom
\PYG{n+nb}{cd} proton\PYGZhy{}ge\PYGZhy{}custom

\PYG{c+c1}{\PYGZsh{} Нативные флаги + дополнительный патч для DXVK позволяющий улучшить производительность видеокарт NVIDIA}
wget \PYGZhy{}qO proton\PYGZhy{}ge\PYGZhy{}aur.patch https://gist.githubusercontent.com/ventureoo/9b89c4799fbc89304f42983c6e90bda0/raw/9f10d463dfecaaa4935be757b48912004c6996fd/proton\PYGZhy{}ge\PYGZhy{}aur
patch PKGBUILD proton\PYGZhy{}ge\PYGZhy{}aur.patch

makepkg \PYGZhy{}sric
\end{sphinxVerbatim}

\sphinxAtStartPar
Дабы использовать Proton\sphinxhyphen{}GE в качестве альтернативы обычному Proton,
после установки Proton\sphinxhyphen{}GE\sphinxhyphen{}Custom вам нужно перезапустить Steam и зайти в Свойства нужной вам игры, прожать в:
\sphinxstyleemphasis{Совместность \sphinxhyphen{}> Принудительно использовать определенный инструмент совместности Steam Play \sphinxhyphen{}> Proton\sphinxhyphen{}6.XX\sphinxhyphen{}GE\sphinxhyphen{}1}. Готово, теперь можно запустить игру.

\index{installation@\spxentry{installation}}\index{wine@\spxentry{wine}}\index{about@\spxentry{about}}\index{prefixes@\spxentry{prefixes}}\ignorespaces 

\subsection{Использование Wine}
\label{\detokenize{source/linux-gaming:wine-usage}}\label{\detokenize{source/linux-gaming:index-9}}\label{\detokenize{source/linux-gaming:id8}}
\sphinxAtStartPar
Использование Wine на деле является довольно простым.
Чтобы запустить любое Windows\sphinxhyphen{}приложение достаточно использовать простую команду:

\begin{sphinxVerbatim}[commandchars=\\\{\}]
wine программа.exe
\end{sphinxVerbatim}

\begin{sphinxadmonition}{danger}{Опасно:}
\sphinxAtStartPar
НИКОГДА НЕ ЗАПУСКАЕТЕ WINE ИЗ ПОД SUDO/ROOT! Это поможет вам избежать проблем в будущем, в том числе с безопасностью.
\end{sphinxadmonition}

\sphinxAtStartPar
Немного иной командой запускаются MSI установщики:

\begin{sphinxVerbatim}[commandchars=\\\{\}]
wine msiexec /i программа.msi
\end{sphinxVerbatim}

\sphinxAtStartPar
При использовании Wine важным понятием является префикс (его также называют бутылкой).
Префикс, это как бы файловая система Windows в миниатюре, а по совместительству это рабочая директория,
где будут устанавливаться/работать все Windows программы которые вы будете запускать из под Wine.
Стоит понимать, что программы запускаемые через Wine по прежнему будут думать что они работают в Windows, хотя на самом деле это не так.
Поэтому Wine и понадобилось воссоздать файловую структуру каталогов Windows внутри Linux (Unix).
Префикс по умолчанию \sphinxhyphen{} это скрытая директория \sphinxstyleemphasis{\textasciitilde{}/.wine} в папке вашего пользователя.
Если вы её откроете то увидите следующее:

\noindent\sphinxincludegraphics{{image3}.png}

\sphinxAtStartPar
Как мы видим, в префиксе находятся файлы с расширением .reg (файлы реестра Windows), директории \sphinxstyleemphasis{dosdevices} и \sphinxstyleemphasis{drive\_c}.
Файлы реестра используются Wine для, собственно, воссоздания работы реестра Windows в Linux.
К ним также будут обращаться программы запускаемые через Wine.
Директория \sphinxstyleemphasis{dosdevices} содержит символические ссылки на примонтированные устройства (разделы) в вашей системе Linux.
Это понадобилось для того чтобы представить их в виде MS\sphinxhyphen{}DOS томов,
ибо Windows приложения опять таки не знают что они работают под Linux, и им нужны привычные им диски D, E и т.д.
Один из таких "виртуальных дисков" располагается в другом каталоге \sphinxhyphen{} \sphinxstyleemphasis{drive\_c} (диск C:).
Если вы его откроете то увидите "замечательную" структуру каталогов Windows:

\noindent\sphinxincludegraphics{{image8}.png}

\sphinxAtStartPar
Именно сюда и будут устанавливаться все Windows программы и работать они как правило тоже будут именно там.

\sphinxAtStartPar
Вы можете переназначить префикс по умолчанию через переменную окружения \sphinxstyleemphasis{WINEPREFIX},
указав Wine использовать другую директорию для его расположения вместо \sphinxstyleemphasis{\textasciitilde{}/.wine}. Например:

\begin{sphinxVerbatim}[commandchars=\\\{\}]
\PYG{n+nv}{WINEPREFIX}\PYG{o}{=}\PYGZti{}/Games wine game.exe \PYG{c+c1}{\PYGZsh{} Если директории не было, он её создаст.}
\end{sphinxVerbatim}

\sphinxAtStartPar
Понятное дело, что при смене префикса через переменную окружения WINEPREFIX не переносится его содержимое,
т.е. программы установленные в одном префиксе не будут скопированы в новый.
Но если вам нужно просто сменить название префикса с сохранением его содержимого,
то просто переименуете название директории, а затем переназначьте переменную, например:

\begin{sphinxVerbatim}[commandchars=\\\{\}]
mv \PYGZti{}/old\PYGZus{}wineprefix \PYGZti{}/new\PYGZus{}wineprefix
\PYG{n+nv}{WINEPREFIX}\PYG{o}{=}\PYGZti{}/.new\PYGZus{}wineprefix wine приложение.exe
\end{sphinxVerbatim}

\sphinxAtStartPar
Префиксы бывают 32\sphinxhyphen{}битные и 64\sphinxhyphen{}битные в соответствии с разрядностью систем Windows (по умолчанию создаются 64\sphinxhyphen{}битные).
Указать разрядность префикса можно через переменную \sphinxstyleemphasis{WINEARCH}.
Для запуска старых видеоигр мы рекомендуем использовать 32\sphinxhyphen{}битный префикс во избежание проблем с совместимостью:

\begin{sphinxVerbatim}[commandchars=\\\{\}]
\PYG{n+nv}{WINEPREFIX}\PYG{o}{=}\PYGZti{}/.wine32 \PYG{n+nv}{WINEARCH}\PYG{o}{=}win32 wine oldgame.exe
\end{sphinxVerbatim}

\sphinxAtStartPar
Если вы уже создали 64\sphinxhyphen{}битный префикс, то переназначить его разрядность через переменную \sphinxstyleemphasis{WINEARCH} не получится. Создайте новый и перенесите нужную вам программу.

\sphinxAtStartPar
Проверить разрядность уже существующего префикса можно командой (можно также проверить по наличию директории \sphinxstyleemphasis{"Program Files (x86)"} внутри префикса):

\begin{sphinxVerbatim}[commandchars=\\\{\}]
grep \PYG{l+s+s1}{\PYGZsq{}\PYGZsh{}arch\PYGZsq{}} \PYGZti{}/.wine/system.reg
\end{sphinxVerbatim}

\sphinxAtStartPar
(Где '.wine' \sphinxhyphen{} путь до нужного вам префикса)

\index{installation@\spxentry{installation}}\index{native\sphinxhyphen{}compilation@\spxentry{native\sphinxhyphen{}compilation}}\index{dxvk@\spxentry{dxvk}}\index{async@\spxentry{async}}\index{lowlatency@\spxentry{lowlatency}}\index{gaming@\spxentry{gaming}}\ignorespaces 

\subsection{DXVK}
\label{\detokenize{source/linux-gaming:dxvk}}\label{\detokenize{source/linux-gaming:index-10}}\label{\detokenize{source/linux-gaming:id9}}
\sphinxAtStartPar
В Linux отсутствует полноценная реализация DirectX по вполне понятным причинам. Но присутствуют альтернативные графические API, работающие под любые платформы.
Прежде всего это OpenGL и Vulkan. В следствии этого в Wine есть так называемый ретранслятор кода \sphinxhyphen{} wined3d.
Он переводит вызовы DirectX в известные любой Linux системе OpenGL вызовы. Однако OpenGL не одно и тоже что и DirectX,
поэтому возникают множественные проблемы. Самая главная из которых \sphinxhyphen{} значительно более худшая производительность OpenGL
по сравнению с DirectX. Именно поэтому если вы запустите любую игру через "голый" Wine вы получите ужасный FPS, т.к. она будет работать через wind3d.
По этой причине был разработан другой ретранслятор кода \sphinxhyphen{} DXVK. Он переводит DirectX вызовы уже не в
OpenGL, а в Vulkan \sphinxhyphen{} более современный графический API, который достигает паритета по своим возможностям и производительности с DirectX.

\sphinxAtStartPar
Установка DXVK \sphinxhyphen{} это первое что должен сделать любой игрок который собирается запустить Windows\sphinxhyphen{}игру под Linux.
Но для любой версии Proton DXVK уже есть из коробки, а вот для Wine его придется действительно устанавливать вручную.

\sphinxAtStartPar
Мы рекомендуем собирать \sphinxhref{https://github.com/loathingKernel/PKGBUILDs/tree/master/public/dxvk-mingw}{dxvk\sphinxhyphen{}mingw}
из GitHub для лучшей производительности и активации асинхронного патча. Если чуть подробнее,
то данный патч позволяет выполнять компиляцию шейдеров в асинхронных потоках.
Такой подход позволяет минимизировать заикания во время игры, которые могут происходить когда вы прогружаете новую локацию или объект на игровой карте,
то есть компилируйте новые шейдеры. В некоторых играх он даже немного повышает FPS и делает график времени кадра более плавным.
Патч не был одобрен разработчиками потому, что он потенциально вызывал проблемы в онлайн\sphinxhyphen{}играх с анти\sphinxhyphen{}чит системами, и теперь
для него требуется отдельная установка.

\sphinxAtStartPar
\sphinxstylestrong{Установка:}:

\begin{sphinxVerbatim}[commandchars=\\\{\}]
git clone https://github.com/loathingKernel/PKGBUILDs
\PYG{n+nb}{cd} PKGBUILDs/public/dxvk\PYGZhy{}mingw
mv PKGBUILD.testing PKGBUILD
sed \PYGZhy{}i \PYG{l+s+s1}{\PYGZsq{}s/patch \PYGZhy{}p1 \PYGZhy{}i \PYGZdq{}\PYGZdl{}srcdir\PYGZdq{}\PYGZbs{}/1582\PYGZbs{}.patch//g\PYGZsq{}} PKGBUILD
sed \PYGZhy{}i \PYG{l+s+s1}{\PYGZsq{}s/patch \PYGZhy{}p1 \PYGZhy{}i \PYGZdq{}\PYGZdl{}srcdir\PYGZdq{}\PYGZbs{}/1582\PYGZhy{}fix\PYGZhy{}include\PYGZbs{}.patch//g\PYGZsq{}} PKGBUILD
sed \PYGZhy{}i \PYG{l+s+s1}{\PYGZsq{}s/\PYGZhy{}O3 \PYGZhy{}march=haswell \PYGZhy{}mtune=haswell \PYGZhy{}pipe/\PYGZhy{}O2 \PYGZhy{}march=native \PYGZhy{}mtune=native \PYGZhy{}pipe/g\PYGZsq{}} PKGBUILD \PYG{c+c1}{\PYGZsh{} Нативные флаги}
makepkg \PYGZhy{}sric \PYG{c+c1}{\PYGZsh{} Сборка и установка}
\end{sphinxVerbatim}

\sphinxAtStartPar
Активировать асинхронную компиляцию шейдеров можно через переменную окружения \sphinxstyleemphasis{DXVK\_ASYNC=1}.

\sphinxAtStartPar
После установки пакета DXVK не задействуется сразу, его библиотеки ещё нужно "распаковать" по отдельности в каждый префикс Wine
(это не относиться к играм запускаемым через Lutris/Proton, в них DXVK включён по умолчанию):

\begin{sphinxVerbatim}[commandchars=\\\{\}]
\PYG{n+nv}{WINEPREFIX}\PYG{o}{=}\PYGZti{}/prefix setup\PYGZus{}dxvk install \PYG{c+c1}{\PYGZsh{} Где \PYGZdq{}prefix\PYGZdq{} \PYGZhy{} это путь до вашего префикса Wine}
\end{sphinxVerbatim}

\begin{sphinxadmonition}{warning}{Предупреждение:}
\sphinxAtStartPar
DXVK осуществляет ретрансляцию вызовов только для игр использующих версии DirectX 9, 10 и 11.
Для DirectX 12 для понадобиться использовать vkd3d. Подробнее о нем вы можете прочитать ниже.
\end{sphinxadmonition}

\begin{sphinxadmonition}{danger}{Опасно:}
\sphinxAtStartPar
С осторожностью используйте \sphinxstyleemphasis{DXVK\_ASYNC=1} в онлайн\sphinxhyphen{}играх!
\end{sphinxadmonition}

\index{installation@\spxentry{installation}}\index{wine@\spxentry{wine}}\index{vkd3d@\spxentry{vkd3d}}\index{gaming@\spxentry{gaming}}\index{native\sphinxhyphen{}compilation@\spxentry{native\sphinxhyphen{}compilation}}\ignorespaces 

\subsection{vkd3d}
\label{\detokenize{source/linux-gaming:vkd3d}}\label{\detokenize{source/linux-gaming:index-11}}\label{\detokenize{source/linux-gaming:id10}}
\sphinxAtStartPar
vkd3d \sphinxhyphen{} это ретранслятор кода, аналогичный DXVK, но уже конкретно для версии DirectX 12.
Стоит отметить, что существует две отдельно разрабатываемые версии vkd3d,
одна из которых разрабатывается командой Wine, а другая \sphinxhyphen{} Valve.
Мы рекомендуем вам использовать ту что от Valve, т.к. она наиболее заточена под современные игры,
а также достаточно хорошо поддерживает Raytracing.

\sphinxAtStartPar
\sphinxstylestrong{Установка vkd3d\sphinxhyphen{}proton}

\sphinxAtStartPar
Для Proton и Lutris установка vkd3d задействован по умолчанию, и никаких дополнительных манипуляций обычно не требуется.
Однако для обычного Wine нужна его отдельная установка. Мы установим vkd3d\sphinxhyphen{}proton из AUR, нативно\sphinxhyphen{}скомпилировав его под свой процессор:

\begin{sphinxVerbatim}[commandchars=\\\{\}]
git clone https://aur.archlinux.org/vkd3d\PYGZhy{}proton\PYGZhy{}mingw.git \PYG{c+c1}{\PYGZsh{} Скачивание исходников}
\PYG{n+nb}{cd} vkd3d\PYGZhy{}proton\PYGZhy{}mingw                                      \PYG{c+c1}{\PYGZsh{} Переход в директорию}
sed \PYGZhy{}i \PYG{l+s+s1}{\PYGZsq{}s/\PYGZhy{}O3 \PYGZhy{}march=nocona \PYGZhy{}mtune=core\PYGZhy{}avx2 \PYGZhy{}pipe/\PYGZhy{}O2 \PYGZhy{}march=native \PYGZhy{}mtune=native \PYGZhy{}pipe/g\PYGZsq{}} PKGBUILD \PYG{c+c1}{\PYGZsh{} Нативные флаги}
makepkg \PYGZhy{}sric                                              \PYG{c+c1}{\PYGZsh{} Сборка и установка}
\end{sphinxVerbatim}

\sphinxAtStartPar
Так же как и в случае с DXVK, после установки пакета, vkd3d нужно предварительно распоковать в нужный Wine префикс:

\begin{sphinxVerbatim}[commandchars=\\\{\}]
setup\PYGZus{}vkd3d\PYGZus{}proton install \PYGZti{}/.wineprefix
\end{sphinxVerbatim}

\sphinxAtStartPar
(Где '\textasciitilde{}/.wineprefix' \sphinxhyphen{} это путь до нужного вам префикса)

\sphinxAtStartPar
Кроме того, обязательно измените версию Windows вашего префикса на \sphinxstyleemphasis{"Windows 10"}:

\begin{sphinxVerbatim}[commandchars=\\\{\}]
\PYG{n+nv}{WINEPREFIX}\PYG{o}{=}\PYGZti{}/.wineprefix winecfg
\end{sphinxVerbatim}

\noindent\sphinxincludegraphics{{vkd3d-configure}.png}

\index{wine@\spxentry{wine}}\index{dxvk@\spxentry{dxvk}}\index{gaming@\spxentry{gaming}}\index{about@\spxentry{about}}\ignorespaces 

\subsection{Полезные ссылки по теме Wine и DXVK}
\label{\detokenize{source/linux-gaming:wine-dxvk}}\label{\detokenize{source/linux-gaming:wine-references}}\label{\detokenize{source/linux-gaming:index-12}}
\sphinxAtStartPar
\sphinxstylestrong{Видео на настройке Бинарной версии Wine.}

\sphinxAtStartPar
\sphinxurl{https://www.youtube.com/watch?v=NKI3dtK7mRI} (Устаревшее видео).

\sphinxAtStartPar
\sphinxstylestrong{Скачать готовые сборки Wine и DXVK}

\sphinxAtStartPar
\sphinxurl{https://mega.nz/folder/pNsTiQyA\#2vur9shHbXvLnhdQTpd3AQ}

\sphinxAtStartPar
\sphinxurl{https://mega.nz/folder/IJdEgIrT\#wXcbgymIDP2mesJ8kE99Qg}

\sphinxAtStartPar
\sphinxurl{https://github.com/Kron4ek/Wine-Builds}

\sphinxAtStartPar
\sphinxurl{https://mirror.cachyos.org/?search=wine}

\sphinxAtStartPar
\sphinxstylestrong{Почитать, что это такое}

\sphinxAtStartPar
\sphinxurl{https://www.newalive.net/234-sborki-dxvk-i-d9vk.html}

\sphinxAtStartPar
\sphinxurl{https://www.newalive.net/231-wine-tk-glitch.html}

\index{gamemode@\spxentry{gamemode}}\index{lutris@\spxentry{lutris}}\index{gaming@\spxentry{gaming}}\ignorespaces 

\section{Дополнительные компоненты}
\label{\detokenize{source/linux-gaming:additional-components}}\label{\detokenize{source/linux-gaming:index-13}}\label{\detokenize{source/linux-gaming:id11}}
\sphinxAtStartPar
Не являются обязательными, но могут помочь повысить производительность системы или облегчить настройку.

\index{installation@\spxentry{installation}}\index{gamemode@\spxentry{gamemode}}\index{lutris@\spxentry{lutris}}\index{gaming@\spxentry{gaming}}\ignorespaces 

\subsection{Lutris}
\label{\detokenize{source/linux-gaming:lutris}}\label{\detokenize{source/linux-gaming:lutris-and-additions}}\label{\detokenize{source/linux-gaming:index-14}}
\sphinxAtStartPar
Lutris \sphinxhyphen{} это удобный графический интерфейс по обслуживанию всей вашей игровой библиотеки
(включая все купленные игры Steam/GOG/Epic Games) в одном приложении.
Через него вы сможете достаточно просто запускать нативные игры, игры запускаемые при помощи эмуляторов, и конечно Wine.
Все это объединено в одном приложении\sphinxhyphen{}комбайне, содержащим много настроек и интеграций с различными сервисами.

\sphinxAtStartPar
\sphinxstylestrong{Установка}

\sphinxAtStartPar
Все проще некуда:

\begin{sphinxVerbatim}[commandchars=\\\{\}]
sudo pacman \PYGZhy{}S lutris
\end{sphinxVerbatim}

\sphinxAtStartPar
Тем не менее, стоит удостовериться что вы установили полный набор зависимостей для Wine. Об этом вы можете прочитать в предыдущих разделах.

\noindent\sphinxincludegraphics{{lutris}.png}

\sphinxAtStartPar
\sphinxstylestrong{Интеграция с GOG/Epic/Steam}

\sphinxAtStartPar
Сразу после установки стоит сделать некоторые базовые вещи. А именно подключить интеграцию с сервисами Steam/GOG/Epic Games.
Это позволит синхронизировать локальную библиотеку Lutris'a вместе с перечисленными площадками и выполнять установку игр в два клика.
Подключать все конечно не обязательно, так что делайте это если считаете нужным.

\sphinxAtStartPar
\sphinxstylestrong{1.} Зайдем в настройки: В правом верхнем углу найдите три горизонтальные полоски и в контекстном меню выберите \sphinxstyleemphasis{"Preferences"}.
После этого выберите \sphinxstyleemphasis{"Services"} и включите те сервисы, которыми вы пользуетесь.

\sphinxAtStartPar
\sphinxstylestrong{1.1}

\noindent\sphinxincludegraphics{{lutris-context-menu}.png}

\sphinxAtStartPar
\sphinxstylestrong{1.2}

\noindent\sphinxincludegraphics{{lutris-preferences}.png}

\sphinxAtStartPar
\sphinxstylestrong{2.} Теперь вернитесь в главное окно и наведите курсор на левую панель в графу \sphinxstyleemphasis{"Sources"}, и ниже выбирите нужную вам платформу.
Справа от курсора будет иконка входа. После этого перед вами появится окно авторизации, после прохождения которой у вас появится
возможность устанавливать и запускать все игры из вашей внешней библиотеки (Steam/GOG/Epic Games).

\sphinxAtStartPar
Пример подключения аккаунта GOG представлен ниже на скриншотах.

\sphinxAtStartPar
\sphinxstylestrong{2.1}

\noindent\sphinxincludegraphics{{lutris-auth-icon}.png}

\sphinxAtStartPar
\sphinxstylestrong{2.2}

\noindent\sphinxincludegraphics{{lutris-gog-auth}.png}

\sphinxAtStartPar
\sphinxstylestrong{2.3}

\noindent\sphinxincludegraphics{{lutris-gog-library}.png}

\sphinxAtStartPar
Аналогичная операция проделывается с Epic Games Store:

\sphinxAtStartPar
\sphinxstylestrong{2.4}

\noindent\sphinxincludegraphics{{lutris-auth-epic-icon}.png}

\sphinxAtStartPar
\sphinxstylestrong{2.5}

\noindent\sphinxincludegraphics{{lutris-epic-auth}.png}

\sphinxAtStartPar
\sphinxstylestrong{2.6}

\noindent\sphinxincludegraphics{{lutris-epic-library}.png}

\sphinxAtStartPar
\sphinxstylestrong{Пример работы с Lutris}

\sphinxAtStartPar
\sphinxurl{https://www.youtube.com/watch?v=ybe0MzJDUvw}

\index{proton@\spxentry{proton}}\index{gaming@\spxentry{gaming}}\index{lutris@\spxentry{lutris}}\index{proton\sphinxhyphen{}ge\sphinxhyphen{}custom@\spxentry{proton\sphinxhyphen{}ge\sphinxhyphen{}custom}}\ignorespaces 

\subsubsection{Использование Proton\sphinxhyphen{}GE\sphinxhyphen{}Custom в Lutris}
\label{\detokenize{source/linux-gaming:proton-ge-custom-lutris}}\label{\detokenize{source/linux-gaming:proton-ge-with-lutris}}\label{\detokenize{source/linux-gaming:index-15}}
\sphinxAtStartPar
Немногие понимают, что Proton по сути является тем же Wine, хоть и с плюшками.
Так вот, зная этот факт, мы можем сказать Lutris использовать Proton в качестве кастомного Wine.
Делается это очень просто:

\begin{sphinxVerbatim}[commandchars=\\\{\}]
mkdir \PYGZhy{}p \PYGZti{}/.local/share/lutris/runners/wine
ln \PYGZhy{}s /usr/share/steam/compatibilitytools.d/proton\PYGZhy{}ge\PYGZhy{}custom/files \PYGZti{}/.local/share/lutris/runners/wine/wine\PYGZhy{}proton\PYGZhy{}ge
\end{sphinxVerbatim}

\sphinxAtStartPar
Затем просто выберите пункт в выборе версии Wine на \sphinxstyleemphasis{"wine\sphinxhyphen{}proton\sphinxhyphen{}ge"} в Lutris для нужной вам игры.

\index{installation@\spxentry{installation}}\index{gamemode@\spxentry{gamemode}}\index{gaming@\spxentry{gaming}}\index{lutris@\spxentry{lutris}}\ignorespaces 

\subsection{Gamemode}
\label{\detokenize{source/linux-gaming:gamemode}}\label{\detokenize{source/linux-gaming:index-16}}\label{\detokenize{source/linux-gaming:id12}}
\sphinxAtStartPar
Gamemode \sphinxhyphen{} утилита для максимальной выжимки системы во время игры.
Установку gamemode можно выполнить следующей командой:

\begin{sphinxVerbatim}[commandchars=\\\{\}]
sudo pacman \PYGZhy{}S gamemode lib32\PYGZhy{}gamemode
\end{sphinxVerbatim}

\sphinxAtStartPar
Lutris, как правило использует gamemode по умолчанию (в случае его наличия в системе), однако вы также можете активировать или деактивировать его в параметрах.

\sphinxAtStartPar
Для запуска игры в ручную с использованием gamemode необходимо выполнить команду:

\begin{sphinxVerbatim}[commandchars=\\\{\}]
gamemoderun ./game
\end{sphinxVerbatim}

\sphinxAtStartPar
Для запуска игр через Steam с использованием gamemode необходимо прописать команду в параметрах запуска игры (находятся в свойствах игры в Steam):

\begin{sphinxVerbatim}[commandchars=\\\{\}]
gamemoderun \PYGZpc{}command\PYGZpc{}
\end{sphinxVerbatim}

\index{amd@\spxentry{amd}}\index{fsr@\spxentry{fsr}}\index{image\sphinxhyphen{}scaling@\spxentry{image\sphinxhyphen{}scaling}}\index{gaming@\spxentry{gaming}}\ignorespaces 

\subsection{AMD FidelityFX Super Resolution в Wine}
\label{\detokenize{source/linux-gaming:amd-fidelityfx-super-resolution-wine}}\label{\detokenize{source/linux-gaming:amd-fsr}}\label{\detokenize{source/linux-gaming:index-17}}
\sphinxAtStartPar
Возможно, вы слышали о волшебной технологии DLSS от Nvidia, которая позволяет поднять FPS почти в два раза и при этом не потратить ни копейки на новое оборудование.
Вот и компания AMD совсем недавно представила похожую технологию, которая получила помпезное название AMD FidelityFX Super Resolution или сокращенно FSR.
Новая технология масштабирования картинки от AMD не требует наличия дорого́й карты или каких\sphinxhyphen{}то аппаратных блоков ускорения,
что в отличие от DLSS, должно позволить использовать технологию везде и совершенно бесплатно.
А благодаря чудесным патчам от энтузиастов для Wine мы можем применять эту волшебную технологию для любой Windows\sphinxhyphen{}игры.

\sphinxAtStartPar
\sphinxstylestrong{I. Установка}

\sphinxAtStartPar
Чтобы установить патч от энтузиастов придется немного помудрить с нашим wine\sphinxhyphen{}tkg.

\sphinxAtStartPar
Его установка описывалась выше, но чтобы задействовать сторонний патч на FSR в Wine нужно отредактировать одну строку в \sphinxstyleemphasis{customization.cfg}:

\begin{sphinxVerbatim}[commandchars=\\\{\}]
nano customization.cfg

\PYG{c+c1}{\PYGZsh{} Найдите строчку \PYGZus{}community\PYGZus{}patches=\PYGZdq{}\PYGZdq{} и добавьте в неё следующее:}

\PYG{n+nv}{\PYGZus{}community\PYGZus{}patches}\PYG{o}{=}\PYG{l+s+s2}{\PYGZdq{}amd\PYGZus{}fsr\PYGZus{}fshack.mypatch\PYGZdq{}}

\PYG{c+c1}{\PYGZsh{} Обязательно оставьте при этом включенными данные параметры:}
\PYGZus{}protonify, \PYGZus{}msvcrt\PYGZus{}nativebuiltin, \PYGZus{}proton\PYGZus{}fs\PYGZus{}hack, \PYGZus{}proton\PYGZus{}rawinput.
Без них ничего работать не будет.
\end{sphinxVerbatim}

\sphinxAtStartPar
И пересоберите ваш wine\sphinxhyphen{}tkg: \sphinxcode{\sphinxupquote{makepkg \sphinxhyphen{}sric}}

\sphinxAtStartPar
\sphinxstylestrong{II. Установка}

\sphinxAtStartPar
Если вам кажется первый способ немного муторным, то вы можете просто использовать уже готовые сборки с FSR патчем в Lutris:

\noindent\sphinxincludegraphics{{linux-gaming-1}.png}

\sphinxAtStartPar
И затем выбрать её для нужной вам игры:

\noindent\sphinxincludegraphics{{linux-gaming-2}.png}

\sphinxAtStartPar
\sphinxstylestrong{III. Установка}

\sphinxAtStartPar
FSR патч также по умолчанию задействован в Proton\sphinxhyphen{}GE\sphinxhyphen{}Custom.
Про его установку вы можете прочитать ниже в соответствующем разделе.

\sphinxAtStartPar
\sphinxstylestrong{Как использовать}

\sphinxAtStartPar
Несмотря на то, что мы выполнили установку патченной версии Wine одним из вышеописанных способов,
технологию FSR ещё нужно активировать.

\sphinxAtStartPar
Сделать это можно руками, через переменные окружения \sphinxstyleemphasis{WINE\_FULLSCREEN\_FSR=1} или в Lutris:

\noindent\sphinxincludegraphics{{linux-gaming-3}.png}

\sphinxAtStartPar
Важно помнить, что эта технология работает \sphinxstylestrong{только в полноэкранном режиме игры}.

\sphinxAtStartPar
Регулировать резкость итогового изображения можно через переменную окружения \sphinxstyleemphasis{WINE\_FULLSCREEN\_FSR\_STRENGTH=N},
где N \sphinxhyphen{} это уровень резкости изображения от 0 до 5. Чем выше значение, тем меньше резкость.
По умолчанию установлено значение \sphinxstyleemphasis{"2"}, мы рекомендуем использовать значение \sphinxstyleemphasis{"3"}.

\sphinxAtStartPar
\sphinxstylestrong{Видеоверсия и демонстрация работы технологии}

\sphinxAtStartPar
\sphinxurl{https://www.youtube.com/watch?v=YNhwAazJODU}

\index{nvidia@\spxentry{nvidia}}\index{dlss@\spxentry{dlss}}\index{proton@\spxentry{proton}}\index{image\sphinxhyphen{}scaling@\spxentry{image\sphinxhyphen{}scaling}}\index{gaming@\spxentry{gaming}}\ignorespaces 

\subsection{Использование DLSS с видеокартами NVIDIA через Proton}
\label{\detokenize{source/linux-gaming:dlss-nvidia-proton}}\label{\detokenize{source/linux-gaming:nvidia-dlss-with-proton}}\label{\detokenize{source/linux-gaming:index-18}}
\sphinxAtStartPar
Для того чтобы использовать DLSS вам потребуется:
\begin{itemize}
\item {} 
\sphinxAtStartPar
Видеокарта поддерживающая данную технологию (видеокарты серии RTX и выше).

\item {} 
\sphinxAtStartPar
Убедиться, что используемая версия Proton не ниже \sphinxstylestrong{6.3\sphinxhyphen{}8}! (\sphinxstylestrong{поддержка DLSS начинается с данной версии!})

\item {} 
\sphinxAtStartPar
Указать параметры запуска игры в свойствах игры Steam \sphinxcode{\sphinxupquote{PROTON\_HIDE\_NVIDIA\_GPU=0 PROTON\_ENABLE\_NVAPI=1}}

\item {} 
\sphinxAtStartPar
Некоторые игры, как правило, которые используют DX11, для корректной работы могут также потребовать включения \sphinxstyleemphasis{dxgi.nvapiHack = False} в \sphinxstyleemphasis{dxvk.conf.}
Для этого выполните инструкции ниже:

\begin{sphinxVerbatim}[commandchars=\\\{\}]
mkdir \PYGZhy{}p \PYGZti{}/.config/dxvk/dxvk.conf
\PYG{n+nb}{echo} \PYG{l+s+s2}{\PYGZdq{}dxgi.nvapiHack = False\PYGZdq{}} \PYGZgt{} \PYGZti{}/.config/dxvk/dxvk.conf
\end{sphinxVerbatim}

\sphinxAtStartPar
После этого не забудьте дописать \sphinxstyleemphasis{DXVK\_CONFIG\_FILE=\textasciitilde{}/.config/dxvk/dxvk.conf} в приведённом ниже примере перед \sphinxcode{\sphinxupquote{\%command\%}}.

\end{itemize}

\sphinxAtStartPar
Пример для использования в Steam:

\begin{sphinxVerbatim}[commandchars=\\\{\}]
\PYG{n+nv}{PROTON\PYGZus{}HIDE\PYGZus{}NVIDIA\PYGZus{}GPU}\PYG{o}{=}\PYG{l+m}{0} \PYG{n+nv}{PROTON\PYGZus{}ENABLE\PYGZus{}NVAPI}\PYG{o}{=}\PYG{l+m}{1} \PYGZpc{}command\PYGZpc{}
\end{sphinxVerbatim}

\begin{sphinxadmonition}{attention}{Внимание:}
\sphinxAtStartPar
Поскольку для DLSS необходимо специальное машинное обучение, то для запуска необходимо чтобы игра поддерживала DLSS, т.е. в настройках игры должен быть параметр включения данной функции. \sphinxstylestrong{Иначе DLSS работать не будет!}
\end{sphinxadmonition}

\index{fps@\spxentry{fps}}\index{monitoring@\spxentry{monitoring}}\index{mangohud@\spxentry{mangohud}}\index{dxvk@\spxentry{dxvk}}\ignorespaces 

\subsection{Мониторинг FPS в играх.}
\label{\detokenize{source/linux-gaming:fps}}\label{\detokenize{source/linux-gaming:fps-monitoring}}\label{\detokenize{source/linux-gaming:index-19}}
\index{installation@\spxentry{installation}}\index{fps@\spxentry{fps}}\index{monitoring@\spxentry{monitoring}}\index{mangohud@\spxentry{mangohud}}\ignorespaces 

\subsubsection{Mangohud}
\label{\detokenize{source/linux-gaming:mangohud}}\label{\detokenize{source/linux-gaming:index-20}}\label{\detokenize{source/linux-gaming:id13}}
\sphinxAtStartPar
Включение мониторинга в играх как в MSI Afterburner.

\noindent{\hspace*{\fill}\sphinxincludegraphics{{image9}.png}\hspace*{\fill}}

\sphinxAtStartPar
\sphinxstylestrong{Установка}

\begin{sphinxVerbatim}[commandchars=\\\{\}]
\PYG{n+nb}{cd} tools                                             \PYG{c+c1}{\PYGZsh{} Переход в заранее созданную папку в домашнем каталоге.}
git clone https://aur.archlinux.org/mangohud.git     \PYG{c+c1}{\PYGZsh{} Скачивание исходников.}
\PYG{n+nb}{cd} mangohud                                          \PYG{c+c1}{\PYGZsh{} Переход в mangohud.}
makepkg \PYGZhy{}sric                                        \PYG{c+c1}{\PYGZsh{} Сборка и установка.}
\end{sphinxVerbatim}

\sphinxAtStartPar
Графический помощник для настройки вашего MangoHud.

\begin{sphinxVerbatim}[commandchars=\\\{\}]
\PYG{n+nb}{cd} tools                                         \PYG{c+c1}{\PYGZsh{} Переход в заранее созданную папку в домашнем каталоге.}
git clone https://aur.archlinux.org/goverlay.git \PYG{c+c1}{\PYGZsh{} Скачивание исходников.}
\PYG{n+nb}{cd} goverlay                                      \PYG{c+c1}{\PYGZsh{} Переход в goverlay\PYGZhy{}bin}
makepkg \PYGZhy{}sric                                    \PYG{c+c1}{\PYGZsh{} Сборка и установка.}
\end{sphinxVerbatim}

\sphinxAtStartPar
Для использования mangohud в играх через Steam необходимо добавить команду в параметры запуска игры (находятся в свойствах игры Steam):

\begin{sphinxVerbatim}[commandchars=\\\{\}]
mangohud \PYGZpc{}command\PYGZpc{}
\end{sphinxVerbatim}

\sphinxAtStartPar
(Для указания нескольких команд необходимо разделять их \sphinxstylestrong{пробелом})

\sphinxAtStartPar
\sphinxstylestrong{Подробней в видео.}

\sphinxAtStartPar
\sphinxurl{https://www.youtube.com/watch?v=4RqerevPD4I}

\index{installation@\spxentry{installation}}\index{fps@\spxentry{fps}}\index{monitoring@\spxentry{monitoring}}\index{dxvk@\spxentry{dxvk}}\ignorespaces 

\subsubsection{Альтернатива: DXVK Hud (\sphinxstyleemphasis{Только для игр запускаемых через Wine/Proton})}
\label{\detokenize{source/linux-gaming:dxvk-hud-wine-proton}}\label{\detokenize{source/linux-gaming:dxvk-hud}}\label{\detokenize{source/linux-gaming:index-21}}
\sphinxAtStartPar
Вы также можете использовать встроенную в DXVK альтернативу для мониторинга \sphinxhyphen{} DXVK Hud.
Он не такой гибкий как MangoHud, но также способен выводить значения FPS, график времени кадра, нагрузку на GPU.
Использовать данный HUD можно задав переменную окружения \sphinxstyleemphasis{DXVK\_HUD}.
К примеру, \sphinxcode{\sphinxupquote{DXVK\_HUD=fps,frametimes,gpuload}} выводит информацию о FPS, времени кадра, и нагрузке на GPU.

\sphinxAtStartPar
Полный список значений переменной вы можете узнать \sphinxhyphen{} \sphinxhref{https://github.com/doitsujin/dxvk\#hud}{здесь}.

\sphinxstepscope

\index{mini\sphinxhyphen{}kernel@\spxentry{mini\sphinxhyphen{}kernel}}\index{modprobed\sphinxhyphen{}db@\spxentry{modprobed\sphinxhyphen{}db}}\index{kernel@\spxentry{kernel}}\index{modules@\spxentry{modules}}\ignorespaces 

\chapter{Сборка мини\sphinxhyphen{}ядра, и с чем это едят.}
\label{\detokenize{source/mini-kernel:mini-kernel}}\label{\detokenize{source/mini-kernel:index-0}}\label{\detokenize{source/mini-kernel:id1}}\label{\detokenize{source/mini-kernel::doc}}
\sphinxAtStartPar
Ядра, что мы скомпилировали выше уже дают существенное повышение производительности системы, однако мы еще выжали не все соки.
По умолчанию ядра собираются для универсального применения на разном оборудовании,
т.е. с наличием различных модулей и драйверов для всякого рода периферии и железа, которого у вас могло никогда и не быть.

\sphinxAtStartPar
\sphinxstyleemphasis{Мини\sphinxhyphen{}ядро} \sphinxhyphen{} Это Linux ядро собранное с минимальным количеством модулей/драйверов необходимых для работоспособности вашего железа.

\sphinxAtStartPar
Плюсы: \sphinxhref{https://wiki.archlinux.org/index.php/Modprobed-db\#Benefits\_of\_modprobed-db\_with\_"make\_localmodconfig"\_in\_custom\_kernels}{Значительное сокращение времени на сборку ядра},
уменьшение размера ядра, повышение производительности.

\sphinxAtStartPar
Минусы: Невозможность использования нового оборудования или портов без повторной пересборки ядра.

\sphinxAtStartPar
Чтобы собрать мини\sphinxhyphen{}ядро, нам нужно:

\sphinxAtStartPar
Установить \sphinxhref{https://aur.archlinux.org/packages/modprobed-db/}{modprobed\sphinxhyphen{}db} по аналогии с другими AUR пакетами.

\sphinxAtStartPar
После установки выполнить:

\begin{sphinxVerbatim}[commandchars=\\\{\}]
systemctl \PYGZhy{}\PYGZhy{}user \PYG{n+nb}{enable} \PYGZhy{}\PYGZhy{}now modprobed\PYGZhy{}db.service \PYG{c+c1}{\PYGZsh{} Это демон для индексирования активно используемых системой модулей ядра}
sudo modprobed\PYGZhy{}db recall \PYG{c+c1}{\PYGZsh{} Сделает дамп используемых системой модулей ядра.}
\end{sphinxVerbatim}

\sphinxAtStartPar
Далее, активно используем всю периферию и железки, что у вас есть пока не соберется достаточное количество модулей (Примерно 2\sphinxhyphen{}3 дня активного пользования системой).

\sphinxAtStartPar
После того как все приготовления сделаны, просто собираем ядро как было указано выше, но перед сборкой (\sphinxstyleemphasis{makepkg \sphinxhyphen{}si}) нужно отредактировать PKGBUILD:

\begin{sphinxVerbatim}[commandchars=\\\{\}]
nano PKGBUILD
\end{sphinxVerbatim}

\sphinxAtStartPar
И меняем значение этой строки (работает почти для любых ядер): \sphinxstyleemphasis{\_localmodcfg=y}

\sphinxAtStartPar
Все, теперь собираем мини\sphinxhyphen{}ядро по аналогии с обычным.

\sphinxAtStartPar
\sphinxstyleemphasis{P.S.} Если при сборке образов уже скомпилированного ядра выдает ошибку с указанием на отсутствующие модули, что\sphinxhyphen{}то в формате: db\_xxx, bd\_xxx \sphinxhyphen{} просто пропишите их в ручную:

\begin{sphinxVerbatim}[commandchars=\\\{\}]
sudo nano \PYGZti{}/.config/modprobed.db
\end{sphinxVerbatim}

\sphinxAtStartPar
Затем выполните:

\begin{sphinxVerbatim}[commandchars=\\\{\}]
sudo modprobed\PYGZhy{}db store
sudo modprobed\PYGZhy{}db recall
\end{sphinxVerbatim}

\sphinxAtStartPar
И снова пересоберите ядро.

\sphinxAtStartPar
\sphinxstylestrong{Видео версия}

\sphinxAtStartPar
\sphinxurl{https://www.youtube.com/watch?v=8GRNN94afyg}

\index{mini\sphinxhyphen{}kernel@\spxentry{mini\sphinxhyphen{}kernel}}\index{problems@\spxentry{problems}}\index{modules@\spxentry{modules}}\index{modprobed\sphinxhyphen{}db@\spxentry{modprobed\sphinxhyphen{}db}}\ignorespaces 

\section{Возможные часто встречаемые проблемы после установки мини\sphinxhyphen{}ядра}
\label{\detokenize{source/mini-kernel:related-issues}}\label{\detokenize{source/mini-kernel:index-1}}\label{\detokenize{source/mini-kernel:id3}}
\sphinxAtStartPar
\sphinxstylestrong{П:} Система не загружается дальше rootfs (частая проблема).

\sphinxAtStartPar
\sphinxstylestrong{Р:} Обычно это означает, что какие\sphinxhyphen{}то системно\sphinxhyphen{}важные модули не были "подхвачены" modprobed\sphinxhyphen{}db.
Почти всегда дело заключается в модулях на поддержку SATA/SCSI, либо ATA и модулей Файловых систем.

\sphinxAtStartPar
Вот список модулей, из\sphinxhyphen{}за отсутствия которых может не грузиться система:
\begin{itemize}
\item {} 
\sphinxAtStartPar
scsi\_mod

\item {} 
\sphinxAtStartPar
sd\_mod

\item {} 
\sphinxAtStartPar
libahci

\item {} 
\sphinxAtStartPar
libata

\item {} 
\sphinxAtStartPar
lzo\_rle

\item {} 
\sphinxAtStartPar
efi\_pstore

\item {} 
\sphinxAtStartPar
evdev

\item {} 
\sphinxAtStartPar
ext4

\item {} 
\sphinxAtStartPar
btrfs

\item {} 
\sphinxAtStartPar
ahci

\item {} 
\sphinxAtStartPar
autofs4

\item {} 
\sphinxAtStartPar
fuse

\item {} 
\sphinxAtStartPar
dm\_cache

\item {} 
\sphinxAtStartPar
dm\_cache\_smq

\item {} 
\sphinxAtStartPar
dm\_mirror

\item {} 
\sphinxAtStartPar
dm\_mod

\item {} 
\sphinxAtStartPar
dm\_snapshot

\item {} 
\sphinxAtStartPar
dm\_thin\_pool

\end{itemize}

\sphinxAtStartPar
Чтобы это исправить просто добавьте эти модули вручную, т.е. отредактировав файл по пути \sphinxcode{\sphinxupquote{sudo nano \textasciitilde{}/.config/modprobed.db}}.
Затем снова пересоберите мини\sphinxhyphen{}ядро как это показано в предыдущем разделе, после пересборки мини\sphinxhyphen{}ядро должно загрузиться.

\sphinxAtStartPar
\sphinxstylestrong{П:} После установки мини\sphinxhyphen{}ядра отсутствует интернет\sphinxhyphen{}подключение.

\sphinxAtStartPar
\sphinxstylestrong{Р:} Обычно это вызвано отсутствием модулей драйвера для сетевой карты,
либо отсутствием важных системных модулей для корректной работы интернет подключения.
Вот список модулей, из\sphinxhyphen{}за которых возможно не работает сеть:
\begin{itemize}
\item {} 
\sphinxAtStartPar
8021q

\item {} 
\sphinxAtStartPar
af\_packet

\item {} 
\sphinxAtStartPar
af\_alg

\item {} 
\sphinxAtStartPar
alx

\item {} 
\sphinxAtStartPar
ecdh\_generic

\item {} 
\sphinxAtStartPar
garp

\item {} 
\sphinxAtStartPar
libphy

\item {} 
\sphinxAtStartPar
r8169

\item {} 
\sphinxAtStartPar
rc\_core

\item {} 
\sphinxAtStartPar
realtek

\item {} 
\sphinxAtStartPar
sch\_fq\_codel

\item {} 
\sphinxAtStartPar
llc

\end{itemize}

\sphinxAtStartPar
Так же, как и в случае с прошлой проблемой, просто пропишите эти модули в ручную, т.е. отредактируйте \sphinxcode{\sphinxupquote{sudo nano \textasciitilde{}/.config/modprobed.db}}.
Обратите внимание, что модуль драйвера для сетевой карты у каждого может быть разный,
и перед тем как прописать какой\sphinxhyphen{}либо модуль драйвера, лучше посмотреть в рабочей системе (\sphinxstyleemphasis{lspci \sphinxhyphen{}v}) какой именно нужен вашей сетевой карте, и прописать его.
После этого, в очередной раз, пересоберите мини\sphinxhyphen{}ядро.

\sphinxAtStartPar
\sphinxstylestrong{П:} После перезагрузки драйвер NVIDIA загружается, но вместо него используется llvmpipe.

\sphinxAtStartPar
\sphinxstylestrong{Р:} Укажите точный путь до модулей драйвера в ваших настройках Xorg, т.е. пропишите в \sphinxstyleemphasis{/etc/X11/xorg.conf} следующее:

\begin{sphinxVerbatim}[commandchars=\\\{\}]
Section \PYG{l+s+s2}{\PYGZdq{}Files\PYGZdq{}}
  ModulePath \PYG{l+s+s2}{\PYGZdq{}/usr/lib/nvidia/xorg\PYGZdq{}}
  ModulePath \PYG{l+s+s2}{\PYGZdq{}/usr/lib/xorg/modules\PYGZdq{}}
EndSection
\end{sphinxVerbatim}

\sphinxAtStartPar
Затем перезагрузитесь.

\sphinxstepscope


\chapter{Оптимизация рабочего окружения (DE)}
\label{\detokenize{source/de-optimizations:de}}\label{\detokenize{source/de-optimizations:de-optimizations}}\label{\detokenize{source/de-optimizations::doc}}
\sphinxAtStartPar
Современные среды рабочего стола стали достаточно прожорливыми и требовательными к аппаратным ресурсам компьютера,
и хотя они и довольно хороши с точки зрения удобства использования,
все же хотелось бы минимизировать потребление той же ОЗУ с их стороны.
Поэтому в этом разделе мы по отдельности и рассмотрим оптимизацию разных рабочих окружений и не только.

\index{x11@\spxentry{x11}}\index{no\sphinxhyphen{}display\sphinxhyphen{}manager@\spxentry{no\sphinxhyphen{}display\sphinxhyphen{}manager}}\index{xorg\sphinxhyphen{}xinit@\spxentry{xorg\sphinxhyphen{}xinit}}\ignorespaces 

\section{Запуск любой DE или WM без экранного менеджера \sphinxstyleemphasis{(Только для X11)}}
\label{\detokenize{source/de-optimizations:de-wm-x11}}\label{\detokenize{source/de-optimizations:launch-without-display-manager}}\label{\detokenize{source/de-optimizations:index-0}}
\sphinxAtStartPar
Почти всегда любое рабочее окружение запускается при помощи экранного менеджера (его ещё называют менеджером входа),
через который вы осуществляете вход в систему и оболочку соответственно.
Говоря ещё проще, это экран входа, где вас  просят пройти аутентификацию (ввести пароль к вашей учетной записи).
Он также выполняет функцию  управления рабочими сессиями в разных окружениях.
Тем не менее он тоже потребляет определенные ресурсы компьютера, а осуществить вход в систему можно и без него,
т.е. через tty, хоть вы и пожертвуете тем самым определенным уровнем удобства.
Для автоматизации запуска любой DE/WM (кроме Wayland сессий) через tty вам понадобиться прописать в ваш \sphinxstyleemphasis{.bash\_profile} или \sphinxstyleemphasis{.zsh\_profile} следующее:

\begin{sphinxVerbatim}[commandchars=\\\{\}]
\PYG{k}{if} \PYG{o}{[}\PYG{o}{[} \PYGZhy{}z \PYG{n+nv}{\PYGZdl{}DISPLAY} \PYG{o}{\PYGZam{}\PYGZam{}} \PYG{k}{\PYGZdl{}(}tty\PYG{k}{)} \PYG{o}{=}\PYG{o}{=} /dev/tty1 \PYG{o}{]}\PYG{o}{]}\PYG{p}{;} \PYG{k}{then}
  \PYG{n+nv}{XDG\PYGZus{}SESSION\PYGZus{}TYPE}\PYG{o}{=}x11 \PYG{n+nv}{GDK\PYGZus{}BACKEND}\PYG{o}{=}x11 \PYG{n+nb}{exec} startx
\PYG{k}{fi}
\end{sphinxVerbatim}

\sphinxAtStartPar
Это запустит X\sphinxhyphen{}сервер сразу при входе в tty1 (терминал по умолчанию).
Обязательным условием при этом является наличие установленного пакета \sphinxhref{https://archlinux.org/packages/extra/x86\_64/xorg-xinit/}{xorg\sphinxhyphen{}xinit},
и заранее настроенный файл \sphinxstyleemphasis{\textasciitilde{}/.xinitrc}, в котором прописана команда запуска вашего DE/WM (например: \sphinxcode{\sphinxupquote{exec gnome\sphinxhyphen{}session}}).
Например: \sphinxcode{\sphinxupquote{exec gnome\sphinxhyphen{}session}} \# Запускает gnome сессию при запуске Xorg сервера.

\index{gnome@\spxentry{gnome}}\index{de\sphinxhyphen{}optimizations@\spxentry{de\sphinxhyphen{}optimizations}}\ignorespaces 

\section{GNOME 3.XX/42}
\label{\detokenize{source/de-optimizations:gnome-3-xx-42}}\label{\detokenize{source/de-optimizations:gnome-optimization}}\label{\detokenize{source/de-optimizations:index-1}}
\sphinxAtStartPar
Сам по себе GNOME \sphinxhyphen{} наверное одна из самых тяжеловесных и требовательных к системным ресурсам оболочка из ныне существующих.
Тем не менее у неё есть свои преимущества перед другими оболочками, за что её и любят пользователи.
Но к сожалению низкое энергопотребление не в их числе, поэтому в этом разделе вы узнаете о том,
как заставить похудеть ваш толстенький gnome\sphinxhyphen{}shell.

\index{garbage\sphinxhyphen{}removal@\spxentry{garbage\sphinxhyphen{}removal}}\index{gnome\sphinxhyphen{}control\sphinxhyphen{}center@\spxentry{gnome\sphinxhyphen{}control\sphinxhyphen{}center}}\index{gnome@\spxentry{gnome}}\ignorespaces 

\subsection{Удаление мусора GNOME}
\label{\detokenize{source/de-optimizations:gnome}}\label{\detokenize{source/de-optimizations:gnome-garbage-removal}}\label{\detokenize{source/de-optimizations:index-2}}
\begin{sphinxVerbatim}[commandchars=\\\{\}]
sudo pacman \PYGZhy{}Rsn epiphany gnome\PYGZhy{}books gnome\PYGZhy{}boxes gnome\PYGZhy{}calculator gnome\PYGZhy{}calendar gnome\PYGZhy{}contacts gnome\PYGZhy{}maps gnome\PYGZhy{}music gnome\PYGZhy{}weather gnome\PYGZhy{}clocks gnome\PYGZhy{}photos gnome\PYGZhy{}software gnome\PYGZhy{}user\PYGZhy{}docs totem yelp gvfs\PYGZhy{}afc gvfs\PYGZhy{}goa gvfs\PYGZhy{}gphoto2 gvfs\PYGZhy{}mtp gvfs\PYGZhy{}nfs gvfs\PYGZhy{}smb gvfs\PYGZhy{}google vino gnome\PYGZhy{}user\PYGZhy{}share gnome\PYGZhy{}characters simple\PYGZhy{}scan eog tracker3\PYGZhy{}miners rygel nautilus evolution\PYGZhy{}data\PYGZhy{}server gnome\PYGZhy{}font\PYGZhy{}viewer gnome\PYGZhy{}remote\PYGZhy{}desktop gnome\PYGZhy{}logs orca
\end{sphinxVerbatim}

\sphinxAtStartPar
\sphinxstylestrong{P.S.} Удаляйте пакеты с осознанием того, что вы делайте.
Несмотря на то, что здесь были собраны наиболее сомнительные по соотношению нужности/прожорливости пакеты,
вы можете найти какой\sphinxhyphen{}либо из данных пакетов полезным и нужным.

\begin{sphinxadmonition}{warning}{Предупреждение:}
\sphinxAtStartPar
Некоторые пакеты из вышеприведенной команды могут быть не найдены в вашей системе.
В таком случае просто выпишите их из команды.
\end{sphinxadmonition}

\sphinxAtStartPar
Для совсем отчаянных парней, после окончательной настройки параметров GNOME,
вы можете удалить самый "тяжелый" пакет \sphinxhref{https://archlinux.org/packages/extra/x86\_64/gnome-control-center/}{gnome\sphinxhyphen{}control\sphinxhyphen{}center} (Параметры GNOME 3/41).

\sphinxAtStartPar
По сути, это графическая обертка для gsettings, которая однако достаточно тяжеловесная, и тянет за собой кучу ненужных зависимостей.

\begin{sphinxVerbatim}[commandchars=\\\{\}]
sudo pacman \PYGZhy{}Rsn gnome\PYGZhy{}control\PYGZhy{}center
\end{sphinxVerbatim}

\index{services@\spxentry{services}}\index{daemons@\spxentry{daemons}}\index{file\sphinxhyphen{}indexing@\spxentry{file\sphinxhyphen{}indexing}}\index{tracker3@\spxentry{tracker3}}\ignorespaces 

\subsection{Отключение Tracker 3 в GNOME}
\label{\detokenize{source/de-optimizations:tracker-3-gnome}}\label{\detokenize{source/de-optimizations:disabling-tracker-3}}\label{\detokenize{source/de-optimizations:index-3}}
\sphinxAtStartPar
Tracker \sphinxhyphen{} это встроенный поисковик для GNOME, который индексирует все файлы на диске и не только.
Как любой индексатор файловых систем, он призван кушать ресурсы и мощности вашего накопителя и висеть в оперативной памяти,
хоть и в гораздо меньшей степени чем конкуренты (До Windows, с их 100\% загруженности на диск, еще как до луны).
Тем не менее, его отключение может положительно повлиять на жизненный цикл вашего HDD (в особенности) или SSD,
поэтому его можно отключить в целях профилактики диска.
Обратите внимание, что после отключения поиск файлов в GNOME может работать некорректно и не так быстро.

\sphinxAtStartPar
\sphinxstylestrong{Инструкция по отключению}

\begin{sphinxVerbatim}[commandchars=\\\{\}]
systemctl \PYGZhy{}\PYGZhy{}user mask tracker\PYGZhy{}miner\PYGZhy{}apps tracker\PYGZhy{}miner\PYGZhy{}fs tracker\PYGZhy{}store
\end{sphinxVerbatim}

\sphinxAtStartPar
После перезагрузки системы выполните:

\begin{sphinxVerbatim}[commandchars=\\\{\}]
rm \PYGZhy{}rf \PYGZti{}/.cache/tracker \PYGZti{}/.local/share/tracker   \PYG{c+c1}{\PYGZsh{} Чистим кэш tracker}
tracker daemon \PYGZhy{}t                                \PYG{c+c1}{\PYGZsh{} Проверяем, должно быть 0 PID}
\end{sphinxVerbatim}

\index{service@\spxentry{service}}\index{daemons@\spxentry{daemons}}\index{gnome\sphinxhyphen{}settings\sphinxhyphen{}daemon@\spxentry{gnome\sphinxhyphen{}settings\sphinxhyphen{}daemon}}\ignorespaces 

\subsection{Отключение ненужных GSD служб GNOME}
\label{\detokenize{source/de-optimizations:gsd-gnome}}\label{\detokenize{source/de-optimizations:disabling-gsd-daemons}}\label{\detokenize{source/de-optimizations:index-4}}
\begin{sphinxadmonition}{attention}{Внимание:}
\sphinxAtStartPar
Способ отключения служб был обновлен.
Крайне рекомендуется использовать именно новый способ через systemd взамен старого, опасного переименования библиотек.
\end{sphinxadmonition}

\sphinxAtStartPar
GSD (gnome\sphinxhyphen{}settings\sphinxhyphen{}daemon) службы, это, как следует из названия, службы настройки GNOME и связанных приложений.
Если отойти от строго определения, то это просто службы\sphinxhyphen{}настройки на все случаи жизни,
которые просто висят у вас в оперативной памяти в ожидании когда вам, или другому приложению, к примеру,
понадобиться настроить/интегрировать поддержку планшета Wacom или других устройств. И другие подобные вещи.

\sphinxAtStartPar
\# Отключение служб интеграции GNOME с графическим планшетом Wacom.
Если у вас такого нет \sphinxhyphen{} смело отключайте.

\begin{sphinxVerbatim}[commandchars=\\\{\}]
systemctl \PYGZhy{}\PYGZhy{}user mask org.gnome.SettingsDaemon.Wacom.service
\end{sphinxVerbatim}

\sphinxAtStartPar
\# Отключение службы уведомления о печати.
Если нет принтера или вам просто не нужны эти постоянные уведомления \sphinxhyphen{} отключаем.

\begin{sphinxVerbatim}[commandchars=\\\{\}]
systemctl \PYGZhy{}\PYGZhy{}user mask org.gnome.SettingsDaemon.PrintNotifications.service
\end{sphinxVerbatim}

\sphinxAtStartPar
\# Отключение службы управления цветовыми профилями GNOME.
Отключив её не будет работать тёплый режим экрана (Системный аналог Redshift).

\begin{sphinxVerbatim}[commandchars=\\\{\}]
systemctl \PYGZhy{}\PYGZhy{}user mask org.gnome.SettingsDaemon.Color.service
\end{sphinxVerbatim}

\sphinxAtStartPar
\# Отключение службы управления специальными возможностями системы.
\sphinxstylestrong{Не отключать людям с ограниченными возможностями!}

\begin{sphinxVerbatim}[commandchars=\\\{\}]
systemctl \PYGZhy{}\PYGZhy{}user mask org.gnome.SettingsDaemon.A11ySettings.service
\end{sphinxVerbatim}

\sphinxAtStartPar
\# Отключает службу управления беспроводными интернет\sphinxhyphen{}соединениями.
Не рекомендуется отключать для ноутбуков с активным использованием Wi\sphinxhyphen{}Fi.

\begin{sphinxVerbatim}[commandchars=\\\{\}]
systemctl \PYGZhy{}\PYGZhy{}user mask org.gnome.SettingsDaemon.Wwan.service
\end{sphinxVerbatim}

\sphinxAtStartPar
\# Отключение службы защиты от неавторизованных USB устройств при блокировке экрана.
Можете оставить если у вас ноутбук.

\begin{sphinxVerbatim}[commandchars=\\\{\}]
systemctl \PYGZhy{}\PYGZhy{}user mask org.gnome.SettingsDaemon.UsbProtection.service
\end{sphinxVerbatim}

\sphinxAtStartPar
\# Отключаем службу настройки автоматической блокировки экрана.
Можете оставить если у вас ноутбук.

\begin{sphinxVerbatim}[commandchars=\\\{\}]
systemctl \PYGZhy{}\PYGZhy{}user mask org.gnome.SettingsDaemon.ScreensaverProxy.service
\end{sphinxVerbatim}

\sphinxAtStartPar
\# Отключение службы настройки общего доступа к файлам и директориям.

\begin{sphinxVerbatim}[commandchars=\\\{\}]
systemctl \PYGZhy{}\PYGZhy{}user mask org.gnome.SettingsDaemon.Sharing.service
\end{sphinxVerbatim}

\sphinxAtStartPar
\# Отключение службы управления подсистемой rfkill, отвечающей за отключения любого радиопередатчика в системе
(сюда же относятся Wi\sphinxhyphen{}Fi и Bluetooth, поэтому данная служба нужна, скорее всего, для так называемого режима в "самолете").

\begin{sphinxVerbatim}[commandchars=\\\{\}]
systemctl \PYGZhy{}\PYGZhy{}user mask org.gnome.SettingsDaemon.Rfkill.service
\end{sphinxVerbatim}

\sphinxAtStartPar
\# Отключение службы управления клавиатурой и раскладками GNOME.
Можно смело отключать если уже настроили все раскладки и настройки клавиатуры заранее,
ибо все предыдущие настройки сохраняются при отключении.

\begin{sphinxVerbatim}[commandchars=\\\{\}]
systemctl \PYGZhy{}\PYGZhy{}user mask org.gnome.SettingsDaemon.Keyboard.service
\end{sphinxVerbatim}

\sphinxAtStartPar
\# Отключаем службу управления звуком GNOME.
Отключает \sphinxstylestrong{ТОЛЬКО} настройки звука GNOME, а не вообще всё управлением звуком в системе.

\begin{sphinxVerbatim}[commandchars=\\\{\}]
systemctl \PYGZhy{}\PYGZhy{}user mask org.gnome.SettingsDaemon.Sound.service
\end{sphinxVerbatim}

\sphinxAtStartPar
\# Отключение службы интеграции GNOME с карт\sphinxhyphen{}ридером.

\begin{sphinxVerbatim}[commandchars=\\\{\}]
systemctl \PYGZhy{}\PYGZhy{}user mask org.gnome.SettingsDaemon.Smartcard.service
\end{sphinxVerbatim}

\sphinxAtStartPar
\# Отключение службы слежения за свободным пространством на диске.
Штука полезная, но если вы предпочитаете следить за этим самостоятельно, то вперед

\begin{sphinxVerbatim}[commandchars=\\\{\}]
systemctl \PYGZhy{}\PYGZhy{}user mask org.gnome.SettingsDaemon.Housekeeping.service
\end{sphinxVerbatim}

\sphinxAtStartPar
\# Отключение службы управления питанием в GNOME.
Можете оставить эту службу включенной, в случае если у вас ноутбук.

\begin{sphinxVerbatim}[commandchars=\\\{\}]
systemctl \PYGZhy{}\PYGZhy{}user mask org.gnome.SettingsDaemon.Power.service
\end{sphinxVerbatim}

\sphinxAtStartPar
\# Отключение служб Evolution для синхронизации онлайн аккаунтов
(Если вы конечно не удалили сам Evolution через команду чистки мусора выше)

\begin{sphinxVerbatim}[commandchars=\\\{\}]
systemctl \PYGZhy{}\PYGZhy{}user mask evolution\PYGZhy{}addressbook\PYGZhy{}factory evolution\PYGZhy{}calendar\PYGZhy{}factory evolution\PYGZhy{}source\PYGZhy{}registry
\end{sphinxVerbatim}

\sphinxAtStartPar
Если после отключения какой\sphinxhyphen{}либо из вышеперечисленных служб что\sphinxhyphen{}то пошло не так,
или просто какую\sphinxhyphen{}либо из них понадобилось снова включить, просто пропишите:

\begin{sphinxVerbatim}[commandchars=\\\{\}]
systemctl \PYGZhy{}\PYGZhy{}user unmask \PYGZhy{}\PYGZhy{}now СЛУЖБА
\end{sphinxVerbatim}

\sphinxAtStartPar
Служба вернется в строй после перезагрузки.

\begin{sphinxadmonition}{attention}{Внимание:}
\sphinxAtStartPar
Если вы по\sphinxhyphen{}прежнему использовали старый способ с переименованием библиотек,
то настоятельно рекомендуется выполнить переустановку пакета gnome\sphinxhyphen{}settings\sphinxhyphen{}daemon, а
затем выполнить отключение ненужных вам служб уже описанным выше способом.
\end{sphinxadmonition}

\index{installation@\spxentry{installation}}\index{gnome\sphinxhyphen{}shell@\spxentry{gnome\sphinxhyphen{}shell}}\index{mutter@\spxentry{mutter}}\index{compositor@\spxentry{compositor}}\ignorespaces 

\subsection{gnome\sphinxhyphen{}shell\sphinxhyphen{}performance и mutter\sphinxhyphen{}performance}
\label{\detokenize{source/de-optimizations:gnome-shell-performance-mutter-performance}}\label{\detokenize{source/de-optimizations:gnome-shell-and-mutter-performance}}\label{\detokenize{source/de-optimizations:index-5}}
\sphinxAtStartPar
Пакеты \sphinxhref{https://aur.archlinux.org/packages/gnome-shell-performance}{gnome\sphinxhyphen{}shell\sphinxhyphen{}performance}
и \sphinxhref{https://aur.archlinux.org/packages/mutter-performance/}{mutter\sphinxhyphen{}performance} \sphinxhyphen{}
это модифицированные версии пакетов GNOME, где упор сделан на плавность и отзывчивость благодаря включению большого количества патчей для повышения производительности DE.

\sphinxAtStartPar
\sphinxstylestrong{Установка gnome\sphinxhyphen{}shell\sphinxhyphen{}performance}

\begin{sphinxVerbatim}[commandchars=\\\{\}]
git clone https://aur.archlinux.org/gnome\PYGZhy{}shell\PYGZhy{}performance.git \PYG{c+c1}{\PYGZsh{} Загружаем исходники}
\PYG{n+nb}{cd} gnome\PYGZhy{}shell\PYGZhy{}performance                                      \PYG{c+c1}{\PYGZsh{} Переход в директорию}
makepkg \PYGZhy{}sric                                                   \PYG{c+c1}{\PYGZsh{} Сборка и установка}
\end{sphinxVerbatim}

\sphinxAtStartPar
\sphinxstylestrong{Установка mutter\sphinxhyphen{}performance}

\begin{sphinxVerbatim}[commandchars=\\\{\}]
git clone https://aur.archlinux.org/mutter\PYGZhy{}performance.git \PYG{c+c1}{\PYGZsh{} Загружаем исходники}
\PYG{n+nb}{cd} mutter\PYGZhy{}performance                                      \PYG{c+c1}{\PYGZsh{} Переход в директорию}
makepkg \PYGZhy{}sric                                              \PYG{c+c1}{\PYGZsh{} Сборка и установка}
\end{sphinxVerbatim}

\sphinxAtStartPar
Также можно выполнить нативную компиляцию пакетов при помощи Clang: \sphinxhref{https://aur.archlinux.org/packages/mesa-git/}{Mesa} (Только для оборудования Intel \& AMD),
\sphinxhref{https://aur.archlinux.org/packages/wayland-git/}{Wayland}, \sphinxhref{https://aur.archlinux.org/packages/wayland-protocols-git/}{Wayland\sphinxhyphen{}protocols},
\sphinxhref{https://aur.archlinux.org/lib32-wayland-git.git}{Lib32\sphinxhyphen{}wayland}, \sphinxhref{https://aur.archlinux.org/egl-wayland-git.git}{Egl\sphinxhyphen{}wayland},
\sphinxhref{https://aur.archlinux.org/packages/xorg-server-git/}{xorg\sphinxhyphen{}server} и многих других.

\sphinxAtStartPar
Более подробную информацию вы можете найти в разделе \sphinxhref{https://ventureoo.github.io/ARU/source/generic-system-acceleration.html\#clang}{"Общее ускорение системы"}.

\index{cosmetics@\spxentry{cosmetics}}\index{gnome@\spxentry{gnome}}\ignorespaces 

\subsection{Бонус: немного косметики}
\label{\detokenize{source/de-optimizations:gnome-cosmetics}}\label{\detokenize{source/de-optimizations:index-6}}\label{\detokenize{source/de-optimizations:id2}}
\sphinxAtStartPar
С обновлением GNOME 42 некоторые приложения на GTK 4 стали использовать тему libadwaita, но из\sphinxhyphen{}за этого
приложения на GTK 3 стали выглядить неоднородными, не говоря уж о Qt.

\sphinxAtStartPar
Чтобы это исправить, установите портированную тему libadwaita для GTK 3.

\sphinxAtStartPar
\sphinxstylestrong{Установка}

\begin{sphinxVerbatim}[commandchars=\\\{\}]
git clone https://aur.archlinux.org/adw\PYGZhy{}gtk3.git \PYG{c+c1}{\PYGZsh{} Скачиваем исходники}
\PYG{n+nb}{cd} adw\PYGZhy{}gtk3                                      \PYG{c+c1}{\PYGZsh{} Переход в директорию}
makepkg \PYGZhy{}sric                                    \PYG{c+c1}{\PYGZsh{} Сборка и установка}

\PYG{c+c1}{\PYGZsh{} Устанавливаем как тему по умолчанию}
gsettings \PYG{n+nb}{set} org.gnome.desktop.interface gtk\PYGZhy{}theme adw\PYGZhy{}gtk3
\end{sphinxVerbatim}

\index{results@\spxentry{results}}\ignorespaces 

\subsection{Результат}
\label{\detokenize{source/de-optimizations:gnome-result}}\label{\detokenize{source/de-optimizations:index-7}}\label{\detokenize{source/de-optimizations:id3}}
\sphinxAtStartPar
По окончании всех оптимизаций мы получаем потребление на уровне современной XFCE,
но в отличие от оной уже на современном GTK4, а также со всеми рабочими эффектами и анимациями.

\noindent\sphinxincludegraphics{{image2}.jpg}

\sphinxAtStartPar
\sphinxstylestrong{Видеоверсия}

\sphinxAtStartPar
\sphinxurl{https://www.youtube.com/watch?v=YlViA-nOzsg}

\sphinxAtStartPar
\sphinxstylestrong{Демонстрация плавности}

\sphinxAtStartPar
\sphinxurl{https://www.youtube.com/watch?v=1TjicRvrFbo}

\index{plasma@\spxentry{plasma}}\index{kde@\spxentry{kde}}\index{de\sphinxhyphen{}optimizations@\spxentry{de\sphinxhyphen{}optimizations}}\ignorespaces 

\section{KDE Plasma 5}
\label{\detokenize{source/de-optimizations:kde-plasma-5}}\label{\detokenize{source/de-optimizations:plasma-optimization}}\label{\detokenize{source/de-optimizations:index-8}}
\sphinxAtStartPar
Несмотря на то, что авторы ARU считают эту оболочку довольно перегруженной,
она по прежнему остается лидером по меньшему энергопотреблению оперативной памяти среди других рабочих окружений.
Однако, "бесконечность \sphinxhyphen{} не предел", поэтому в этом разделе мы сделаем так,
чтобы ваша plasma\sphinxhyphen{}shell кушала еще меньше ресурсов, и применим на ней другие твики.

\index{garbage\sphinxhyphen{}removal@\spxentry{garbage\sphinxhyphen{}removal}}\index{plasma\sphinxhyphen{}pa@\spxentry{plasma\sphinxhyphen{}pa}}\ignorespaces 

\subsection{Удаление мусора из Plasma 5}
\label{\detokenize{source/de-optimizations:plasma-5}}\label{\detokenize{source/de-optimizations:plasma-garbage-removal}}\label{\detokenize{source/de-optimizations:index-9}}
\begin{sphinxVerbatim}[commandchars=\\\{\}]
sudo pacman \PYGZhy{}Rsn kwayland\PYGZhy{}integration kwallet\PYGZhy{}pam plasma\PYGZhy{}thunderbolt plasma\PYGZhy{}vault powerdevil plasma\PYGZhy{}sdk kgamma5 drkonqi discover oxygen bluedevil plasma\PYGZhy{}browser\PYGZhy{}integration plasma\PYGZhy{}firewall
\PYG{c+c1}{\PYGZsh{} Не удаляйте powerdevil если у вас  ноутбук, а bluedevil если используете bluetooth соответственно.}

sudo pacman \PYGZhy{}Rsn plasma\PYGZhy{}pa     \PYG{c+c1}{\PYGZsh{} Удаляем виджет управления звуком.}
sudo pacman \PYGZhy{}S kmix            \PYG{c+c1}{\PYGZsh{} Замена виджету plasma\PYGZhy{}pa, совместим с ALSA.}
\end{sphinxVerbatim}

\sphinxAtStartPar
\sphinxstylestrong{P.S.} Удаляйте пакеты с осознанием того, что вы делайте.
Несмотря на то, что здесь были собраны наиболее сомнительные по соотношению нужности/прожорливости пакеты,
вы можете найти какой\sphinxhyphen{}либо из данных пакетов полезным и нужным.

\begin{sphinxadmonition}{warning}{Предупреждение:}
\sphinxAtStartPar
Некоторые пакеты из вышеприведенной команды могут быть не найдены в вашей системе.
В таком случае просто выпишите их из команды.
\end{sphinxadmonition}

\index{services@\spxentry{services}}\index{daemons@\spxentry{daemons}}\index{file\sphinxhyphen{}indexing@\spxentry{file\sphinxhyphen{}indexing}}\index{baloo@\spxentry{baloo}}\ignorespaces 

\subsection{Отключение Baloo в Plasma}
\label{\detokenize{source/de-optimizations:baloo-plasma}}\label{\detokenize{source/de-optimizations:disabling-baloo}}\label{\detokenize{source/de-optimizations:index-10}}
\sphinxAtStartPar
Baloo \sphinxhyphen{} это файловый индекстор в Plasma, аналог Tracker в GNOME, который однако
\sphinxhref{https://sun9-71.userapi.com/impg/BfaY4aziS81VH2i839oSLOx87oezAyryVyeBRA/Jpv5mJGJ7X4.jpg}{ОЧЕНЬ прожорливый},
и ест довольно много ресурсов процессора и памяти, вдобавок фоном нагружая ваш диск, в отличии от того же Tracker 3.
Поэтому, мы рекомендуем отключать его в любом случае, HDD у вас, или SSD.
Хоть разработчики и пытались исправить ситуацию с его непомерным потреблением ресурсов,
по прежнему \sphinxhref{https://sun9-23.userapi.com/impg/dREwZKZRK80G5sASKacn7mLpQ00-9I1KUncXWg/SDEoiKFoS4M.jpg}{осталась проблема}
"утечки" оперативной памяти среди подпроцессов Baloo.

\sphinxAtStartPar
\sphinxstylestrong{Инструкция по отключению:}

\begin{sphinxVerbatim}[commandchars=\\\{\}]
systemctl \PYGZhy{}\PYGZhy{}user mask kde\PYGZhy{}baloo.service           \PYG{c+c1}{\PYGZsh{} Полное отключение}
systemctl \PYGZhy{}\PYGZhy{}user mask plasma\PYGZhy{}baloorunner.service
\end{sphinxVerbatim}

\sphinxAtStartPar
Или:

\begin{sphinxVerbatim}[commandchars=\\\{\}]
balooctl \PYG{n+nb}{suspend}                  \PYG{c+c1}{\PYGZsh{} Усыпляем работу индексатора}
balooctl disable                  \PYG{c+c1}{\PYGZsh{} Отключаем Baloo}
balooctl purge                    \PYG{c+c1}{\PYGZsh{} Чистим кэш}
\end{sphinxVerbatim}

\sphinxAtStartPar
Его точно так же можно отключить в графических настройках Plasma:

\noindent\sphinxincludegraphics{{image91}.png}

\index{debug@\spxentry{debug}}\index{plasma@\spxentry{plasma}}\index{kdebugdialog5@\spxentry{kdebugdialog5}}\ignorespaces 

\subsection{Отключение отладочной информации в KDE 5}
\label{\detokenize{source/de-optimizations:kde-5}}\label{\detokenize{source/de-optimizations:disabling-kde-debug}}\label{\detokenize{source/de-optimizations:index-11}}
\sphinxAtStartPar
Слышали о таких настройках отладки в KDE? Нет? Вот и мы не слышали, а они есть.
Так как рядовой пользователь почти не видит этой самой "отладочной информации",
мы считаем что лучше отключить её вывод и не тратить на это процессорное время.
Чтобы это сделать, введите в терминал или меню запуска приложений команду \sphinxcode{\sphinxupquote{kdebugdialog5}}.
Перед вами появиться диалоговое окно, где вам нужно поставить галочку на пункте \sphinxstyleemphasis{"Отключить вывод любой отладочной информации"}.
Затем, просто нажимаете \sphinxstyleemphasis{"Применить"} и \sphinxstyleemphasis{"ОК"}.

\sphinxAtStartPar
Сбор отладочной информации теперь отключен.

\noindent\sphinxincludegraphics{{image5}.png}

\index{service@\spxentry{service}}\index{daemons@\spxentry{daemons}}\index{plasma@\spxentry{plasma}}\ignorespaces 

\subsection{Отключение ненужных служб Plasma}
\label{\detokenize{source/de-optimizations:plasma}}\label{\detokenize{source/de-optimizations:disabling-plasma-daemons}}\label{\detokenize{source/de-optimizations:index-12}}
\sphinxAtStartPar
По аналогии с GNOME, у Plasma тоже есть свои службы настройки, которые хоть и гораздо менее требовательны к ресурсам.
Тем не менее, это по прежнему солянка из различных процессов, которые вам далеко не всегда пригодятся,
а отключая ненужные из них вы можете чуть снизить потребление оперативной памяти вашей оболочкой, т.к. по умолчанию все службы включены.

\sphinxAtStartPar
Настройка служб происходит в графических настройках Plasma, в разделе "\sphinxstyleemphasis{Запуск и завершение}" \sphinxhyphen{}> \sphinxstyleemphasis{"Управление службами"}

\noindent\sphinxincludegraphics{{image12}.png}

\sphinxAtStartPar
\sphinxstylestrong{Список служб к отключению:}

\sphinxAtStartPar
\sphinxstyleemphasis{Монитор устройств Thunderbolt} \sphinxhyphen{}> Отключаем, если вы не используйте Thunderbolt

\sphinxAtStartPar
\sphinxstyleemphasis{Запуск системного монитора} \sphinxhyphen{}> Отключаем, довольно бесполезная служба.

\sphinxAtStartPar
\sphinxstyleemphasis{Напоминание, об установке расширения браузера} \sphinxhyphen{}> Еще более бесполезная служба, отключаем.

\sphinxAtStartPar
\sphinxstyleemphasis{Настройка прокси\sphinxhyphen{}серверов} \sphinxhyphen{}> Отключайте если не используете прокси/системный VPN.

\sphinxAtStartPar
\sphinxstyleemphasis{Bluetooth} \sphinxhyphen{}> Отключайте если не используйте bluetooth
(Если удален bluedevil, этого пункта может и не быть).

\sphinxAtStartPar
\sphinxstyleemphasis{Учётные записи} \sphinxhyphen{}> Нужна только если у вас больше одной учетной записи на компьютере.

\sphinxAtStartPar
\sphinxstyleemphasis{Сенсорная панель} \sphinxhyphen{}> Отключаем если её нет или вы ей не пользуйтесь.

\sphinxAtStartPar
\sphinxstyleemphasis{KScreen 2} \sphinxhyphen{}> Нужна только мультимониторным конфигурациям,
если у вас один монитор \sphinxhyphen{} отключайте.

\sphinxAtStartPar
\sphinxstyleemphasis{Обновление местоположения для коррекции цвета} \sphinxhyphen{}> Нужна для "теплого режима" экрана, аналог Redshift.
Если не пользуетесь или в ваш монитор встроен этот режим \sphinxhyphen{} отключайте.

\sphinxAtStartPar
\sphinxstyleemphasis{Модуль шифрования папок рабочей среды Plasma} \sphinxhyphen{}> Нужна только если вы параноик.
Впрочем, параноики используют более тяжёлые средства шифрования, поэтому отключаем.

\sphinxAtStartPar
\sphinxstyleemphasis{Слежение за изменениями в URL} \sphinxhyphen{}> Работает только в сетевых папках,
если вы ими не часто пользуетесь \sphinxhyphen{} отключаем.

\sphinxAtStartPar
\sphinxstyleemphasis{Слежение за свободным местом на диске} \sphinxhyphen{}> Вещь полезная, но это вы можете сделать и самостоятельно через виджеты,
поэтому Откл./Оставлять по желанию.

\sphinxAtStartPar
\sphinxstyleemphasis{SMART} \sphinxhyphen{}> Тоже довольно полезная служба, отключайте на свое усмотрение.

\sphinxAtStartPar
\sphinxstyleemphasis{Диспетчер уведомлений о состоянии} \sphinxhyphen{}> Нужна для правильной работы лотка и трея.

\sphinxAtStartPar
\sphinxstyleemphasis{Служба синхронизации параметров GNOME/GTK} \sphinxhyphen{}> Осуществляет смену GTK темы на лету.
Если отключите, смена GTK темы будет применяться только после перезагрузки.

\sphinxAtStartPar
\sphinxstyleemphasis{Фоновая служба клавиатуры} \sphinxhyphen{}> Служба для отображения раскладки в системном лотке.

\sphinxAtStartPar
\sphinxstyleemphasis{Служба локальных сообщений} \sphinxhyphen{}> Следит в общении между терминалами через команды wall и write.
Это очень специфично, поэтому отключаем.

\sphinxAtStartPar
\sphinxstyleemphasis{Модуль для управления сетью} \sphinxhyphen{}> Добавляет системный лоток виджет для управления сетевыми подключениями.
Отключайте, если не используете NetworkManager.

\sphinxAtStartPar
\sphinxstyleemphasis{Состояние сети} \sphinxhyphen{}> Оповещает приложения в случае неработоспособности интернет\sphinxhyphen{}соединения.
Тоже довольно нишевая служба, можно отключить.

\sphinxAtStartPar
\sphinxstyleemphasis{Подключение внешних носителей} \sphinxhyphen{}> Автоматически примонтирует внешние устройства при их подключении.
Например, такие как USB\sphinxhyphen{}флешки. Отключайте на свое усмотрение.

\sphinxAtStartPar
\sphinxstyleemphasis{Часовой пояс} \sphinxhyphen{}> Информирует другие приложения об изменении  системного часового пояса.
Довольно редко применимо, можно отключить.

\sphinxAtStartPar
\sphinxstyleemphasis{Обновление папок поиска} \sphinxhyphen{}> Автоматически обновляет результат поиска файлов.
Отключаем на свое усмотрение. Кроме того, судя по всему работает только в Dolphin.

\sphinxAtStartPar
\sphinxstyleemphasis{Действия} \sphinxhyphen{}> Обеспечивает работу специально назначенных действий в настройках.
Если вы не используйте кастомные бинды, можете отключить.

\sphinxAtStartPar
\sphinxstyleemphasis{Фоновая служба меню приложений} \sphinxhyphen{}> Странная служба.
По своей функции она осуществляет обновление Меню Приложений при появлении новых ярлыков,
однако даже при её отключении этот функционал работает.
Отключайте на свое усмотрение.

\index{lowlatency@\spxentry{lowlatency}}\index{compositor@\spxentry{compositor}}\index{kwin@\spxentry{kwin}}\index{vsync@\spxentry{vsync}}\ignorespaces 

\subsection{Настройка работы KWin для увеличения плавности}
\label{\detokenize{source/de-optimizations:kwin}}\label{\detokenize{source/de-optimizations:lowlatency-kwin}}\label{\detokenize{source/de-optimizations:index-13}}
\sphinxAtStartPar
До недавнего времени у Plasma были определенные проблемы с качеством отрисовки и работой композитора в целом.
Были и серьёзные проблемы при работе с закрытым драйвером NVIDIA. Правда, начиная с версии плазмы 5.21, ситуация значительно улучшилась,
но по прежнему довольно нестабильна.
Напомним, что композитор, и одновременно оконный менеджер, в Plasma это kwin \sphinxhyphen{} и он отвечает за:
\begin{enumerate}
\sphinxsetlistlabels{\arabic}{enumi}{enumii}{}{.}%
\item {} 
\sphinxAtStartPar
Управление окнами, и все что с ними связано.

\item {} 
\sphinxAtStartPar
Различные графические эффекты и визуальные "приблуды" (Прозрачность, тени, размытие и проч.)

\item {} 
\sphinxAtStartPar
Плавность отрисовки и бесшовность отображаемой картинки, т. е. обеспечивает синхронизацию между кадрами (Vsync), предотвращает тиринг (разрывы экрана).

\end{enumerate}

\sphinxAtStartPar
Вообщем, делает довольно много интересных вещей.

\sphinxAtStartPar
Но нас интересует только третья и немного вторая его функции.

\sphinxAtStartPar
Итак, чтобы обеспечить наилучшую плавность и визуальное качество отклика, нам нужно провести грамотную его (композитора) настройку.
Для этого мы перейдем в соответствующий раздел настроек Plasma, т. е. в \sphinxstyleemphasis{Экран} \sphinxhyphen{}> \sphinxstyleemphasis{Обеспечение Эффектов}.

\noindent\sphinxincludegraphics{{image41}.png}

\sphinxAtStartPar
Что\sphinxhyphen{}ж, давайте по порядку.

\sphinxAtStartPar
\sphinxstylestrong{"Включать графические эффекты при входе в систему"}

\sphinxAtStartPar
Данная опция отвечает за то, будет ли композитор брать на себя роль за отрисовку графических эффектов, и синхронизации кадров соответственно.
Т. е. будет ли он выполнять свои две последние функции (См. выше) сразу после запуска оболочки.
Вы можете отключить этот параметр, в случае крайней экономии аппаратных ресурсов,
т.к. это снимет с композитора роль за граф. эффекты и вертикальную синхронизацию,
то это также может уменьшить его потребление ресурсов компьютера вдвое,
и он просто станет лишь менеджером управления окнами.

\sphinxAtStartPar
\sphinxstylestrong{"Механизм отрисовки"}

\sphinxAtStartPar
Отвечает за то, средствами какого API\sphinxhyphen{}бэкенда будет производиться отрисовка.
OpenGL механизм дает больше возможностей для обеспечения различных графических эффектов, и лучшую синхронизацию кадров.
Принципиальной разницы между OpenGL 2.0 и OpenGL 3.1 \sphinxhyphen{} нет.
Поддержка OpenGL 2.0 нужна и остается только для работы со старыми видеокартами, у которых нет поддержки OpenGL 3.1.
XRender механизм считается морально устаревшим, он не поддерживает такое же количество граф. эффектов как OpenGL,
поэтому не удивляетесь что какие\sphinxhyphen{}то из них не будут работать на этом механизме отрисовки.
Кроме того, с этим бэкендом не работает синхронизация кадров, т. е. Vsync автоматически отключается при выборе данного механизма, и может появиться тиринг.
Тем не менее, XRender обеспечивает практически минимальное потребление оперативной памяти компьютера со стороны композитора,
и полагается в основном на ресурсы центрального процессора, практически не задействуя видеокарту и не создавая задержки ввода.
Поэтому он может эффективно использоваться в комбинации с включенной \sphinxstyleemphasis{"Tearfree"} опцией открытого драйвера AMD/Intel исправляющей тиринг,
и  \sphinxstyleemphasis{"ForceCompostionPipeline"} закрытого драйвера NVIDIA
(Что, впрочем, не очень рекомендуется при наличии OpenGL бэкенда с поддержкой Vsync) или NVIDIA PRIME Sync
(В таком случае даже рекомендуется его использовать, т.к. это может исправить проблему высокой задержки на ноутбуках с поддержкой NVIDIA PRIME,
а проблема тиринга при этом будет решаться использованием самой технологии PRIME Sync).
И конечно для AMD Freesync и Nvidia Gsync.

\sphinxAtStartPar
\sphinxstylestrong{"Задержка отрисовки"}

\sphinxAtStartPar
Параметр напрямую влияющий на плавность отрисовки и синхронизацию между кадрами.
Он задает с какой задержкой композитор перейдет к композитингу и синхронизации следующего кадра.
Соответственно, чем меньше задержка между этими событиями, тем быстрее композитор сможет нарисовать последующие кадры,
благодаря чему и достигается такое расплывчатое понятие, как "плавность" картинки,
отсутствие высокой задержки ввода (input lag) и в тоже время бесшовность картинки, т.е. отсутствие тиринга.
Лучшим вариантом для закрытого драйвера NVIDIA будет, и настоятельно рекомендуется \sphinxhyphen{} \sphinxstyleemphasis{"Принудительно низкая задержка"}.
Для открытых драйверов Intel/AMD не все так однозначно, и с принудительно низкой задержкой могут возникать артефакты отрисовки.
Тем не менее, все также рекомендуется \sphinxstyleemphasis{"Предпочитать низкую задержку"}.

\sphinxAtStartPar
\sphinxstylestrong{"Предотвращение разрывов (VSync)"}

\sphinxAtStartPar
Здесь, мы выбираем метод с которым будут синхронизироваться наши кадры (VSync).
Лучше всего отдать его предпочтение автоматическому выбору самого композитора под ваш видеодрайвер, т. е. \sphinxstyleemphasis{"Автоматически"}.
Можно также отдать предпочтение методу \sphinxstyleemphasis{"При минимуме затрат"}, где следуя из названия, будут достигаться минимальные затраты на синхронизацию кадра.
Однако, этот метод работает только при обновлении всего экрана, например при воспроизведении видео.
Поэтому при его использовании может \sphinxstyleemphasis{"проявляться"} тиринг в некоторых местах при частичном обновлении экрана.
Другие методы могут ухудшать производительность, либо в целом, либо для определенных видеодрайверов
(\sphinxstyleemphasis{"Повторное использование"} ухудшает производительность при использовании с драйверами Mesa, т.е. на оборудовании с Intel/AMD).

\sphinxAtStartPar
\sphinxstylestrong{"Разрешить приложениям блокировать режим с графическими эффектами"}

\sphinxAtStartPar
Не всегда, и не во всех приложениях нужно осуществлять композитинг и отрисовку графических эффектов,
поэтому была сделана эта опция чтобы дать разрешение на их блокировку другими приложениями.
В целом, блокировка графических эффектов нужна в основном для полноэкранных видеоигр,
чтобы не создавать для них лишней задержки ввода и немного улучшить их производительность.
Настоятельно рекомендуется оставлять включенным данный параметр.

\sphinxAtStartPar
\sphinxstylestrong{"Метод масштабирования"}

\sphinxAtStartPar
Из названия понятно, что это метод с которым у вас будет масштабироваться интерфейс.

\sphinxAtStartPar
\sphinxstyleemphasis{"Простое растяжение пикселов"} \sphinxhyphen{} Самый производительный метод, но в тоже время самый топорный по качеству.

\sphinxAtStartPar
\sphinxstyleemphasis{"Со сглаживанием"} \sphinxhyphen{} оптимальный вариант, и рекомендуется большинству конфигураций.

\sphinxAtStartPar
\sphinxstyleemphasis{"Точное сглаживание"} \sphinxhyphen{} Лучший вариант с точки зрения качества, но при этом жертвуете некоторой производительностью,
и этот метод может работать не со всеми видеокартами и приводить к артефактам отрисовки.

\index{lowlatency@\spxentry{lowlatency}}\index{compositor@\spxentry{compositor}}\index{x11\sphinxhyphen{}unredirection@\spxentry{x11\sphinxhyphen{}unredirection}}\index{kwin@\spxentry{kwin}}\ignorespaces 

\subsection{Отключение композитинга для полноэкранных окон}
\label{\detokenize{source/de-optimizations:kwin-full-screen-unredirection}}\label{\detokenize{source/de-optimizations:index-14}}\label{\detokenize{source/de-optimizations:id6}}
\sphinxAtStartPar
\sphinxhref{https://store.kde.org/p/1502826/}{kwin\sphinxhyphen{}autocomposer} \sphinxhyphen{} расширение для Kwin, которое позволяет
полностью отключить композитинг для полноэкранных окон в X11 сессии Plasma. Это помогает
исправить дрожание фреймтайма во время игры и понизить задержки.

\sphinxAtStartPar
Для Wayland сессий Plasma с версии 5.22 отключение композитинга полноэкранных окон происходит
по умолчанию.

\sphinxAtStartPar
\sphinxstylestrong{Установка}

\sphinxAtStartPar
Зайдите в настройки, затем в раздел \sphinxstyleemphasis{Диспетчер окон} \sphinxhyphen{}> \sphinxstyleemphasis{Сценарии Kwin}.

\noindent\sphinxincludegraphics{{kwin-autocomposer-1}.png}

\sphinxAtStartPar
Внизу найдите кнопку \sphinxstyleemphasis{"Загрузить новые сценарии"}

\noindent\sphinxincludegraphics{{kwin-autocomposer-2}.png}

\sphinxAtStartPar
Найдите в представленном катологе \sphinxstyleemphasis{"Autocomposer"} выоплните его установку.

\noindent\sphinxincludegraphics{{kwin-autocomposer-3}.png}

\sphinxAtStartPar
После этого перезагрузите рабочее окружение. Готово.

\index{lowlatency@\spxentry{lowlatency}}\index{compositor@\spxentry{compositor}}\index{kwin@\spxentry{kwin}}\index{effects@\spxentry{effects}}\ignorespaces 

\subsection{Отключение ненужных графических эффектов Plasma}
\label{\detokenize{source/de-optimizations:disabling-kwin-effects}}\label{\detokenize{source/de-optimizations:index-15}}\label{\detokenize{source/de-optimizations:id7}}
\sphinxAtStartPar
Plasma предоставляет возможность использовать много различных графических эффектов (С включенным методом отрисовки OpenGL естественно).
Но далеко не все из них нужны, и, по сути, являются сугубо декоративным элементом,
которые при этом потребляют некоторые мощности оперативной памяти и GPU на их отрисовку.
Поэтому, если вы хотите минимизировать потребление этих ресурсов,
рекомендуется либо полностью, либо частично отключить графические эффекты.
Осуществить это можно, либо как уже говорилось выше, сняв галочку с \sphinxstyleemphasis{"Включать графические эффекты при входе в систему"} в настройках Plasma \sphinxstyleemphasis{"Экран \sphinxhyphen{}> Обеспечение эффектов"},
либо можно частично отключить определенные граф. эффекты в настройках \sphinxstyleemphasis{"Поведение рабочей среды"} \sphinxhyphen{}> \sphinxstyleemphasis{"Эффекты"}.
Какие из них оставлять, а какие нет \sphinxhyphen{} решать только вам, но чем меньше эффектов будет включено, тем меньше потребление ресурсов.

\index{results@\spxentry{results}}\ignorespaces 

\subsection{Результат}
\label{\detokenize{source/de-optimizations:plasma-result}}\label{\detokenize{source/de-optimizations:index-16}}\label{\detokenize{source/de-optimizations:id8}}
\noindent\sphinxincludegraphics{{image1}.jpg}

\index{xfce@\spxentry{xfce}}\index{xfce4@\spxentry{xfce4}}\index{de\sphinxhyphen{}optimizations@\spxentry{de\sphinxhyphen{}optimizations}}\ignorespaces 

\section{Xfce4}
\label{\detokenize{source/de-optimizations:xfce4}}\label{\detokenize{source/de-optimizations:xfce-optimization}}\label{\detokenize{source/de-optimizations:index-17}}
\sphinxAtStartPar
Xfce, или мышонок в простонародье, является примером "старой школы" среди всех рабочих окружений.
Он до сих пор сохранил свою незамысловатость и простоту, однако с последними выпусками и переходом на GTK3 к сожалению потерял свою легковесность.
Поэтому в этом разделе, мы поговорим об оптимизации Xfce.

\index{garbage\sphinxhyphen{}removal@\spxentry{garbage\sphinxhyphen{}removal}}\index{xfce@\spxentry{xfce}}\ignorespaces 

\subsection{Удаление потенциально ненужных компонентов Xfce}
\label{\detokenize{source/de-optimizations:xfce}}\label{\detokenize{source/de-optimizations:xfce-garbage-removal}}\label{\detokenize{source/de-optimizations:index-18}}
\sphinxAtStartPar
Честно говоря, в Xfce довольно мало откровенно "ненужных" пакетов. И, по сути, все сводиться к личным предпочтениям, какие пакеты вам нужны, а какие нет.
Поэтому рассматриваете указанные ниже инструкции по удалению на свой лад.

\begin{sphinxVerbatim}[commandchars=\\\{\}]
\PYG{c+c1}{\PYGZsh{} Удалит менеджер питания Xfce. Нужен только если у вас ноутбук и нужно настроить энергосбережение. На ПК можно считать это лишним фоновым процессом который висит у вас в памяти.}
sudo pacman \PYGZhy{}Rn xfce4\PYGZhy{}power\PYGZhy{}manager

\PYG{c+c1}{\PYGZsh{} Пожалуй единственный, действительно мусорный пакет, который весит процессом на случай если вам нужно будет \PYGZdq{}найти приложение\PYGZdq{}, которые вы можете и сами найти в соответствующем меню.}
sudo pacman \PYGZhy{}Rsn xfce4\PYGZhy{}appfinder

\PYG{c+c1}{\PYGZsh{} Набор тем для Xfwm (Оконного менеджера по умолчанию в Xfce). Удаляйте по желанию.}
sudo pacman \PYGZhy{}Rsn xfwm4\PYGZhy{}themes

\PYG{c+c1}{\PYGZsh{} Дополнение к Thunar, и фоновый процесс для удобного и скорого управления различными съемными устройствами при их подключении,}
например такими как USB\PYGZhy{}флешки, CD диски, камера и пр.. Если такими устройствами не пользуетесь, или делаете это не часто \PYGZhy{} можете удалять.
sudo pacman \PYGZhy{}Rsn thunar\PYGZhy{}volman

\PYG{c+c1}{\PYGZsh{} Создает превью изображений различных форматов для Thunar. Довольно прожорливая штука, поэтому если хотите можете его удалить.}
sudo pacman \PYGZhy{}Rsn tumbler

\PYG{c+c1}{\PYGZsh{} Терминал по умолчанию для Xfce. Является довольно прожорливым, поэтому можете заменить его на менее энергозатратные аналоги.}
sudo pacman \PYGZhy{}Rsn xfce4\PYGZhy{}terminal

\PYG{c+c1}{\PYGZsh{} Графическая обертка для главной панели настроек Xfce. По желанию можете удалить, и использовать вместо неё xfconf\PYGZhy{}query.}
sudo pacman \PYGZhy{}Rsn xfce4\PYGZhy{}settings

\PYG{c+c1}{\PYGZsh{} Демон отображения уведомлений в Xfce. Можете удалить и заменить на более легковесные аналоги (например, dunst), не забудьте при этом добавить замену в автозагрузку.}
sudo pacman \PYGZhy{}Rsn xfce4\PYGZhy{}notifyd
\end{sphinxVerbatim}

\index{service@\spxentry{service}}\index{daemons@\spxentry{daemons}}\index{xfce@\spxentry{xfce}}\ignorespaces 

\subsection{Отключение ненужных служб и приложений автозапуска}
\label{\detokenize{source/de-optimizations:disabling-xfce-daemons}}\label{\detokenize{source/de-optimizations:index-19}}\label{\detokenize{source/de-optimizations:id9}}
\sphinxAtStartPar
В Xfce также не так много различных фоновых служб, скорее их очень мало.
Тем не менее, они есть, и не все они лично вам могут быть нужны.
Настроить их вы можете в настройках \sphinxstyleemphasis{"Сеансы и запуск"} \sphinxhyphen{}> \sphinxstyleemphasis{"Автозапуск приложений"}.
Отключить вы можете почти все, они не очень важны для работоспособности оболочки.
Единственное, что вы можете оставить \sphinxhyphen{} это \sphinxstyleemphasis{"PolicyKit Authentication Agent"}, для приложений требующих пароль на выполнение действий из под sudo/root.
Служба \sphinxstyleemphasis{"Tracker FIle System Miner"} \sphinxhyphen{} это встроенный файловый индексатор Xfce, его можете либо включить для корректной работы поиска в оболочке и Thunar,
либо отключить в целях экономии ресурсов компьютера.

\noindent\sphinxincludegraphics{{image11}.png}

\index{lowlatency@\spxentry{lowlatency}}\index{compositor@\spxentry{compositor}}\index{xfwm@\spxentry{xfwm}}\index{x11\sphinxhyphen{}unrediction@\spxentry{x11\sphinxhyphen{}unrediction}}\index{vsync@\spxentry{vsync}}\ignorespaces 

\subsection{Настройка композитора Xfwm4}
\label{\detokenize{source/de-optimizations:xfwm4}}\label{\detokenize{source/de-optimizations:lowlatency-xfwm}}\label{\detokenize{source/de-optimizations:index-20}}
\sphinxAtStartPar
Композитор по умолчанию в Xfce это Xfwm.
К сожалению, порой он достаточно неэффективно выполняет функцию синхронизации кадров (Vsync),
поэтому нужно выполнить самостоятельную настройку его работы для исправления проблем тиринга.
Сделать это можно в \sphinxstyleemphasis{"Редакторе Настроек"} \sphinxhyphen{}> \sphinxstyleemphasis{"xfwm4"}.
Здесь нас интересуют три опции, а именно: \sphinxstyleemphasis{"vblank\_mode"}, \sphinxstyleemphasis{"unredirect\_overlays"} и \sphinxstyleemphasis{"use\_compositing"}. Теперь подробнее.

\sphinxAtStartPar
\sphinxcode{\sphinxupquote{xfconf\sphinxhyphen{}query \sphinxhyphen{}c xfwm4 \sphinxhyphen{}p /general/unredirect\_overlays \sphinxhyphen{}s true}} \# Параметр на отвязку полноэкранных окон от работы композитора.
В разделе c Plasma эта тема освещалась более подробно.
В основном, это применимо к полноэкранным видеоиграм, чтобы не создавать для них лишнюю задержку ввода и немного улучшить их производительность.

\sphinxAtStartPar
\sphinxcode{\sphinxupquote{xfce\sphinxhyphen{}query \sphinxhyphen{}c xfwm4 \sphinxhyphen{}p /general/use\_compositing \sphinxhyphen{}s true}} \# Параметр для переключения работы графических эффектов и вертикальной синхронизации композитора.
Если отключите (\sphinxstyleemphasis{false}), то Xfwm больше не будет выполнять ни вертикальную синхронизацию, ни отрисовку граф. эффектов, и станет просто оконным менеджером.
В целях уменьшения потребления ресурсов, это рекомендуется выключить, однако может снова возникнуть проблема тиринга.
Как её решить без применения вертикальной синхронизации было указано ниже,
но вы также можете использовать сторонний композитор для решения этой проблемы, например такой как Picom.
Чтобы это сделать нужно отключить графические эффекты Xfwm, т.е. как раз выключить параметр \sphinxstyleemphasis{use\_compositing},
и установить \sphinxhref{https://archlinux.org/packages/community/x86\_64/picom/}{picom} (\sphinxstyleemphasis{sudo pacman \sphinxhyphen{}S picom}).
И затем добавить его в автозагрузку (См. приложение). Вот и все.

\noindent\sphinxincludegraphics{{image13}.png}

\sphinxAtStartPar
vblank\_mode задает через какие средства будет осуществляться вертикальная синхронизация кадров. Всего есть три возможных значения:
\begin{enumerate}
\sphinxsetlistlabels{\arabic}{enumi}{enumii}{}{.}%
\item {} 
\sphinxAtStartPar
\sphinxcode{\sphinxupquote{xfconf\sphinxhyphen{}query \sphinxhyphen{}c xfwm4 \sphinxhyphen{}p /general/vblank\_mode \sphinxhyphen{}s glx}} \# Композитинг и синхронизация кадров при помощи OpenGL.
Самый надежный вариант для исправления проблем тиринга, как для открытых драйверов, так и (в особенности) для закрытого драйвера NVIDIA.
Может создавать некоторую задержку ввода.

\item {} 
\sphinxAtStartPar
\sphinxcode{\sphinxupquote{xfconf\sphinxhyphen{}query \sphinxhyphen{}c xfwm4 \sphinxhyphen{}p /general/vblank\_mode \sphinxhyphen{}s xpresent}} \# Морально устаревший бэкенд отрисовки, который почти не использует ресурсы видеокарты,
и перекладывает основную нагрузку за отрисовку эффектов и синхронизации кадров на процессор.
В целом, потребление ресурсов с ним меньше чем под glx, и он не создает лишней задержки ввода.
И все же, он довольно плохо решает проблему тиринга, поэтому порой он может проявляться.
С Закрытым драйвером NVIDIA вертикальная синхронизация при xpresent вообще не будет работать.

\item {} 
\sphinxAtStartPar
\sphinxcode{\sphinxupquote{xfconf\sphinxhyphen{}query \sphinxhyphen{}c xfwm4 \sphinxhyphen{}p /general/vblank\_mode \sphinxhyphen{}s off}} \# Отключение вертикальной синхронизации кадров.
Этот вариант можно рассмотреть, в случае если вы компенсируете проблему тиринга через опции драйвера \sphinxstyleemphasis{"Tearfree"} для Intel/AMD,
и \sphinxstyleemphasis{"ForceCompistionPipiline"} для закрытого драйвера NVIDIA или NVIDIA PRIME Sync
(Что даже рекомендуется, т.к. NVIDIA PRIME Sync это единственный возможный способ полного исправления проблемы тиринга на ноутбуках с NVIDIA PRIME,
и никакая дополнительная синхронизация обычно не нужна).
Также эта опция настоятельно рекомендуется пользователям технологий AMD Freesync и Nvidia G\sphinxhyphen{}Sync.

\end{enumerate}

\index{results@\spxentry{results}}\ignorespaces 

\subsection{Результат}
\label{\detokenize{source/de-optimizations:xfce-result}}\label{\detokenize{source/de-optimizations:index-21}}\label{\detokenize{source/de-optimizations:id10}}
\noindent\sphinxincludegraphics{{image81}.png}

\index{cinnamon@\spxentry{cinnamon}}\index{de\sphinxhyphen{}optimizations@\spxentry{de\sphinxhyphen{}optimizations}}\ignorespaces 

\section{Cinnamon}
\label{\detokenize{source/de-optimizations:cinnamon}}\label{\detokenize{source/de-optimizations:cinnamon-optimization}}\label{\detokenize{source/de-optimizations:index-22}}
\sphinxAtStartPar
Cinnamon, или дословно корица, это форк GNOME 3, который был создан разработчиками Linux Mint для исправления проблем своего родителя,
когда последний был в крайне нестабильном состоянии.
И отчасти им это удалось, но одну из главных проблем GNOME она (корица), к сожалению, унаследовала \sphinxhyphen{}
это большое потребление оперативной памяти и других ресурсов компьютера.
Поэтому здесь мы поговорим об оптимизации нашей булочки с корицей.

\index{service@\spxentry{service}}\index{daemons@\spxentry{daemons}}\index{cinnamon\sphinxhyphen{}settings\sphinxhyphen{}daemon@\spxentry{cinnamon\sphinxhyphen{}settings\sphinxhyphen{}daemon}}\ignorespaces 

\subsection{Отключение ненужных CSD служб (НОВЫЙ СПОСОБ)}
\label{\detokenize{source/de-optimizations:csd}}\label{\detokenize{source/de-optimizations:disabling-cinnamon-daemons}}\label{\detokenize{source/de-optimizations:index-23}}
\sphinxAtStartPar
Будучи форком GNOME 3, Cinnamon также имеет свой аналог GSD служб, которые называются CSD службами (Cinnamon Settings Daemon).
Принципиальных различий от GSD служб у них по сути нет, просто другое название и немного измененный состав.

\begin{sphinxVerbatim}[commandchars=\\\{\}]
\PYG{n+nb}{cd} \PYGZti{}/.config/autostart \PYG{c+c1}{\PYGZsh{} Переходим в директорию автозагрузки}
cp \PYGZhy{}v /etc/xdg/autostart/cinnamon\PYGZhy{}settings\PYGZhy{}daemon\PYGZhy{}*.desktop ./ \PYG{c+c1}{\PYGZsh{} Копируем автозагрузку служб}
\end{sphinxVerbatim}

\sphinxAtStartPar
\# Отключение служб интеграции Cinnamon с графическим планшетом Wacom.
Если у вас его нет \sphinxhyphen{} смело отключайте.

\begin{sphinxVerbatim}[commandchars=\\\{\}]
\PYG{n+nb}{echo} \PYG{l+s+s2}{\PYGZdq{}Hidden=true\PYGZdq{}} \PYGZgt{}\PYGZgt{} cinnamon\PYGZhy{}settings\PYGZhy{}daemon\PYGZhy{}wacom.desktop
\end{sphinxVerbatim}

\sphinxAtStartPar
\# Отключение службы интеграции принтера в Cinnamon.

\begin{sphinxVerbatim}[commandchars=\\\{\}]
\PYG{n+nb}{echo} \PYG{l+s+s2}{\PYGZdq{}Hidden=true\PYGZdq{}} \PYGZgt{}\PYGZgt{} cinnamon\PYGZhy{}settings\PYGZhy{}daemon\PYGZhy{}print\PYGZhy{}notifications.desktop
\end{sphinxVerbatim}

\sphinxAtStartPar
\# Отключение службы настройки цветовых профилей в Cinnamon.:

\begin{sphinxVerbatim}[commandchars=\\\{\}]
\PYG{n+nb}{echo} \PYG{l+s+s2}{\PYGZdq{}Hidden=true\PYGZdq{}} \PYGZgt{}\PYGZgt{} cinnamon\PYGZhy{}settings\PYGZhy{}daemon\PYGZhy{}color.desktop
\end{sphinxVerbatim}

\sphinxAtStartPar
\# Отключение служб настройки "Специальных Возможностей" в Cinnamon.
\sphinxstylestrong{Не отключать людям с ограниченными возможностями!}

\begin{sphinxVerbatim}[commandchars=\\\{\}]
\PYG{n+nb}{echo} \PYG{l+s+s2}{\PYGZdq{}Hidden=true\PYGZdq{}} \PYGZgt{}\PYGZgt{} cinnamon\PYGZhy{}settings\PYGZhy{}daemon\PYGZhy{}a11y\PYGZhy{}settings.desktop
\PYG{n+nb}{echo} \PYG{l+s+s2}{\PYGZdq{}Hidden=true\PYGZdq{}} \PYGZgt{}\PYGZgt{} cinnamon\PYGZhy{}settings\PYGZhy{}daemon\PYGZhy{}a11y\PYGZhy{}keyboard.desktop
\end{sphinxVerbatim}

\sphinxAtStartPar
\# Отключение службы настройки автоматической блокировки экрана.

\begin{sphinxVerbatim}[commandchars=\\\{\}]
\PYG{n+nb}{echo} \PYG{l+s+s2}{\PYGZdq{}Hidden=true\PYGZdq{}} \PYGZgt{}\PYGZgt{} cinnamon\PYGZhy{}settings\PYGZhy{}daemon\PYGZhy{}screensaver\PYGZhy{}proxy.desktop
\end{sphinxVerbatim}

\sphinxAtStartPar
\# Отключаем службу управления звуком Cinnamon.
Отключает \sphinxstylestrong{ТОЛЬКО} настройки звука Cinnamon, а не вообще все управление звуком в системе.

\begin{sphinxVerbatim}[commandchars=\\\{\}]
\PYG{n+nb}{echo} \PYG{l+s+s2}{\PYGZdq{}Hidden=true\PYGZdq{}} \PYGZgt{}\PYGZgt{} cinnamon\PYGZhy{}settings\PYGZhy{}daemon\PYGZhy{}sound.desktop
\end{sphinxVerbatim}

\sphinxAtStartPar
\# Отключение службы интеграции Cinnamon с картридером.

\begin{sphinxVerbatim}[commandchars=\\\{\}]
\PYG{n+nb}{echo} \PYG{l+s+s2}{\PYGZdq{}Hidden=true\PYGZdq{}} \PYGZgt{}\PYGZgt{} cinnamon\PYGZhy{}settings\PYGZhy{}daemon\PYGZhy{}smartcard.desktop
\end{sphinxVerbatim}

\sphinxAtStartPar
\# Отключение службы настройки клавиатуры и раскладок Cinnamon.
Можно смело выключать если вы уже настроили все раскладки и настройки клавиатуры.

\begin{sphinxVerbatim}[commandchars=\\\{\}]
\PYG{n+nb}{echo} \PYG{l+s+s2}{\PYGZdq{}Hidden=true\PYGZdq{}} \PYGZgt{}\PYGZgt{} cinnamon\PYGZhy{}settings\PYGZhy{}daemon\PYGZhy{}keyboard.desktop
\end{sphinxVerbatim}

\sphinxAtStartPar
\# Выключаем службу настройки мониторов Cinnamon.
Смело отключайте если у вас нет более одного монитора (ноутбук) и вы настроили герцовку уже имеющихся мониторов.

\begin{sphinxVerbatim}[commandchars=\\\{\}]
\PYG{n+nb}{echo} \PYG{l+s+s2}{\PYGZdq{}Hidden=true\PYGZdq{}} \PYGZgt{}\PYGZgt{} cinnamon\PYGZhy{}settings\PYGZhy{}daemon\PYGZhy{}xrandr.desktop
\end{sphinxVerbatim}

\sphinxAtStartPar
\# Отключаем службу автоматического монтирования внешних, подключаемых устройств.
Например таких как USB\sphinxhyphen{}флешки, CD диски и прочие внешние носители.

\begin{sphinxVerbatim}[commandchars=\\\{\}]
\PYG{n+nb}{echo} \PYG{l+s+s2}{\PYGZdq{}Hidden=true\PYGZdq{}} \PYGZgt{}\PYGZgt{} cinnamon\PYGZhy{}settings\PYGZhy{}daemon\PYGZhy{}automount.desktop
\end{sphinxVerbatim}

\sphinxAtStartPar
\# Отключаем службу слежения за свободным пространством на диске.

\begin{sphinxVerbatim}[commandchars=\\\{\}]
\PYG{n+nb}{echo} \PYG{l+s+s2}{\PYGZdq{}Hidden=true\PYGZdq{}} \PYGZgt{}\PYGZgt{} cinnamon\PYGZhy{}settings\PYGZhy{}daemon\PYGZhy{}housekeeping.desktop
\end{sphinxVerbatim}

\sphinxAtStartPar
\# Отключаем службу настройки ориентацией дисплея. Если у вас нет сенсорного экрана или поддержки переворота дисплея \sphinxhyphen{} отключайте.:

\begin{sphinxVerbatim}[commandchars=\\\{\}]
\PYG{n+nb}{echo} \PYG{l+s+s2}{\PYGZdq{}Hidden=true\PYGZdq{}} \PYGZgt{}\PYGZgt{} cinnamon\PYGZhy{}settings\PYGZhy{}daemon\PYGZhy{}orientation.desktop
\end{sphinxVerbatim}

\sphinxAtStartPar
\# Отключение службы настройки мыши и тачпада Cinnamon.

\begin{sphinxVerbatim}[commandchars=\\\{\}]
\PYG{n+nb}{echo} \PYG{l+s+s2}{\PYGZdq{}Hidden=true\PYGZdq{}} \PYGZgt{}\PYGZgt{} cinnamon\PYGZhy{}settings\PYGZhy{}daemon\PYGZhy{}mouse.desktop
\end{sphinxVerbatim}

\sphinxAtStartPar
\# Отключение службы настройки энергосбережения Cinnamon. Можете оставить эту службу если у вас НЕ ноутбук.:

\begin{sphinxVerbatim}[commandchars=\\\{\}]
\PYG{n+nb}{echo} \PYG{l+s+s2}{\PYGZdq{}Hidden=true\PYGZdq{}} \PYGZgt{}\PYGZgt{} cinnamon\PYGZhy{}settings\PYGZhy{}daemon\PYGZhy{}power.desktop
\end{sphinxVerbatim}

\sphinxAtStartPar
\# Отключаем службу интеграции работы буфера обмена c Cinnamon.

\begin{sphinxVerbatim}[commandchars=\\\{\}]
\PYG{n+nb}{echo} \PYG{l+s+s2}{\PYGZdq{}Hidden=true\PYGZdq{}} \PYGZgt{}\PYGZgt{} cinnamon\PYGZhy{}settings\PYGZhy{}daemon\PYGZhy{}clipboard.desktop
\end{sphinxVerbatim}

\sphinxAtStartPar
Если после отключения какой\sphinxhyphen{}либо из вышеперечисленных служб что\sphinxhyphen{}то пошло не так, или просто какую\sphinxhyphen{}либо из них понадобилось снова включить, просто пропишите::

\begin{sphinxVerbatim}[commandchars=\\\{\}]
rm \PYGZhy{}rf \PYGZti{}/.config/autostart/cinnamon\PYGZhy{}settings\PYGZhy{}daemon\PYGZhy{}СЛУЖБА.desktop
\end{sphinxVerbatim}

\sphinxAtStartPar
Это вернет нужную службу в строй после перезагрузки.

\begin{sphinxadmonition}{attention}{Внимание:}
\sphinxAtStartPar
Если вы по\sphinxhyphen{}прежнему использовали старый способ с переименованием библиотек,
то настоятельно рекомендуется выполнить переустановку пакета cinnamon\sphinxhyphen{}settings\sphinxhyphen{}daemon, а
затем выполнить отключение ненужных вам служб уже новым способом.
\end{sphinxadmonition}

\index{lowlatency@\spxentry{lowlatency}}\index{compositor@\spxentry{compositor}}\index{muffin@\spxentry{muffin}}\index{x11\sphinxhyphen{}unrediction@\spxentry{x11\sphinxhyphen{}unrediction}}\index{vsync@\spxentry{vsync}}\ignorespaces 

\subsection{Настройка композитора Muffin}
\label{\detokenize{source/de-optimizations:muffin}}\label{\detokenize{source/de-optimizations:lowlatency-muffin}}\label{\detokenize{source/de-optimizations:index-24}}
\sphinxAtStartPar
По традиции, настроим композитор оболочки. В случае с Cinnamon это Muffin.
Он не содержит много настроек, и его нельзя заменить на другой композитор как мы это делали с Xfwm.
По сути, вся настройка Muffin сводиться к двум банальным, и уже нам знакомым, параметрам:
\sphinxstyleemphasis{"Метод Vsync (Вертикальная Синхронизация)"} и \sphinxstyleemphasis{"Отключение композитора для полноэкранных окон"}.

\noindent\sphinxincludegraphics{{image10}.png}

\sphinxAtStartPar
\sphinxstyleemphasis{"Отключение композитора для полноэкранных окон"} \sphinxhyphen{} Это уже знакомая вам опция, где из названия все понятно.
Вкратце, нужна для уменьшения задержек в видеоиграх создаваемых композитором.

\sphinxAtStartPar
\sphinxstyleemphasis{"Метод Vsync"} \sphinxhyphen{} параметр задающий метод синхронизации кадров.

\sphinxAtStartPar
Впрочем, в случае с Muffin, скорее не метод, а ее поведение. Всего есть четыре возможных значения:
\begin{enumerate}
\sphinxsetlistlabels{\arabic}{enumi}{enumii}{}{.}%
\item {} 
\sphinxAtStartPar
"None" \sphinxhyphen{} Отключение вертикальной синхронизации.
Более подробно мы рассматривали применимость этого значения в разделе с Plasma и Xfce.
Наиболее рекомендуется пользователям ноутбуков с активированным NVIDIA PRIME Sync или обладателям AMD Freesync и NVIDIA G Sync.
Помогает избегать высоких задержек и input lag’a.

\item {} 
\sphinxAtStartPar
\sphinxstyleemphasis{"Fallback / Classic"} \sphinxhyphen{} Классический метод вертикальной синхронизации, используемый в ранних версиях Cinnamon.

\item {} 
\sphinxAtStartPar
\sphinxstyleemphasis{"Swap Throttling"} \sphinxhyphen{} Обеспечивает вертикальную синхронизацию с учетом родной частоты обновления вашего монитора.
Лучше всего совместим с не\sphinxhyphen{}дисплеями (т.е. мониторами).

\item {} 
\sphinxAtStartPar
"Presentation Time" \sphinxhyphen{} Может осуществлять вертикальную синхронизацию сразу нескольких устройств с разной частотой обновления (Герцовкой).
Рекомендуется включить, если вы используете более одного монитора или дисплея.

\end{enumerate}

\index{lowlatency@\spxentry{lowlatency}}\index{compositor@\spxentry{compositor}}\index{muffin@\spxentry{muffin}}\index{effects@\spxentry{effects}}\ignorespaces 

\subsection{Отключение ненужных эффектов Muffin}
\label{\detokenize{source/de-optimizations:disabling-muffin-effects}}\label{\detokenize{source/de-optimizations:index-25}}\label{\detokenize{source/de-optimizations:id11}}
\sphinxAtStartPar
К сожалению, по умолчанию в Muffin отсутствует опция отключения сразу всех графических эффектов в оболочке (т.е. композитинга).
Поэтому, нам нужно отключить их поочередно в соответствующем разделе настроек \sphinxstyleemphasis{"Эффекты"}.

\noindent\sphinxincludegraphics{{image6}.png}

\sphinxAtStartPar
Желательно, в целях максимальной экономии аппаратных ресурсов, отключить все имеющийся здесь эффекты.
Но вы можете сделать это также и выборочно. И как обычно: Чем меньше эффектов включено \sphinxhyphen{}> Тем меньше потребление ресурсов ОЗУ и VRAM.

\index{results@\spxentry{results}}\ignorespaces 

\subsection{Результат}
\label{\detokenize{source/de-optimizations:cinnamon-result}}\label{\detokenize{source/de-optimizations:index-26}}\label{\detokenize{source/de-optimizations:id12}}
\noindent\sphinxincludegraphics{{image31}.png}

\sphinxstepscope


\chapter{Полезные программы}
\label{\detokenize{source/useful-programs:useful-programs}}\label{\detokenize{source/useful-programs:id1}}\label{\detokenize{source/useful-programs::doc}}
\sphinxAtStartPar
Программы разного назначения, однако могут быть полезными.

\index{useful\sphinxhyphen{}programs@\spxentry{useful\sphinxhyphen{}programs}}\index{stacer@\spxentry{stacer}}\index{garbage\sphinxhyphen{}removal@\spxentry{garbage\sphinxhyphen{}removal}}\ignorespaces 

\section{Stacer}
\label{\detokenize{source/useful-programs:stacer}}\label{\detokenize{source/useful-programs:index-0}}\label{\detokenize{source/useful-programs:id2}}
\sphinxAtStartPar
Помощник в обслуживании и чистке системы.

\noindent\sphinxincludegraphics{{generic-system-acceleration-3}.png}

\sphinxAtStartPar
\sphinxstylestrong{Установка}:

\begin{sphinxVerbatim}[commandchars=\\\{\}]
git clone https://aur.archlinux.org/stacer.git \PYG{c+c1}{\PYGZsh{} Скачивание исходников.}
\PYG{n+nb}{cd} stacer                                      \PYG{c+c1}{\PYGZsh{} Переход в stacer.}
makepkg \PYGZhy{}sric                                  \PYG{c+c1}{\PYGZsh{} Сборка и установка.}
\end{sphinxVerbatim}

\index{useful\sphinxhyphen{}programs@\spxentry{useful\sphinxhyphen{}programs}}\index{bleachbit@\spxentry{bleachbit}}\index{garbage\sphinxhyphen{}removal@\spxentry{garbage\sphinxhyphen{}removal}}\ignorespaces 

\section{Bleachbit}
\label{\detokenize{source/useful-programs:bleachbit}}\label{\detokenize{source/useful-programs:index-1}}\label{\detokenize{source/useful-programs:id3}}
\sphinxAtStartPar
Аналог CCleaner для Linux, помогает выполнить очистку системы от накопившегося в ней мусора.

\sphinxAtStartPar
Советуем выполнять чистку системы уже после проведения всех оптимизаций.

\noindent\sphinxincludegraphics{{generic-system-acceleration-4}.png}

\sphinxAtStartPar
\sphinxstylestrong{Установка + дополнительные фильтры}:

\begin{sphinxVerbatim}[commandchars=\\\{\}]
sudo pacman \PYGZhy{}S bleachbit

\PYG{c+c1}{\PYGZsh{} Дополнительные фильтры}

git clone https://aur.archlinux.org/cleanerml\PYGZhy{}git.git \PYG{c+c1}{\PYGZsh{} Загрузка исходников.}
\PYG{n+nb}{cd} cleanerml\PYGZhy{}git                                      \PYG{c+c1}{\PYGZsh{} Переход в cleanerm.}
makepkg \PYGZhy{}sric                                         \PYG{c+c1}{\PYGZsh{} Сборка и установка.}
\end{sphinxVerbatim}

\index{useful\sphinxhyphen{}programs@\spxentry{useful\sphinxhyphen{}programs}}\index{mouse@\spxentry{mouse}}\index{settings@\spxentry{settings}}\ignorespaces 

\section{Piper}
\label{\detokenize{source/useful-programs:piper}}\label{\detokenize{source/useful-programs:paper}}\label{\detokenize{source/useful-programs:index-2}}
\sphinxAtStartPar
Позволяет выполнить более тонкую настройку вашей мышки, в том числе переназначить DPI, настроить подсветку и собственные действия на дополнительные кнопки.

\noindent\sphinxincludegraphics{{piper-resolutionpage}.png}

\sphinxAtStartPar
\sphinxstylestrong{Установка}

\begin{sphinxVerbatim}[commandchars=\\\{\}]
sudo pacman \PYGZhy{}S piper
\end{sphinxVerbatim}

\begin{sphinxadmonition}{attention}{Внимание:}
\sphinxAtStartPar
Поддерживаются только некоторые из моделей мышек от Logitech/Razer/Steelseries.
Полный список поддерживаемых устройств вы можете найти по ссылке:

\sphinxAtStartPar
\sphinxurl{https://github.com/libratbag/libratbag/wiki/Devices}
\end{sphinxadmonition}

\index{useful\sphinxhyphen{}programs@\spxentry{useful\sphinxhyphen{}programs}}\index{usb@\spxentry{usb}}\index{security@\spxentry{security}}\ignorespaces 

\section{pam\_usb}
\label{\detokenize{source/useful-programs:pam-usb}}\label{\detokenize{source/useful-programs:index-3}}\label{\detokenize{source/useful-programs:id4}}
\sphinxAtStartPar
Позволяет сделать из вашей USB\sphinxhyphen{}флешки ключ для авторизации в вашу систему.
Совместим с экранными менеджерами входа GDM и KDM.

\sphinxAtStartPar
Существует несколько режимов работы:
\begin{enumerate}
\sphinxsetlistlabels{\arabic}{enumi}{enumii}{}{.}%
\item {} 
\sphinxAtStartPar
Использовать флешку вместо пароля, при условии её подключения (если подключение отсутствует \sphinxhyphen{} нужно вводить пароль)

\item {} 
\sphinxAtStartPar
Требовать наличие подключенного USB\sphinxhyphen{}носителя вместе с вводом пароля.

\end{enumerate}

\sphinxAtStartPar
\sphinxstylestrong{Установка}

\begin{sphinxVerbatim}[commandchars=\\\{\}]
git clone https://github.com/mcdope/pam\PYGZus{}usb.git
\PYG{n+nb}{cd} pam\PYGZus{}usb/arch\PYGZus{}linux
makepkg \PYGZhy{}sric
\end{sphinxVerbatim}

\index{useful\sphinxhyphen{}programs@\spxentry{useful\sphinxhyphen{}programs}}\index{wine@\spxentry{wine}}\index{prefixes@\spxentry{prefixes}}\index{gaming@\spxentry{gaming}}\ignorespaces 

\section{Bottles}
\label{\detokenize{source/useful-programs:bottles}}\label{\detokenize{source/useful-programs:index-4}}\label{\detokenize{source/useful-programs:id5}}
\sphinxAtStartPar
Удобный менеджер по управлению бутылками (префиксами) в Wine. Альтернатива Lutris, имеет приятный и понятный интерфейс,
возможность графической установки зависимостей (DLL библиотек) и поддерживает изоляцию из коробки.

\sphinxAtStartPar
\sphinxstylestrong{Демонстрация}
\begin{enumerate}
\sphinxsetlistlabels{\arabic}{enumi}{enumii}{}{.}%
\item {} 
\sphinxAtStartPar
Окно выбора бутылки

\end{enumerate}

\noindent\sphinxincludegraphics{{generic-system-acceleration-5}.png}
\begin{enumerate}
\sphinxsetlistlabels{\arabic}{enumi}{enumii}{}{.}%
\setcounter{enumi}{1}
\item {} 
\sphinxAtStartPar
Создание новой бутылки

\end{enumerate}

\noindent\sphinxincludegraphics{{generic-system-acceleration-6}.png}
\begin{enumerate}
\sphinxsetlistlabels{\arabic}{enumi}{enumii}{}{.}%
\setcounter{enumi}{2}
\item {} 
\sphinxAtStartPar
Управление бутылкой

\end{enumerate}

\noindent\sphinxincludegraphics{{generic-system-acceleration-7}.png}
\begin{enumerate}
\sphinxsetlistlabels{\arabic}{enumi}{enumii}{}{.}%
\setcounter{enumi}{3}
\item {} 
\sphinxAtStartPar
Установка зависимостей (DLL библиотек)

\end{enumerate}

\noindent\sphinxincludegraphics{{generic-system-acceleration-8}.png}

\sphinxAtStartPar
\sphinxstylestrong{Установка}

\begin{sphinxVerbatim}[commandchars=\\\{\}]
git clone https://aur.archlinux.org/bottles.git \PYG{c+c1}{\PYGZsh{} Скачиваем исходники}
\PYG{n+nb}{cd} bottles                                      \PYG{c+c1}{\PYGZsh{} Переход в директорию}
makepkg \PYGZhy{}sric                                   \PYG{c+c1}{\PYGZsh{} Сборка и установка}
\end{sphinxVerbatim}



\renewcommand{\indexname}{Алфавитный указатель}
\printindex
\end{document}